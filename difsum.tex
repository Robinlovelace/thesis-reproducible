\documentclass[a4paper, 11pt]{article}  % Based on the ECS Thesis style
\usepackage{makeidx}
\usepackage{multirow}
\usepackage{threeparttable}
\usepackage{idxlayout}
% Making R code work!
\usepackage{listings}
\usepackage{color}
\usepackage{hyperref}
\hypersetup{urlcolor=blue, colorlinks=false, hypertexnames=true}  % Colours hyperlinks in blue, but this can be distracting
\usepackage{cleveref}
\definecolor{dkgreen}{rgb}{0,0.6,0}
\definecolor{gray}{rgb}{0.5,0.5,0.5}
\definecolor{mauve}{rgb}{0.58,0,0.82}

\setlength{\parindent}{0cm}
\setlength{\parskip}{0.3cm}

\lstset{ %
  language=R,                % the language of the code
   basicstyle=\normalsize\ttfamily,           % the size of the fonts that are used for the code
%   numbers=left,                   % where to put the line-numbers
%   numberstyle=\tiny\color{gray},  % the style that is used for the line-numbers
%   stepnumber=2,                   % the step between two line-numbers. If it's 1, each line
                                  % will be numbered
%   numbersep=5pt,                  % how far the line-numbers are from the code
%   backgroundcolor=\color{white},      % choose the background color. You must add \usepackage{color}
%   showspaces=false,               % show spaces adding particular underscores
%   showstringspaces=false,         % underline spaces within strings
%   showtabs=false,                 % show tabs within strings adding particular underscores
   frame=false,                   % adds a frame around the code
   rulecolor=\color{white},        % if not set, the frame-color may be changed on line-breaks within not-black text (e.g. commens (green here))
%   tabsize=2,                      % sets default tabsize to 2 spaces
%   captionpos=b,                   % sets the caption-position to bottom
%   breaklines=true,                % sets automatic line breaking
%   breakatwhitespace=false,        % sets if automatic breaks should only happen at whitespace
%   title=\lstname,                   % show the filename of files included with \lstinputlisting;
                                  % also try caption instead of title
  keywordstyle=\color{blue},          % keyword style
  commentstyle=\color{dkgreen},       % comment style
  stringstyle=\color{mauve},         % string literal style
  escapeinside={\%*}{*)},            % if you want to add a comment within your code
  morekeywords={*,...}               % if you want to add more keywords to the set
}

% Include any extra LaTeX packages required
\usepackage[round,]{natbib}  % Use the "Natbib" style for the references
\usepackage{verbatim}  % Needed for the "comment" environment to make LaTeX comments
\usepackage{wallpaper}
\usepackage{cases}
\makeindex
%% ----------------------------------------------------------------
%DIF PREAMBLE EXTENSION ADDED BY LATEXDIFF
%DIF UNDERLINE PREAMBLE %DIF PREAMBLE
\RequirePackage[normalem]{ulem} %DIF PREAMBLE
\RequirePackage{color}\definecolor{RED}{rgb}{1,0,0}\definecolor{BLUE}{rgb}{0,0,1} %DIF PREAMBLE
\providecommand{\DIFaddtex}[1]{{\protect\color{blue}\uwave{#1}}} %DIF PREAMBLE
\providecommand{\DIFdeltex}[1]{{\protect\color{red}\sout{#1}}}                      %DIF PREAMBLE
%DIF SAFE PREAMBLE %DIF PREAMBLE
\providecommand{\DIFaddbegin}{} %DIF PREAMBLE
\providecommand{\DIFaddend}{} %DIF PREAMBLE
\providecommand{\DIFdelbegin}{} %DIF PREAMBLE
\providecommand{\DIFdelend}{} %DIF PREAMBLE
%DIF FLOATSAFE PREAMBLE %DIF PREAMBLE
\providecommand{\DIFaddFL}[1]{\DIFadd{#1}} %DIF PREAMBLE
\providecommand{\DIFdelFL}[1]{\DIFdel{#1}} %DIF PREAMBLE
\providecommand{\DIFaddbeginFL}{} %DIF PREAMBLE
\providecommand{\DIFaddendFL}{} %DIF PREAMBLE
\providecommand{\DIFdelbeginFL}{} %DIF PREAMBLE
\providecommand{\DIFdelendFL}{} %DIF PREAMBLE
%DIF END PREAMBLE EXTENSION ADDED BY LATEXDIFF
%DIF PREAMBLE EXTENSION ADDED BY LATEXDIFF
%DIF HYPERREF PREAMBLE %DIF PREAMBLE
\providecommand{\DIFadd}[1]{\texorpdfstring{\DIFaddtex{#1}}{#1}} %DIF PREAMBLE
\providecommand{\DIFdel}[1]{\texorpdfstring{\DIFdeltex{#1}}{}} %DIF PREAMBLE
%DIF END PREAMBLE EXTENSION ADDED BY LATEXDIFF

\title{Thesis corrections: summary of changes}
\author{Robin Lovelace}

\begin{document}

\maketitle

\section{Introduction}

Many changes have been made to the thesis following the viva, all of them
minor typos and formatting issues. These can be split into 3 main parts:
corrections suggested by Charles Pattie, corrections from Michael Batty
and additional changes.

Additions are highlighted in \DIFadd{blue} text. Deletions are \DIFdel{crossed-out in red}.

A number of issues applied throughout the entire document, so not every instance of a change is
mentioned here. These systematic changes include:
\begin{itemize}
 \item ``fig.'' replaced by ``figure'' throughout
 \item ``individual-level'' replaced by ``individual level'' throughout and other hyphenation mistakes corrected
 \item data are now treated as plural, although there is an ongoing debate on the subject e.g.
 \href{http://www.theguardian.com/news/datablog/2010/jul/16/data-plural-singular}{http://www.theguardian.com/news/datablog/2010/jul/16/data-plural-singular}
 \item ``1990's'' replace by ``1990s'' throughout and for all decades
\end{itemize}

More changes than those described here have been made:
this document is a summary of changes. To see all changes
see the file ``dif.pdf'' in the
``thesis-reproducible'' GitHub repository ( \href{https://github.com/Robinlovelace/thesis-reproducible}{https://github.com/Robinlovelace/thesis-reproducible} ).


\section{Corrections from Charles Pattie}


 \DIFdelbegin \DIFdel{temping }\DIFdelend \DIFaddbegin \DIFadd{tempting}\DIFaddend, page 2.

\DIFdelbegin \DIFdel{On }\DIFdelend \DIFaddbegin \DIFadd{In }\DIFaddend addition ... Page 9.

\pounds900 bicycle through \DIFdelbegin \DIFdel{them}\DIFdelend \DIFaddbegin \DIFadd{it}\DIFaddend ... Page 15.

\DIFdelbegin \DIFdel{bound-up }\DIFdelend \DIFaddbegin \DIFadd{bound up }\DIFaddend... Page 24.

The energy costs of commuting \DIFdelbegin \DIFdel{is }\DIFdelend \DIFaddbegin \DIFadd{are }\DIFaddend... Page 25.

\DIFdelbegin \DIFdel{social-theory }\DIFdelend \DIFaddbegin \DIFadd{social theory }\DIFaddend... Page 25.

Clearly, it is not money per se that affects commuting energy costs, but
its indirect influence on behaviour\DIFdelbegin \DIFdel{, which is rarely direct}\DIFdelend ... Page 32.

A wide range
\DIFaddbegin \DIFadd{of }\DIFaddend individual and geographical
factors. Page 35.

Building on these findings, \DIFdelbegin \DIFdel{\mbox{%DIFAUXCMD
\citep{Levinson2012}
}%DIFAUXCMD
}\DIFdelend \DIFaddbegin \DIFadd{\mbox{%DIFAUXCMD
\citet{Levinson2012}
}%DIFAUXCMD
}\DIFaddend returned to the question ... Page 36.


\citet{Simonsen2011} \DIFdelbegin \DIFdel{harnesses }\DIFdelend \DIFaddbegin \DIFadd{harness }\DIFaddend the knowledge that energy use
and greenhouse gas emissions are two sides of the same coin ... Page 37.

on transport \DIFdelbegin \DIFdel{an }\DIFdelend \DIFaddbegin \DIFadd{and }\DIFaddend energy... Page 38.

since the \DIFdelbegin \DIFdel{1800's }\DIFdelend \DIFaddbegin \DIFadd{1800s }\DIFaddend... Page 39.

\citet{Gabrielli1950} and \DIFdelbegin \DIFdel{its }\DIFdelend \DIFaddbegin \DIFadd{their }\DIFaddend successors... Page 39.

[The impacts] \DIFdelbegin \DIFdel{was }\DIFdelend \DIFaddbegin \DIFadd{were }\DIFaddend determined ... Page 42.

\citet{Sexton2011} \DIFdelbegin \DIFdel{paper }\DIFdelend set out to ... Page 43.

the need to \DIFdelbegin \DIFdel{asses }\DIFdelend \DIFaddbegin \DIFadd{assess }\DIFaddend potential future impacts... Page 43.

The authors tested their results against the \DIFdelbegin \DIFdel{another }\DIFdelend \DIFaddbegin \DIFadd{radiation }\DIFaddend model and
found that \DIFdelbegin \DIFdel{their model }\DIFdelend \DIFaddbegin \DIFadd{theirs }\DIFaddend ``yields significantly better results''
\citep[p.~6]{Lenormandplosone2012}\DIFdelbegin \DIFdel{than than the `radiation model'}\DIFdelend . It is to
this \DIFaddbegin \DIFadd{radiation }\DIFaddend model, another scientific approach to commuting,
that attention is directed below... Page 45.

\DIFdelbegin \DIFdel{read-through }\DIFdelend \DIFaddbegin \DIFadd{read through }\DIFaddend Page 46.


In \DIFaddbegin \DIFadd{the }\DIFaddend sustainability
literature, \DIFaddbegin \DIFadd{the }\DIFaddend term is rarely quantified ... Page 50.

In the context of transport \DIFdelbegin \DIFdel{system}\DIFdelend \DIFaddbegin \DIFadd{systems}\DIFaddend ... Page 50.

is \DIFdelbegin \DIFdel{is }\DIFdelend central to the thesis. ... Page 53.

frequently used \DIFdelbegin \DIFdel{in }\DIFdelend throughout the thesis... Page 53.

\DIFdelbegin \DIFdel{short-term}\DIFdelend \DIFaddbegin \DIFadd{short term}\DIFaddend ... Page 56.

\DIFdelbegin \DIFdel{make-do }\DIFdelend \DIFaddbegin \DIFadd{make do }\DIFaddend ... Page 58.

empirical data seldom \DIFdelbegin \DIFdel{fits }\DIFdelend \DIFaddbegin \DIFadd{fit }\DIFaddend into any neat model and therefore
\DIFdelbegin \DIFdel{distracts }\DIFdelend \DIFaddbegin \DIFadd{distract }\DIFaddend from explanation.
This point was made as early as the \DIFdelbegin \DIFdel{1970's}\DIFdelend \DIFaddbegin \DIFadd{1970s}\DIFaddend ... Page 59.

the
\DIFdelbegin \DIFdel{the
}\DIFdelend car ... Page 62.

the field \DIFaddbegin \DIFadd{of }\DIFaddend statistical
mechanics ... Page 63.

to \DIFdelbegin \DIFdel{the }\DIFdelend tackle the research challenge ... Page 66.

\DIFdelbegin \DIFdel{eduction}\DIFdelend \DIFaddbegin \DIFadd{level of education}\DIFaddend ... Page 69.

\DIFdelbegin \DIFdel{ground-up }\DIFdelend \DIFaddbegin \DIFadd{ground up }\DIFaddend ... Page 74.

used for \DIFaddbegin \DIFadd{a }\DIFaddend range of applications ... Page 75.

During
the 3$^{rd}$ quarter of 2012, there were 29.86 \DIFaddbegin \DIFadd{million
}\DIFaddend employed
people in the UK ... Page 79.

\DIFdelbegin \DIFdel{be harder
more difficult }\DIFdelend \DIFaddbegin \DIFadd{harder
}\DIFaddend to analyse and visualise ... Page 86.

\DIFdelbegin \DIFdel{split-up }\DIFdelend \DIFaddbegin \DIFadd{split up }\DIFaddend ... Page 88.

The NTS dataset has an impressive response rate to key question which tend to
have a lot of NA values, and are very patchy\DIFdelbegin \DIFdel{)}\DIFdelend ... Page 95.

\DIFdelbegin \DIFdel{an }\DIFdelend \DIFaddbegin \DIFadd{a }\DIFaddend geo-referenced ... Page 97.

\DIFdelbegin \DIFdel{could-well }\DIFdelend \DIFaddbegin \DIFadd{could well }\DIFaddend ... Page 97.

The \DIFdelbegin \DIFdel{data }\DIFdelend \DIFaddbegin \DIFadd{datasets }\DIFaddend presented so far, on energy use of personal travel and commuting
behaviour, \DIFdelbegin \DIFdel{is }\DIFdelend \DIFaddbegin \DIFadd{are }\DIFaddend sufficient ... . \emph{And changed throughout}. Page 97.

`\DIFdelbegin \DIFdel{drop-off }\DIFdelend \DIFaddbegin \DIFadd{drop off }\DIFaddend the kids' ... Page 97.

as buses and railed vehicles can only follow
pre-defined \DIFdelbegin \DIFdel{path}\DIFdelend \DIFaddbegin \DIFadd{paths}\DIFaddend ... Page 98.

\DIFdelbegin \DIFdel{\mbox{%DIFAUXCMD
...
}%DIFAUXCMD
}\DIFdelend \DIFaddbegin \DIFadd{\mbox{%DIFAUXCMD
\citep{Johnston1993, blien1998entropy}
}%DIFAUXCMD
}\DIFaddend ... Page 110.

\DIFdelbegin \DIFdel{download }\DIFdelend \DIFaddbegin \DIFadd{downloaded }\DIFaddend from census dissemination
portal Casweb ... Page 110.

perfect \DIFaddbegin \DIFadd{(}\DIFaddend fig. 4.14 \DIFaddbegin \DIFadd{)}\DIFaddend ... Page 118.

(see 4.25 \DIFdelbegin \DIFdel{, above}\DIFdelend )... Page 122.

household incomes \DIFdelbegin \DIFdel{and }\DIFdelend \DIFaddbegin \DIFadd{at }\DIFaddend the MSOA level ... Page 124.

to \DIFdelbegin \DIFdel{to }\DIFdelend a fixed randomising factor ... Page 126.

Because simulation, almost by definition, estimates something that is not
otherwise \DIFdelbegin \DIFdel{know}\DIFdelend \DIFaddbegin \DIFadd{known}\DIFaddend  ... Page 128.

\DIFdelbegin \DIFdel{Intergerisation }\DIFdelend \DIFaddbegin \DIFadd{Integerisation }\DIFaddend ... Page 130.

\DIFdelbegin \DIFdel{tops-up }\DIFdelend \DIFaddbegin \DIFadd{tops up }\DIFaddend ... Page 132.

does one only include the
chemical energy stored in the petrol burned in the pistons? \DIFdelbegin \DIFdel{or }\DIFdelend \DIFaddbegin \DIFadd{Or }\DIFaddend do we also
include the primary energy consumed ... Page 151.

Brackets added around figure. Page 152.

Inserted brackets around table ... Page 155.

Although \DIFaddbegin \DIFadd{the }\DIFaddend Defra dataset  ... Page 155.

The closeness of the average energy costs of driving reported by \DIFdelbegin \DIFdel{\mbox{%DIFAUXCMD
\citep{MacKay2009}
}%DIFAUXCMD
}\DIFdelend \DIFaddbegin \DIFadd{\mbox{%DIFAUXCMD
\citet{MacKay2009}
}%DIFAUXCMD
}\DIFaddend ... Page 157.

he first paper that formalised this problem in the context of \DIFdelbegin \DIFdel{system-level
}\DIFdelend \DIFaddbegin \DIFadd{system level
}\DIFaddend energy costs of transport was \DIFdelbegin \DIFdel{\mbox{%DIFAUXCMD
\citep{Fels1975}
}%DIFAUXCMD
}\DIFdelend \DIFaddbegin \DIFadd{by \mbox{%DIFAUXCMD
\citet{Fels1975}
}%DIFAUXCMD
}\DIFaddend ... Page 160.

exceptions include
\DIFdelbegin \DIFdel{\mbox{%DIFAUXCMD
...
}%DIFAUXCMD
}\DIFdelend \DIFaddbegin \DIFadd{\mbox{%DIFAUXCMD
\citealp{Treloar2004, Lenzen1999, MacKay2009, Lovelace2011-assessing}
}%DIFAUXCMD
)}\DIFaddend Page 160.

It is all \DIFdelbegin \DIFdel{to }\DIFdelend \DIFaddbegin \DIFadd{too }\DIFaddend easy ... Page 161.

Inserted brackets around figure Page 175.

model was \DIFdelbegin \DIFdel{ran }\DIFdelend \DIFaddbegin \DIFadd{run }\DIFaddend ... Page 175.

hotels \DIFdelbegin \DIFdel{, }\DIFdelend \DIFaddbegin \DIFadd{and }\DIFaddend restaurants ... Page 176.

students trace roadways on paper maps'' \citep[189]{bosco2012circs}.
\DIFaddbegin \DIFadd{The original is not cited in the quote because a hard copy of the book
could not be found. Therefore the quotation, from \mbox{%DIFAUXCMD
is used.
}\DIFaddend } ... Page 183.

Building on these findings, \DIFdelbegin \DIFdel{\mbox{%DIFAUXCMD
\citep{Levinson2012}
}%DIFAUXCMD
}\DIFdelend \DIFaddbegin \DIFadd{\mbox{%DIFAUXCMD
\citet{Levinson2012}
}%DIFAUXCMD
}\DIFaddend returned to the question of the factors affecting commute time ... Page 183.

\DIFdelbegin \DIFdel{The }\DIFdelend \DIFaddbegin \DIFadd{An }\DIFaddend indirect (yet somehow more tangible) impact of poor road quality on
transport energy use is that it can discourage \DIFdelbegin \DIFdel{from low energy vehicles}\DIFdelend \DIFaddbegin \DIFadd{people from buying
a low powered and energy efficient car}\DIFaddend ... Page 190.

altering the values for
one \DIFdelbegin \DIFdel{modes }\DIFdelend \DIFaddbegin \DIFadd{mode }\DIFaddend whilst leaving the others unchanged ... Page 191.

\DIFdelbegin \DIFdel{between }\DIFdelend in  the decade following 1973... Page 192.

more than 10\% below the 3 MJ/km figure calculated from
\DIFdelbegin \DIFdel{\mbox{%DIFAUXCMD
\citep{Defra2011}
}%DIFAUXCMD
}\DIFdelend \DIFaddbegin \DIFadd{\mbox{%DIFAUXCMD
\citet{Defra2011}
}%DIFAUXCMD
}\DIFaddend ... Page 192.

Bracket inserted. Page 195.

Additional strain on \DIFdelbegin \DIFdel{a }\DIFdelend \DIFaddbegin \DIFadd{an }\DIFaddend ailing electricity grid ... Page 199.

This is the argument made powerfully by \DIFdelbegin \DIFdel{\mbox{%DIFAUXCMD
\citep{plowden2008cars}
}%DIFAUXCMD
}\DIFdelend \DIFaddbegin \DIFadd{\mbox{%DIFAUXCMD
\citet{plowden2008cars}
}%DIFAUXCMD
}\DIFaddend ... Page 199.

Which of these
approaches to projecting fleet efficiencies \DIFaddbegin \DIFadd{is most is context specific and
}\DIFaddend depends on the aims of the research ... Page 200.

\DIFdelbegin \DIFdel{YatH }\DIFdelend \DIFaddbegin \DIFadd{Yorkshire and the Humber }\DIFaddend was taken ... Page 205.

Brackets added. Page 207.

come \DIFdelbegin \DIFdel{for }\DIFdelend \DIFaddbegin \DIFadd{from }\DIFaddend official data sources ... Page 208.

``Having considered the limitations of the data, and weighed up the pros additional of
complexity with the advantages of simplicity''  replaced by ``Having considered the limitations of the data,
and weighed up the costs and benefits of complexity'' ... Page 212.

```card', `carp', and `moto' refer to
car driver, car passenger and motorbike respectively.'' Added to explain the acronyms. Page 214.

It is interesting to \DIFdelbegin \DIFdel{not }\DIFdelend \DIFaddbegin \DIFadd{note }\DIFaddend ... Page 219.

The highest and lowest (outside London)
values are found in Rutland (the geographic centroid of which is located 109
km from central London, and which was the last county in England to have
a direct trainline to London) \DIFaddbegin \DIFadd{and }\DIFaddend the City of Kingston upon Hull\DIFaddbegin \DIFadd{, respectively}\DIFaddend 
... Page 219.

there is a tendency for
areas located close to \DIFaddbegin \DIFadd{railways
to }\DIFaddend be associated with a high proportion of per trip energy
use to be composed of rail travel ... Page 221.

\DIFdelbegin \DIFdel{course }\DIFdelend \DIFaddbegin \DIFadd{coarse }\DIFaddend levels of aggregation ... Page 223.

something \DIFdelbegin \DIFdel{for }\DIFdelend which ... Page 223.

Local \DIFdelbegin \DIFdel{authorities }\DIFdelend \DIFaddbegin \DIFadd{Authorities }\DIFaddend ... Page 224.

When distance band and mode of travel are \DIFdelbegin \DIFdel{know }\DIFdelend \DIFaddbegin \DIFadd{known }\DIFaddend ... Page 227.

the lowest classes \DIFdelbegin \DIFdel{live }\DIFdelend \DIFaddbegin \DIFadd{tend to work }\DIFaddend closer to home ... Page 231.

Another issue was finding \DIFdelbegin \DIFdel{Geographical }\DIFdelend \DIFaddbegin \DIFadd{geographical }\DIFaddend ... Page 233.

\DIFdelbegin \DIFdel{the }\DIFdelend also ... Page 236.

That energy use per commute is greater in the Netherlands
\DIFdelbegin \DIFdel{is
higher }\DIFdelend than in England is an interesting result in itself ... Page 237.

Brackets inserted. Page 243.

as aggregate data \DIFdelbegin \DIFdel{tells
}\DIFdelend \DIFaddbegin \DIFadd{tell
}\DIFaddend us little. Page 244.

\DIFdelbegin \DIFdel{the }\DIFdelend Sheffield028 (an MSOA zone) is more unequal ... Page 255.


Figure 7.14 size increased. Page 261.

the proportion of car trips \DIFdelbegin \DIFdel{that shift to car }\DIFdelend \DIFaddbegin \DIFadd{replaced by bicycle }\DIFaddend trips
... Page 267.

cycling uptake is driven largely \DIFdelbegin \DIFdel{be }\DIFdelend \DIFaddbegin \DIFadd{by }\DIFaddend the
young ... Page 269.

Suburbia in its current form gradually vanishes, and communities
 will become ```villagised' so people could meet more of their needs from their
neighbourhood without commuting'' \citep[p.~591]{North2010585}.
 \DIFdelbegin \DIFdel{around urban areas, while avoidable long distance travel would be
 }\DIFdelend ... Page 276.

 \DIFdelbegin \DIFdel{R$^2$ values }\DIFdelend \DIFaddbegin \DIFadd{Pearson's coefficient of correlation
(r) }\DIFaddend ranged from -0.59 to -0.22 ... Page 283.

\section{Corrections from Michael Batty}

due \DIFdelbegin \DIFdelend \DIFaddbegin \DIFadd{to}\DIFaddend lack of jobs ... Page iv.

roughly equivalent to   \DIFdelbegin \DIFdel{country}\DIFdelend \DIFaddbegin \DIFadd{county}\DIFaddend level in the UK ... Page 35.

``than those'' added to the following sentence: It was found that people living in more sparsely populated areas
were more likely to travel far to work than those living in dense areas. Page 35.

Moreover, it is assumed that the former is a \DIFdelbegin \DIFdel{a }\DIFdelend close enough proxy of the latter ... Page 37. 

This is a recurring \DIFdelbegin \DIFdel{them }\DIFdelend \DIFaddbegin \DIFadd{theme }\DIFaddend ... Page 39.

In \DIFaddbegin \DIFadd{the }\DIFaddend sustainability
literature, \DIFaddbegin \DIFadd{the }\DIFaddend term is rarely quantified ... Page 50.

changes \DIFdelbegin \DIFdel{on }\DIFdelend \DIFaddbegin \DIFadd{in }\DIFaddend the real spatial microdata ... Page 56.

Whole cases
are \DIFdelbegin \DIFdel{are }\DIFdelend generated using integerisation. Page 58.

\DIFdelbegin \DIFdel{microsimualtion }\DIFdelend \DIFaddbegin \DIFadd{microsimulation }\DIFaddend ... Page 75.

translating the total \DIFdelbegin \DIFdel{in }\DIFdelend into commuter energy use ... Page 84.

ggplot2 has been used throughout this thesis for plotting with help from key
 \DIFdelbegin \DIFdel{reference }\DIFdelend \DIFaddbegin \DIFadd{references }\DIFaddend ... Page 109.

 Because simulation, almost by definition, estimates something that is not
otherwise \DIFdelbegin \DIFdel{know}\DIFdelend \DIFaddbegin \DIFadd{known}\DIFaddend ... Page 128.

the true value is
only really \DIFdelbegin \DIFdel{know the value }\DIFdelend \DIFaddbegin \DIFadd{known }\DIFaddend to one significant figure. Page 168.

Another pattern that emerges is the relationship
between the very low energy costs of commuting in London, and the relatively high
costs of areas within a $\sim$100 km radius \DIFdelbegin \DIFdel{surround }\DIFdelend \DIFaddbegin \DIFadd{surrounding }\DIFaddend the centre ...
Page 219.

When distance band and mode of travel are \DIFdelbegin \DIFdel{know }\DIFdelend \DIFaddbegin \DIFadd{known }\DIFaddend ... Page 227.

Despite these possibilities, \DIFaddbegin \DIFadd{it }\DIFaddend is important to remember that the results are
\emph{simulated} ... Page 261.

\section{Additional changes}

Many small additional changes were made to the thesis, reflecting typos,
style issues and grammatical mistakes and inconsistencies found in the text.
A selection of these are highlighted below. Page numbers are not shown;
a full list of changes can be found in the file ``dif.pdf'' in the
``thesis-reproducible'' GitHub repository ( \href{https://github.com/Robinlovelace/thesis-reproducible}{https://github.com/Robinlovelace/thesis-reproducible} ).

\DIFdelend \DIFaddbegin \DIFadd{, except
the }\DIFaddend Cretaceous-Tertiary boundary event generally attributed to impact of an
asteroid with the Earth'' \citep{Hay2011}. 

This is because transport systems are inherently mobile,
therefore requiring \DIFaddbegin \DIFadd{a }\DIFaddend high energy density \DIFdelbegin \DIFdel{energy storage}\DIFdelend \DIFaddbegin \DIFadd{power source}\DIFaddend .

Energy is the `master resource' from which all others (including more
energy) can be obtained; emissions are the end result \DIFdelbegin \DIFdel{result }\DIFdelend of energy use.

Most greenhouse gas emissions stem from \DIFdelbegin \DIFdel{burning }\DIFdelend fossil fuel use, and once
extracted, these fuels are invariably burned. This has led to the conclusion

Overall commuting is
the \DIFdelbegin \DIFdel{the }\DIFdelend most time-consuming reason for personal travel in the UK

results to be replicated by anyone provided with the same input data as used in
the \DIFdelbegin \DIFdel{this }\DIFdelend thesis. To this end numerous script files are provided which allow many
of the analyses performed to be re-run on any computer using free
--
This recognition of the potential applications of the research
is reflected in the \DIFdelbegin \DIFdel{the }\DIFdelend aims and objectives.

The energy costs of commuting \DIFdelbegin \DIFdel{is }\DIFdelend \DIFaddbegin \DIFadd{are }\DIFaddend therefore of critical importance to
the ability of modern economies to sustain themselves.

characterising and
modelling active travel patterns \DIFdelbegin \DIFdel{also}\DIFdelend \DIFaddbegin \DIFadd{as well}\DIFaddend

e.g.~Chris Fisher's decision not to move to Hereford because
commuting to the \DIFdelbegin \DIFdel{Tyrrel}\DIFdelend \DIFaddbegin \DIFadd{Tyrrell}\DIFaddend 's crisp factory would then become too expensive

The energy \DIFdelbegin \DIFdel{costs }\DIFdelend \DIFaddbegin \DIFadd{cost }\DIFaddend of commuting

Theories are hypotheses about how the world \emph{should be}, based on
\DIFdelbegin \DIFdel{past }\DIFdelend experience

The \DIFdelbegin \DIFdel{the }\DIFdelend law is falsifiable (and has been falsified on numerous occasions!)

efficiency --- and related concepts of fuel economy and energy
\DIFdelbegin \DIFdel{economy
}\DIFdelend intensity --- is well established in research on the energy requirements
of \DIFdelbegin \DIFdel{of }\DIFdelend freight transport \citep{Kamakate2009}.

At present however, this quantitative branch of
the resilience concept lacks empirical application. The term \DIFdelbegin \DIFdel{will }\DIFdelend \DIFaddbegin \DIFadd{is }\DIFaddend harnessed to
discuss the \DIFdelbegin \DIFdel{the long-sustainability }\DIFdelend \DIFaddbegin \DIFadd{long term sustainability }\DIFaddend of commuter systems and their capacity to
function in the event of oil shortages.

\DIFdelbegin \DIFdel{Energy use in }\DIFdelend \DIFaddbegin \DIFadd{The energy costs of }\DIFaddend transport, and \DIFdelbegin \DIFdel{its }\DIFdelend \DIFaddbegin \DIFadd{their }\DIFaddend underlying causes, have been explored at a
range of different scales.

for energy and transport studies focussed more locally. Moreover,
because the factors affecting commuting behaviour operate at many \DIFdelbegin \DIFdel{different }\DIFdelend levels

In this chapter they
are \DIFdelbegin \DIFdel{are }\DIFdelend grouped together under the broad term `urban modelling'

the impacts of policy or other
changes \DIFdelbegin \DIFdel{on }\DIFdelend \DIFaddbegin \DIFadd{in }\DIFaddend the real spatial microdata
methods of analysis and \DIFdelbegin \DIFdel{and }\DIFdelend conventions of mathematical notation

one
advantage they seem to have had \DIFdelbegin \DIFdel{was }\DIFdelend a clear theoretical focus

strategic \DIFdelbegin \DIFdel{planing }\DIFdelend \DIFaddbegin \DIFadd{planning }\DIFaddend

hierarchy of entities for inclusion was established, in descending
order \DIFdelbegin \DIFdel{order }\DIFdelend of priority:

 Thus ILUTE can \DIFdelbegin \DIFdel{can }\DIFdelend be used
 
for spatial microsimulation, at various levels of geographic aggregation.
The \DIFdelbegin \DIFdel{disadvantage of census
data }\DIFdelend \DIFaddbegin \DIFadd{main disadvantage of the census
dataset }\DIFaddend is that it only provides information about a small number of variables
compared with more specific surveys that have lower samples sizes.

in addition to enabling statistical operations, \DIFdelbegin \DIFdel{it�}\DIFdelend \DIFaddbegin \DIFadd{it'}\DIFaddend s a general programming
language, so that you can

The mathematical properties of IPF
have \DIFdelbegin \DIFdel{have }\DIFdelend been described in several papers

Neighbourhood Statistics data using \DIFdelbegin \DIFdel{a }\DIFdelend \DIFaddbegin \DIFadd{an }\DIFaddend ordinary least squares (OLS) regression
\DIFaddbegin \DIFadd{model}\DIFaddend.

In second and third place respectively were the proportional probabilities
and TRS approaches, which took a \DIFdelbegin \DIFdel{a }\DIFdelend couple of seconds longer for a single
integerisation run for all areas.


\bibliographystyle{model2-names}  % Use the "unsrtnat" BibTeX style for
% \bibliography{library, lincluded}  % The references (bibliography) information are stored
\bibliography{custom,link-geo,link-pstar,library}  % The
% Replace w. link-pstar or link-ps depending

\end{document}

