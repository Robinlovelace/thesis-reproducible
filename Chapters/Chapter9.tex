% Chapter 9

\chapter{Conclusions} % Write in your own chapter title
\label{Chapter9}
% \lhead{Chapter 9. \emph{Conclusions}} % Write in your own chapter title to set
\fancyhead[RO,LE]{Chapter 9. Conclusions} %2side
\fancyhead[RE,LO]{\thepage}
% - What has been done?
% - What has been learned?
% - What can the new methods provide?
% - What are the political implications of this study?
% What does this thesis add to %!!!
% the rapidly growing and diverse field of transport and energy?
This thesis has investigated the energy costs of commuting and how they vary
between people and over space.
Motivated by the major problems of climate change, peak oil and
social inequality, the research set out to offer
evidence, and tools, to policy makers tackling these issues in the realm
of personal travel.
To complete the task, the methodology had to provide insight into the
spatial distribution of commuter energy use,
inequalities in its social distribution and the likely social and spatial
impacts of different intervention options.
Based on reviews of previous transport studies (in \cref{Chapter2})
and individual level methodologies (\cref{Chapter3}), it was decided that
a \emph{spatial microsimulation approach} was most appropriate,
due to the maturity of the techniques involved, flexibility of application and
ease of use.
A spatial microsimulation model was developed and tested, building on previous
work and implemented in the free and open source programming
language R (\cref{Chapter4}). The model
was used to combine geographically aggregated count data from the UK's 2001 National Census
with individual level data from the national Understanding Society dataset,
resulting in simulated \emph{spatial microdata}: individual
records which have been selectively sampled based on `constraint variables' shared
between the individual and aggregate level datasets.

Spatial microdata form the foundation of the spatial microsimulation approach.
Yet it is during the subsequent \emph{analysis} of this spatial microdata that
value for decision makers is generated: the interrogation of spatial microdata enables
calculation of energy costs at high geographical resolution (\cref{sindvar}),
analysis of social and spatial inequalities in the distribution of this energy
use (\cref{Chapter7}) and the development of quantitative
`what if' scenarios to model the impacts of change (\cref{Chapter8}).
Thus the spatial microsimulation approach developed here
includes not only the generation of spatial microdata but
analysis, visualisation, testing and modelling as well.
% To produce estimates of energy use at the
% individual level and at small geographical scales a spatial microsimulation
% model was developed. This model

This thesis provides, for the first time,
estimates of the energy costs of commuting at a range of geographic scales in
the UK, and an exploration of its social and spatial
variability. Some of the methods
used to achieve this result are already well established. What is new
methodologically is the way that these methods, and datasets on which
they depend, have been integrated with one another in novel ways
to provide results that are reproducible
and consistent regardless of the scale of analysis.

This chapter summarises what has
been learned during the research project:
methodological  contribution (\cref{smethcont}), its policy relevance
(\cref{sprel}) and the central findings (\cref{sumfind}). The research opens many
new pathways for further research which are discussed in \cref{sfurther}.
Finally, the thesis is evaluated in terms of the original aims and objectives,
in \cref{ssummary}. It is worth reflecting on the conclusions in the
context of the two
main aims of the thesis, introduced in \cref{s:aims}:
\begin{itemize}
 \item[A1] Investigate the energy cost of transport to work, its variability
at individual and geographic levels, drivers, and policy implications.
  \item[A2] Explore and evaluate the potential of spatial microsimulation
models for the social and spatial analysis of the energy costs of commuting.
\end{itemize}

\section{Methodological contribution} \label{smethcont}
The main methodological contribution of this thesis is the application of spatial
microsimulation to the social and spatial analysis of the energy costs of commuting.
It is concluded that commuting
research is an area that can benefit from this increasingly accessible
technique. Individual level analysis is becoming the norm in
transport modelling (\cref{Chapter3}) but often these
omit distributional impacts of new policies.
From the geographical literature, the vast majority of analysis into the spatial variability
of transport energy use and commuting patterns operates solely at aggregate levels.
Spatial microsimulation has several practical advantages over these
aggregate approaches, enabling outcomes that are otherwise inaccessible.
% Three central advantages of the approach, from the perspective of
% evaluating the energy costs of commuting patters from UK census data, are as follows:
More specifically, the four central methodological achievements of the
work are as follows:
\begin{itemize}
 \item The development and testing of algorithms to `integerise' the 
 weight matrices generated by iterative proportional fitting, allowing analysis
 to be conducted on whole individuals rather than fractions of individuals
(\cref{s:integerisation}).
 \item The calculation of energy costs per commuter trip in zones for which
distance/mode cross tabulated count data are unavailable (e.g.~output area levels)
from official sources.
 \item Insight into the intra-zone variability of commuting energy costs and
the links between commuter energy use and other socio-demographic variables,
based on analysis of spatial microdata.
\item The manipulation of this dataset to achieve goals outside the reach of
aggregate level studies, such as the targeting of specific groups in what-if
scenarios of the future, and assessment of the distributional impacts of
localised transport interventions.
\end{itemize}
Each of these points highlights the advantages of the spatial microsimulation
to analysing the energy costs of commuting and modelling travel to work.
Although spatial microsimulation has not been used to generate every energy
cost estimate presented in this thesis (it has been demonstrated that per trip
energy use can be estimated based on geographical data that provides
mode/distance cross-tabulations), the approach has been critical to achieving the four
outcomes listed above. These are arguably the most important outcomes
from a policy and methods perspective, hence the title of this thesis as a
\emph{spatial microsimulation} approach. During some sections (the
national level results presented in parts \cref{Chapter6} and \cref{Chapter8}),
a simpler `spatial approach' has been used to assess energy costs.
Yet, as illustrated in \cref{smshift}, the two approaches are not incompatible.
On the contrary, the scenario of modal shift shows that aggregate level
analysis can be useful for a rapid assessment of the basic determinants of
change (in this case mode and distance categories) and for generating
national level results (which would be overly resource consuming
using spatial microsimulation). The progression from aggregate to micro level
undertaken in this scenario illustrates the benefits of using a micro level
approach in tandem with preliminary aggregate level analysis.
The individual level implementation of the scenario, based on spatial microsimulation,
allowed greater sophistication: new variables
(age and distance as a continuous variable in this case)
were taken into account when estimating the extent of modal shift;
the results were displayed at a higher resolution,
and information about the socio-demographics of those affected was generated.
% Do this!!!

In the process of moving from an aggregate to a micro level model of
modal shift many new possibilities were opened up, not
all of which were implemented (\cref{stfurther}). The decision to commute,
how far and by what mode, is ultimately determined
by individuals (\cref{s:commuting}), so a micro level approach makes sense in theory.
Of course, transport infrastructure and other geographic factors also have a major
influence, and the spatial microsimulation approach would enable the interaction
between geographical and individual level factors to be included.
The reason for choosing the topic
were not only academic, but related to issues that
require an urgent policy response. Policy-makers often lack the tools and skills needed to
evaluate which policies would actually work to reduce energy
use and emissions, let alone at local levels and taking consideration
of the social distribution of these changes \citep{Banister2008, Tribby2012}.

In light of the evidence presented throughout the thesis, the kinds of question that
the spatial microsimulation approach helps answer
seem to be precisely those that policy makers should be asking before implementing new
strategies to meet climate change targets in fair way. Will the policy work?
Are there more effective alternatives? and which types of areas will be most
affected, and is this fair? The thesis cannot answer these questions in general terms,
but the results show that the 
methods can provide important evidence to aid the evaluation process, if the policy options are
clearly defined. The policy relevance of this work is one of its major strengths.

\section{Policy relevance and limitations} \label{sprel}
Climate change, resource depletion and standard of living provide the underlying
motivation for this research. One of the broad conclusions is that
methods of calculating energy costs of everyday activities are
highly relevant to policy makers concerned with sustainability. The
`sustainable mobility' paradigm requires new tools of assessment
as well as new concepts if it is to move out of pure academic discussions and
into practice around the world \citep{Banister2008}. In this respect, the
research presented in this thesis has much to offer. Too often, academic
research into the energy and climate impacts of transportation operates solely
at the level of entire nations or regions (\cref{s:energy}). Yet actual
transport
policies are often implemented locally.\footnote{The recently announced
\pounds77 million funding to promote cycling in cities and
national parks has been allocated to 7 specific urban areas and particular
routes within 4 national parks
\citep{RimeMinister'sOfficea}. \pounds 20 million of this funding is
allocated to Manchester alone, for 56 km of new cycle paths, amongst other
facilities. The question of where to invest these funds for the greatest social
and environmental benefit is of great policy importance.
}
The spatial microsimulation approach can help bridge such a `scale gap'
between academics and practitioners, by making individual and local level
analysis of personal travel patterns accessible.

% Granted, not all transport
% planners or other local decision makers can be expected to tailor dozens of
% lines of R code to their particular requirements. And granted, commuters will
% only represent a fraction of total traffic, let alone energy use, in any
% particular area. % Scrapped by dad
Not all local transport policy makers will have the time, skills or
desire to apply the methods advocated in this thesis to their local areas
and problems. However, some may be prepared to use techniques, with potential gains
in their ability to evaluate different scenarios of change. Would
increasing the cycling rate have greater impacts in location A or B? This kind
of question can be answered using the simple what-if scenarios presented in
\cref{Chapter8}, and refined to provide insight into the distributional impacts
using spatial microdata.

The spatial microsimulation approach is not without limitations:
it is complex,\footnote{Spatial
microsimulation is complex relative to
simplistic cost-benefit scenarios, but \emph{not}
compared with some transport models currently used in local government such as
SATURN \citep{SATURN2012}.
}
requires specialist knowledge to implement and produces simulated results that
may be prohibitively expensive to verify. For these reasons, it has been
emphasised that spatial microsimulation results should build on, rather than
replace, simpler aggregate level analyses for corroboration. There is a real
danger that, without proper understanding of the assumptions on which spatial
microsimulation is based, the approach could lead to incorrect interpretation
of results or, in worst case scenarios, fudging of results for political
purposes \citep{Openshaw1978}.
For this reason the reproducibility of the method and results is of utmost
importance if spatial microsimulation does
become widespread for evaluating real (and not just hypothetical)
interventions in transport systems. Following best practice guidelines
\citep{Peng2006}, government or private analyses can be made both transparent
and reproducible. Using free, open source and cross-platform programs such as R
can give analyses on which transport decisions are made attributes
vitally important in the democratic system: accessibility and transparency.
% Results generated
% through properly commented code and publicly available data should, in theory,
% be reproducible by anyone willing to learn how the model works.
% Another potential benefit is the use of local analyses in education. 

\section{Summary of findings} \label{sumfind}
Returning to energy in transport, a range of interesting results have been
generated using the methods developed during the PhD project.
No single,
overriding factor that determines commuter energy has been found. In broad
terms the findings presented in \cref{Chapter6}
support the conclusions of past research that energy use in transport is
complex, varies on a range of scales, and appears to be affected by many
factors, especially urban form \citep{Levtnson1997, smith2011polycentricity,
Levinson2012}.
More specifically, it has been found that at the regional level London
is the `greenest' area in terms of commuter energy use, but that this is
partly offset by the surrounding regions which have the nation's most
energy intensive average commute. This finding provides tentative support
to the `compact city' hypothesis \citep{Breheny1995}, but suggests that
the energy use in surrounding areas may be pushed up beyond the average
due to long-distance commuting to concentrated employment centres.

Nationally, it was calculated that
commuting uses 4.1\% of direct energy use in England. Commuting was found to
account for almost 15\% of transport energy use, representing an
important and relatively inelastic contribution to the total.
Individual level variability was also explored in the
same chapter (\cref{sindvar}). It was found that in urban centres
the 20\% top energy consuming commuters can account for over 90\% of commuter
energy use, a very high level of inequality.

At lower geographical levels, the variability in average commuting energy
costs increases as would be expected, and a clear spatial pattern, in which
urban centres and their direct surroundings have low energy costs compared
with the rural surroundings. However, commuting energy costs still vary
greatly between many areas that are similar `on paper' at the level of
statistical wards (\cref{sregional}). At the local level, the pattern appears
to be more complex still, with a tendency for large city centres to be associated
with above commuter energy costs greater than their surroundings in South
Yorkshire. Later, in \cref{c7results} this finding is replicated in terms
of the relationship between areas' distance to the nearest employment
centre and average energy costs across Yorkshire and the Humber, adding
further evidence to suggest that the compact city hypothesis, in its simplest
form, is over simplistic. 

% Say what you actually found here!!!
In agreement with \citet{Boussauw2010}, the average distance between home and work,
which in itself depends on a range of social and geographical factors,
seems to be the major driver of energy intensive commuting: when distances
are large, the possibilities for modal shift are greatly reduced, and
telecommuting can only be seen as a realistic solution for certain types of
jobs, many of which are out of the reach of the most vulnerable
(\cref{Chapter8}).
% Correlation between distance and energy use! !!!
Further modelling work could contribute to the debate about the factors
underlying transport energy use, providing statistical evidence about the range
of factors at play. But the focus here has been policy, not theory.
To summarise, the most important policy relevant findings are as follows:
\begin{itemize}
 \item Energy use for commuting varies at all geographical levels
 and is distributed highly unevenly between individuals in most zones.
% The gini index of commuting is higher than for income %!!!
 Even between areas
 that appear to have similar levels of energy use at the aggregate level, there
 are great differences in how commuter energy use is divided up between their
 inhabitants (\cref{c7results}).
 \item At the scale of cities,
 there is a tendency for highest energy costs to appear furthest from the city (around
 60 km in the case of London), which tends to fall towards the city centre, but then
 rising again in the city centre (\cref{fig:dis-e}).
 \item At the international level, England appears to have lower per-trip energy
 costs than the Netherlands, despite Holland's reputation for excellence in
 environmentally benign transport planning.
 \item In terms of modes of travel, cars were found to completely dominate the
energy costs of commuting in most areas. This can be easily overlooked based on
existing statistics that focus on modal split by number of trips and distance.
In Yorkshire and the Humber over 95\% of energy use for commuting was found to
be due to cars (\cref{c7results}), implying that environmentally aware policy
makers there should focus on reducing private car use as a priority rather than
the current focus on modal shift. 
 \item The energy impacts of an ambitious scenario of modal shift from cars to
 bicycles would be relatively modest, compared with telecommuting, which is
 rarely framed as a transport policy. Active travel policies need to be supplemented
 by policies encouraging car sharing, reducing demand for long-distance travel and,
 in the long-term, reducing average home-work distances.
\end{itemize}

Each of these findings has implications for transport planning strategies
in the UK in broad terms. Exploring what these implications are on a
case-by-case basis is outside the scope of this thesis, and further exploration
of the most policy relevant overall findings provides a strong
incentive for further work at the local level in different case study areas.
Because of the applied nature of this research, it is suggested that much of
it is conducted by policy makers. In terms of opportunities for
building on the thesis in the academic context, there is also much scope for further
work, as outlined below.

\section{Further work} \label{sfurther}
The work undertaken has provided new contributions to knowledge, both empirical
and methodological. The latter contribution, used appropriately,
could outlast the former: the spatial microsimulation approach has the potential
to generate many more interesting results than are presented in the preceding
chapters. The empirical results also raise important research questions, by
challenging conventional wisdom about the energy costs of commuting and how
these costs can be best be reduced.

It is therefore hoped that the thesis is not seen simply as an `end product' or `final
result' but as a tool for stimulating and enabling further lines of study into
energy and transportation. It is up to other researchers to
judge how best to use the methods for their own purposes, so the concluding
remarks in this section are intended to provide general guidance, rather than a
prescriptive research agenda. It was decided that the following research areas,
in rough descending order of priority, would benefit from further investigation,
building on the methods and findings presented in this thesis:
\begin{itemize}
 \item The use of spatial microdata as an input into agent-based
transport models: the recent advances in microsimulation in urban and transport
models outlined in \cref{s:urbanmodel} make modelling techniques simultaneously
more accessible to transport planners and much more
powerful.\footnote{In this
regard MATSim in particular
seems to hold great promise for `open sourcing' transport modelling for the
evaluation of specific schemes, due to its uptake by US planning authorities.
Yet environmental/energy and distributional impacts are still under-reported in
scheme evaluation. Combining the socio-demographic variables contained within
simulated spatial microdata with models such as MATSim therefore has great
potential to further enhance the use of models for practitioners.
}
Starting from
the other side of the spatial microsimulation versus transport
planning/modelling
divide, the addition of agent-based models with inbuilt capability to load and
interpret the road network (e.g.~from Open Street Map data), has the potential
to vastly improve the ease with which infrastructure interventions can be
assessed by academics already acquainted with spatial microsimulation. This
approach could be far more advanced (and potentially user
friendly) than the crude methods presented in \cref{s:workdes}.
\item Extend the spatial microsimulation methods presented in \cref{Chapter4} so
that they are capable of classifying individuals into family units
(\citealp{Pritchard2012}, see \cref{sreweight}) and allocating their home and
work locations to precise geographical coordinates (as described in
\cref{s:workdes}).
\item Development of more realistic and localised `what if' scenarios: 
the modal shift scenario presented in \cref{smshift} is useful to gauge the
potential magnitude and spatial distribution of cycling uptake in the UK, but
is unlikely to be realistic as the same proportion of short-distance car
drivers are expected to shift in every area. In reality, most transport
interventions are localised. The recent allocation of \pounds 77 million to
cycling cities schemes \citep{BBc2013-cycling}, for example, will inevitably be
spent locally. Localised scenarios of different expenditure options could help
planners maximise the benefits resulting from this expenditure.
\item Prediction of energy use: variation in energy use variable has been
explained intuitively as the result of a few key factors: wealth, distance to
employment centre and the nature of the surrounding transport network all seem
to have an influence (\cref{Chapter6}). The next logical step forward would
be the creation of a predictive model to estimate energy use based on
underlying geographical drivers. This could include flow data
\citep{Simini2012} as well as more conventional explanatory variables such as
topology, wealth and connectivity measures. Such a predictive model would be
useful academically, enhancing understanding of the geographical drivers of
energy use \citep{Steemers2003} and practically, as a basis to project the
energy impacts of future change.
\item The application of the method to more countries at more time periods, to
investigate the generality of the findings and provide further guidance to
policy makers based on the international evidence.
\end{itemize}
This is a diverse set of recommendations that can be explored using a variety
of methods. It is therefore suggested that resulting research does not 
need to fit into the `spatial microsimulation approach' advocated
in this thesis to build on its findings. However, approach
may offer certain advantages as a way of framing the research methodologically.
Returning to the central policy issue of energy use in transport
it is recommended, if an overriding agenda or
paradigm is deemed beneficial at all (it may not be), that future research
in this area uses the sustainable mobility paradigm \citet{Banister2008}.

\section{Thesis evaluation and summary} \label{ssummary}
To evaluate the thesis by its own standards, we return to the aims and
objectives introduced at the end of the opening chapter (\cref{s:aims}), and
discuss to what extent they have been accomplished.  The first aim (A1) was to
``Investigate the energy cost of transport to work, its variability
at individual and geographic levels, drivers, and policy implications.'' This
aim was mostly accomplished in \cref{Chapter6}, in which national commuter
energy costs were estimated in terms of both energy use per trip and energy use
per year per commuter. In the same chapter commuter energy use was also found
to vary at all geographical scales, with the range of average values
unsurprisingly increasing at lower geographies and the spatial pattern becoming
more complex at the local level. In terms of individual level variability, it
was shown in \cref{sindvar} and throughout \cref{Chapter7} that the distribution
of energy use across the population varies greatly from place to place and that
socio-economic factors play an important role in determining an individual's
use of energy to travel to work that is likely to be missed in analyses that
operate only at the aggregate level. 

Sub aims 1.1, 1.2 and 1.3 relate to the variability of commuting energy costs;
the factors most closely associated with high and low energy use; and how the
spatial microsimulation approach can be used to inform policies using scenarios
of change, respectively. The following bullet points summarise
progress in achieving these aims:
\begin{itemize}
 \item The quantification of the variability of commuter energy costs at various
levels has been a major output of the research, as detailed above. However, the
variability over time has received less attention due to data
constraints.\footnote{The
observation that energy costs have increased tenfold
over the past century (\cref{s:eff-imps}, \cref{flongip}) was based on a small
sample and crude assumptions about average distances travelled by, and
efficiencies of, different modes of transport. Still, this is an interesting
result. Also, the changing distribution of car dominance for the trip to work,
illustrated in \cref{f1981}, is an interesting finding that likely relates to
changes in the spatial distribution of energy
intensive commuting over time.
}
Aim 1.1 was also to investigate household level variability. This has not been
achieved in the thesis, although pointers of how to do this have been
suggested.\footnote{See
the second bullet point in the list of further research
in the previous section.
}
\item The explanation of this variability set out in aim 1.2 was largely
achieved. At the aggregate level, distance from employment centre was found to
account for much of the variability in average commuter energy use, although
this was not formalised as a predictive model or linked to additional
geographical factors such as the road network. At the individual level it has
been shown that average commuting behaviour also varies depending on age,
number of cars in household and, more importantly for policy makers, by
socio-economic class and income (\cref{c7results}).
\item Regarding the formulation of models for change (Aim 1.3), a number of
`what if' scenarios were considered in \cref{Chapter8}. Only 2 of
these (high cycling and telecommuting scenarios, based on evidence from Holland
and Finland) were quantified, but the results were interesting, policy relevant
and surprising. As stated in the previous section, there is great potential for
further research in this area.
\end{itemize}
The second main aim was methodological, to test the potential of spatial
microsimulation for the ``social and spatial analysis of the energy costs of
commuting.'' It is concluded that the thesis has succeeded in meeting this aim:
spatial microsimulation has for the first time been applied to the
investigation of this issue and the methodology has been developed in a way
that should be reproducible by others based on code and documentation that has
been made available to others.\footnote{In
the
{\color{blue}\href{https://github.com/Robinlovelace/thesis-reproducible}
{`thesis-reproducible'}} repository
and other personal repositories hosted on the social coding site github.com
}
It is also concluded that the benefits of using the spatial microsimulation
approach outweigh the additional complexity, computing and time costs of the
individual level methodologies compared with more common aggregate level
approaches. The ability to target specific groups in scenarios of change, to
explore the interaction of individual and geographical factors in influencing
travel behaviours and to investigate the distributional impacts of change
suggests the approach has great potential as a tool for policy makers and
academics. Overall the thesis has achieved most aspects of all of its original
aims, although further work is needed to include household level impacts and
better explain the variability of energy use based on a wider range of
variables than those used here.

In summary, this thesis has contributed methods and findings to the emerging
area of energy use in transport. The research was motivated by the seemingly
intractable socio-environmental problems of climate change and resource
depletion, leading to a focus on pragmatic policy relevance rather than theory.
The methodological innovations of integerisation and allocation of home-work
locations in the context of spatial microsimulation are relatively minor
achievements academically, yet their application to real-world transport
planning decisions could yield major benefits for policy makers.

Some of the findings were unexpected and challenge conventional wisdom about
what constitutes `good' transport policy environmentally. The current
emphasis on bicycles, for example, is at odds with its relatively
minor potential for large emissions cuts (although health and social
considerations should also play their part in transport policy, areas
in which the bicycle has much more to offer).
The key message for policy-makers wanting to reduce fossil fuel dependence
is that policies that can reduce the consumption
of the most energy intensive areas and individuals (such as telecommuting)
should take priority over policies that will further reduce energy use in
places that are already quite energy efficient in terms of travel to work.
This finding was reinforced by the comparison between 
commuter energy use in England and the Netherlands, where the Dutch were
unexpectedly found to use \emph{more} energy for commuting. 

These findings not only challenge wishful thinking in the area of energy and
transport, they lay the
foundations for further work from which additional results can be
generated. The findings are also important in their own right: they
provide insight into the interventions that would be needed if reducing energy
use in personal transport
becomes a political priority. The impacts of this research may thus
depend more on the extent to which the approach is adopted by practitioners,
than its direct influence in
academia. In terms of social and environmental impact, a single well-designed
intervention in the transport system resulting from this research could be worth
several thousand words.

% The methodological `black boxes', that process the
% input data and generate results may only be seen as a means to an end, but they
% are fundamental to the results and how reality is represented. It is therefore
% important that they
% are critically placed in the spotlight.
% % Reproducible research is one of the
% % cornerstones
% % of scientific advancement, so this is probably
% % the most important chapter in the Thesis from an academic perspective.
% % Others can build on the analysis and take it further.
% Too many times have researchers had to `start from scratch', to implement
% methods that are already widely used. This problem is accute in Transport
% Modelling, where a small number of proprietary (and extremely expensive)
% software packages dominate the market, reducing potential for transparency,
% reproducibility and inter-organsiational collaboration
% \citep{Tamminga2012}. These problems can largely be overcome
% by an open source approach, to both data and methods \citep{Ince2012}.
% Therefore, in addition to presenting the methods clearly and concisely,
% a sub-aim of this chapter is to ensure that all of the results are reproducible
% based only on data referred to in this thesis.