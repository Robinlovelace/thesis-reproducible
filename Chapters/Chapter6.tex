% Chapter 6

\chapter{The energy costs of commuting}
\label{Chapter6}
%The volatility of oil prices \cref{fig:oilprices}).
% This is the money shot chapter: maps of behaviour culminating in energy maps
% generated by msim
% Compare energy use estimates derived through msim vs agg methods (it allows
% estimates at a small geo. scale right???!!!)
% Put national maps in national section
% \lhead{Chapter 6. \emph{The energy costs of commuting}}
\fancyhead[RO,LE]{Chapter 6. The energy costs of commuting} %2side
\fancyhead[RE,LO]{\thepage}
The preceding two chapters have demonstrated that there are both detailed
data (at various levels) on travel to work in the UK \emph{and} methods
that can be used to convert this information on behaviour
into estimates of energy use. Based
on these foundations, this chapter illustrates the main results, in terms
of overall energy use. 
Estimates of
energy use at national (\cref{snational}), regional (\cref{sregional}) and
in comparison with other sectors (\cref{stotalcomp}) levels are presented.
The approach follows the
principle of Occam's razor, whereby additional complexity is only added when
necessary, in contrast to agent-based approaches, where complexity is inherent at the
outset \citep{batty2012perspectives}.
Therefore the high level results are based on the simpler
aggregate level methods. Results that emerge from spatial microsimulation
(and which would be inaccessible using aggregate level methods alone) are presented
later on, for a smaller case study region. South Yorkshire is used here
as the case study region here and in subsequent chapters for consistency
(\cref{sindvar}).\footnote{The
reasons for choosing this case study area explained in \cref{soyoref}.
}
In this section the spatial distribution of energy use for commuting is
illustrated at a low level. Indicators of how the energy use
in each zone is distributed between different members of society are also
presented.
The international applicability of the
methods for calculating the energy costs of work travel
is tested in \cref{sinternational}, which
compares the energy intensity of commuting in England and the Netherlands.
% Estimates of changes in the energy costs of work travel over time
% are presented in \cref{stime}.
% 
% After presenting these estimates, attention is directed towards
% factors accounting for the variability observed at all levels.
% \Cref{sexplanation}, explores possible causes of energy intensive
% commuter patterns with reference to the literature, the energy use estimates
% present in this chapter and 
% additional data sources. The aim here is to move beyond description and
% towards explanation, to
% % so \cref{sexplanation} sets !!! re-add
% set out the scene for policy implications:
% it is the \emph{understanding} of an issue that allows it to be tackled
% successfully, rather than mere knowledge of its existence
% (see \citealp{Berners-Lee2013}).
In the final section the results are discussed with
reference to the debate on energy use and urban form, introduced in
\cref{s:energy}. %!!! really?
% \section{Commuting behaviour}
% Although available transport technologies and infrastructure affect the energy
% use of travel to work patterns over space and time, as presented in
% \cref{Chapter5}, it is behaviour that determines how much energy using devices
% are used. This section therefore analyses the aggregate level transport to work
% data, to guide the subsequent analysis of energy costs.

\section{Commuter energy use at the national level} \label{snational}
Based on the data and discussion of it presented until now, we are
well-placed to perform a preliminary estimate of energy use at the aggregate level.
This approach, starting simple to understand the fundamentals and most important
factors influencing the system before later adding details, follows the
recommendation of \citet{batty1976urban}.

Having considered the limitations of the data, and weighed up the
costs and benefits of complexity, it was decided to
primarily calculate $ET$ at the aggregate level,
as a function of only two parameters: mode and distance travelled.
(These are the cross-tabulated categorical variables provided as geographically
aggregated count data at administrative levels down to ST Wards ---
see \cref{t:agdata}). This can be expressed for any particular area as
\begin{equation}
 ET = \sum_m \sum_d{2dR_{(d,m)} \times E_m}
 \label{eqet1}
\end{equation}
where ET is the total work-day energy costs for all commuter trips that happen
in that area, d and m are distance and mode categories, dR is the mean average
route distance inferred from the mode-distance combination and E an
estimate of the
energy cost per unit distance (direct or indirect), presented for each mode
in \cref{tfinale}.

An alternative way to express this would be based on commuter flow data.
If one know the approximate origins (i) and destinations (j) of every commuter
trip, this can be expressed in a different way:
\begin{equation}
 Et_i = \sum_j \sum_m {n_{(i,j)} \times 2Q \times dE_{(i,j)} \times Ef_m}
 \label{eqet2}
\end{equation}
where $Q$ is the circuity factor which translates the Euclidean distance between
two places into an approximation of the network distance, defined by \cref{eq:circ}.
Summing Et for all the origin areas in the region of interest would provide
an overall estimate of energy costs.

Clearly, neither \cref{eqet1} nor \cref{eqet2}
tell the entire story, as they omit frequency of travel: how many days
per week people travel to work (this is covered in \cref{sfreq}).
They also omit a number of other complicating factors that are discussed in
the previous chapter.  However, they are enough to begin with, to create maps
that capture the spatial variability of energy costs of commuting at
a coarse geographical resolution.
The approach summarised by \cref{eqet1} is used, because the input data
is much simpler, smaller and easier to manage. (\Cref{eqet2} could be used
to verify the estimates.)

The input variable into \cref{eqet1} that has not yet been quantified is
dR. Route distance by mode and distance band
is needed to account for the fact that Census data on distance is presented in
categories (with breaks at 2, 5, 10, 20, 30, 40 and 60 km), whereas distance
itself is continuous. The simplest way around this problem would be to
assume that route distance sits in the centre of the bins (i.e.~1, 3.5, 7.5, ...
km). However, this would be a very gross simplification because the route distance
is certain to be greater than the Euclidean distances calculated from
home-work postcode pairs. Also, because each mode has a different
distance-frequency distribution,
% !!! figure???
it is safe to say that the average route distance will also vary depending
on the mode of travel.\footnote{One would, for example,
expect people who walk 2 to 5
km in Euclidean distance to travel on average less far than those who drive
between 2 to 5 km, as `impedance' of walking rises rapidly after the first
kilometre whereas the additional personal effort of driving
an extra kilometre or two is much lower
\citep{Iacono2010}, discussed in \cref{Chapter2}.
}
To take this into account, distance data from Understanding Society was used.
% !!! major *** should have used NTS --- do after = priority!
First, the values were converted into estimates of Euclidean distance and
split into the Census bands. Next, these were re-converted into the original
route distances, and the average was taken for each distance band/mode
combination. The results, which are presented in \cref{tdboxes} and visualised in 
\cref{fdboxes} and \cref{fdboxes2} for motorised and non-motorised modes,
provide strong evidence of inter-mode variation in distance travelled within
the same distance band. However, these results are problematic due to the
low quality of the input data (n = 5,000 but less than 5 individuals
were present for unusual
categories such as people walking more than 5 km to work) and were not entirely
as expected. The anomalies are summarised as follows:
\begin{itemize}
 \item Bus journeys appear to be longer than the equivalent journeys by train,
 which was expected to be associated with the longest trips (although train
 journeys are in second place).
 \item The average bicycle trip was expected to be longer than walking trips
 in all cases. This did apply in the 0-2 and 2-5 km categories, but after that
 the trend reversed. This can be explained by sample size: a few unusual people
 walk far to work, whereas cyclists, as expected, tend to cluster around the lower
 ends of the 5-10 and 10-20 km bins.
 \item The `inverse U' shape of the bottom graphs in both cases were unexpected.
 This could be explained by the tendency of people to round to 10: the
 average distance travelled in the 30-40 km bin was the closest to the upper
 bound in all cases, perhaps a result of people rounding to 25 miles for many
 trip distances in the 20s (just under 40 km in Euclidean distance).
\end{itemize}
It would be desirable to corroborate these findings with other individual level
data on travel to work. For the purposes of assessing the relative energy
costs of commuting in different areas, however, these estimates suffice:
the concepts and code behind the estimates would produce slightly different
values given different input data, but, at present, this is not our concern.
With evidence-based estimates of $dR_{(d,m)}$ in place,
we can proceed to estimate the relative energy costs of commuting in different
places. 

\begin{table}[htbp]
\caption[Average distance travelled by mode and distance band]
{Average distance travelled by mode and distance band (km),
from USd data.}\label{tdboxes}
\begin{center}
\begin{tabular}{lrrrrrrrr}
\toprule
Upper limit & 2.0 & 5.0 & 10.0 & 20.0 & 30.0 & 40.0 & 60.0 & 250.0 \\ \midrule
Car driver & 1.6 & 3.9 & 7.9 & 15.0 & 26.0 & 35.8 & 50.3 & 102.6 \\
Car passenger & 1.5 & 3.9 & 7.9 & 15.2 & 26.5 & 36.4 & 48.0 & 95.0 \\
Motorbike & 1.4 & 4.1 & 7.0 & 15.2 & 23.5 & 36.0 & \multicolumn{1}{l}{NA} & \multicolumn{1}{l}{NA} \\
Bus & 1.8 & 3.8 & 7.7 & 13.9 & 27.7 & 40.0 & 56.0 & 110.5 \\
Train & 1.5 & 4.2 & 8.1 & 15.1 & 26.4 & 37.6 & 53.2 & 98.8 \\
Metro & 1.7 & 4.0 & 8.1 & 14.7 & 25.8 & \multicolumn{1}{l}{NA} & \multicolumn{1}{l}{NA} & 65.0 \\
Cycle & 1.5 & 3.9 & 7.5 & 11.5 & \multicolumn{1}{l}{NA} & \multicolumn{1}{l}{NA} & \multicolumn{1}{l}{NA} & \multicolumn{1}{l}{NA} \\
Walk & 1.2 & 3.5 & 8.0 & 13.7 & 25.0 & \multicolumn{1}{l}{NA} & \multicolumn{1}{l}{NA} & \multicolumn{1}{l}{NA} \\
Other & 1.0 & 4.3 & 7.6 & 13.5 & 27.8 & 37.5 & 42.0 & 130.0 \\
Taxi & 1.7 & 3.0 & 9.0 & 12.0 & \multicolumn{1}{l}{NA} & \multicolumn{1}{l}{NA} & \multicolumn{1}{l}{NA} & \multicolumn{1}{l}{NA} \\
\bottomrule
\end{tabular}
\end{center}
\end{table}

\begin{figure}[htbp]
\begin{center}
    \includegraphics[width=9cm]{dboxes}\end{center}
  \caption[Distance bands and average distance travelled for motorised modes]
  {Distance bands and average distance travelled for motorised modes, expressed
  as the relationship between lower bound and average distance (top)
  and that between lower bound and the ratio of upper bound to average distance
  (below), from Understanding Society data. `card' and `carp' refer to
car driver and car passenger respectively.} %%% could update with NTS!!! should have used centre...
  \label{fdboxes}
\end{figure}

\begin{figure}[htbp]
\begin{center}
    \includegraphics[width=9cm]{dboxes-walk}  \end{center}
  \caption[Distance bands and average distance travelled for active modes]
  {Distance bands and average distance travelled for non-motorised modes, expressed
  as the relationship between lower bound and average distance (top)
  and that between lower bound and the ratio of upper bound to average distance
  (below).} %%% could update with NTS!!! should have used centre...
 \label{fdboxes2}
\end{figure}

% As a result, it can be assumed that the number of commuter occupants in each
% car is 1 (so $EI_car$ = 2.98, not 2.4) for all those who
% ticked the option ``...'': Those who car share will be captured in option x:
% car passenger trips are counted a unique mode of transport to work. Calculated
% at the individual level, a sample of the input data and resulting values
% for Et are illustrated in table xx and xx for xxx, the first MSOA area in
% the input dataset.
% 
% Even if more accurate results can be obtained by, and assumptions behind, the data
% used to calculate energy costs per trip

Based on these categories, and the values of Ef reported in the previous section,
the 99 distance-mode variables of the cross-tabulated
census table ST121 can each be allocated an average energy
costs.
% !!! figure here right!
Originally the energy cost associated with the number of people in
each distance/mode category was calculated using the LibreOffice Calc
spreadsheet software. However, this soon became unwieldy so the analysis
was transferred into R. The main script file used to convert the raw
count data (\cref{frcount}) into energy estimates is available in the
{\color{blue} \href{https://github.com/Robinlovelace/thesis-reproducible}
{thesis-reproducible}} folder associated with this
thesis.\footnote{Code %!!! make available!!! - in intro!
and output were also embedded in RMarkdown, to show the output from R.
Every step of this process is illustrated on the author's RPubs website
(\href{http://rpubs.com/RobinLovelace/7178}{rpubs.com/robinlovelace)}.}
The benefit of this script is that it can take input data of the type
displayed in \cref{frcount}, regardless of the number or scale of the
geographic units.
% Starting at the largest (regional) geography, the results
% are displayed in \cref{fgoren} to \cref{fwarden}.

At the national level, the distribution of trips by mode and distance is
displayed in \cref{fengmodedis}. This graph shows the
dominance of car drivers for all trip distances, except for the 0-2 km bin.
As expected, bicycle and walking trips are dominant in the lowest
distance categories and tail off to essentially zero after the 20 km mark.
Another result that was expected was the tendency of train journeys to
be longer, probably due to the possibility of working on the train and
the use of this mode by high-income workers travelling to London.

\begin{figure}[htbp]
    \centering{\includegraphics[width=12cm]{England-ttw-mode-distances}}
  \caption[Mode and distance categories of commute in England]
  {Mode and distance categories of commuter trips in England, 2001.}
  \label{fengmodedis}
\end{figure} %!!! re-add? (hopefully!)

According to the methodology described above,
this data was translated into energy costs at the national level of
Wales and England (the data table ``ST121'' is unavailable for Scotland and
Northern Ireland). As illustrated in \cref{few}, the energy costs of commuting
in Wales are higher per trip, by 10\% (34.5 MJ in England, 38.0 in Wales).
In practice, it is probably not worth plotting this information geographically,
as there is very little geographical information to report:
the values are aggregated over a very wide area, so a choropleth map
of the results makes little sense. However, the purpose of  \cref{few}
is primarily to introduce the subsequent geographical plots, which
are of increasingly small geographic zones.

\begin{figure}
 \centering{
 \includegraphics[width=8cm]{ew} }
 \caption{Comparison of commute energy costs between England and Wales.}
 \label{few}
\end{figure}



\section{Regional and sub-regional patterns} \label{sregional}
The average energy costs of commuter trips in England
are illustrated at the regional level in \cref{fgoren}, to provide an
overall impression of its spatial variability at the coarsest geography.
The high degree of geographical aggregation masks much of the variability,
yet there is still a substantial difference between regions. As expected,
London is the region with the lowest energy costs per commute at
20.8 MJ per one-way trip or 40\% below the average for all regions.
Excluding London, energy costs were lowest in the North West
and highest in the East of England (closely followed by the South East).
The variability between these regions was less noticeable:
they were 10\% below and 12\% below the national average respectively.


\begin{figure}[htbp]
\begin{center}
    \includegraphics[width=10cm]{goren}  \end{center}
  \caption[Average energy use per trip (Etrp, in MJ) in English regions]
  {Average energy use per trip (Etrp, in MJ) in English regions, based
  on cross-tabulated distance/mode geographically aggregated count data.}
 \label{fgoren}
\end{figure}

\begin{figure}[htbp]
\begin{center}
    \includegraphics[width=13cm]{rawcount}  \end{center}
  \caption[Raw count data of commuters by mode and distance.]
  {Raw count data of commuters by mode and distance, the first 5 columns of
  regional level data, from Casweb table ST121. Data displayed in RMarkdown
  format, illustrating the reproducibility of the results (see
  \href{http://rpubs.com/robinlovelace}{www.RPubs.com}).}
 \label{frcount}
\end{figure}

To gain more insight into the spatial pattern of commuter energy costs,
the same data was re-plotted at lower geographical scales, down to the ward
level for the nation. \Cref{fcountyen} shows the distribution of energy costs
at the county level, constituting 88 polygons (42 counties and an additional 46
Local Authorities to make-up areas not covered by counties). This is a useful
level for identifying case study cities and areas that have unusually high
or low levels of energy use, given their surroundings. As a general pattern,
large and high-density urban areas tend to have lower energy use, with
the three largest built-up areas in England (Inner London, Greater Manchester
and the West Midlands built-up area) all having average commuter energy costs
below 30 MJ (the mean is 36). Another pattern that emerges is the relationship
between the very low energy costs of commuting in London, and the relatively high
costs of areas within a $\sim$100 km radius surrounding the centre: commuters in Bedford, Essex and
Kent, all of which contain `commuter belts' feeding London, for example, use
on average 45 MJ per trip to work. The highest and lowest (outside London)
values are found in Rutland (the geographic centroid of which is located 109
km from central London, and which was the last county in England to have
a direct trainline to London) and the City of Kingston upon Hull, respectively.
Comparison of these two counties could make an interesting case study
to explore the reasons for underlying reasons behind high and low
energy costs of commuting in England.

\begin{figure}[htbp]
\begin{center}
    \includegraphics[width=13cm]{countyen}  \end{center}
  \caption[Average energy use per commuter trip at the county level]
  {Average energy use per commuter trip at the county level. The
  letter strings are abbreviations of the full county names (e.g. Dv is Devon).}
 \label{fcountyen}
\end{figure}

The results for districts, of which there are 308 in England,
are presented in \cref{fdistricten}.
As is apparent from the large and relatively homogeneous
area of bright green in London (and
knowing its high population density), the districts with the lowest
commuter energy costs are found in the capital. In fact,
9 out of 10 of the districts with the lowest energy costs per
commuter trip are located in London (the lowest is found in the
Isles of Scilly, with an average of 7.6 MJ/trip). The district with the
highest energy use per commuter trip (60 MJ/trip, 10\% more than the second
highest zone) is South Northamptonshire, visible
in \cref{fdistricten} as the red zone in the far south corner of the
East Midlands. The standard deviation of average energy use per trip at
this level of geographic aggregation was 9.0 MJ, 50\% higher than the
6.0 MJ/trip standard deviation observed at the regional level.

\begin{figure}[htbp]
\begin{center}
    \includegraphics[width=13cm]{districten}  \end{center}
  \caption[Average energy use per commuter trip at the district level]
  {Average energy use per commuter trip at the district level.}
 \label{fdistricten}
\end{figure}

The same results are presented in \cref{fengplotnm}, at the ward level.
Here, much greater variability is apparent (note the increased range of
values represented in the colour scale). The standard deviation is 11.6
and values range all the way from 5.1 to 88 MJ per trip.
It is interesting to note where these extreme values are found:
the former is located in the central London ward of
Portsoken, where walking is the most common mode of travel to work,
followed closely by catching the tram. The latter was
found in Park Farm North,
a suburban ward located in the far South East of England, just south of
Ashford, where car drivers account for 68\% of all commutes. The complex
patchwork of average  commuter energy costs displayed in \cref{fengplotnm}
suggests that regional level assessments, such as those
presented in \cref{fgoren}, are not able to capture the full geographical
variability of the variable at all well: there is much more variability
within zones than between them. One pattern that stands out from the ward level
analysis is the tendency of settlements to be directly surrounded by green areas
associated with low energy costs. Although only large cities (those with
populations in excess of 100,000) are displayed in
\cref{fengplotnm}, it seems that many towns and cities are immediately surrounded
by areas of low commuter energy costs. Haverhill (located in the East of
England, roughly half-way between Cambridge and Chelmsford),
Hereford (in the south-west of the West Midlands) and a number of coastal
towns such as  Sheringham ($\sim$40 km north of Norwich) and Scarborough
(in Yorkshire and the Humber) are examples of this.
%%% Link here to analysis showing cor between remoteness and ecosts
\begin{figure}[htbp]
\begin{center}
    \includegraphics[width=16cm]{engplotnm}  \end{center}
  \caption[Average energy use per trip (Etrp, in MJ) in English wards]
  {Average energy use per trip (Etrp, in MJ) in English wards.
  The black dots are large (100,000 people or more) cities (from
  \citet{Brownrigg2013}).}
 \label{fengplotnm}
\end{figure}

The method used to calculate energy costs creates estimates
that are disaggregated by mode and distance. This allows the
aggregate energy use result in each area to be subdivided.
A policy-relevant example of this would be those areas in which
short-distance car journey constitute
a large proportion of the energy costs of work travel (these areas
may benefit from improved walking and cycling infrastructure). Another example
is the proportion of commuter trip energy use
in each area used by trains. The result is interesting in itself, and
provides confidence that the calculations are working correctly:
it is clear from \cref{ftrainen} that there is a tendency for
areas located close to railways
to be associated with a high proportion of per trip energy
use to be composed of rail travel. Also as expected, areas with fast rail
connections to London seem to have high energy use for this mode of travel.

\begin{figure}[htbp]
\begin{center}
    \includegraphics[width=15cm]{trainen}  \end{center}
  \caption[Proportion of energy use caused by train trips]
  {Proportion of energy use caused by train trips, plotted alongside the rail
network (black lines). Only areas above the national average (3\%) are
plotted.}
 \label{ftrainen}
\end{figure}

\section{Total commuting energy use and comparisons with other sectors}
\label{stotalcomp}
In \cref{Chapter5}, reasons and methods for calculating commuter energy use
on an annual level were laid out. In this section, total energy use
for commuting is presented, based on the average frequency counts presented
in \cref{tdistable} and the assumption that people work on average for
44 weeks per year. As acknowledged in \cref{sfreq}, these are quite crude
assumptions that could be updated if the true distribution of part and
full time jobs in each area were known and using spatial microdata.
However, geographical breakdowns of energy use from other sectors are
provided only at coarse levels of aggregation, so using the spatial
microsimulation approach in this case seemed unnecessary. Moreover, total
energy use for commuting is something that would be useful to estimate at
the national level, something which the spatial microsimulation methods
described in \cref{Chapter4} cannot
handle.\footnote{If small samples of the
spatial microdata were used (e.g. a 1\% sample), a spatial microsimulation
model would be possible for the whole of England, although the loss of
information from sampling may negate the benefits.
}

Using the script file `districten-yr', the total energy costs of commuting
across all of England in 2001 was estimated to be 220 PJ, or 61 TWh.
To put these large numbers into context, total electricity usage in the UK
(not just England) is 400 TWh \citep{MacKay2009}. Overall, this represented
4.1\% of total energy in England from all sectors and 14.4\% of total transport
energy use, based on the DECC's 2003 NUTS level 4
estimates.\footnote{This
dataset is available from {\color{blue} \href{https://www.gov.uk/government/statistical-data-sets/total-final-energy-consumption-at-regional-and-local-authority level-2005-to-2010}{https://www.gov.uk/government/statistical-data-sets/}}
and includes breakdowns of energy use by sector (industry \& commercial, domestic and transport)
and primary energy source (from coal to renewables).
Because the national level commuting dataset I was using operated at the
Local Authority level, while the DECC data was presented as NUT 4 zones,
which are slightly different. Joining by zone name, 16 of the 354 Local
Authorities were left blank, as shown in \cref{fpropten}.
}
As expected, commuting was found to be a large energy user.

Because commuter energy use scales with population, it was decided to represent
total energy use not in absolute terms, but relative to total energy use,
in each area. \Cref{fpropten} illustrates the spatial distribution of
the proportion of energy use across England. It shows that although the
average is just over 4\%, in some areas it approaches 10\%. Four areas
were identified in which commuter energy use accounted for over 9\% of total
energy use: Castle Point (a wealthy area in South
Essex),\footnote{Hints to its high commuter energy use, relative to its total
can be found on its Wikipedia page:
``Levels of home and car ownership in Hadleigh and Canvey are very high,
social deprivation is relatively low.'' `Commuters' are also
identified as a major economic group in the area {\color{blue}
\href{http://tinyurl.com/qfkb9ta}
{(see wikipedia link embedded in pdf)}}.
}
Maldon (another wealthy zone in Essex),
Rushmore (East Hampshire) and Tamworth (an urban area on the Northern
outskirts of Birmingham). Whether or not these areas can be classified as
`commuter belts' or if there are other reasons for their high energy use was
not explored and remains an interesting question for future research.
The only two Local Authorities in which commuting was found to account
for less than 1\% of total energy use were both in Central London.
A similar picture is painted when the proportion of total transport
energy use consumed by commuting is plotted (\cref{fproptrans}).
It inspires confidence that when total transport energy use was plotted
against commuter energy use, there was a strong positive correlation
(r = 0.75). This correlation was slightly higher than when
the simpler energy use per trip (Etrp) metric was used.
This correlation increased slightly when compared with
total road energy use. Surprisingly, the correlation
was even greater between total commuting energy use and total energy use
(r = 0.82). No explanation for this finding could be found.

\begin{figure}
 \centering{
 \includegraphics[width=15 cm]{prop-total-energy}}
 \caption[Proportion of total energy use in the UK consumed by commuting]
 {Proportion of total energy use in the UK consumed by commuting.
 Grey areas represent zones for which the DECC `NUTS 4' level did not coincide with
 Local Authorities from the census.
 }
 \label{fpropten}
\end{figure}

\begin{figure}
 \centering{
 \includegraphics[width=15 cm]{prop-trans-energy}}
 \caption{Proportion of transport energy use in the UK consumed by commuting.}
 \label{fproptrans}
\end{figure}


It is also interesting to compare the energy use estimates presented in the
previous section with official emission data, which have recently been
released as 2005 estimates (the closest to 2001 available) at the Local Authority
level.\footnote{These datasets can be accessed from {\color{blue}
\href{https://www.gov.uk/government/publications/local-authority-emissions-estimates}
{https://www.gov.uk/government/publications}}.
}
It was found that the total per trip costs were closely correlated to the
official estimate of total transport energy (r = 0.78) and that emissions
from minor roads were most closely correlated (\cref{tco2cor}). 
It is interesting to note that the variable most highly correlated
with per person energy commuter energy costs was transport emissions
from motorways. This can be explained by considering that areas near to
motorways tend to have longer commutes. There was also a fairly strong
positive correlation (r = 0.48) between per capita commuter energy use
and per capita transport use.

\begin{table}[htbp]
\caption[Correlation matrix of energy use for commuting and emissions]
{Correlation matrix of energy use for commuting and emissions at the Local
Authority level in England. ET and EAV are total and per capita commuter energy
costs, respectively.}
\begin{tabular}{lrrrrrr}
\toprule
 & \multicolumn{1}{l}{ET} & \multicolumn{1}{l}{EAV} & \multicolumn{1}{l}{A roads} & \multicolumn{1}{l}{M ways} & \multicolumn{1}{l}{Minor roads} & \multicolumn{1}{l}{Trans. Total} \\
 \midrule
ET & 1 &  &  &  &  &  \\
ETrp & 0.06 & 1 &  &  &  &  \\
A roads & 0.62 & 0.13 & 1 &  &  &  \\
M ways & 0.36 & 0.25 & 0.16 & 1 &  &  \\
Minor roads & 0.85 & -0.08 & 0.55 & 0.25  & 1 &  \\
Trans. Total & 0.78 & 0.18 & 0.71 & 0.74 & 0.74 & 1 \\
\bottomrule
\end{tabular}
\label{tco2cor}
\end{table}

In the policy context, commuter energy use has been
quantified at the national level and disaggregated by Local Authority.
It appears to be closely correlated with official data on transport energy
use and emissions. In the intuitive units recommended by \citet{MacKay2009},
commuting has been found to use, on average, 7.9 kWh/p/d for each commuter or
3.7 kWh/p/d for every man, woman and child living in England. In terms of
the total energy use figures developed by David MacKay (which includes
embodied energy and services such as defence), this equates to
only 1.9\% of per capita energy use. (The system boundaries in the
DECC analysis are far narrower, accounting for the differences between
MacKay's figures and theirs.) Even without including the system level
energy costs of commuting described in \cref{Chapter5}, this is a large
energy user for something that is so integral to a functioning society as
getting to work. However, the aggregate level is limited, and masks the
large differences that exist within statistical zones.
For this reason, the next section investigates the variability of commuter
energy costs at the individual level.


\section{Local and individual level variability} \label{sindvar}

As with any research in which geographical zones are the unit of analysis,
the maps of energy use presented above mask individual level variability within
zones. If interpreted incorrectly, conclusions resulting from such analyses
may be `ecological fallacies', where knowledge generated
at one level of understanding is incorrectly applied to another.
To provide an example, the strength of the correlation between wealth and the energy
costs of work travel at the ward level is unlikely to be the same as the
strength of the correlation at the level of individuals. The process of
geographic aggregation smooths relationships, often making correlations seem
greater and simpler that they really are \citep{Openshaw1983}. 

Spatial microsimulation can also be used to generate estimates of
geographically aggregated variables such as income, hence the use of the term
`small area estimation' used to describe some spatial microsimulation models
(see \cref{Chapter3}). Regarding the energy use of travel to work, spatial
microsimulation can help overcome a major data constraint at some geographical
levels: energy use is roughly a function of mode and distance of travel, yet
in some cases no cross tabulations on this matter are provided.
Even if average distances of travel to work are provided, it may be
impossible to know which modes of travel are responsible for high values.
When distance band and mode of travel are known but no cross-tabulations
are provided between them (as is the case with Super Output Area administrative
geographical levels from the data portal Casweb),
spatial microsimulation can be used to `fill in the gaps'.


A final potential issue with the ward level analysis of the entire nation, as presented
above, is the assumption that relationships are constant over space.
In many cases this assumption may justified (e.g. for the relationship between
population density and travel-to-work distance, which can be assumed to be
more-or-less universal), but sometimes relationships vary
substantially from place to place. This is a central motivation behind
geographically weighted regression \citep{Fotheringham2002}.

\subsection{A case study from South Yorkshire} \label{soyoref}
To illustrate the results of the spatial microsimulation model
in terms of energy use, a case study of South Yorkshire is used.
This county case study is used rather than the entirety of England because
processing time and memory demands were found to be problematic for
larger areas.\footnote{The
model was run for
Yorkshire and the Humber, which contains just over 2 million commuters.
Results were generated (as shown in \cref{svul}), but the time between
IPF iterations, and the tendency of the computer to lock-up after all
available RAM had been used --- on a computer with 12 Gb ---
led to a smaller case study region being selected.
}
The reasons for selecting South Yorkshire over other counties included the
clearly defined cities of Sheffield and Barnsley, as well as the region
between Sheffield, Rotherham and Doncaster that may be described as
the `South Yorkshire conurbation' \citep{barker1978perthes} --- it has
a diverse range of settlements from rural to urban and suburban.
In addition, social inequalities are quite clearly inbuilt into South Yorkshire's
geography. One can see, for example, where traits associated
with wealthy (to the west of Sheffield city centre, bordering
the Peak District) and more deprived (in the South-East of Sheffield,
for example) are located by visual inspection. The final reason
is that the author is well-acquainted with this area of England,
although a different case study region could equally have been used:
the purpose is to show the kinds of result that the
spatial microsimulation method can generate.
For continuity, 
South Yorkshire is also used as a case study region in the subsequent chapters.
% Because the majority of the analysis was done in R (with the R's
% improving geo-spatial data handling packages, all of the analysis
% could be done in this environment), some of the results are
% reported as listings, to illustrate how the results are accessed. %wtf?

After running the spatial microsimulation model outlined in
\cref{Chapter4}, constraining by age/sex, mode, distance of commute and
social class, an R object called a list is created. The list is a collection
of data tables, one for each administrative zone; each contains a number of
rows corresponding to the number of commuters in the area of interest.
The results for the first six individual in the first MSOA area
in South Yorkshire in the list (``Barnsley 001'') are displayed in
\cref{tintallh}.

\begin{table}[htbp]
\caption[Sample of the spatial microsimulation model output]
{Sample of the spatial microsimulation model output for South
Yorkshire. The table was saved as a comma-delimited file with the command
``intall[[1]]'', which refers to the data table corresponding to the
first zone in Sheffield. In total, the R object
``intall'' contains 532,130 individuals from 176 MSOA zones.}
\begin{tabular}{rrrrlrrlllr}
\toprule
\multicolumn{1}{l}{} & \multicolumn{1}{l}{a\_hidp} & \multicolumn{1}{l}{a\_pno} &
\multicolumn{1}{l}{pidp} & sex & \multicolumn{1}{l}{age} & \multicolumn{1}{l}{dis}
& mode & nssec8 & urb & \multicolumn{1}{l}{ncars} \\
\midrule
18 & 68041483 & 2 & 68041491 & male & 35 & 71 & Car (d) & Other & rural & 2 \\
18 & 68041483 & 2 & 68041491 & male & 35 & 71 & Car (d) & Other & rural & 2 \\
200 & 68303283 & 1 & 68303287 & male & 41 & 125 & Car (d) & Other & urban & 1 \\
200 & 68303283 & 1 & 68303287 & male & 41 & 125 & Car (d) & Other & urban & 1 \\
219 & 68323003 & 1 & 68323007 & male & 53 & 71 & Car (d) & Other & urban & 1 \\
219 & 68323003 & 1 & 68323007 & male & 53 & 71 & Car (d) & Other & urban & 1 \\
\bottomrule
\end{tabular}
\label{tintallh}
\end{table}

From the household and personal ids (a\_hidp and a\_pidp) can be joined a
wide range of additional variables (\cref{tintcar}).
Binding the information representing
in \cref{tintallh} for all 176 zones (using the command \verb do.call() )
results in a single table representing all five hundred thousand commuters
in South Yorkshire. From here, energy use data can be produced for each
individual, using the same technique described for the calculation of
aggregate energy use. The additional refinement added at this individual level
was the size of car: large cars were allocated a higher value (3.9 MJ/km)
than small cars (2.5 MJ/km).\footnote{13.6\% of
responses to this question were ``inapplicable'' or some other `NA' value,
even amongst those who drove a car. In these cases the energy costs were
set equal to those of a medium-sized car.}
\begin{table}[htbp]
\caption[Sample of individual level spatial microsimulation output]
{Sample of individual level microsimulation output. The number of cars
in the individuals' household and the engine size of their primary car
are extracted using the merge() ~function applied to the ID codes,
that are also present in \cref{tintallh}}
\begin{tabular}{rrrrrlr}
\toprule
\multicolumn{1}{l}{} & \multicolumn{1}{l}{a\_hidp} & \multicolumn{1}{l}{a\_pno} & \multicolumn{1}{l}{pidp} & \multicolumn{1}{l}{N.~cars} & Engine size & \multicolumn{1}{l}{Et} \\ \midrule
18 & 68041483 & 2 & 68041491 & 2 Sheffield& medium engine - 1.4 - 1.9999 & 312.3 \\
18 & 68041483 & 2 & 68041491 & 2 & small engine - 1.0 - 1.3999 & 268.3 \\
200 & 68303283 & 1 & 68303287 & 1 & inapplicable & 743.8 \\
200 & 68303283 & 1 & 68303287 & 1 & small engine - 1.0 - 1.3999 & 471.7 \\
219 & 68323003 & 1 & 68323007 & 1 & inapplicable & 423.0 \\
219 & 68323003 & 1 & 68323007 & 1 & medium engine - 1.4 - 1.9999 & 312.3 \\
\bottomrule
\end{tabular}
\label{tintcar}
\end{table}

The impact of car engine size on the relative average energy use of each
zone was found to be very small and the correlation between values calculated
that did not take car size into account and values that did was very high
(r = 0.9985). The resulting spatial distribution of energy costs of
commuting at the MSOA level is plotted in \cref{fsoyoen}. This illustrates how
spatial microsimulation can be used to create estimates of energy use at the
aggregate level when cross-tabulated distance/mode datasets are unavailable.
At the individual level, the standard deviation in per trip
energy use is much greater than at the geographical level in this
case study: 95 MJ between individuals compared with only 11 MJ between
MSOA areas. This reflects the impact of geographical smoothing and also
provides an indication of the high level of inequality in energy use for
work travel between commuters living in the same area.

\begin{figure}[htbp]
\begin{center}
    \includegraphics[width=14cm]{soyoen}  \end{center}
  \caption[Commuter energy use in South Yorkshire.]
  {Energy use (direct and indirect) per commuter trip at the
  MSOA level in South Yorkshire.}
 \label{fsoyoen}
\end{figure}

\begin{figure}[htbp]
\begin{center}
    \includegraphics[width=14cm]{prop20-top}  \end{center}
  \caption[Proportion of energy used for commuting by the top 20\%]
  {Proportion of energy used for commuting by the top 20\% of commuters.
  Highest and lowest areas labelled for future reference.}
 \label{fineq20}
\end{figure}

The individual level results are well-illustrated by plotting the proportion
of energy use consumed by different groups. The example plotted in
\cref{fineq20} represents the proportion of energy use for commuting
consumed by the 20\% most energy-intensive commuters, which is also a
proxy for inequality. This plot shows a very clear spatial pattern,
with city centres being associated with the most unequal distribution
of commuter energy costs. We will return to this point in the subsequent
chapter --- for now suffice to say it is an interesting result.
To illustrate the method's ability to disaggregate
by socio-economic categories, \cref{ftopprop} shows the ratio of energy
used for commuting by the top social classes (1.1 and 1.2) compared with
the average energy cost per commute in each area. It is interesting to
note that in all areas the value is above 1.4, reaching more than 3 times
the average in some areas.

In fact, one can use the simulated spatial microdata
to cross-tabulate any combination of variables within any area.
This is illustrated in \cref{tindenergy}, which shows the
link between socio-economic class and commuter energy use
for 3 geographical zones: South Yorkshire overall, as well as the same
relationships in the most and least unequal areas, defined in \cref{fineq20}.
The results indicate that in the centre of Sheffield (`Sheffield 031'),
the lowest classes tend to work closer to home, on average, than the averages for
their class overall and that distance travelled is highly unequally distributed.
In North Stocksbridge (`Sheffield 001'), by contrast, there is much less
difference between different classes. It is also interesting to note that
the average energy intensity of trips in the city centre is lower for all
classes than in Stocksbridge. This can be explained by the proximity to
tram and rail stations and the higher proportion of walking and cycling.
We build on these insights in \cref{Chapter7} to further explore the
inequalities in commuting and commuter energy use in the study region.

\begin{table}[htbp]
\caption[Commuter energy use in South Yorkshire areas by class]
{Average commuter energy use (MJ/trip), distance (km) and energy intensity (MJ/km)
in South Yorkshire (SOYO) by socio-economic class.
The three areas are SOYO and the most and least
unequal zones in terms of the distribution of individual energy use (see
\cref{fineq20})}
\begin{tabular}{rl|rrr|rrr|rrr}
\toprule
\multicolumn{2}{c}{Area $\rightarrow$} &  \multicolumn{3}{|c|}{SOYO}  & \multicolumn{3}{c}{Shef 031} & \multicolumn{3}{|c}{Shef 001} \\
\multicolumn{1}{l}{} & Employment class & \multicolumn{1}{l}{Etrp} & \multicolumn{1}{l}{Dis} & \multicolumn{1}{l}{EI} & \multicolumn{1}{|l}{Etrp} & \multicolumn{1}{l}{Dis} & \multicolumn{1}{l}{EI} & \multicolumn{1}{|l}{Etrp} & \multicolumn{1}{l}{Dis} & \multicolumn{1}{l}{EI } \\
\midrule
% 1 & other & 62 & 14.5 & 4.3 & 87 & 21.0 & 4.2 & 60 & 15.2 & 3.9 \\
 & large employers  & 111 & 27.5 & 4.1 & 141 & 39.6 & 3.6 & 119 & 28.4 & 4.2 \\
 & higher professional & 73 & 17.8 & 4.1 & 102 & 27.3 & 3.7 & 86 & 19.9 & 4.3 \\
 & lower management  & 56 & 14.5 & 3.8 & 66 & 21.9 & 3.0 & 59 & 16.3 & 3.6 \\
 & intermediate & 29 & 8.1 & 3.6 & 17 & 7.5 & 2.3 & 47 & 12.1 & 3.9 \\
 & lower supervisory  & 39 & 10.5 & 3.7 & 16 & 8.8 & 1.9 & 58 & 14.8 & 3.9 \\
 & semi-routine & 20 & 8.4 & 2.4 & 10 & 11.1 & 0.9 & 28 & 13.7 & 2.0 \\
 & routine & 26 & 8.1 & 3.2 & 9 & 5.8 & 1.6 & 42 & 12.7 & 3.3 \\
 \bottomrule
\end{tabular}
\label{tindenergy}
\end{table}

More detailed analysis at the individual
level is presented in \cref{Chapter7}. The results presented in this section
demonstrate that individual level variability in commuter energy use
is important and in some cases potentially more so than inter-zone variation.



\begin{figure}[htbp]
\begin{center}
    \includegraphics[width=14cm]{topprop}  \end{center}
  \caption[Relative energy use by top social classes]
  {Relative energy use by top social classes in South Yorkshire.}
 \label{ftopprop}
\end{figure}

\section{A comparison of commuter energy use in England and the Netherlands}
\label{sinternational}
In order to demonstrate that the methods can be used internationally,
this section provides a short case study, comparing the energy costs
of home-work travel in England and the Netherlands. These countries
were chosen for the following reasons:
\begin{itemize}
 \item Geographically aggregated data could be found for both.
 \item There are reasons to expect the Netherlands to have commuting energy costs
 substantially different from those in England. The working hypothesis we
 set out to test was that the Netherlands would have lower energy costs, primarily
 due to the high uptake of cycling, for which the nation is famous.
 \item The countries are similar `on paper', in terms of population density,
 GDP per capita and culture.
\end{itemize}
The final point is illustrated in \cref{tcompare}, which shows the extent to
which England and the Netherlands are similar according to a handful of basic
measures. One major difference between the two countries is in terms of
income inequality, with England being substantially more unequal.
If only \cref{tcompare} were considered, one would assume that the energy
costs of commuting would be roughly the same in the two countries. However,
a couple of factors led to the hypothesis that commuting in the Netherlands
would be less energy-intensive: its relative size (42,000 km$^2$ vs 130,000 km$^2$
for England) and its famously high rate of cycling, which account for
27\% of trips nationwide and above 50\% of trips in some cities
\citep{Pucher2008}.

\begin{table}[htbp]
\caption{Comparison of basic national attributes in England and the Netherlands}
\begin{center}
\begin{tabular}{llrl}
\toprule
Attribute & England & \multicolumn{1}{l}{Netherlands } & Units \\
\midrule
Population density & \multicolumn{1}{r}{407} & 406 & ppl/km2 \\
GDP & \multicolumn{1}{r}{50000} & 46000 & \$/capita \\
Income inequality & 34 (UK) & 31 & Gini Index \\
Wellbeing & 0.875 (UK) & 0.921 & UN HDI \\
\bottomrule
\end{tabular}\end{center}
\label{tcompare}
\end{table}

\subsection{Data, method and results} \label{sdutchdata}
The input dataset for the Netherlands came in a different form from
that of England. The English data, downloaded from the Census,
provided 88 key columns from which energy values were generated:
8 distance bins for 11 modes of transport. Based on average route distances
estimated for each of the 8 Euclidean distance bins for the 8 modes whose
energy costs are described in \cref{sfinal}, the energy costs per one-way
trip were calculated for each cell in all of the 88 columns. The values in
each of the cells of the English data are people counts, constraining the
number of people in each distance/mode category.
The Dutch dataset, on the other hand, provided proportions, average distances
and average times for 8 modes of transport in a wide format (\cref{tdutch}).
The first challenge upon receiving this dataset was to understand the
table's structure and translate the column headings into English.
Another issue was finding geographical data for Dutch provinces and their
populations (this allowed for the energy costs per province to be weighted,
to provide an accurate estimate of average energy costs per commuter trips
nationwide). This data was provided by the open-data initiative
Natural Earth.\footnote{\href{http://www.naturalearthdata.com/}
{http://www.naturalearthdata.com/}
}

\begin{table}[htbp]
\caption[Sample of the raw Dutch commuting data]
{Sample of the first 4 columns of the raw Dutch commuting data. A further 54
columns on the proportions travelling by and average time and distances of
trips by 9 modes of transport are not shown.}
\centering{\begin{tabular}{lrrr}
\toprule
Perioden & 2010 & 2010 & 2010 \\
Vervoerwijzen & \multicolumn{1}{l}{Totaal} & \multicolumn{1}{l}{Auto (bestuurder)} & \multicolumn{1}{l}{Auto (passagier)} \\
Regio's & \multicolumn{1}{l}{aantal} & \multicolumn{1}{l}{aantal} & \multicolumn{1}{l}{aantal} \\
\midrule
Nederland & 0.48 & 0.25 & 0.03 \\
Groningen (PV) & 0.44 & 0.22 & 0.03 \\
Friesland (PV) & 0.45 & 0.24 & 0.02 \\
Drenthe (PV) & 0.46 & 0.29 & 0.03 \\
Overijssel (PV) & 0.48 & 0.26 & 0.03 \\
Flevoland (PV) & 0.51 & 0.28 & 0.04 \\
Gelderland (PV) & 0.47 & 0.26 & 0.03 \\
Utrecht (PV) & 0.5 & 0.23 & 0.03 \\
Noord-Holland (PV) & 0.48 & 0.22 & 0.03 \\
Zuid-Holland (PV) & 0.49 & 0.23 & 0.03 \\
Zeeland (PV) & 0.47 & 0.27 & 0.03 \\
Noord-Brabant (PV) & 0.47 & 0.28 & 0.04 \\
Limburg (PV) & 0.46 & 0.28 & 0.03 \\
\bottomrule
\end{tabular}}
\label{tdutch}
\end{table}

Finally, the commuting dataset was matched to the geographical shapefile
data in %!!! add reference of the files where this is done.
R.\footnote{Initially
this stage was problematic, as was discovered when the regions were
plotted with their name codes highlighted: the names were not associated
with the correct geographical areas. The R code used was reviewed at
each stage and it was discovered that the error was introduced through
the ``merge()'' function, which allocated the tabular data to the
geographical data by matching the zone codes. It was found that the
default (silent) default argument of ``merge()''~is ``sort=TRUE'' .
This meant that the function was re-ordering the geographical data
alphabetically. Adding ``sort=F'' ~into the command solved the problem.
}
Despite these data preparation issues, the Dutch dataset
was in fact easier to convert into average energy costs per trip than
the UK data, as it was simply the product of mode efficiency ($Ef$), average
route distance ($dR$) and modal split ($p$) for each mode:
\begin{equation}
 Etrp = \sum_m p_m \times Ef_m \times \bar{dR}_m
\end{equation}
This formula was applied to Dutch regional data, and aggregate energy costs
were calculated for England using the method described in \cref{snational}.
The results, illustrated in \cref{fdutchen}, came as a surprise:
energy use for commuting is \emph{higher} in the Netherlands, which is relatively
small, bicycle-friendly and has a low GDP, than in England. The difference
is not as great as that represented in \cref{fdutchen} (a 14\% difference,
when energy use per trip is averaged across all zones), because the zones
are not of equal population or size. When commuter energy costs are
weighted by population, the overall average energy cost per commuter trip
is still higher in the Netherlands, but less so --- 8\%:
37.5 MJ/trip in the Netherlands against 34.5 MJ/trip in England.

\begin{figure}
\centering{
\includegraphics[width=12cm]{dutchen}}
 \caption[Comparison of commuter energy use in England and the Netherlands]
 {Comparison of commuter energy use in England and the
Netherlands.} \label{fdutchen}
\end{figure}
\subsection{Explaining Dutch commuter energy use}
To explore this non-intuitive result, the first stage was to look at the modal
split of commuting in England and the Netherlands (\cref{fdutchmode}).
As expected, Dutch commuters are far more likely to travel to work by bicycle.
However, they are also less likely to travel to work by walking, as a car
passenger or by metro (due primarily to the London Underground) --- all low-energy
modes --- than UK commuters. The proportion of people travelling by car,
the most energy-intensive personal travel mode, is  only slightly lower in the
Netherlands (57\%) than in England (60\%) despite the 27\% of trips made by
bicycle. Modal split cannot account for
unexpectedly high Dutch commuter energy costs.

\begin{figure}
\centering{
\includegraphics[width=12cm]{envsnl-modesplits}}
 \caption[Modal split of commuter trips in England and the Netherlands]
 {Modal split of commuter trips in England and the
Netherlands.}\label{fdutchmode}
\end{figure}

The next variable explored was distance. The average Dutch
commute for the major forms of transport %%% The overall average is 17.6km!!!
is 1 km further than the English average at 15.5 km, from the data.
This may seem like a small amount, yet it is almost 7\% further, accounting for
most of the variability in energy use. When we
break this figure down by mode, as in \cref{favdistnl}, it becomes clear that
car trips are the reason for the increased distance of travel to work
in the Netherlands: all other modes are associated with shorter trip distances,
whereas the average commuter trip by car, the most energy intensive transport
mode, is \emph{30\%} further than in England (24.6 km in
the Netherlands, compared with 18.7 km). It therefore seems that
the prevalence of one particular trip type --- long car trips --- explains why
commuter energy use in the Netherlands is greater, per person, than in the UK.

To explore the underlying reason for these high-distance car commutes,
the length of motorway in each country was found. In the Netherlands
there are 2631 km of motorways whereas in the
England there are 3673 (Eurostat, 2013, via the UK Data Service). These
values equate to roughly 150 km of motorway per million people in the Netherlands,
compared with only 70 km per million in England, less than half. Despite this
advanced road network, and the bicycle infrastructure for which Holland is
famous, road congestion is a known problem \citep{OECD2010}. The average time for commutes
in the Netherlands is longer than for any other nation in the Organisation for
Economic Cooperation and Development, something that has been attributed to
high population density and a rigid housing market: ``more than just transport
policies are required to solve these problems'' \citep[p.~8]{OECD2010}.
% !!! spatial distribution???

\begin{figure}
\centering{
\includegraphics[width=12cm]{avdist-nl-en}}
 \caption[Distance of commuting by mode, England and the Netherlands]
 {Average distance of commuter trip by mode in England and the
Netherlands.}\label{favdistnl}
\end{figure}

Regarding the spatial distribution of energy-intensive commuting,
there is no clear pattern at this coarse level of geographical aggregation.
A pattern does emerge when energy use is plotted against
population density (\cref{fepdensnl}), which shows a strong negative correlation
(r = -0.7, p $<$ 0.001) between the two variables. The two clear outliers in
terms of energy use are London (20.8 MJ/trip) and Flevoland (54.8 MJ/trip),
which are also on opposite ends of the population density scale.
\Cref{fepdensnl} is also useful as it shows there is a large amount of
overlap in commuter energy between the two countries, even at this high
level of geographical aggregation. Three English regions
(the South East, East of England and the East Midlands) have average
commuter energy costs above the Dutch national average; interestingly
each of these zones is quite wealthy, with strong links to London
(implying commuting to London may be a cause of high energy use here).
The only Dutch province with average commuter energy costs below the
English average is Zuid (meaning South) Holland. This area has a very high
population density and includes large cities including the Hague and
Rotterdam.

\begin{figure}
 \centering{\includegraphics[width=12cm]{epdensnl}}
 \caption[Population density against commuter energy use]
 {Population density against commuter energy use, in Netherlands and England.}
 \label{fepdensnl}
\end{figure}

\subsection{Data inconsistencies and caveats}
A problem with the preceding national level comparison is that the
data come from different years, 2001 and 2010 for England and the Netherlands
respectively. One could argue that this is not an issue
from the perspective of demonstrating the international applicability of the
methods. However, it is a major problem if the aim is to use the empirical results
to inform policy.
for example to argue that a focus on modal split alone may not be  effective at
increasing the sustainability of personal travel, if distance is not considered
as well. %!!! add a link here
That energy use per commute is greater in the Netherlands
than in England is an interesting result in itself and merits
corroboration with additional data to confirm this result.

\Cref{fcommuterdistime} shows that the length of commuter trips in Great
Britain (including Wales and Scotland) has remained steady over
time. It increased by only 5\% between 1995/1997 and 2009 and only by
1\% between 2002 (the closest data point to 2001) and 2009. In addition,
\cref{fmode-time-dft2011} demonstrates that the modal split of commuter trips
has also been relatively steady, with slight declines in car use suggesting that
energy use may have even declined. 

\begin{figure}
 \centering{\includegraphics[width=12cm]{commuter-trip-dis-time}}
 \caption[Average commuter trip distance over time in Great Britain]
 {Average commuter trip distance over time in Great Britain. Data from
 \citet[table 9]{DfT2011-commuting}, n $>$ 15,000 for every year.} \label{fcommuterdistime}
\end{figure}

\begin{figure}
 \centering{\includegraphics[width=12cm]{mode-time-dft2011}}
 \caption[Modal split of commuter trips, Great Britain 1995 - 2009]
 {Modal split of commuter trips, Great Britain 1995 - 2009. Data from
 \citet[table 9]{DfT2011-commuting}, n $>$ 15,000 for every year.} \label{fmode-time-dft2011}
\end{figure}

Another issue is data quality. While both datasets
are from official sources, the Dutch dataset is far less detailed and provides
only two significant figures for the proportions of people travelling by each
mode (e.g. 0.01). Thus, error up to 0.5\% in these figures is possible.
Further, average distances were not provided for all modes of transport in all
areas, in which case the mode's average figure for the areas that were reported
were used to fill in the gap. Finally, the figures for the proportion of people
travelling by train seemed very low, given that the Netherlands has an
advanced rail network. As outlined in \cref{Chapter4}, %!!! really???
there are also issues with the UK dataset. The translation of
Euclidean distance
categories into average route distances is a particularly risky
activity and may introduce error in excess of the difference between
Dutch and English average commuter trip energy costs reported above.

In light of these caveats, it is concluded that a 
more robust dataset from the Netherlands is needed to resolve the
enigma of high Dutch commuter energy use. The basic method used to calculate
energy costs has been shown to be applicable to another country,
although more refinements (e.g.~alterations in the average energy
intensity of Dutch cars) will be needed if this result is to be
seen as robust. If it holds up to further investigation, it is an interesting
and policy relevant result: it would illustrate that promotion of urban
cycling alone is not enough to reduce the overall energy costs of personal
transport nationwide.


% \section{Changes over time}
% % \section{Changes in energy use over time} \label{stime}
% % Mothballed for now - now that interesting, and not needed atm
% As seen in the gradual improvements in car fleet efficiencies reported in
% \cref{Chapter5}, the rate of energy use in transport systems is in constant
% flux due to changing technologies. While efficiency improvements tend
% to be gradual and predictable (due to regulation, and the long lead times
% in car model development), human behaviour is not. It can respond rapidly
% and unexpectedly to external factors, such as the oil price
% spike of 2008 \citep{Sexton2011} and can be shifted
% by emergency measures such as food rationing, conscription and
% enforced agricultural and mine labour policies in World War II.
% 
% More mundane shifts in behaviour have affected the energy costs of commuting
% in recent years, including the rise of `telecommuting', `flexi-time', increased
% labour mobility and the tendency of people to start families far from their
% home. Over the last 300 years the most important change
% in commuting behaviour has been its emergence as a common activity:
% before the industrial revolution and
% the emergence of centralised factories,
% many people worked in `cottage industries' at home. A common arrangement in rural
% areas in the
% 1800s, for example, was for the men to work long hours (50 + per week) on
% neighbouring farmland, while women worked at home \citep{groves1949sharpen}.
% From the historical literature
% it is clear that the energy costs of transport to work
% before the industrial revolution were very low indeed, consisting primarily of
% people walking a mile or two to work and a relatively small number of
% horse-drawn carriages for the wealthy whilst the majority of
% of the population worked from or very near to home.
% 
% To gain insight into commuting energy use before the advent of
% large official databases on work and travel to work
% behaviour, anecdotal and archival evidence must be relied upon. Over the
% past 100 years there has been a trend towards official
% data collection, analysis, storage and dissemination. In the last 50 years,
% this tendency has been greatly accelerated and automated by the `digital
% revolution', as mentioned in \cref{Chapter3}.
% Still, before 1971 (when travel to work is first made available as a
% Census variable) the best source of commuting data that could be
% found was from a retrospective survey of elderly people, asked about their
% past travel to work habits. %!!!
% The longest available dataset on commuting that could be found was collected by
% \citet{Turnbull2000}, in a retrospective questionnaire about commuting habits over
% the past 100 years...

% \section{Explaining high and low energy use} \label{sexplanation} %!!! re-add
% \index{decomposition framework}
% Regardless of technology and the various complicating factors discussed
% in \cref{svariable}, the primary determinants of the energy costs of
% personal transport are mode and distance. Focussing simply on one or the other
% (as other authors have done) omits a substantial part of the picture because it
% is the combination of an energy intensive mode and an energy intensive trip that
% leads to high energy costs. In England as a whole, the most energy intensive
% form of transport (single occupancy cars) is the most common form of transport
% to work for all but the shortest trips \cref{fengmodedis}.

% The aggregate estimate of energy use is interesting in itself, allowing
% commuting energy used to be placed in context of other phenomena.
% % It should be clear that commuting is more important than many other
% % ``energy issues'' such as the efficiency of lights and solar panels,
% % which have received relatively more attention from an energy perspective from
% % policy makers and academicis. !!! put commuter energy use in perspective - fig
% However, one of the aims (A1.2, \cref{s:aims}) was also to explain why energy
% use in transport to work is as it is. As illustrated in previous literature
% \cref{Chapter2},
% many factors contribute to the total energy use in transport systems.
% To recap, these include number, distance and frequency of trips, occupancy,
% mode, fleet efficiencies, infrastructure impacts and behavioural factors such
% as driving style. %!!! refs.
% 
% This section will formalise these considerations using a framework: the
% decomposition framework \citep{}. In terms of total energy use in the economy,
% three main factors (activity, structural and energy intensity effects)
% can be modelled to explain and project growing or declining energy use
% \citep{farla2000physical}.
% This framework can be applied to the decomposition of energy use in, as
% illustrated in \cref{f:components}. Starting with decomposition by mode, the
% formulae to describe each element of the decomposition analysis are formalised
% in equations \ref{e:component1} to \ref{e:componentn}.
% 
% \begin{equation}
%  Etot = {\displaystyle \sum^m_{m=1}
% N_{trips,m} \times \overline{d}_{R,m} \times \overline{E}_{F,m}}
% \end{equation}  


% \section{Uncertainties and complications} \label{suncertainties}
% 
% ``correctness is usually expensive, and high correctness is often
% \emph{disproportionately} more expensive'' \citep[p.~153]{janert2010data}.

% !!! Links with section in ch.4.

% \subsection{The relevance of flow data for the energy costs of travel to work}
% The route taken by a given vehicle can have a large impact on its energy use.
% This can be illustrated by considering cases of bus travel, on various
% locations of the energy efficiency spectrum:
% \begin{itemize}
%  \item dhf
% 
% \end{itemize}
% 
% On one hand,  extreme example could be the
% comparative efficiency per person of an old London bus carrying 20 people
% through clogged streets c

\section{Discussion}
In this chapter the methods and data presented in
\cref{Chapter4} have been combined with the estimates of energy use by mode presented in
\cref{Chapter5} to calculate the energy costs of commuting at a range of scales.
The main unit of measurement used to present these results is
energy use per one-way commuter trip. This is a useful measure, as it is
robust to variations in the employment rate and makes no assumptions about
frequency of trip. If the aim is to compare commuting with other energy-using
activities, however, the results would be more usefully presented as energy
costs per person per day. This approach was undertaken
by \citet{Boussauw2009}, which would allow
direct comparisons between commuter energy use and other `essential'
energy costs such as electricity and gas use in the house and (depending on
data availability), other travel costs. %!!! Add here results of e.use/zone!!!

Despite these limitations, the findings are still useful in their own right.
From inspection of the district and ward level maps, it is clear that dense
urban areas tend to have lower average commuting costs than the countryside.
London is the extreme manifestation of this tendency, and has achieved
commuting energy costs below the national average throughout most of its
wards. However, many of the areas within roughly 100 km but outside
Greater London have unusually high  average energy costs per commute.
This is likely to be due to long-distance commuters and `commuter belts'
which serve London's vast service sector. It is concluded from this
pattern that citywide personal transport costs should not be evaluated
only in terms of the internal flows within them: flows from the surrounding
areas should also be considered.

The results presented in this chapter provide much scope for further research.
The pattern of London as a centre of relative commuting sustainability surrounded by
a ring of high energy costs, for example, raises the following question:
are cities, overall, associated
with lower commuting energy costs than rural settlements, once long-distance
commuting has been taken into account? This question feeds into the ongoing
debate about compact cities and urban forms that are conducive to reduced energy
use \citep{Levinson2012}. Moreover, the descriptive results require explanation.
Is there a model that can successfully explain the variability in energy
use observed, based solely on population distribution and infrastructure?
If so, this would have implications for planning policy, as the energy impacts
of new settlements (e.g.~housing estates) and transport infrastructure could
be predicted.

This potential for policy relevance leads on to the tentative
finding that Dutch commuter trips are, on average, more energy intensive than
English ones. This, if it was confirmed, would strongly suggest that simply
trying to emulate the Netherlands in terms of rates of urban cycling
would not guarantee environmental and other benefits of lowered energy use.
The finding supports the conclusion of \citet{Boussauw2009}, that
interventions aiming to reduce the distance between
home and work may be more effective than those aimed at changing
modal split.

Before exploring some of these broad policy-relevant questions 
in \cref{Chapter8}, the next chapter zooms-in, to a single case-study area.
This is to illustrate the ability of the spatial microsimulation approach to
explore local commuting patterns and evaluate specific transport interventions.

