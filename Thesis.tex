%% ----------------------------------------------------------------
%% Robin Lovelace's thesis (based on Sunil Patel's design from http://uk.tug.org/)
%% ----------------------------------------------------------------


% Set up the document
\documentclass[a4paper, 11pt, twoside]{Thesis}  % Based on the ECS Thesis style
\usepackage{makeidx}
\graphicspath{{Figures/}{/home/robin/Dropbox/vul-meth/Figures/}
{/home/robin/Dropbox/vulnerability/Figures/}}  % Location of the graphics files
\usepackage{multirow}
\usepackage{threeparttable}
\usepackage{idxlayout}
% Making R code work!
\usepackage{listings}
\usepackage{color}
\usepackage{hyperref}
\hypersetup{urlcolor=blue, colorlinks=false, hypertexnames=true}  % Colours hyperlinks in blue, but this can be distracting 
\usepackage{cleveref}
\definecolor{dkgreen}{rgb}{0,0.6,0}
\definecolor{gray}{rgb}{0.5,0.5,0.5}
\definecolor{mauve}{rgb}{0.58,0,0.82}

\lstset{ %
  language=R,                % the language of the code
   basicstyle=\normalsize\ttfamily,           % the size of the fonts that are used for the code
%   numbers=left,                   % where to put the line-numbers
%   numberstyle=\tiny\color{gray},  % the style that is used for the line-numbers
%   stepnumber=2,                   % the step between two line-numbers. If it's 1, each line
                                  % will be numbered
%   numbersep=5pt,                  % how far the line-numbers are from the code
%   backgroundcolor=\color{white},      % choose the background color. You must add \usepackage{color}
%   showspaces=false,               % show spaces adding particular underscores
%   showstringspaces=false,         % underline spaces within strings
%   showtabs=false,                 % show tabs within strings adding particular underscores
   frame=false,                   % adds a frame around the code
   rulecolor=\color{white},        % if not set, the frame-color may be changed on line-breaks within not-black text (e.g. commens (green here))
%   tabsize=2,                      % sets default tabsize to 2 spaces
%   captionpos=b,                   % sets the caption-position to bottom
%   breaklines=true,                % sets automatic line breaking
%   breakatwhitespace=false,        % sets if automatic breaks should only happen at whitespace
%   title=\lstname,                   % show the filename of files included with \lstinputlisting;
                                  % also try caption instead of title
  keywordstyle=\color{blue},          % keyword style
  commentstyle=\color{dkgreen},       % comment style
  stringstyle=\color{mauve},         % string literal style
  escapeinside={\%*}{*)},            % if you want to add a comment within your code
  morekeywords={*,...}               % if you want to add more keywords to the set
} 

% Include any extra LaTeX packages required
\usepackage[round,]{natbib}  % Use the "Natbib" style for the references
\usepackage{verbatim}  % Needed for the "comment" environment to make LaTeX comments
\usepackage{wallpaper}
\usepackage{cases}
\makeindex
% \renewcommand{\includegraphics}[2][]{\fbox{#2}} %omits images

 \AtBeginDocument{%
    \crefname{equation}{equation}{equations}%
    \crefname{chapter}{chapter}{chapters}%
    \crefname{section}{section}{sections}%
    \crefname{appendix}{appendix}{appendices}%
    \crefname{enumi}{item}{items}%
    \crefname{footnote}{footnote}{footnotes}%
    \crefname{figure}{figure}{figures}%
    \crefname{table}{table}{tables}%
    \crefname{theorem}{theorem}{theorems}%
    \crefname{lemma}{lemma}{lemmas}%
    \crefname{corollary}{corollary}{corollaries}%
    \crefname{proposition}{proposition}{propositions}%
    \crefname{definition}{definition}{definitions}%
    \crefname{result}{result}{results}%
    \crefname{example}{example}{examples}%
    \crefname{remark}{remark}{remarks}%
    \crefname{note}{note}{notes}%
}
%% ----------------------------------------------------------------
\begin{document}
\pagenumbering{roman}
% \ThisCenterWallPaper{1}{wallpaper}%
\frontmatter	  % Begin Roman style (i, ii, iii, iv...) page numbering

% Set up the Title Page
\title  {The energy costs of commuting: a spatial microsimulation approach}
\authors  {\texorpdfstring
            {\href{mailto:rob00x@gmail.com}{Robin Lovelace, MSc BSc}}
            {Robin Lovelace}
            }
\addresses  {\groupname\\\deptname\\\univname}  % Do not change this here, instead these must be set in the "Thesis.cls" file, please look through it instead
\date       {\today}
\subject    {}
\keywords   {}

\maketitle
%% ----------------------------------------------------------------

\setstretch{1.3}  % It is better to have smaller font and larger line spacing than the other way round

% Define the page headers using the FancyHdr package and set up for one-sided printing
\fancyhead{}  % Clears all page headers and footers
% \rhead{\thepage}  % Sets the right side header to show the page number
% Above edited based on
% http://www.sunilpatel.co.uk/thesis-template/template-faq/ 4 2 page print!!!
\fancyhead[LE,RO]{\thepage}
\fancyfoot{}
\lhead{}  % Clears the left side page header

\pagestyle{fancy}  % Finally, use the "fancy" page style to implement the FancyHdr headers

%% ----------------------------------------------------------------
% Declaration Page required for the Thesis, your institution may give you a different text to place here
% \Declaration{
% 
% \addtocontents{toc}{\vspace{1em}}  % Add a gap in the Contents, for aesthetics
% 
% I, Robin Lovelace, declare that this thesis titled, `The energy costs of commuting: a spatial microsimulation approach'
% and the work presented in it are my own. I confirm that:
% 
% \begin{itemize}
% \item[\tiny{$\blacksquare$}] This work was done wholly or mainly while in candidature for a research degree at this University.
% 
% \item[\tiny{$\blacksquare$}] Where any part of this thesis has previously been submitted for a degree or any other qualification at this University or any other institution, this has been clearly stated.
% 
% \item[\tiny{$\blacksquare$}] Where I have consulted the published work of others, this is always clearly attributed.
% 
% \item[\tiny{$\blacksquare$}] Where I have quoted from the work of others, the source is always given. With the exception of such quotations, this thesis is entirely my own work.
% 
% \item[\tiny{$\blacksquare$}] I have acknowledged all main sources of help.
% 
% \item[\tiny{$\blacksquare$}] Where the thesis is based on work done by myself jointly with others, I have made clear exactly what was done by others and what I have contributed myself.
% \\
% \end{itemize}
% 
% 
% Signed:\\
% \rule[1em]{25em}{0.5pt}  % This prints a line for the signature
% 
% Date:\\
% \rule[1em]{25em}{0.5pt}  % This prints a line to write the date
% }
\clearpage  % Declaration ended, now start a new page

%% ----------------------------------------------------------------
%%%%% Uncomment from here!
%% ---------------------------------------------------------------- 
% The "Funny Quote Page"
\pagestyle{empty}  % No headers or footers for the following pages

\null\vfill
% Now comes the "Funny Quote", written in italics
\textit{``A finales del siglo XX, y gracias a su automovil privado, un simple
trabajador pod\'{i}a residir en un lugar determinado pero desempe\~nar su trabajo,
diariamente, en otro lugar que se encuentra a 50 o 60 km de distancia. Este
hecho, que para tal ciudadano formaba parte de la rutina de su vida
cotidiana, constituye, sin duda, uno de los m\'{a}s grandes enigmas de la
antropolog\'{i}a y la historia''}

\begin{flushright}
Jos\'{e} Ardillo, \emph{El Salario del Gigante}
\end{flushright}

\textit{``Towards the end of the 20$^{th}$ century, and thanks to the private
automobile, a simple worker could live in one place but carry out their work,
daily, 50 to
60 km away. This fact, which for the citizen formed part of their everyday
routine, constitutes, without doubt, one of the greatest enigmas of Anthropology
and History''}

\begin{flushright}
Author's translation
\end{flushright}



\vfill\vfill\vfill\vfill\vfill\vfill\null
\clearpage  % Funny Quote page ended, start a new page
%% ----------------------------------------------------------------

% The Abstract Page
\addtotoc{Abstract}  % Add the "Abstract" page entry to the Contents
\abstract{
\addtocontents{toc}{\vspace{1em}}  % Add a gap in the Contents, for aesthetics
Commuting is a daily 
ritual for a large proportion of the world's population.
It is important materially,
consuming large amounts of time, money and natural resources.
% Yet, as with many routine activities, travel to work is often
% taken for granted. In terms of inflexible energy use ---
% consumption that cannot easily be cut through reduced demand --- 
% commuting is of great interest.
As with many routine activities travel to work is often taken for
granted but its energy consumption is of particular interest due to
its heavy reliance on fossil fuels and
the inflexibility of the demand for commuting.
This understudied area of knowledge, the
energy costs of travel to work, forms the basis of the thesis.

% The variability of energy use for commuting over time has been
% indicative of wider shifts in personal travel. During the 20$^{th}$
% century the trend has been for trips made by foot, rail % too woolly! rm
% or bus to be replaced by the relatively energy intensive car.
% % This represents an `energy transition',
% % a long-term shift in the way that energy is used in the human economy.
% % Since the oil price shock of the summer 2008 and subsequent global economic
% % crisis, certain statistics suggest that this particular transition may have
% % gone into reverse. Car kilometres peaked in 2008, and show no signs of
% % returning to their former levels. What this means overall remains an open question:
% % more evidence on energy supplies, patterns of behaviour and the
% % ultimate unknown --- policy responses --- will be needed before any firm
% % conclusions can be drawn. Making judgements in the relatively narrow area
% % of travel to work is a far less ambitious
% The direction of change into the 21$^{st}$ century is uncertain, however,
% especially since car use has declined since 2008. Improved understanding of commuting
% patterns could therefore benefit efforts to reduce greenhouse gas emissions,
% and hence energy use, in the transport sector overall.

There is much research into commuting and transport energy use as separate
fields, but they have rarely been combined in the same analysis, let alone
at high levels of geographical resolution.
The well-established field of spatial microsimulation
offers tools for investigating commuting patterns in detail at local and
individual
levels, with major potential benefits for transport planning.
% In this project
% the method is applied, for the first time, to investigate variability in energy
% use both between and within small administrative zones.
For the first time this method is deployed to study commuter energy
use between and within small administrative zones.
% and
% the scenarios investigated in this PhD suggest that a wide range of phenomena,
% from future oil shocks to the localisation of economic activity could lead
% the energy use of commuting to fall. C
% ommuting is also

% The study most closely related to this thesis published to date, \citep{Boussauw2009},
% found that the energy costs of commuting are highly variable over space and to
% some extent predictable.
The maps of commuter energy use presented in this thesis
illustrate this variability at national, regional and local levels.
Supporting previous research, the results suggest that a
range of geographical factors influence energy use for travel.
This
has important policy implications: when high transport energy use in commuting
is due to lack of jobs in
the vicinity, for example, modal shift (e.g.~from cars to bicycles)
on its own has a limited potential to reduce
energy costs. Such insights are quantified using existing aggregate data.
The main methodological contribution of this work, however, is to add
individual-level factors to
the analysis --- creating the potential for policy makers to also assess the
distributional
impacts of their interventions and target specific types of
commuters having high transport energy costs,
rather than treat areas as homogeneous blocks. This potential is
demonstrated with a case study of South
Yorkshire, where commuting energy use is cross-tabulated by
socio-economic variables and disaggregated over geographical space.
% The areas where commuting energy use
% is most unevenly distributed (in urban centres) will likely benefit most from
% policies that target
% the specific groups that have the greatest impact.
The areas where commuting energy use is less evenly distributed across the population, 
for example in urban centres, are
likely to benefit most from policies that target the specific groups.
Areas where commuter
energy use is more even, such as
Stocksbridge (in Northwest Sheffield), will benefit from more universal policies.

% The findings
% confirm the common-sense notion that those on high-incomes are more
% energy intensive in their travel to work behaviours and quantify by
% how much (the top income quintile uses around 3 times more energy
% for commuting than the bottom, although this varies from place to place).
% In terms of vulnerability, the research has identified deprived rural areas
% isolated from large employment centres as most at risk. Individual-level risk
% factors could include ease of relocating work more locally, opportunities
% for lift sharing or, in extreme cases, physical fitness and access to
% a suitable bicycle. None of these ideas are new, but the strength
% of evidence to support them is.

The thesis contributes to human knowledge new information about
the energy costs of commuting, its variability
at various levels and insight into the implications.
New methods of generating and analysing individual-level
data for the analysis of commuter energy use have also been developed.
% These methods are reproducible and should be of use to others % dad'd
% aiming to investigate the energy security, resource efficiency
% and long-term welfare benefits of interventions in personal travel systems.
These are reproducible (see the GitHub repository ``\href{https://github.com/Robinlovelace/thesis-reproducible}{thesis-reproducible}''
for example code and data) and will be of interest
to researchers and policy makers investigating
the energy security, resource efficiency and potential welfare impacts
of interventions in personal travel systems.
}

\clearpage  % Abstract ended, start a new page
%% ----------------------------------------------------------------

\setstretch{1.3}  % Reset the line-spacing to 1.3 for body text (if it has changed)

% The Acknowledgements page, for thanking everyone
\acknowledgements{
\addtocontents{toc}{\vspace{1em}}  % Add a gap in the Contents, for aesthetics

It should
be acknowledged at the outset that some parts of the thesis have been published:
\begin{itemize}
 \item Parts of \cref{setsim} have been published in \emph{Computers, Environment and Urban Systems} \citep{Lovelace2013-trs}.
 \item The tutorial ``Spatial microsimulation in R'', a supplement to \citet{Lovelace2013-trs},
 is based on \Cref{simplementing}.
 \item The results presented in \cref{Chapter7} have been
 published in the \emph{Journal of Transport Geography} \citep{Lovelace2013-jtrg}.
 \item Results presented in \cref{Chapter8} have been published in \emph{Geoforum} \citep{Lovelace2014-vul}.
\end{itemize}

Thanks to my supervisors Dimitris Ballas, Matt Watson and Stephen Beck for
unceasing encouragement and guidance throughout. Dimitris has
been instrumental in developing the methodological direction of the PhD project.
I will be forever grateful for the guidance provided in the research and beyond.

Many thanks to Carlota for keeping my spirits up throughout.
To Engineers Without Borders for allowing me to get my hands dirty,
a feature too often missing from modern research. To my house-mates
for providing a fun and homely habitat in Sheffield.
To my parents, who instigated trips into the Peak District ---
the ultimate antidote to square-eyes. To my dear friends in
Sheffield, especially James Folkes for providing pedal-powered
entertainments and Joseph Moore for `moore' distractions.

Thanks to the E-Futures Doctoral Training Centre.
E-Futures was vital to this PhD, not only for providing funding
that allowed its students financial security to dedicate themselves to study.
% money and economic violence of the open labour market
% for four years.
% investigate possible solutions to the unsustainable nature of
% our current energy system.
E-Futures also provided a forum for debate.
The encouragement from peers and across disciplines was inspirational.
Neil Lowrie deserves special mention here,
as he helped channel my energy away from confrontations
with coal-fired power station operators and towards research. Thanks.

Thanks to the Department of Geography, for providing an academic
home and a quiet desk. Members of the Social
and Spatial Inequalities group (SASI), especially, provided
feedback on my work, and encouraged the investigation of
how commuting affects people, not just energy. I thank Luke Temple and Mark
Green in particular in this regard.
% for welcoming me to the department and for keeping me on
% the straight and narrow.

% Thanks to the nameless and often un-thanked people
% who allowed the
% PhD course run roughly to plan. The cleaners, porters, IT staff,
% secretaries and finance people kept things ticking over, creating the
% conditions for the impartial pursuit of knowledge.
% It should be clear, though, that this thesis is not a
% purely disinterested investigation of the energy costs of work travel.
% It is proudly driven by concern for the future
% of a civilisation facing depletion of its defining resource
% --- fossil fuels.
Thanks to the open source software movement in general and to the developers
of R and~\LaTeX~(in which the document was written) in particular.
Hadley Wickham stands out in this
regard, whose own thesis \citep{Wickham2008}, led to the ggplot2 package
used for many of the visualisations.
Thanks to Github for hosting code and data that should make the methods
and results more accessible and reproducible for
others.\footnote{Sample code
and data used can be found on {\color{blue} \href{https://github.com/Robinlovelace/}
{github.com/Robinlovelace/}}. In particular, reproducible versions of
the results can be found in the {\color{blue} \href{https://github.com/Robinlovelace/thesis-reproducible}
{thesis-reproducible}} repository. }

The thesis has benefited from the feedback of people who read 
early drafts of various sections and chapters: Milan Delor,
Ian Philips, Jake Gower, Chris Hunter, Charlotte Bjork and my father 
David Lovelace. Dan Olner's input was especially beneficial in the final 
stages. Thanks to all for providing additional feedback and support outside
of the usual academic channels.

My penultimate thank you is for writers who awoke my interest in this topic: Ivan
Illich, John Michael Greer, Howard T.~Odum, George Monbiot and
Vaclav Smil.

The final thank you is to the examiners of the thesis, Charles Pattie and
Michael Batty. 
}
\clearpage  % End of the Acknowledgements
%% ----------------------------------------------------------------
%% ----------------------------------------------------------------
%%%%% To from here!
%% ---------------------------------------------------------------- 

\pagestyle{fancy}  %The page style headers have been "empty" all this time, now use the "fancy" headers as defined before to bring them back
%% ----------------------------------------------------------------
\setcounter{tocdepth}{1} % TOC depth
\setcounter{tocdepth}{2} % TOC depth
\phantomsection
\addcontentsline{toc}{chapter}{Table of Contents}
% \lhead{\emph{Contents}}  % Set the left side page header to "Contents"
\fancyhead[LO,RE]{\emph{contents}}
\fancyfoot{}
\tableofcontents  % Write out the Table of Contents

%% ----------------------------------------------------------------
%%%%% Uncomment from here!
%% ---------------------------------------------------------------- 

%% ----------------------------------------------------------------
% \lhead{\emph{List of Figures}}  % Set the left side page header to "List
\fancyhead[LO,RE]{\emph{List of Figures}}
\fancyfoot{}

\listoffigures  % Write out the List of Figures

%% ----------------------------------------------------------------
% \lhead{\emph{}}  % Set the left side page header to
\fancyhead[LO,RE]{\emph{List of Tables}}
\fancyfoot{}
\listoftables  % Write out the List of Tables

%% ----------------------------------------------------------------
\setstretch{1.5}  % Set the line spacing to 1.5, this makes the following tables easier to read
\clearpage  % Start a new page
% \lhead{\emph{Abbreviations}}  % Set the left side page header to
\fancyhead[LO,RE]{\emph{Abbreviations}}
\fancyfoot{}

\listofsymbols{ll}  % Include a list of Abbreviations (a table of two columns)
{
{Acronym} & What it Stands For \\
%\textbf{LAH} & \textbf{L}ist \textbf{A}bbreviations \textbf{H}ere \\
kJ & kilojoules (10$^3$ J) \\
MJ & megajoules (10$^6$ J) \\
GJ & gigajoules (10$^9$ J) \\
TJ & terajoules (10$^{12}$ J) \\
PJ & petajoules (10$^{15}$ J) \\
EJ & exajoules (10$^{18}$ J) \\
\\
kWh & kilowatt hour (3.6 MJ) \\
\\
CO$_2$ & Carbon dioxide \\
EROI & Energy return on (energy) investment \\
IPF & Iterative proportional fitting \\
pkm & passenger-kilometres\\
vkm & vehicle-kilometres\\
TRS & Truncate replicate sample (integerisation method) \\
% PLF & passenger load factor (average occupancy of vehicle / n. seats)
\\
DECC & Department of Energy and Climate Change\\
Defra & Department for Environment Food \& Rural Affairs\\
NTS & National Travel Survey\\
ONS & Office for National Statistics \\
OSM & Open Street Map \\
USd & Understanding Society dataset\\
\\
% GOR & Government Office Region \\
% MSOA & Medium super output area \\
% OA & Output area \\
% TTWA & Travel to work areas \\
}

%% ----------------------------------------------------------------
\clearpage  %Start a new page
\label{nomen}
% \lhead{\emph{Symbols}}  % Set the left side page header to "Symbols"
\fancyhead[LO,RE]{\emph{Symbols}} 
\fancyfoot{}
\listofnomenclature{lll}  % Include a list of Symbols (a three column table)
{ 
 symbol & name & unit \\
$dE$ & Euclidean distance & km \\
$dR$ & route distance & km \\
$Etrp$ & direct primary energy use per trip & J \\
$ET$ & total energy use of all commuter trips in a given area & GJ \\
$ETyr$ & total primary energy per year & GJ/yr \\
$Esys$ & total primary energy use (direct and indirect) & MJ/km \\
$Ef$ & Direct fuel (including electricity and food) energy use per kilometre &
MJ/vkm \\
$Efp$ & Energy costs of fuel production & MJ/vkm \\
$Ev$ & Energy costs of vehicle production per unit distance & MJ/vkm \\
$Eg$ & Energy costs of guideway construction per unit distance & MJ/vkm \\

$EMv$ & embodied energy of vehicle production & GJ/vehicle \\
$EMg$ & embodied energy of guideway production & GJ/km \\

$EI$ & energy intensity of transport per passenger kilometre & MJ/pkm\\
$FE$ & fuel economy of vehicle & L/100 vkm \\
$Lf$ & load factor of vehicle or mode &  \\
$Lg$ & lifespan of guideway & vehicle passes \\%(unles
$Lv$ & lifespan of vehicle & vkm \\%(unles

$m$ & mode of transport (e.g. car, train) & \\%(unles
$Oc$ & occupancy, the number of people in each vehicle & people/vehicle\\
$P$ & power & W (Js$^{-1}$) \\
$Q$ & circuity: route distance divided by Euclidean distance &\\
% Gap to separate the Roman symbols from the Greek
$\eta$ & energy conversion efficiency ($\frac{Energy\ in}{Energy\ out}$) & \\

$Toe$ & tonnes of oil equivalent &
}

%% ----------------------------------------------------------------
% End of the pre-able, contents and lists of things
% Begin the Dedication page
%% ----------------------------------------------------------------
%%%%% To here!
%% ---------------------------------------------------------------- 

\setstretch{1.3}  % Return the line spacing back to 1.3

\pagestyle{empty}  % Page style needs to be empty for this page
%\dedicatory{Dedicated to my mum and dad}

%\addtocontents{toc}{\vspace{2em}}  % Add a gap in the Contents, for aesthetics


%% ----------------------------------------------------------------
\mainmatter	  % Begin normal, numeric (1,2,3...) page numbering
\pagestyle{fancy}  % Return the page headers back to the "fancy" style

% Include the chapters of the thesis, as separate files
% Just uncomment the lines as you write the chapters

% Chapter 1

\chapter{Introduction} % Write in your own chapter title
\label{Chapter1}
% \lhead{Chapter 1. \emph{Introduction}} % Write in your own chapter title to
\fancyhead[RO,LE]{Chapter 1. Introduction} % for double sided printing
\fancyhead[RE,LO]{\thepage}
% Task: port stuff to other chaps; make this ultra-short.
The research presented in this thesis focuses on commuting and its energy costs.
UK datasets from the beginning of the 21$^{st}$ century form the empirical
foundation of the work. Travel to work statistics are described, analysed and in later
chapters modelled to assess the variability of energy use for this commuting.
The underlying motivations are broader and play an important role
throughout the thesis, from the choice of methodology (\cref{Chapter4}) to the
specification for scenarios of change (\cref{Chapter8}).
% The premise of the research, that energy use is an important measure of
 % travel systems, is based on evidence from a range of disciplines.
% including
% climate science, `peak oil' theory and social sciences.
% The thesis is framed in terms of these big issues
% The thesis is policy-driven.
It is therefore important to lay out these wider issues at the outset, before
highlighting the impact of commuting at the individual and national scale (in
sections \ref{s:realities} and \ref{snimportance}). These `big picture'
motivations also inform the research aims and objectives (\cref{s:aims}).

\section{The `Big Picture'}
% In the grand scheme of things the topic of this work ---
%  commuting and its energy impacts --- may seem mundane.
% The issues that it relates to are big, however.
Our increasingly interconnected global civilisation is facing challenges
that are unique in the history of humankind. Environmental
and social-economic changes are occurring to a greater extent and faster
than ever before \citep{Rifkin2011a, ehrlich2013can}. Perhaps more
importantly, this generation is in the privileged position of being able to monitor,
predict and respond to these changes as they occur \citep{Evans1998,
Smil2008, IPCC2007}.
This work is firmly situated in the context of these changes and aims to
contribute to humanity's understanding of them.
Following the academic tendency for specialisation
 whilst avoiding the pitfalls of dogmatic allegiance to any particular
discipline or worldview \citep{kates1986geography},
this thesis focuses on one `bite-sized' yet important part of these wider
issues.

Energy intensive transport
contributes to pressing environmental, social and economic problems of the
21$^{st}$ century. Climate change, resource depletion, and growing levels of
economic inequality are global problems aggravated by energy use.
Travel is a major energy consumer. Yet transport
systems powered by fossil fuels have become integral to modern life:
by the 1970s `automobility' was central to social change \citep{Illich1974}
and since then motorised transport has become even more central to
modern life \citep{Rodrigue2009}.
This means that policy-makers, businesses, and individuals
 will have to make difficult decisions in the coming decades.
According to some the situation is urgent: ``Rapid decisions now need to
be made
so that the impacts of transport on the environment can be minimised and fossil
fuel resources conserved'' \citep[p.~354]{Chapman2007}. \emph{Rapid} decisions are not
always \emph{good} decisions, however: rational choices depend on good
information about the world. 
 %Expand and quote from Illich hh 

Because of the scale and complexity of the previously mentioned global
problems, it is tempting to focus solely on the detail of energy use in commuting
 as one aspect of personal travel about which good datasets are available.
% it is
%tempting to focus on the detail --- in this case
%the abstract and reductionist energy costs of one source of personal travel:
%transport to work. 
It is however important to understand the wider context of 
transport energy use in order to decide the most useful applications of and
directions for future research in this area.
%However, an understanding of the context in which interest
%in transport energy use has grown is vital to deciding which research
%directions to pursue. Understanding of this wider context also informs
%how the methods and understanding contained in this
%narrow field of knowledge can best be mobilised.
%  to
% inform people who must make decisions now that will based on an uncertain future.
An introduction to the broader context that motivates this research is therefore
provided, focussing on the three `big issues' of climate change, peak oil,
and economic inequality which are also
long-term political priorities in the UK \citep{UKERC2010}.

\subsection{Climate change}
The Earth's climate has always changed: it is a complex system with non-linear
responses to internal and external drivers and a number of feedback loops
\citep{IPCC2007}. The changes during the 20$^{th}$ and 21$^{st}$ centuries are,
however, different from those observed in the paleoclimate record: ``It is
important to realize that the
current change in atmospheric CO$_{2}$ is proceeding at a rate more than 200
times faster than any natural change in Earth's past history with the exception
of the Cretaceous-Tertiary boundary event generally attributed to impact of an
asteroid with the Earth'' \citep{Hay2011}. The other major difference is that
today climate change is caused by the combustion of fossil fuels by humans.
Commuting, composed of millions of motorised trips to work and back each day, is
a small yet important contributor. The desire to reduce these emissions, for the
maintenance of a ``safe operating space for humanity'' \citep{Rockstrom2009}
provides an important motivation for this research. An underlying aim is to
contribute ideas and information to the ongoing debate about how to mitigate
anthropogenic climate change \citep{Matschoss2011}.


This aim appears to be shared by others:
academic interest in transport emissions has proliferated in recent years
\citep{Akerman2006, Chapman2007, Schwanen2011}, although less so in the
specific area of commuting (\cref{Chapter2}).
Because energy use is directly related to greenhouse gas
emissions \citep{MacKay2009}, this research is also about climate change.
%%% The following is superfluous bollocks!
% Before presenting the other major policy driver of this research, energy
% security, it is important to understand how emissions relate to the UK's
% climate change policy. This is the subject of the next subsection. 

\subsubsection*{UK greenhouse gas emissions}
At the UK level, the emissions associated with commuter energy are subsumed
within `transport emissions'. These include emissions from shipping, aviation
and military transport, as well as the road and rail sectors
\citep{Decc2011t}. Road transport dominates,
accounting for more than 90\% of the UK's transport emissions
(\cref{fig:cc-trans}).

\begin{figure}[h]
 \centering
 \includegraphics[width=14cm]{cc-trans.png}
 % cc-trans.png: 1113x529 pixel, 72dpi, 39.26x18.66 cm, bb=0 0 1113 529
 \caption{UK transport emissions by source in 2009 \citep{Decc2011t}.}
 \label{fig:cc-trans}
\end{figure}

An interesting feature of the UK's emissions reporting strategy is that
`transport' is generally presented as a monolithic category (e.g.
\citealp{Decc2010}), despite the wide variety of transport modes and purposes
presented in \cref{fig:cc-trans}. This makes it difficult to identify the
specific drivers of growth in UK transport emissions since 1970
\citep{Gasparatos2009} and stagnation since 1990 (\cref{fig:cc-ems}).
What is clear in
both cases is that energy use and hence emissions from transport have increased
(since 1970) or stagnated (since 1990) while those of other sectors have
declined. Between 1990 and 2010, transport was the only sector other than
housing in which emissions increased; transport now accounts for just over 20\%
of UK emissions (\cref{table:mtco2}, below). This research project quantifies
the contribution of commuting to this total in terms of energy use, and provides evidence
about which strategies may be effective for reducing the emissions due to
transport to work.

The UK's climate change commitments are unambiguous, agreed upon by all major
parties, and legally binding: emissions in 2050 must be below 20\% of their 1990
level \citep{ClimateChangeAct2008}. This means that the \emph{total}
permitted emissions in 2050 across all sectors are roughly equal to the emissions from
just the \emph{transport sector} today. This fact underlines the scale of the
proposed changes: transport to work represents a small but important component
of this challenge that affects millions of working people every day.

\begin{figure}[h]
 \centerline{
 \includegraphics[width=17 cm,bb=0 0 792 612]{cc-ems.pdf}}
 % cc-ems.pdf: 792x612 pixel, 72dpi, 27.94x21.59 cm, bb=0 0 792 612
\caption{UK greenhouse gas emissions by sector. Data from
 \citet{Decc2011ff}.}
 \label{fig:cc-ems}
\end{figure}

\begin{table}[htbp]
\caption[Top 5 UK sectors in terms of greenhouse gas emissions, 1990-2010]{Top 5
UK sectors in terms of greenhouse gas emissions, 1990-2010
(MtCO$_{2}$e). Data from \citet{Decc2011ff}}
\centering{
\begin{tabular}{lrrrrr}

 & 1990 & 2000 & 2010 & \multicolumn{1}{l}{\% change} & \multicolumn{1}{l}{\%
emissions (2010)} \\ \hline
Energy Supply & 273.4 & 220.1 & 204.3 & -25.3 & 34.8 \\ \hline
Transport & 121.5 & 126.7 & 121.9 & 0.3 & 20.7 \\ \hline
Residential & 80.8 & 90.1 & 89.9 & 11.3 & 15.3 \\ \hline
Business & 113.2 & 111.3 & 89 & -21.4 & 15.1 \\ \hline
Agriculture & 63.1 & 58 & 50.7 & -19.7 & 8.6 \\ \hline
Other & 117.4 & 65.8 & 32 & -72.7 & 5.4 \\ \hline
Total & 769.4 & 672 & 587.8 & -23.6 & 100.0 \\ \hline
\end{tabular}
}
\label{table:mtco2}
\end{table}

\subsubsection*{Emissions from transport to work}
% Something about transport being a challenging sector for emissions cuts
% Something about freight vs personal transport energy use and emissions
Of the 20\% of UK emissions that arise from transport, only a small fraction
are due to transport to work. How small? No official breakdowns of emissions are
provided by reason for trips, but estimates can be made by
analysing the make-up of the transport sector. As shown in 
\cref{fig:cc-trans}, 5\% of transport emissions can be accounted for by military
vehicles, aviation and shipping: none of these are usually involved in transport to
work. Also, 31\% of road transport emissions arise from goods vehicles (HGVs and
LGVs); the remaining 69\% arise from road vehicles for personal transport --
buses, motorcycles and cars \citep{DECC2011a}. From these figures, it is
possible to estimate that ~80 MtCO$_{2}$e result from personal travel in the UK.
19.5\% of passenger kilometres travelled by all personal transport modes in the
UK are due to travel to work \citep{DfT2011-why}. Transport to work
can be estimated to cause $\sim$16 MtCO$_{2}$e of emissions or around 3\% of the
UK's total. (In \cref{stotalcomp} a more refined estimate of commuter energy
use is presented, based
on geographically disaggregated data: commuting was found to account for
4.1\% of total energy use and 14.4\% of transport energy use.)

It is important to undertake such `back of the
envelope' calculations at the outset of research into emissions reduction
strategies or sustainable energy to ensure that time is not wasted on negligible
issues such as phone chargers \citep{MacKay2009}. David MacKay, Chief Scientific
Advisor at the Department of Energy and Climate Change (DECC), puts this
argument in lay terms by proposing a rule for
energy-saving interventions: ``A gizmo may be discussed only if it could lead to
energy savings of at least 1\% ... because the public conversation about energy
surely deserves to be
focussed on bigger fish'' \citep{MacKay2009-energyplates}. Applying this
reasoning more broadly to areas of energy use, transport to work clearly
deserves attention according to
this rule, although emissions cuts in commuting will have to be matched in all
other sectors for targets to be met. However, there are reasons to believe that
making cuts in the transport sector generally, and in transport to work in
particular, will be especially difficult, and therefore worthy of dedicated
investigation. These include:
\begin{itemize}
\item The transport sector is overwhelmingly dependent on petrol and diesel:
motorised transport (which accounts for most trips and the vast majority of the
distance travelled, as shown in \cref{Chapter5}) is 95\% %really!!!
dependent on refined oil products \citep{Woodcock2007}. This is problematic
because there are no commercially viable, low emissions alternatives to crude
oil for liquid fuels. Biofuels are the only `renewable' option on the table,
but their potential contribution is low \citep{Patzek2006,Michel2012}, they
can conflict with
food production \citep{Pimentel2009}, and currently used crops may increase
greenhouse gas emissions due to land use change \citep{Fargione2008}.
\item Linked with the previous point, low carbon technology is far less
promising in the transport sector than in other large emitting sectors.
% In electricity generation and residential heat demand, for example the %dad'd
For electricity generation and residential heating the technologies
for renewable alternatives are becoming more commercially viable \citep{Chu2012}.
By contrast,
the penetration of electric, hydrogen, and biofuel-powered cars may be slow,
largely due to their high cost \citep{Proost2011,AdamVaughan2011}.
\item The current transport system is built around road (and to a lesser extent
rail) infrastructure that took many decades and large capital investments to
complete. The dependence of society on the car is deeply embedded, yet a
low-energy (and hence low emissions) transport system may require a shift away
from personal ownership of automobiles altogether
\citep{MacKay2009,Moriarty2010}, something that will take decades to accomplish.
\end{itemize}
These difficulties make de-carbonising transport systems problematic
compared with the other large energy users --- electricity and heat
production.\footnote{These can convert more easily to renewable
sources --- e.g.~via stationary wind turbines
and solar hot water panels --- than can transport systems.
This is because transport systems are inherently mobile,
therefore requiring high energy density energy storage.
Fossil fuels are unrivalled in terms of their energy density ---
almost 100 times greater than the best non-agrofuel commercial alternative:
lithium ion batteries. %!!!reference
Hydrogen fuel cells have been proposed as a solution, but these are
still far from commercial viability, and have been 
precluded by DECC's Chief Scientific Advisor on the
grounds that they are highly inefficient \citep{MacKay2009}.}
Despite these issues, transport is rarely framed in terms of energy use and
greenhouse gas emissions (\cref{Chapter2}). In addition to
its impacts on climate change via direct and indirect greenhouse gas emissions,
commuting is also vulnerable to the effects of climate change,
as discussed in \cref{s:uncertainties}.

\subsubsection{Climate change and energy}
Most studies looking at the impact of one aspect of the economy on climate
change do so through the emissions that it produces.
These studies generally measure environmental impact in terms of kilograms of
carbon or  CO$_2$ equivalent caused by different modes of travel.
This seems logical if one is concerned about climate change:
it is the greenhouse gases that trap the heat \citep{Houghton1990}.
However, others have suggested
that the best way to tackle the problem is from an energy perspective:
``climate change is an energy problem'', as a group of 18 prominent US
academics put it \citep[p.~981]{Hoffert2002}. What is meant by this is that
energy use and greenhouse gas emissions are currently two sides of the same
coin. More than 80\% of commercial total primary energy supply (TPES)
worldwide is provided by fossil fuels \citep{Smil2008} and in the
transport sector this is even higher.
It is true that not all forms of energy have the same emissions. Yet,
as illustrated in \cref{fgco2}, CO$_2$ emissions per unit energy are
in fact surprisingly similar across a wide range of transport fuels.
In addition, even if it were possible to decarbonise electricity
production in the near-term, the fact remains that uptake of low-energy sources
will almost certainly be gradual \citep{smil2010energy}. Another issue is
that technologies that have low emissions per unit of energy use during the
usage phase of their lifecycle often have an energy intensive production
phase. Because much modern food production depends upon fossil fuel energy,
the energy approach can also help in the
assessment of wide-boundary energy impacts.
Some environmental impacts of transport such as noise, road-kill and
the need to frequently resurface roads pummelled by powerful vehicles are not
included in most emissions estimates. Energy use can to some degree
encapsulate these additional impacts.

 \begin{figure}[htbp]
  \centerline{
    \includegraphics[width = 12 cm]{gco2}}
  \caption[The greenhouse gas emissions per unit energy of various fuels]
{The greenhouse gas emissions per unit energy of various fuels. Data
taken from \citet{Defra2011} (additional sources for
\href{http://www.ipcc.ch/pdf/special-reports/sroc/Tables/t0305.pdf}{\color{blue}electricity}
and
\href{http://www.biomassenergycentre.org.uk/portal/page?_pageid=75,163182&_dad=portal&_schema=PORTAL}{\color{blue}biofuel}
emissions were used)
and converted into SI units. The dominant
transport fuels are black for emphasis.  }
  \label{fgco2}
\end{figure}

The reasons for advocating a focus on energy use,
and not emissions directly, can be summarised as follows:
\begin{itemize}
 \item Emissions can be variable depending on the energy/fuel source, whereas
 energy is constant across fuel sources.
 \item If energy use is reduced overall, carbon-intensive forms can be phased out.
However, if emissions from one sector fall, they may well rise in another as
fossil energy resources are freed-up.\footnote{For
example, imagine if transport emissions rapidly dropped to zero
due to electrification and rapid uptake of renewables. The additional
load on the grid caused by this new user \citep{dyke2010impact} could
lead to an increase in the emissions stemming from space heating because
the total supply of renewable energy is fundamentally
limited by the laws of physics \citep{MacKay2009}. \citet{Berners-Lee2013}
describe this problem with emission reduction plans overall as squeezing
a balloon: savings in one area tend to bulge out in another.}
 \item Energy is the `master resource' from which all others (including more
energy) can be obtained; emissions are the end result result of energy use.
 \item It can be argued that energy use is at the root of the linked `big picture'
problems mentioned in this chapter, not just climate change. Therefore
tackling the energy problem could have numerous co-benefits.
\end{itemize}
All this suggests that the climate debate should be much more closely
linked to the energy debate. Specifically, the carbon content of proven
fuel reserves should be compared with the carbon dioxide content that
can safely be burned. Doing this analysis, based on recently released
data on fossil fuel assets, has led to an alarming finding: ``for all the
talk about finite resources and peak oil, scarcity is resoundingly not the
problem. From the climate's perspective, there is far too much fossil fuel''
\citep[p.~29]{Berners-Lee2013}.
% In fact, to keep the chances of a
% global temperature increase below two degrees centigrade
% (the point beyond which many studies show would be dangerous),
% above 75\%, \citet{Berners-Lee2013} showed that humanity can burn only
% around a half of economically viable reserves.
\citet{Berners-Lee2013} show that for there to be at least a 75\% chance
that the global temperature increase remains below two degrees humanity can
burn only around a half of economically viable reserves.
In terms of personal transport, this means
phasing out petrol and diesel and avoiding carbon-intensive electricity sources:
a fundamental shift.

Most greenhouse gas emissions stem from fossil fuel use, and once
extracted, these fuels are invariably burned. This has led to the conclusion
amongst some that the solution must be top-down:
fossil fuel companies must be forced to leave most of their assets untapped.
This can be achieved 
either through plummeting prices of fossil fuels or through regulation.
The former case is currently highly unlikely due to the surge of fuel demand from
emerging economies, combined with the sheer utility of fossil
fuels.\footnote{However,
if governments, in coordination,
prioritise minimising energy use while maximising
uptake of renewable energy, the former possibility would become more feasible.
}
The latter also seems unlikely, following the failure of UN talks in
Copenhagen to arrive at a consensus on legally binding
and enforceable emission targets for the major emitter.
This research is relevant in any case: if fuel prices remain high there
is a strong economic incentive to reduce energy imports. If leaders worldwide
agree to tackle climate change through top-down or bottom-up
policies, there will clearly be a strong interest in how best to
reduce reliance on fossil fuels in every sector that is vital for
well-being. Regardless of the level of regulation (whether it
occurs at the point of extraction or use of fuel), it implies high consumer prices
for fuels, through policies such as taxes, a
`carbon cap' or even energy rationing.\footnote{Interestingly, high prices of fossil
fuels is also the end result of many scenarios of resource depletion, which
has historically been another major driver of research into energy and
transport \citep{Fels1973}.}
% At this stage, it is worth noting that some people,
% including a number of politicians, believe that
% anthropogenic climate change should not be a political priority. Climate
% contrarians, some funded by the fossil fuel industry itself,
% have managed to confuse large sections of the public.
Another pragmatic benefit of the energy approach is that even if one questions
the need to tackle climate change, the arguments to reduce dependence on
finite fossil fuels for other reasons are very strong.

\subsection{Peak oil and resource depletion}
In addition to the impacts of climate change,
depletion of our fossil energy resources is another non-negotiable
reason for transition away from fossil fuels, to a ``post-carbon'' economy
\citep{Heinberg2005, Heinberg2009, Heinberg2010, Kunstler2006}.
Oil is the most rapidly depleting resource yet motorised
transport is almost entirely dependent on liquid fossil fuels
derived from it \citep{Gilbert2008}.
Multinational personal transport industries tend to downplay or deny the risks
of peak oil, pointing to non-conventional oil resources and technological
advance as reasons not to worry. Prototype biofuels, electric cars and hydrogen
fuel cells are often cited as ways of overcoming high prices. This is ironic
because each technology is highly dependent on oil for resource extraction,
manufacture, distribution and waste disposal stages of their life-cycle: high
oil prices could make the batteries for electric cars, to take one example, even more
expensive, far out of the reach of the median global citizen.
%!!! income?
Each technology is still in the research phase of development,
relies on scarce public subsidies to be commercially viable and cannot operate
on the scale needed within modern transport infrastructures even if production
lines producing them were scaled up before a major oil shock. Biofuels, to take
the most heavily subsidised example, can only ever produce a small fraction of
current transport energy demand even if all available resources were exploited
to the maximum (\cref{f:biofools}).

 \begin{figure}[htbp]
  \centerline{
    \includegraphics[width = 12 cm]{biofuels-cont}}
  \caption[Biofuels' contribution to global transportation energy use]
{Biofuels' current (2010) and potential contribution to
global transportation energy use \citep[p.~228]{Aleklett2012-peeking}.
Image used with permission of author. Data originally presented in
\citet{Johansson2010-ag-fuel+food}.}
  \label{f:biofools}
\end{figure}


For this reason peak oil is a major motivation for research into energy and
transport. How will transport systems operate beyond 2050,
when oil production will be a fraction of its current level? 
\citep{Aftabuzzaman2011}.
How will people get to work in the event of shortages?
\citep{Noland2006}. These are just a couple of examples of
the kinds of questions that are being asked in preparation for
declining oil supply. A parallel question
(explored in \cref{svul}) is: how will commuters be affected by
oil price shocks, depending on where they live and their socio-demographic
characteristics? The potential problems posed
by peak oil for motorised transport systems are severe and include
collapse of complex economic activity due to the
highly inter-dependent nature of the global economy \citep{Friedrichs2010,
Korowicz2011}.
For this reason an introduction to peak oil, and how it relates
to commuting, will help to place this research in the wider
context. \citet{Gilbert2008} provide a comprehensive reference
on the subject, from a North American perspective.

Peak oil is the point at which global oil production
enters terminal decline due to depletion of large oil fields
\citep{Greer2008}. It is an inevitable event during the 21$^{st}$
century, as oil is a finite resource, approximately half of which
has been used \citep{Aleklett2010}. However, there remains controversy
about the exact timing of the peak \citep{Smil2008}.
An in-depth review by the UK's Energy Research Centre \citep{UKERC2009}
found that the weight of evidence suggests a peak in the near-term,
before 2030. This is
well before the 20 years that the famous Hirsh Report \citep{Hirsch2005}
indicated would be needed to prepare for declining supplies of liquid fuel.
The implications are stark: if peak oil does occur before 2030, as
the evidence reviewed by \citet{UKERC2009} suggests, urgent preparations
must begin now.

As economists have long indicated \citep{Solow1974},
it is not only the amount of oil left in the ground
that directly affects peoples' lives. It is the \emph{price} of oil that
affects transport systems, with knock-on impacts on human lives.
Price is also affected by changes in demand and technologies for
extraction and substitution \citep{Perman2003}.
Over the past decade there has been increasing evidence that depletion
plays a major role in determining global oil prices, however,
with high and volatile prices likely in the future \citep{Aleklett2012-peeking}.
The price of crude oil during the past 20 years has shown both volatility
and (when a smoothed by a rolling average function) a near inexorable
upward trend \cref{fig:oilprice}. 


 \begin{figure}[htbp]
  \centerline{
    \includegraphics[width = 12 cm]{oilprices}}
    \rule{35em}{0.5pt}
  \caption[Average prices of Brent Crude oil spot prices, 1992 -- 2012]
  {Average prices of Brent Crude oil spot prices per week,
January 1992 until October 2012 (dots) and a 2 year rolling average
(blue line) Data from the U.S.~Energy
Administration (\url{http://www.eia.gov/dnav/pet/pet_pri_spt_s1_d.htm})
plotted using the R package ggplot2.}
  \label{fig:oilprice}
\end{figure}

Despite these upward trends, UK government energy policies
are still largely based on the assumption that oil prices will
remain below \$100 per barrel into the 2020s \citep{UKERC2010}.
Thus methods that estimate the oil-reliance of households based
on readily available commuter statistics could be highly relevant
to politicians and planners making long-term decisions. The ability to
quantitatively explore
the impact of high oil prices and other scenarios of change
at the individual level is an output of this
research that could have applications in transport policy evaluation and 
development. See \cref{Chapter7}.

\subsection{Inequality and well-being}
Peak oil and climate change are important because we
depend on the resources and processes of the natural environment to survive.
Humans also depend on the relationships between each other, not simply for
survival, but for quality of life. ``It is only in the backward countries of
the world'', wrote John Stuart Mill, ``that increased production is an
important object; in those most advanced, what is needed is a better
distribution'' (Mill 1857, in \citealt{Perman2003}: p. 6).

With more than 150 years of hindsight, Mill's statement seems all but
Utopian: economic growth is still the number one priority of most governments
worldwide, even in wealthy countries such as the UK where evidence
suggests that further growth may do more harm than good, for people and
the environment \citep{Latouche2008}.
To such an extent does economic growth dominate modern decision making,
regardless of consideration of how growth is distributed,
that authors such as Charles Eisenstein and John Michael Greer
refer to it as the founding story of our age \citep{Eisenstein2011, Greer2009}.
In contrast to this dogmatic growth focus, evidence suggests that
other things, including equality of economic and social opportunities, lead
to quality of life \citep{Jackson2008, Jackson2009}.

The growth-at-all-costs mentality, combined with our debt-based
capitalist economy\footnote{As explained by \citet{Eisenstein2011},
the very existence of positive interest rates ensures that those who
have money tend to have more. According to this view, growing
levels of economic inequality is built into the monetary system,
and can only revert back to low levels with crises such as
wars or depressions, planned debt annulments or
(preferably for Eisenstein) negative interest rates.}
has caused inequalities to grow worldwide
\citep{OECD2011}. The UK has one of the highest levels of
inequality in Europe (\cref{fig:ineqs}).

\begin{figure}[htbp]
  \centerline{
    \includegraphics[width = 15 cm]{oecd}}
    \rule{35em}{0.5pt}
  \caption{UK Gini index for market and disposable income
  in context \citep{OECD2011}.}
  \label{fig:ineqs}
\end{figure}

This problem is important in the context of the energy costs
of commuting because employment opportunities are greatly
affected by one's ability to find and affordably travel to work.
Variable transport opportunities amplify social and economic
inequalities: 38\% of jobseekers say transport problems prevent
them from getting a job \citep{SocialExclusionUnit2002}⁠.  ``No jobs nearby'' and
``lack of personal transport'' were the first and second most
frequently cited barriers to getting or keeping a job in a survey
of young people in the UK \citep{Bryson2000a}⁠.

% \subsubsection*{Well-being}
% \label{s:well}
% In addition to the readily quantifiable impacts of commuting on climate change,
% rates of non-renewable resource depletion and (to a lesser extent) economic
% opportunities and inequality, the daily trip to work also has a direct impact
% on well-being. This body of literature can be divided into three main areas:
% commuting and physical health, psychological impacts and how the impacts
% of commuting vary between people, for example based on gender and class.
% Key papers in each of these areas are highlighted below.
% 
% Commuting involves a substantial amount of time, effort and expenditure. It
% is (for the xx\% of employed people in Britain who do not work at home)
% an essential component of daily economic life which.
% It is worth, at this stage, reflecting briefly on employment because if
% employment is an optional extra, the well-being impacts of commuting could
% simply be avoided by not working. However work appears to be integral
% to well-being for most adults.
Paid employment, and the
economic independence it brings, is a foundation for life satisfaction
\citep{Jahoda1982}. Work is ``a principal source of identity for most adults''
\citep{Tausig1999} and can promote good health (if the work is satisfying)
\citep{Graetz1993}. By corollary unemployment, the proportion of working-aged
people without a proper job, ``is a crucial indicator of the welfare and
economic performance of different areas'' \citep[141]{Coombes1982}. Yet without
accessible means of travelling to and from work each day, these benefits are
impossible to reach.

Given the importance of work, and the high proportion of work that is
undertaken outside the home, it should come as no surprise that
people will commute even if it an arduous task damaging to their health.
Taking a broad definition of health, these impacts range from those
narrowly associated with breathing urban air to more subjective consequences for
mental health including stress. From a human ecology perspective
commuting can be understood as a stressful relocation from one's
`domestic habitat' to a more hostile, hierarchical workplace. %%%%% ref!!!
The trip to get there will often coincide with thousands of other
commuters, all using the same road, railway or path. With these factors
in mind, the finding that, ``For most people,
commuting is a mental and physical burden'' should come as little surprise
\citep{Stutzer2007}.\footnote{The question
``how much of a burden'' is open to debate, however.
The finding of \citet{Stutzer2008}, that subjective well-being
declines proportionally with time, was not replicated in a
recent analysis of data from the BHPS \citep{Mumford2012}.}
The entrenched issue of inequality is tackled from
the perspective of commuting by measuring it in energy
(as opposed to purely monetary) terms (\cref{sindvar}) and providing
methods for assessing the distributional impacts of future
what-if scenarios (\cref{Chapter7} and \cref{Chapter8}).

% \subsection{Why energy?}
% The 

\section{Commuter energy use: everyday realities}
\label{s:realities}
The large scale processes of change mentioned above tend to be thought of in the
abstract, using inevitably simplified versions of reality. They are
often best represented through statistics, inherently simplified and
aggregated for visualisation. Seeing the issues quantitatively and
at `arms length' may be necessary to
gain an objective understanding of their evolution. Yet this may also lead to lack
of understanding of their local level manifestations and poor retention in memory:
although physical
reality may be best understood through numbers, human brains seem better able to
retain information that has emotional or personal content
\citep{Laird1982, Green2012}.
% \footnote{This enhanced recall for emotional information
% appears to be linked to emotional sensitivity in general
% which is generally higher for women:
% ``An individual’s level of emotional sensitivity
% was a stronger predictor of their emotional recall
% than their gender, suggesting that memory for
% emotional information is not determined by
% gender alone, but instead reflects a person’s
% sensitivity to emotional information in their
% environment. Thus, gender differences in memory
% for emotional information observed in the present
% study most likely reflect that women are, on
% average, more sensitive than men to the emotional
% aspects of their environment''
% \citep[p.~204]{Bloise2007}.
% }
When explaining my research to others, the following question
has been found to
effectively transform a purely academic and boring issue into something
interesting and relevant:
``What would a doubling of global oil prices mean for your family?''
For this reason, and to introduce some themes that
are used throughout this thesis in `layman's terms', this section is
based on a brief personal story: that of Chris Fisher.

Chris was born and bred in Weobley, a small town nestled between Hereford,
Leominster and Kington (\cref{fig:hereford}). Since finishing
at Weobley secondary school he has worked in a
wide range of jobs in the local area, including for Weobley's largest employer
(and sponsor of the village football team) Primasil and a local restaurant
called Joules. His current job, held for over 3 years now, is
to provide manual labour in Tyrrell's crisp factory.

\begin{figure}[htbp]
  \centerline{
    \includegraphics[width = 16 cm]{hnew}}
    \rule{35em}{0.5pt}
  \caption[Commuting options to Tyrrell's crisp factory]{Commuting options to
Tyrrell's crisp factory for Chris Fisher if he lives in Weobley (7 km one way)
or Hereford (13 km one way), as illustrated by the thick red lines.}
  \label{fig:hereford}
\end{figure}

Commuting and the economic cost it exacts has a large impact on Chris's life.
Ideally he would like to move to Hereford as that is where more of his
friends live and because there is more going on in the city than in Weobley.
However, Chris feels
bound to continue living with his mum in Weobley due to the costs of commuting.
The numbers work out like this: it's an 8 to 9 mile round trip to work from
Weobley, whereas the distance would approximately double if he lived in Hereford.
The location of his job also essentially forces car ownership: there are no buses
between Weobley and the Tyrrell's crisp factory, car sharing options are limited
and relying on a bicycle does not seem feasible for winter shifts that end at 6 am.
In addition to location, other downsides include long hours (12 hour shifts for
everyone, 4 days on, 4 days off), poor pay (\pounds8 per hour) and unpleasant
working conditions (the factory contains no windows, meaning that during
some day shifts you do not see the sun for 4 days in a row). For these reasons
Chris was tempted to quit when Tyrrell's decided to move towards 24 hour
production following increased demand from the USA: previous to this change
8 hour shifts were the norm; afterwards 12 hour shifts were implemented, broken up by
three 20 minute breaks.
% \footnote{This industrial regime is ironic given that
% part of Tyrrell's appeal is their `local' feel: on the company's website this is
% hammered home in the following blurb:
% 
% ``We're proud Herefordshirians [sic]
% (is that a word?), so we only use potatoes
% from local farmers, our favourites being Lady Rosetta and Lady Claire.
% They're the names of the potatoes, not the farmers.
% Just to be clear.''
% 
% In contrast to this artisan advertising, the management seem
% to treat the local employees as cheap labour rather than a treasured and
% unique group of people. For example, wage rises were promised following the
% implementation of 12 hour shifts but 6 months after the salary remains
% fixed at \pounds8 per hour, this time with no prospect of earning overtime.
% }


Despite these issues Chris has so far
decided to stay on at Tyrell's because
``if you live in Weobley, there are not many jobs.''
This context is important, because it illustrates how commuting
interacts with everyday life dilemmas, in this case
between moving house or staying put and
between quitting an exploitative job or finding a new one.
Ideally, Chris would like to sell his car, get a job in Hereford and be able
to walk to work each day. However, he's adapted to the new shifts, and
enjoys the 4 days of freedom he is allocated out of every 8, using them
to climb mountains, go to gigs and relax. The need to own a car
(on which 20\% of his income goes) and the expenditure on commuting
(5 to 10\% of his income) are disadvantages that can be endured for now.

Chris almost always drives to work. He has cycled a few times in nice weather
and would like to cycle to work more frequently. However, the
prospects for \emph{modal shift} are not great at present: his bike is not
much good, and the prospect of cycling 5-odd miles at 6 in the morning
after a physically punishing 12 hour shift is not attractive.
Chris is very interested in the cycle to work scheme, and believes he
would cycle more if he had a decent bike --- a friend was able to
get a \pounds900 bicycle through it. That's the semi-solution that
will be pursued in the short-term, and that goes well with Chris's
fitness hobbies.
When asked about the impact of the commute on his quality of life
Chris gave a short answer: ``not a lot really.'' For him commuting
is simply a means to an end --- to get to paid employment --- which
in itself is just a way to earn a living.

The sheer complexity of commuting on a national scale is well illustrated
by considering that Chris's commuting behaviour, plans and experiences
are just one data point out of hundreds of thousands. Subtleties of
his current behaviour,
% \footnote{For example he occasionally cycles and lived `car free'
% during 7 months last year.
% }
let alone the transient nature of his working
hours, shift patterns, home location and employment status are not
picked up by questions in the census or, to varying degrees, in the
national travel surveys (see \cref{Chapter4}). Nevertheless, the things that
Chris allocated importance to --- the distance to work, the time and money
costs of the commute and the availability of alternative modes --- indicate
that quantitative analysis of these aspects of the problem
of commuting is appropriate and relevant to everyday life.

There are certainly many unknown and highly varied individual circumstances,
such as Chris's that can never be squeezed into simple numerical models.
However, the variables about which good geographical data are available
(mode and distance) and the variables which can be calculated
with varying levels of uncertainty (e.g.~economic costs,
potential for modal shift), match the factors that held most
sway for Chris, with the exception of the location of his friends.
% \footnote{Even this could in theory be estimated, based on central place theory.}

% \section{Commuting as a global phenomenon (dodgy)}
% Chris Fisher's experience illustrates the importance of distance,
% available modes of travel and economics in the humble journey to work for one
% individual. On the global scale, commuting is vast. It is a near ubiquitous
% indicator of a formal economy, and consumes huge
% amounts of time, resources, and money (fig. on country commuter distances
% would be mint here). A `back of the envelope' calculation suggests that ~ xxx TJ
% of primary energy are consumed worldwide during 2 billion daily commuter trips,
% excluding indirect energy costs such as vehicle production, fossil fuel
% extraction, and road maintenance.\footnote{
% %
% Add method here
% %
% }
%  To provide a sense of scale, this translates to 2\% of global primary energy
% use (xx EJ) and 4\% of oil consumption across the globe. Commuting is also
% complex, taking place in myriad contexts
% and by a plethora of different modes. Charcoal burners... ,
% Factory workers in ..., The challenges faced by Chris Fisher regarding his  8
% mile car trip to the Tyrrel's crisp factory in Herefordshire may not be
% representative of commuting worldwide, but its importance in everyday life and
% the problems it poses for welfare and sustainability found everywhere.
% 
% To reiterate, journeys to work are dynamic and
% diverse: their \emph{length}, \emph{mode}, and \emph{costs} vary depend on a
% range of range of factors that also change over time and space. These are the
% fundamental characteristics of commuting and the building blocks of this
% research. These three variables also determine another important characteristic
% of commuter trips: their energy costs. This abstract yet quantifiable variable,
% based in physical reality, is increasingly important in a world of emissions
% targets, environmental awareness, and depleting oil fields. A review of previous
% work on commuting indicates that its quantification is also often absent
% in transport and energy policy. For this reason the energy costs of transport to
% work is the main concern of this PhD: understanding its variability across
% space, time and between people and interpretting what these results may mean
% for future transport policies is the broad goal. Each section of this
% thesis contributes to this goal. Before describing the contribution of each
% chapter, it is important to understand a little about the case study region,
% and why it was chosen.
% 
% Could include section on Getting to work in a 'post carbon' future here.

\section{The importance of commuting} \label{snimportance}
The previous two sections have illustrated the importance of commuting
in terms of its impact at the individual level,
and in the global context. In many countries, however, the importance of
commuting can be investigated using a more detailed source of information:
national transport statistics. This section introduces
aggregate level travel to work statistics from the UK Census,
which form the foundation of analysis in the coming sections, and outlines
the variability of commuting patterns nationally.
Based on these statistics, it also illustrates the importance of
commuting in comparison with other reasons for travel.

% These statistics demonstrate that there is no single way to measure the relative
% importance of travel to work compared with other reasons for personal travel
% such as shopping and `the school run'.
% 
% The relative importance of commuting
% as a reason for trip will depend on how you value personal travel, whether number
% of trips, distance, time or energy use are considered as appropriate measures
% of ``importance'', or what constitutes `essential' travel.
% Do I mention each of these points?

% Overall, commuter flow data tends to be the most reliable
% source of personal travel statistics. This is because
% travel to work is regular and predictable in time and over space.
% Distance can also be estimated quickly, based only on matched
% home-work postcodes.

% These, and other defining characteristics of work travel such as
% rates of occupancy and  multi-purpose trips are also explored, towards the
% sections end. These national level data, placed in context, help to introduce
% commuting as a unique and important trip type and outline gaps in our knowledge.

\subsection{Trips}
Trips are the basic unit of travel, ``a one-way course of travel
with a single main purpose'' \citep[p.~6]{Dft2011-notes}. The data presented in
\cref{fig:trip-nums-gb} (and henceforth)
therefore counts the daily journey to work and
back as two trips. The value for commuting provided by this dataset
(150 trips per year) may therefore
seem surprisingly low, implying that people only work an average of 75 days per year
--- \citet{hall2011tourism} estimate that roughly
400 commuter two-way trips are made per capita
per year worldwide. However,
the National Travel Survey samples all citizens, including children and the
elderly; the average number of trips made by commuters --- the focus in this
thesis --- is estimated to be double this figure, around 320 (\cref{sfreq}).

\begin{figure}[htbp]
  \centerline{
    \includegraphics[width = 14 cm]{./Figures/trip-nums-gb}}
    \rule{35em}{0.5pt}
  \caption{Average number of trips per person per year across Great Britain.}
  \label{fig:trip-nums-gb}
\end{figure}

\subsection{Distance}
% The total distance of trips made by each trip type is equal to the number of
% trips made, multiplied by the average distance. Commuter trips averaged 14.2 km
% in 2009/10, slightly longer than the average trip distance for all trips Great
% Britain (11.3 km). Commuter trips are the third longest type of trips in the
% UK, following holiday and business trips, whose average values are greatly
% increased by flying. %Dad'd
The distance made by all trips is their number multiplied by their average distance.
Commuter trips averaged 14.2 km in 2009/10, slightly longer than the 11.3 km
average for all trips in Great Britain and the third longest,
following holiday and business trips. The average length of the latter
are greatly increased by flying.
This information
% , as well as the evolution over time,
are illustrated in \cref{fig:dist-trip-gb}.

\begin{figure}[htbp]
  \centerline{
    \includegraphics[width = 14 cm]{./Figures/dist-trip-gb}}
    \rule{35em}{0.5pt}
  \caption{Average trip length by purpose in Great Britain.}
  \label{fig:dist-trip-gb}
\end{figure}

The average distance of each trip helps characterise commuting as relatively
long-distance compared with other trip purposes such as shopping (6.9 km).
However, total travel distance is more important from an energy perspective:
long leisure trips, for example, are comparatively unimportant in energy terms if
they are infrequent. The data shows that leisure travel\footnote{Leisure
trips include holidays and social trips, in the 2010 National Travel
Survey \citep{Dft2011-notes}.} dominates trip distances, despite the sporadic
nature of international holidays. Commuting is in second place, responsible for
2160 km of personal travel each year for UK citizens, including those under 16.
For commuters, the average total distance of commute would be approximately
double this value (\cref{fig:dist-purp-gb}).

\begin{figure}[htbp]
  \centerline{
    \includegraphics[width = 14 cm]{./Figures/dist-purp-gb}}
    \rule{35em}{0.5pt}
  \caption{Total distance travelled by mode in Great Britain.}
  \label{fig:dist-purp-gb}
\end{figure}

\subsection{Time}
\index{trip duration}
From the commuter's perspective, the number and distance of commuter trips made
may seem relatively unimportant: in the formal economy, time is money and
people are increasingly rushed to face up to professional and family
commitments \citep{Eisenstein2011}. Therefore, time is another measure of importance that
should receive attention in any introduction to commuting. Overall commuting is
the most time-consuming reason for personal travel in the UK, accounting
for 19\% of trip time, consuming 70 hours per year. Because both the numerator
and the denominator in this measure (hours per year) have time units, travel to
work can also be presented as the percentage of one's life spent travelling to
and from work\footnote{This
is a potentially poignant metric for those who
spend more than 5 hours per working day or more than 10\% of their life simply
getting to work and then turning around going home again!}
(\cref{fig:t-commuting}).

\begin{figure}[htbp]
  \centerline{
    \includegraphics[width = 14 cm]{./Figures/time-commuting}}
    \rule{35em}{0.5pt}
  \caption[Time spent commuting in Great Britain]{The average time spent by
citizens of Great Britain travelling to work and back each year. The right hand
axis illustrates the same information, this time as a proportion (data
source: \citealp{NTS2012-time-travel}).}
  \label{fig:t-commuting}
\end{figure}

There is pronounced regional variation in the average time spent travelling to
work. This variation is linked to the average time per commuter trip (high
total work travel time values are influenced by how frequently people work),
the distance to workplace, and, of prime importance, levels of congestion.
% (\cref{fig:really?}).

% \subsection{Energy} % Include this!
% The energy costs of commuting is a complex subject with variability over
% time, space and from person-to-person. However, its relative importance
% compared with other sectors of the economy can be estimated rapidly,
% using simple `back of the envelope' calculations. 

% \subsection{How `essential'  is commuting?}
% The premise of this thesis is that commuting is essential for healthy
% functioning of modern civilisation, yet fundamentally unsustainable due
% to its use of energy. Certain trips for leisure and holidays are clearly
% discretionary, and could be discontinued with relatively minor
% consequences for the individual.
% Stopping travelling to work, in contrast, could result in unemployment ---
% with major impacts on well-being, as described in Section \ref{s:well}.
% However, one could equally argue that trips for shopping, the `school
% run' and to even to visit distant family members is essential for
% quality of life.
% Here, we consider different ways of classifying `essential trips' and
% the consequences for the importance of travel to work in the overall
% drive towards a sustainable national economy.
% 
% The simplest approach is to define commuting as essential and all other
% reasons for trips as superfluous. Based on this definition, travel to work,
% accounts for 100\% of essential trips in the UK. Under this (highly
% unrealistic) assumption, the best way to reduce the energy costs of
% personal transportation would be to reduce demand for other reasons for trips.
% Under this extreme scenario,
% children could be educated at home, food could be delivered by vans
% directed by internet transactions and communication with friends and family
% could occur online. Demand for trips to work, however ---
% seen as 100\% essential --- would remain unchanged.
% 
% Instead of this dichotomy (image of dichotomy!!!)
% 
% 
% 
% \section{The case study areas}


\section{Thesis overview}
The thesis is divided into 9 chapters which can be classified into four parts:
introduction, methods, results and conclusions.
Chapters 1, 2 and 3 provide background to the research. The present chapter
provides context. The purpose is to show how the thesis is motivated by
and informs some of the grand debates of the 21$^{st}$ century: environmental,
economic and social.
Chapter 2 is a more conventional academic literature review, focusing on the
research that is most closely related to the thesis topic rather than its wider
context. Chapter 2 tackles the following questions: what is the range of methods used
to investigate energy use in transport from a policy perspective? To what extent
is the literature coherent in its assessment of the reasons for energy intensive
transport behaviour and appropriate solutions? \Cref{Chapter3}
is the methodological literature review. It traces the various
incarnations and uses of spatial microsimulation and related methods.
The purpose is to illustrate
the reasons for choosing to apply the technique to the research questions
outlined in chapter 2.

Chapters 4 and 5 are methodological. The data available to analysts interested
in commuting are explained in detail in chapter 4, with reference to an ideal
dataset. Later in the same chapter, the underlying theory and computer code
developed and
used to generate spatial microdata is described in detail. The aim is to allow the
results to be replicated by anyone provided with the same input data as used in
the this thesis. To this end numerous script files are provided which allow many
of the analyses performed to be re-run on any computer using free
software.\footnote{Sample code
and data used can be found here: {\color{blue}
\href{https://github.com/Robinlovelace/}
{github.com/Robinlovelace/}}
}
Chapter 5 describes and analyses the factors affecting energy use in personal
transport. Methods for converting CO$_2$ emissions data (the best official
source on the matter) into energy cost values per unit distance are described
and put to work on the best available data. \Cref{Chapter5}
culminates in a table summarising the best estimates for the
efficiency of each commonly used mode of travel to work.

The subsequent three chapters present the results and conclusions.
\Cref{Chapter6} harnesses the data and methods described in
previous chapters to
calculate the energy costs of travel to work at a range of levels, in England
and within the case study region of South Yorkshire. (A brief detour in
\cref{sinternational} compares English and Dutch commuter energy use
to illustrate the international applicability of the methods.) There is some discussion
of the links between energy use and other variables under investigation such as
home-work distance, mode of travel, age, sex and socio-economic class. However,
most of the results at this stage are descriptive: no
attempt is made here to evaluate political implications of the results. The
desirability of the commuting patterns that have been observed is more the
topic of \cref{Chapter7}, which discusses inequalities in commuter patterns.
% . The numbers are
% made to `speak for themselves' using a range of
% visualisation techniques.
In \cref{Chapter8} the attention is turned to the future. The analysis
is informed by `what if' scenarios made possible through spatial microsimulation
and a case study of `oil vulnerability' in Yorkshire and the
Humber.
The former creates quantitative scenarios to describe futures
of high cycling uptake and a shift to Finnish levels of telecommuting.
%!!! update this when c8 done!!!
Based on these assumptions, the total energy savings from each scenario is
estimated and the spatial and social distribution of the impacts
analysed. 
The latter investigates the likely impacts of high oil prices on
different social groups and places and is designed to show the policy-relevance
and usefulness of the methods.
% This links in with Chapter 7, which investigates
% inequalities in commuter energy use, and the related issue of oil
% vulnerability, in more general terms. These scenarios are
% based on evidence of reduced travel in a range of contexts, and how changing
% travel behaviours influence different sections of society.

Chapter 9 draws
together the various threads of the thesis to arrive at overall conclusions
about the energy costs of commuting: current patterns are not as simple
as first-impression thinking may indicate and neither are the solutions.
A particularly surprising result for the author was that cycling
can only make small savings in the current context compared with
the relatively overlooked options of telecommuting and car sharing.
% The methods allow
% other to check the results for themselves and tailor solutions to
% their specific situation.
% The aims and objectives that guide the thesis are as follows:

\section{Aims and objectives} \label{s:aims}
This chapter has argued that the energy costs of commuting is an
important and policy-relevant area of research, that links with
some of the major issues of the age.
This recognition of the potential applications of the research
is reflected in the aims and objectives. These, which have
helped to guide the research throughout, are as follows:

\subsection{Aims}
\begin{itemize}
 \item[A1] Investigate the energy cost of transport to work, its variability
at individual and geographic levels, drivers, and policy implications.
  \begin{itemize}
   \item[A1.1] Examine the variation of energy cost of trips to work, at
      geographic,	household and individual levels, and over time.
    \item[A1.2] Identify and explain the geographic and socio-economic factors
most closely  associated with high and low energy use.
    \item[A1.3] Formulate and analyse scenarios of change to inform decision
    makers about how commuter energy use can be reduced.
  \end{itemize}
  \item[A2] Explore and evaluate the potential of spatial microsimulation
models for the social and spatial analysis of the energy costs of commuting.
\end{itemize}

\subsection{Objectives}
\begin{itemize}
 \item[O1] Conduct a review of literature pertaining to the socio-economic and
geographical factors of
energy use and identify studies most relevant to the aims of this thesis.
  \item[O2]Calculate the energy costs of transport to work at different
geographic levels and interpret the results.
  \item[O3]Develop and use a spatial microsimulation model to simulate the
characteristics of different types of commuter and estimate the variability of
energy costs at the individual level.
  \item[O4] Identify the links between individual characteristics, geographic
variables and energy use
and analyse them further using the microsimulation model.
  \item[O5]Apply the energy use formula described by (Fels 1975) to individual
level commuting data to create estimates of the energy costs of transport to
work in Yorkshire (A1, O2).
  \item[O6] Formulate and test `what if' scenarios of future change in
variables
associated with commuter behaviour with the use of microsimulation and identify
the likely energy impacts of policy measures for commuters.
  \item[O7]Discuss the results in the context of high future energy prices and
the desire for reduced dependence on fossil fuels.
\end{itemize}

\subsection{Methods}
\begin{itemize}
 \item[M1] Descriptive statistics, time-series analysis, and GIS mapping (A1.1,
O2).
\item[M2]Development of a spatial microsimulation model (A1, A2, O3, O4).
\item[M3]Use the spatial microsimulation to investigate the impact of
change on commuter behaviour and energy consumption (A1.3, A2, O6, O7).
\end{itemize}







 % Introduction
%
% Chapter 2

\chapter{Personal transport, energy and commuting}
\label{Chapter2}
\lhead{Chapter 2. \emph{Personal transport, energy and commuting}}
% \lhead{Chapter 2. \emph{Commuting and its energy implications}} %
% \lhead{Chapter 2. \emph{The Energy Costs of Transport: A Review}}
\fancyhead[RO,LE]{Chapter 2. Personal transport, energy and commuting} %2side
\fancyhead[RE,LO]{\thepage}
%%% Really happy with this introductory text: reads nicely.
%%% Could I write this as a paper? May be worth it...

\begin{quote}
\textit{The traditional preoccupation with the supply side of transport policy
---
the provision of additional road, air and rail infrastructures --- is no longer
appropriate socially, economically and environmentally.}
\flushright{\citep[p.~5]{Peake1994}}
\end{quote}

Any review of research into the energy consumption of commuters is bound to
encounter wider issues such as transport infrastructure,
% --- which once in
% place, can affect transport flows for decades \citep{Whitelegg1987} ---
the spatial characteristics of labour markets \citep{Ballas2006}, population
densities of settlements \citep{Breheny1995}
and the price of oil \citep{Sexton2011}. Transport research is often
multidisciplinary \citep{Hoyle1992modern}. This element is even more important in
the present study because
commuting and energy use in transport are not academic disciplines, or even
established fields, of their own right. Rather they are issues, tackled from a
range of perspectives using various methods. 

As illustrated by the quote that opens this chapter,
research into energy in transport is
contested. Almost 20 years since it was written
there has undoubtedly been much more focus on the demand side;
social and environmental considerations have increasingly been
taken into account; and transport studies have become more
multi-disciplinary. Yet fundamental differences in the methods used by
researchers persist. Battle lines can be seen emerging in the literature,
for example, between those
who advocate a greater role for the social sciences \citep{Schwanen2011} and
those who advocate a scientific approach 
\citep{Simini2012, Marshall2008}. The transport-energy nexus has also received
attention from disciplines not traditionally associated with either issue,
such as computer science, physics and psychology.  It is therefore necessary to
impose some kind of order on the mass of work that is related to the topic.
With this aim in mind, the literature reviewed is divided into six sections:
\begin{itemize}
 \item the `sustainable mobility' paradigm (\cref{ssus})
 \item commuting research, at various scales   (\cref{s:commuting})
 \item energy use and emissions in personal transport generally (\cref{s:energy})
 \item energy impacts of commuting specifically (\cref{sdisciplines})
\item `tools of the trade' --- methods for studying energy and commuting
(\cref{s:tools})
\item key concepts in energy and commuting (\cref{skeyconcepts})
\end{itemize}
These sections initially deal with commuting and
transport energy use as separate entities,
because they have rarely overlapped. The studies that
do tackle the interface between these issues are generally conducted from
within pre-existing disciplines, such as economics or transport geography,
rather than adopting a completely multidisciplinary approach or attempting
to start a new field in `transport and energy', let alone
`energy use in commuting studies'.
\Cref{sdisciplines} therefore focuses on two studies
that deal with energy and commuting from two different perspectives:
transport geography and economics.
% have most to say in this
% regard, but drawing disciplinary boundaries in the field is probably
% not useful, so the section is structured by the research focus
% (life-cycle analysis, energy use of different modes etc.).
Because this
research area is quite specific, the section is the only one in which
comprehensive coverage is attempted. The other sections attempt
only to outline influential strands of research and highlight findings
of direct relevance to this project. \Cref{s:tools} provides
an overview of the techniques used in the research areas covered, and introduces
one of the main methods: spatial microsimulation.
(The spatial microsimulation literature is covered in more detail in
\cref{Chapter3}.)
The current chapter
concludes with a summary of important knowledge gaps in the area of
commuter energy costs, and promising research directions that
are related to the thesis (\cref{sc2sum}).
%%% This where we've hit the end (the section is only 1 paragraph long - Jan 2013)
%%% Ideally each section should be ~2,000 words long (10,000 / 4), less with images.

\section{The sustainable mobility paradigm} \label{ssus}
As outlined in \cref{Chapter1}, energy use in transport is bound up with a
number of issues --- climate change, energy, inequality.
Diverse as these are, they all fall within
the umbrella term of sustainability. It is not surprising, therefore, that much
of the work linking transport and energy use has been conducted within the context of
sustainability, especially since the 1990s when sustainability became a buzzword
in politics and academia. Here is not the place to discuss
of what sustainability does and does not mean.\footnote{See \citep{Pezzey1997} for an
attempt to define the term rigorously or \citep{Steg2005} for a discussion
of `sustainable transportation'.
}
For the purposes of this section, suffice to
say that sustainability relates to \emph{long-term} environmental, social and
economic well-being. According to \citet{Banister2008}, in a paper with the
same title as this section, sustainable mobility is an approach to transport
research and policy that differs from conventional transport planning priorities
in the following ways:
\begin{itemize}
 \item its focus on people and social outcomes rather than infrastructure,
 vehicles and traffic
 \item localised and specific in its approach to intervention, rather than large
 scale and homogeneous
 \item a focus on potential scenarios of the future rather than univariate `modelling'
 \item travel modes placed in a hierarchy with pedestrians and cyclists at the top,
 rather than a focus on motorised transport
 \item multi-criteria assessment methods used for project assessment
 rather than just economic valuation
\end{itemize}
On all counts, the world-view adopted in this research project
fits firmly into the sustainable mobility
paradigm, so this is the starting point for the literature review. Energy use
in personal transport may seem a technical consideration, suitable for consideration
only by traffic engineers and natural resource economists. Yet the energy intensity
of transport systems has a direct impact on resource depletion (and therefore
economic sustainability), the natural environment and, by amplifying inequalities
in access to physical and cultural resources, people's lives.
The energy costs of commuting are therefore of critical importance to
the ability of modern economies to sustain themselves.

Probably the most high-profile UK government report written from the perspective
of the sustainable mobility was published by the Sustainable Development
Commission (SDC) \citep{Kay2011}.\footnote{This report, incidentally, was
published just before the SDC was dissolved
by the coalition government in March 2011. No follow-up research
in the area has been conducted.
}
`Fairness in a Car
Dependent Society' takes
a broad perspective when analysing personal transport.
As advocated by \citet{Banister2008},
it focuses on people rather than traffic and infrastructure, while also mentioning
the potential for environmental and (long-term) economic gain.
The report urges the prioritisation of
``quality of life, safety and the environment'' for all members of society
affected by personal travel systems over the speed and convenience of
wealthy travellers \citep[p.~5]{Kay2011}. The report's findings are
especially powerful because it provided a very large body of evidence
to support its findings, rather than to simply repeat the `anti-car' mantra
expounded by some based on the strength of rhetoric, social theory and
a smattering of technical facts (e.g.~\citealp{Dennis2009}).
% An example of this insistence on evidence to support the argument
% is presented in \cref{fcrashnssec}: the simple visualisation
% vividly illustrates the problem in an politically neutral way, for
% maximum impact. Although traffic-related deaths and injuries are outside
% the scope of this PhD, the approach to quantitative evidence embodied in
% \cref{fcrashnssec} is something to which this thesis aspires.
% 
% \begin{figure}[htbp]
%   \centerline{
%     \includegraphics[width = 14 cm]{crashesnssec}}
%   \caption[The relationship between class and traffic-related deaths]
%   {The relationship between social class and traffic-related deaths
%   \citep{Kay2011}}
%   \label{fcrashnssec}
% \end{figure}

\citet{Kay2011} is also useful as a source of inspiration about
future interventions, as it provides %!!! mention kay in concs.
strong and specific policy recommendations. The most general of these, that 
can be applied to nearly every intervention affecting transport, is that a clear order
of priorities should be followed by transport policy-makers (\cref{fsdc}).
Incidentally, this is the same order of priorities that would be
followed if reducing energy use were the primary objective of
transport policy, as the evidence presented in chapter 1 suggests it should be.

This thesis is therefore closely related to the SDC study (and
the sustainable mobility paradigm more generally) in a number of ways.
It begins from the same world-view as \citet{Banister2008}, but
focuses on energy as a way to include all the various
factors affecting sustainability. The purpose of this research mirrors
that of \citet{Kay2011}:  to highlight the wider impacts of
personal mobility.
The methods are quite different, however: based \index{fairness}
on the knowledge that a range of social, economic and environmental ills are
associated with energy intensive transport highlighted in \cref{Chapter1}, the focus
is on
% not on the fairness and social outcomes favoured by \citet{Banister2008}
% and \citet{Kay2011}, but
energy use. This thesis does also
highlight the wider costs to society of personal travel advocated
in the `sustainable mobility paradigm', but indirectly,
via energy use, and with a focus on only one type of trip: commuting.

\begin{figure}[htbp]
  \centerline{
    \includegraphics[width = 10 cm]{Sustainable-trans-hierarchy}}
  \caption{The sustainable transport hierarchy \citep{Kay2011}.} %%% could update with methods
  \label{fsdc}
\end{figure}

\subsection{Active travel} \label{sactive}
Although not always explicitly part of the sustainable mobility
paradigm, many of the studies from the loosely defined `active travel'
literature\footnote{This
area of
research has also been referred to `non-motorised transport', or simply
`walking and cycling'. The term `active travel' is preferred as it is more
concise and encapsulates all methods of travel to work that rely on human
muscles rather than mass-produced motors as prime-movers (see
\citealp{Smil2008} for more on the contrasts and surprising similarities between
the two). The rare but growing category of muscle-motor hybrid vehicles such as
electric bicycles is ambiguous is in this regard: as the ratio of motive energy
provided by personal exertion and inanimate energy sources will vary between
zero and infinity from case to case. The approach taken here is to exclude it
from active travel completely as motors and their energy supply must be
included for a realistic energy assessment.
%ref
}
make reference to the sustainability benefits of walking and cycling. For
the purposes of this literature review, research into non-motorised modes is
therefore considered as part of sustainable mobility,
although the term has been used in different
contexts.\footnote{Lawrence Burns, who directs the Program on Sustainable
Mobility at Columbia University's Earth Institute,
uses `sustainable mobility' primarily to describe shifts in
car technology and use, including driver-less cars and electrification \citet{Burns2013}.
\citet{Aftabuzzaman2011} uses the term to describe a transport system
resilient in the face of peak oil.
}
Much
of the active travel literature
has a clear health agenda (e.g.~\citealp{Jarrett2012}); here the focus is
on studies that also report energy and emissions implications.

\citet{Woodcock2007} investigated the links between transport, the environment
and health by projecting the rate of active travel up to 2030 in London.
The outcome of policies to encourage cycling were found to be wide ranging,
including positive impacts on road injury rates (a `neglected epidemic'), physical
inactivity and associated degenerative diseases, climate change and pollution,
`community severance', as well as difficult-to-measure impacts on energy
security and rates of transmission of infectious diseases. Clearly it is not
possible to accurately measure each of these impacts in a single study, but
it is useful to bear in mind the broader benefits of walking and cycling, which
are also particularly energy efficient. In a similar vein, \citet{Jacobsen2009}
provided evidence to suggest that as well as competing with healthier
and lower-energy active travel modes
for trips and space, motorised traffic also discourages walking and cycling
through perceived danger levels. Although their methodology was relatively
rudimentary (a review of statistics from the academic and policy literature),
\citet{Jacobsen2009} provide the basis for an interesting hypothesis:
that strategies to reduce
car use may be more effective than pro-active travel measures in terms of energy
and health outcomes. The case study comparing commuter energy use between
the UK and the Netherlands presented in \cref{sinternational} provides some
empirical support for this hypothesis.

With the emergence of newly available datasets from GPS devices, mobile phones
and bicycle rental schemes, more sophisticated methods have emerged in
the realm of active travel research. \citet{Ogil-cambridge2010}, for example,
provide details of how GPS measurements for individuals can be used
estimate both physical activity levels and CO$_2$ savings of active travel.
In-depth questionnaires were also used to estimate
``physical activity energy expenditure (PAEE) and total energy expenditure (TEE)''
\citep[p.~7]{Ogil-cambridge2010}. GPS data was combined with
accelerometer data by \citet{Cooper2010} to estimate physical activity. Although this
metabolic energy consumption of the human body is not
generally seen in the same light as energy use by vehicles, both can be measured
in the same units and compared directly. It is argued in \cref{Chapter5} that
this fact is a further benefit of the energy approach to commuting: substituting
motorised energy use with muscular energy has a direct impact on obesity and
chronic inactivity levels. Thus energy measurements can
encapsulate (to some degree) health as well as environmental impacts of travel.

In line with this new abundance of data, advances have been made in
characterising and 
modelling active travel patterns as well. \citet{Millward2013} used GPS
data to supplement survey findings on walking trip characteristics in a US
city. The combination allowed for accurate characterisation of both
quantitative variables such as speed, time and distance of travel as well
as qualitative information about the reason for the trip.
Of particular relevance to scenarios of
future change, is work looking at the `impedance functions' of active travel
modes with respect to distance under various conditions \citep{Iacono2010}.
Here, impedance refers to the disincentive to make trips by active travel per
unit distance.
Impedance influences $p$, the proportion trips that
take place between A and B made by walking or cycling. Due to the impedance or 
`resistance' to travel associated with these modes being highly dependent on distance
compared with faster and 
less physically demanding motorised modes, the proportion of trips made by them
can be expressed as a function of distance ($p = f(d)$). Based on this reasoning
$p$ should be high for the shortest
trips, dropping rapidly as the distance increases beyond a few kilometres
and levelling-off towards 0\%  after around 5 km for walking and 15 km for cycling. 
This hypothesis has indeed been born-out in practice.
Based on travel survey data, \citet{Iacono2010} calculated the rate
at which the proportion of trips made by bicycle and walking decreases
with increasing distance for different trip reasons, including shopping and
commuting \cref{fimpedance}.
The average proportion of trips ($p$) made by a particular mode in a particular context
(e.g.~bicycles for shopping in a given settlement) was found by \citet{Iacono2010}
to take the following functional form:
\index{impedance}
\begin{equation}
 p = \alpha \times e^{- \beta \times d}
 \label{eimpedance}
\end{equation}
where $\alpha$, the proportion of made for the shortest distances
and $\beta$, the rate of decay
are parameters to be calculated from empirical evidence. 
This equation is interpreted in \cref{Chapter8} as a proxy for the probability
of car-bicycle modal shift.


\begin{figure}[htbp]
  \centerline{
    \includegraphics[width = 10 cm]{impedance}}
    \rule{35em}{0.5pt}
  \caption[Proportion of trips by active travel by distance and mode]
  {Proportion of trips made by active travel by distance and mode. Functional form
  from \cref{eimpedance}; parameter values taken from \citep{Iacono2010}.}
  \label{fimpedance}
\end{figure}

In summary, the sustainable mobility literature provides a strong foundation
for investigating energy costs in commuting. The emerging field of
active travel also has a strong interest in energy, although this is rarely
linked to the energy use of motorised modes. Sustainable mobility provides both
a world-view and methodological guidance for the thesis, yet is still only a
minor influence on commuting research overall, as shown in the subsequent section.

\section{Commuting research: individual to national levels} %%% only 1-2000 words
\label{s:commuting}
The energy costs of commuting depend on commuting behaviour.
As \citet[p.~297]{smith2011polycentricity}
put it regarding CO$_2$ emissions from travel to work, they are
``essentially a weighted combination of the mode-choice and travel distance
patterns.'' Understanding
the factors driving travel behaviour is key, therefore, to understanding
energy costs. `Behaviour' can be understood from a range of
perspectives, from the internal workings of the mind to the macro-economic
forces driving the type and spatial distribution of jobs
(\cref{fig:com-pyramid}). This section is structured to
reflect the multiple levels that affect commuter patterns.

Many important factors influencing the decision of whether, how and
how far to travel to work depend on the global economy, which is
largely beyond anyone's control \citep{Eisenstein2011}:
the price of crude oil, industrial
production\footnote{Production of
cars, trains and machinery, for example, is a prerequisite
for the construction and maintenance of transport infrastructure.
}
are all determined outside the sovereignty of any person or even
country, yet these factors, determined by the global economic system,
clearly have large knock-on effects on commuting patterns. National-scale
physical factors also play a role.
The transport network, shifting vehicle fleet efficiencies and the nation's
topography all help determine the ease with which
different commutes are undertaken, and their energy costs.
Large-scale political and
economic processes, such as congestion charges, fuel taxes
and house price gradients also affect commuting behaviour.
Zooming in on the local scale, the strength and nature of the local
economy will decide whether suitable jobs are available locally or
whether one's job search must go further afield.
Community and family ties could both make commuting distances shorter
(by providing support to family and friends searching for work --- the
``home-field advantage'' identified by \citealt[p.~100]{Simini2012}), or longer
(by creating a disincentive for people to move closer to where they
work \citealp{Green-1999-ld-commute}).
At the simplest level, however, the decision to get up in the morning
and commute to work is ultimately made by individuals \cref{fig:com-pyramid}).

\begin{figure}[htbp]
  \centerline{
    \includegraphics[width = 10 cm]{commuting-research-image2}}
  \caption{Schematic
for organising research commuting research by scale.} %%% could update with methods
  \label{fig:com-pyramid}
\end{figure}

\subsection{Personal factors: psychology, family and community}
As Chris Fisher's story demonstrated (\cref{s:realities}), human beings are not
merely economic machines motivated solely by money. We make decisions based on
a wide and interrelated range of factors \citep{Pinker1997}. Some are instinctive, others
are carefully planned \citep{Kahneman2012}. 
%Within this array of factors
%money tends to play an important role, as do family considerations
%and proximity to friends and one's home town.
While money plays an important role, it is within an array of 
factors along with family considerations and proximity to friends and home.

In some ways, long-distance commuting is the ultimate
manifestation of the conflict between
work and family life. If money were the only objective, people would be far more
mobile, willing to pack their bags and leave to live near better salaried jobs
whenever opportunities arrive. This is obviously not the case:
``job relocation almost always involves a move not only of one
individual's job, but also of his/her household's home and of jobs/schools for other
household members''
\citep[p.~52]{Green-1999-ld-commute}.\footnote{This
decision, to move for personal reasons rather than work,
is also well-expressed in everyday speech:
``I'd much rather
have a crap job and be with Richard than have a good job and be miserable'', as one
person told me (Emma, 2013, personal communication).
}
Over the past 50 years, perhaps due to the perceived social costs
of this upheaval, job relocation has increasingly
\emph{not} led to house relocation, but longer commutes instead
\citep{Green-1999-ld-commute, Nielsen2008}.

This trend has been labelled the `commuting paradox' due to the
seeming irrationality of the decision 
to spend much of one's time travelling to work and back \citep{Stutzer2008},
in face of evidence of negative impacts on well-being \citep{novaco1990objective}.
Approaching the problem at the individual level makes sense:
people are not economic machines,
yet assuming that people make a personal cost-benefit analysis for each
available option allows the powerful tools of microeconomics to be used. Applied to
commuting, each individual would evaluate all work-home (and
hence commuting) options and select the
best \citep{Stutzer2008}.\footnote{Mysteriously,
as the authors of the `commuting paradox' point out, this cost-benefit analysis is often
performed in a less that rational way, leading to commuting costs
(predominantly on unquantified well-being) that far outweigh the benefits
in many cases \citep{Stutzer2008}.
}

Research into commuting at the individual level generally uses psychology
(e.g.~\citealp{van1998social}) or microeconomic theory (e.g.~\citealp{van1999job}) to explain
\emph{why} people choose their commuting behaviours.
Yet the level of analysis is generally weaker when it comes to describing
\emph{how} commuting patterns --- the aggregate pattern of many individual flows ---
are configured and how much energy or other resources these patterns
use relative to other activities.
The relationship between commuting and larger scale processes
is generally not considered in individual level studies, although
there is a move towards more holistic understanding of individuals.
One study that analysed both environmental and psychological
determinants of individual level commuting behaviour found conclusive evidence
(from a sample of 130 university students) that ``cognitive variables play
a more important role in the prediction of active commuting than do environmental
variables'' \citep[p.~9]{Lemieux2009}. Because of the non-geographical nature
of this study and its small sample size, however, it provides little evidence
on the factors related to \emph{aggregate level} variability in commuter flow
patterns. Local, regional and national level studies are needed.

% The psychological literature on commuting sheds some light on
% this commuting paradox.
% \citep{Brand2013}
% \citep{Gatersleben2010}
% \citep{Roberts2011}
\subsection{Behavioural economics and its impacts on commuting}
Behavioural economics seeks to explain a large part of human behaviour in advanced
capitalist societies where making money is often (implicitly or otherwise) seen
as the number one \emph{raison d'etre} of life \citep{Eisenstein2011}.
The underlying assumption that human beings are rational beings
has of course come under attack from many quarters. To take one 
example, ``There is probably no other hypothesis about human behaviour [than
economic rationality] so thoroughly discredited on empirical grounds that still
operates as a standard working assumption in any discipline''
(Anderson, 2000; cited in \citealp[p.~34]{Exel2011-b-ec}). 
Despite these criticisms it is easier to create testable models in economics 
than the social sciences (Perman, 2003).
%Despite
%these
%criticisms, and the fact that the idea of an `economic man' is abhorrent to many
%people's better instincts, a clear advantage of economics over other social
%sciences is the possibility to formulate and test quantitative relationships,
%to evaluate the extent of the models' deviance from reality \citep{Perman2003}.

Indeed, many economists would be quick to point out that the term `economics'
has been conflated with what is in fact `neoclassical economics' in the public
consciousness and in other academic disciplines.
It has been argued that it is only with the recent
focus on money exclusively (instead of the physical reality that underpins
its value) that utility and profit have been conflated \citep{porritt2007capitalism,
Eisenstein2011}.
Clearly, it is not money per se that affects commuting energy costs, but
its indirect influence on behaviour. It is for this
reason that behavioural economics is the
branch of the discipline with most insight into travel to work patterns.
At its most tempered, modern behavioural economics completely accepts that much
of human behaviour follows a rationality other than the profit
motive. Many behavioural economists acknowledge the findings of Nobel
Laureate Daniel Kahneman, neatly summarised in the
book \emph{Thinking, Fast and Slow} \citep{Kahneman2012}, which
explains that humans are servants to both cool rational thought processes
(when `system 2' is dominant) and also to quick-fire decisions based on
spontaneous urges and heuristic reasoning (when `system 1' is dominant). The
caveat in the quantitative analysis underlying economic analyses
becomes ``when humans are acting
rationally, with the objective of maximising profit'' which is only some of the
time.
% (the rest of the time providing a convenient explanation for model errors).

If these limitations are understood, behavioural economics can provide a powerful
framework for explanation. The
framework is consistent with anecdotal evidence about the reasons behind travel
behaviours (e.g.~Chris Fisher's decision not to move to Hereford because
commuting to the Tyrrell's crisp factory would then become too expensive) and
the observed behaviour that people react predictably to price signals.
The framework can also be called upon to explain more general (and less
testable) trends, such as the increasing dominance of the car throughout the
20$^{th}$ century: ``One important reason for the automobile's increasing
dominance in passenger transport is that ... the price of car travel relative
to public transport has largely remained steady while the (system) quality of
car travel has considerably increased relative to public transport''
\citep[p.~149]{Exel2011-b-ec}. % p.149 in the book in case it's different
Far from assuming humans are soulless economic machines,
such explanations, taken as descriptors of aggregate behaviour,
% (not that of specific individuals),
assume citizens are simply careful with their cash.
Such explanations are supported by multiple studies of transport elasticity
(e.g.~\citealp{goodwin2004elasticities}).

\subsection{The local and regional economy}
The idea that localised environmental factors can
influence behaviour patterns has a strong tradition
in geography.
In terms of the impact of local factors on commuting, existing
research has focussed on transport infrastructure, the built
environment,\footnote{The
built environment is defined as ``equipment, facilities or infrastructures in
one's environment'' that influence travel behaviour by \citet[p.~2]{Lemieux2009}.
The built environment can thus be seen as a superset of transport infrastructure,
which includes features such as parks, street lights and even showers designed
to encourage running or cycling to work.
}
topography and local economies, as well as the more abstract concept of
`urban form'.

A common research strategy for exploring these links is to take aggregate travel
behaviour in different areas as the dependent variable and set-up a
multiple regression model to identify which factors can best explain its variation.
This strategy has provided a number of insights into commuting
behaviour and its dependence on geographical factors:
\begin{itemize}
 \item \citet{Buehler2012} ran a logistic regression model and found that
 the provision of showers and bicycle parking by employers (which
 had not previously been included in regression models of commuter behaviour)
 were significantly related to the chances of respondents cycling to work.
 The provision of bicycle lanes and free car parking also had large
 impacts on the odds ratio of a person cycling in the expected
 direction, supporting past literature on the matter. Significantly, this
 study also combined household level variables; it was found that a high number
 of bicycles (and low number of cars) per household member also increased the
 propensity to cycle, as did high income and `white' ethnicity.
 \item  \citet{Titheridge2006} used distance of commute as the dependent variable
 in their study of commuter patterns in the East of England. It was found that
 distance from London, social class and level of car ownership in each ward
 affected distance in the expected ways. Population density, which would
 be expected to be associated with lower energy costs based on the `compact
 city' concept, was positively associated with commuting distance in their model.
 This contrasts the idea that bunched-up living is a panacea for travel costs and
 was explained by \citet{Titheridge2006} in terms of accessibility to transport
 infrastructure.
 \item \citet{Muniz2005} performed a regression analysis exploring the impacts of urban form on
 the `ecological footprint' (which is closely related to energy use) of commuting
 in Barcelona Metropolitan Region. It was found that, for the 163 municipalities
 that constituted the case-study area, low population densities, high `accessibility'
 (which seems to have been defined simply as distance from central Barcelona)
 and high average income all were positively associated with the dependent variable.
 Although this study was conducted at only one scale (it may suffer from the
 ecological fallacy and does not prove causality), the authors concluded that
 factors relating to urban form ``have a greater capacity to explain municipal
 ecological footprints variability than other factors'' \citep[p.~511]{Muniz2005}.
\end{itemize}

Such studies, which use geographical zones as the unit of analysis,
have revealed some of the factors that are closely related to certain commuting
patterns. Some of these, such as propensity to cycle and distance to workplace,
have important energy implications. When the independent variables include
factors over which policy makers have some degree of influence, such as
employers' provision of showers investigated by \citet{Buehler2012}, the
findings can be used to predict changes resulting from new policies. Even in
cases where the independent variables are largely beyond anyone's control ---
such as population density and home-work distances ---
regression analysis can be useful: it can be used to identify anomalies
where commuting patterns differ greatly from what would be expected based on
explanatory variables alone. In these cases, it must be acknowledged that
other processes are in operation, which can lead to new avenues for research.
However, regression analysis used in this way is limited:
causality is not proved; relationships may not hold at different levels of
analysis; and standard regression does not take space into account
(spatially weighted regression can be used to tackle this problem).
Partly to overcome these limitations, a number of other strategies have
been used to explore the geographical determinants of commuting behaviour.

In a study of commuting behaviour in northern Sweden, descriptive statistics
and maps were used to characterise commuter patterns in the region \citep{Sandow2008}.
Making use of the abundant
anonymous spatial microdata made available by the Swedish state, an
individual level logit model, with long or short distance commute
set as the binary variable,
was used to explore the reasons for and impacts of the observed patterns.
It was found that people living in more sparsely populated areas
were more likely to travel far to work than those living in dense areas.
This was as expected (but in contrast to \citet{Titheridge2006}).
The individual level data allowed for the investigation of socio-demographic
variables: education and income were associated with longer commutes.
Interestingly (in contrast to UK data), commuting distance decreases
with every age group above the 16-25 band. Gender differences were also
apparent: men travelled further than women and the impact of marriage and
children on the probability of commuting far was greater on females. Thus
it was concluded that family
commitments ``constrain women to a higher extent than men''
\citep[p.~24]{Sandow2008}.

% Urban form: \citep{Pooley2000commuting}
% \citep{Levtnson1997}
% \citep{North2010585}
% The importance of infrastructure has been noted in a number of studies.
% \citep{Titheridge2006} 

\subsection{National and global considerations}
While regional approaches have tended to focus on detailed sub-regional factors
affecting commuting, national approaches tend to be broader. The large
quantity of data available (albeit often at a high level of spatial
aggregation and low temporal resolution) make the national level
well suited to analysing shifts over time and persistent patterns within commuter
flows. Larger study areas also shift attention towards universal concepts,
that should, in theory, apply anywhere with similar underlying conditions.

In the context of the compact city debate, an individual level regression
model involving 47,000 people across the US was undertaken by
\citet{Levtnson1997} to ascertain the impact of population density on
travel to work distance and time (and hence average speed also). A wide range
of individual and geographical
factors (the latter aggregated at the level of Metropolitan Statistical
Areas (MSA), roughly equivalent to county level in the UK) were
used as explanatory variables. These were
carefully selected based on theory and previous findings. They
included a measure of polycentricity (the number of `activity centres' --- meaning
employment centres --- in each MSA), population growth rate and three variables
to quantify the transport technology in use in each area. It was found that for
car drivers, travel speed and distance were negatively associated with density. Time,
which had received little attention in the compact city debate previously, was found to be
negatively associated  increased residential density up to a certain limit
and then actually increase above this threshold. It was concluded that
this indicates diminishing returns as the density of settlements increased 
if cars are the main form of transport, due to congestion. Public
transport users, by contrast, ``displayed a negative relationship between travel
time and density both above and below the 10,000 ppsm density threshold'',
suggesting that these modes are less affected by traffic (and hence more attractive)
in dense urban areas \citep[p.~168]{Levtnson1997}.

Building on these findings, \citet{Levinson2012}
returned to the question of the factors affecting commute time in US
MSAs with updated datasets and more sophisticated tools for analysis.
It was found that accessibility was the major determining factor of
travel to work characteristics at the MSA level, and had a strong negative
association with average time and mode share of cars. Accessibility
(a slightly refined version of which was used in the final model) was defined,
for given time thresholds, as follows:
\begin{equation}
 a_t = \pi \times \left[ \frac{V_n \times t}{Q} \right]^2 \times p_{emp}
\end{equation}
where $V_n$ is average network velocity, $Q$ is circuity ---
see page xix for definition and \cref{fig:routes} for illustration ---
and $p_{emp}$ is the urban density (measured in jobs per km$^2$). A number of other
mathematical entities were used to define the transport network, the most
influential of which were treeness (roughly speaking, the proportion of the network going to
new places), connectivity (measured in five metrics, from alpha to gamma) and
circuity. The relevance of \citep{Levinson2012} for this thesis is that it
provides strong evidence to suggest key aspects of the journey to work are
influenced by road and settlement factors, and a set of tools for measuring
and assessing the effects of these factors. These techniques are not
used in a model of commuter energy use in the case studies presented in
this thesis, but could be in the future.

% Also at the national scale, \citet{Turnbull2000} used retrospective
% questionnaires to assess the changing nature of travel to work over time.
% The sample size was small 

Commuting has been studied and understood from a wide range of
perspectives. For the purposes of this thesis, insights are taken from economics,
ecology, and transport geography. The first assumes commuters to
be free thinking utility maximisers \citep{Sexton2011}; the second sees humans
as ``mobile, interacting animals'' who ``are no different from our fellow
species'' \citep[p. 40]{Brockmann2012}. Transport geography tends to be
agnostic in its explanatory framework, taking insights from the spatial
structure of transport networks, supply and demand centres, and the physical
environment \citep{Rodrigue2009}.
Interestingly, considering the ubiquity of commuting worldwide, no
research into commuting as a global phenomenon could be found, let alone
systematic comparisons between nations.
This suggests that there is a research gap in the area of international
commuting studies, which may be partially filled by a comparison of the
UK and the Netherlands later in this thesis \cref{sinternational}, as
recommended in the conclusions (see \cref{sfurther}).

\section{Energy use and CO$_2$ in transport studies}
\label{s:energy}
The traditional reasons for interest in commuting and  personal transport
more generally include its links to urban structure, industrial location,
productivity of workers and  quality of life. Economic factors have
tended to be dominant in past research,
but energy use and its environmentally destructive impacts,
predominantly quantified in the form of greenhouse gas emissions,
% \footnote{The term
% `evil twin' is used here because emissions almost always result from
% energy use in transport, yet only the latter is seen as a `bad' thing.
% Energy use can be seen as a good thing when used as a measure of economic
% activity.}
are increasingly becoming a focus for transport researchers \citep{Chapman2007}.
Although CO$_2$ production is a direct result of energy consumption,
depending on emission factors (\citealp{Defra2011}; see \cref{fgco2}), some studies continue to
treat them as separate issues. \citet{Boussauw2009}, for example,
calculate the energy costs of commuting in Flanders, but nowhere does
the paper mention the link to climate change: results are also, in essence,
a map of CO$_2$ emissions due to commuting, relevant to EU targets.
On the other hand,
it is possible and equally valid (if one's primary concern is climate change)
to only quantify CO$_2$ emissions and acknowledge
that the results essentially show energy use \citep{smith2011polycentricity}.

\citet{Simonsen2011} harness the knowledge that energy use
and greenhouse gas emissions are two sides of the same coin to use the
same energy analysis model to quantify both. In their analysis of cars in Norway,
it was found that only electric vehicles powered by renewable
sources (hydro-electric plants in this case, which are bountiful in Norway)
performed well. The approach taken in this thesis follows
\citet{Simonsen2011} in seeing the link
between energy and emissions.
% but takes it even further: the former is seen
Moreover, it is assumed that the former is a close enough proxy of the latter
at the system level that only energy use needs to be
calculated to gain an understanding of
both.\footnote{`At
the system level'
in this context means emissions arising from knock-on impacts of
interventions in the transport system are taken into account.
For example, if rapid uptake of electric cars leads to slower phasing
out of fossil fuel fired power plants, this would constitute
additional emissions at the system level that are not included in
official emissions inventories.
}
This prevents the complexity of having to report two (very highly correlated)
sets of indicators for the energy and emissions impacts. They are assumed to
be essentially the same thing.

Underlying drivers of this interest in energy use in transport and
associated emissions include peak oil and climate
change (\cref{Chapter1}). This attention has led to 
methods and findings directly related to the thesis.
% The work described in the following section is therefore of practical use.
% Describing the methods of, and questions raised by, this body of literature are
% therefore priorities of the subsequent section.
Although there has been a recent proliferation of interest in the contribution of
transport energy use to climate change \citep{Schwanen2011}, the topic
has  received attention, intermittently, over many years. Interest seems to
have peaked during the 1970s,
following the major oil crises of that decade \citep{Greer2009}. Since then the
topic has largely been confined to the following fields:
\begin{itemize}
 \item Urban sprawl:
 the phenomenon of low density housing, also known as suburbia, is highly car
dependent and has attracted attention investigating its impacts on transport energy
use. The antithesis to this is the `compact city'. Investigation of continuum between
these two extremes has led to many insights on the impact of urban form on transport energy use.
\item The energy costs of transport modes: quantifying which modes of transport
use most, and least energy per unit distance, typically per passenger, vehicle or
tonne kilometre: $pkm$, $vkm$ or $Tkm$.
% \item The `compact city' debate, in which the hypothesis that high density
% settlements are more energy efficient, due largely to transport.
% \item Life cycle analysis (LCA) and studies of the relative importance of
% embodied energy in transport systems.
\item The climate impacts of transport, usually quantified through estimates
of the quantity of CO$_2$ directly emitted by vehicles.
% \item The field of `energy and equity', which investigates the impact of
% unequal access to powerful machines for personal travel and social inequalities.
\end{itemize}
Transport and energy use is a broad area of research, so it
is inevitable that not all of it fits neatly into these four categories. A fifth
category, miscellaneous studies on transport and energy, will emphasise this
diversity of approaches, and touch on the interdisciplinary nature of the work.

\subsection{The energy costs of urban form: urban sprawl and compact cities}
The links between urban form and consumption of fossil fuels (primary energy)
have been of interest since at least the 1940s, especially amongst utopian town
planners \citep{Steadman1977}. Of the various types of urban form under
consideration, from the fictional `City of Efficient Consumption'
\citep{Goodman1947} to the `compact city' \citep{Breheny1995}, none have
received more critical attention than that of urban sprawl \citep{Marshall2008}.
Urban sprawl has long been identified as an energy intensive settlement
pattern, with social and environmental knock-on effects: ``Urban sprawl not only
consumes more natural ecosystems and has a higher cost per unit of development
in both money and materials, but once completed it requires higher inputs of
energy and generates more air and water pollution'' \citep{Bormann1976}.

Such statements may seem obvious, yet without evidence questions about
the extent of the
problem, and how to mitigate it, remain unanswered. This is a key motivation
behind methods which seek to measure aggregate energy use over space, and
provide breakdowns of how much energy is used where, and insights into why.
% (Example of overall energy use)
One implicit assumption underlying much of this research is that
energy use is \emph{the} defining
variable of a settlement and hence requires most attention.
This reasoning was stated explicitly by
\citet{Marique2012}, who note that despite the primacy of the transport sector
in driving up energy use in sprawling suburbs, ``transport energy consumption
is rarely taken into account'' (p.~1). In response to this negligence, the authors
quantify the average transport energy costs in four settlements, based on travel
statistics. Their analysis shows commuting to be the most
important determinant of transport energy consumption in Belgium. Commuting
consumes more than double the amount of energy (4000 to 6000 kWh/p/yr) 
than the next largest transport energy user (trips to school)
\citep{Marique2012}. These findings lend support to the topic of this thesis and
encourage further analysis of energy use in personal travel overall.

Despite the use of census data, \citet{Marique2012} present their findings only
at high levels of aggregation, for entire settlements. The \emph{distribution}
of energy consumption within the areas is not considered. Nor are the
\emph{types} of people responsible for high energy use for commuting. These
gaps in their research suggest more detail would be welcome: providing a method
to calculate the energy costs of commuting at lower geographies that is capable
of providing breakdowns of energy use at the individual level would constitute
a step forward for this research.

\subsection{The energy costs of different transport modes}
The relative energy use of different ways of travelling per unit distance
or time has been of interest
to researchers at least since the 1800s when \citet{tredgold1835practical}
was taking measurements from railway engines to ascertain their coal
consumption. A more universal approach to energy use in transportation
was taken by \citet{Gabrielli1950}, who characterised the energy performance of
different modes, for given speeds and loads. This model included jet fighters,
helicopters and even a horse, as well as more traditional vehicles such as
cars, bicycles and trains. Although largely unnoticed by the academic
community (it has been cited 11 times according to Google Scholar), this
paper was seminal in its approach to comparing widely varying forms of
transport, and the findings still largely hold today (although efficiency
gains have been made) \citep{yong2005price}. An updated analysis, which uses
a simpler energy performance metric, kilogram-metres per Joule, multiplied
by speed ($kg*m^2/J/s$) applied the method
to a wide range of modern vehicles, confirming the relatively poor energy
performance of cars in comparison with trains and bicycles (\citealp{Radtke2008},
\cref{fgabrielli}). This is a recurring theme in \cref{Chapter5}.

\begin{figure}[htbp]
  \centerline{
    \includegraphics[width = 13 cm]{gabrielli}}
  \caption{Energy performance of different modes, from \citep{Radtke2008}.} %%% could update with methods
  \label{fgabrielli}
\end{figure}

\citet{Gabrielli1950} and their successors made large advances in understandings
of the relative energy costs of widely different transport modes.
It is therefore surprising that methods and findings stemming from this
work are not more frequently used in transport studies.
One limitation of the research area is that it omits indirect energy impacts from
the analysis.
This is problematic because vehicle and infrastructure manufacture obviously
require large amounts of energy: inclusion of direct energy costs only
``might lead to serious faults in estimating environmental impacts of new
infrastructure or modal shift policies'' \citep[p.~23]{Wee2005}.
A pioneering paper that sought to overcome this issue quantified
both the direct and indirect
energy costs per unit kilometre of the main US modes of personal travel shortly
after the 1973 oil shock \citep{Fels1975}.

In hindsight, Fels' research seems to have stood at the beginning of a
research area, dedicated to assessing the wide-boundary energy impacts of
personal travel. Key papers in this area include \citet{Lenzen1999}, who used
updated versions of Fels' early methodology to calculate the total energy and
emissions impacts of the Australian transport system and  \citet{Ramanathan2000}
used a new method (`data envelope analysis') to investigate
the relative energy costs of Indian road and rail transport.
% The analysis was
% novel in that it took into account both freight and passenger transit in
% the analysis, leading to the finding that rail efficiencies have continuously
% improved, while road transport efficiency has plateaued since the late 1980s.
Another group of researchers have researched essentially the same
issue, but with different methodologies and terminologies
(the `well-to-wheels' approach) from the
life cycle analysis (LCA) perspective (e.g.~\citealp{wang2002fuel};
see \cref{sfuelem}).
Because research rooted in LCA tends to be
concerned with emissions rather than energy use per se, it is
of slightly less relevance to this thesis.
Surprisingly, there seems to be
limited overlap between the well-to-wheels approach and the aforementioned
system level energy use studies.
Despite the activity of these research areas, 
there has been limited uptake of system level energy
cost estimates in transport studies overall. Direct emissions and
their climate impacts have received more attention.

\subsection{The climate impacts of transport}
Since 1985, when Professor James Hansen of NASA's Goddard centre testified
to the US congress about the threat posed by climate change, there has
been a growing concern about the issue from all quarters, including the media
\citep{Boykoff2007}. While media insistence on `balance' seems to have actually
led to bias in climate change reporting, providing excessive coverage to contrarian
views \citep{boykoff2004balance}, academia has largely risen to
the challenge in practical terms. A multitude of articles has been written on how to
reduce emissions in everything ranging from catering \citep{gossling2011food}
to the Indian cement industry \citep{kumar2010environmental}.
Acknowledging that transport is responsible for roughly a quarter of emissions,
researchers in the sector have been no exception.
Modelling scenarios of future change proposing new policies
for emissions reductions are now common themes in the transport literature
(see reviews by \citealp{Chapman2007} \citealp{ross2010analysis}).

Without delving further into this large and diverse body of literature,
a few generalised criticisms of it can serve to
highlight where improvements can be made. It is acknowledged that
these observations do not apply to all research into
transport and climate change. The reason for voicing these concerns,
summarised in the bullet points below, is that they
help focus attention on areas within the field lacking in
coverage.
% Their consideration has contributed to the energy approach
% to commuting presented in this thesis in the following ways:
\begin{itemize}
 \item Transport and emissions studies have tended to focus exclusively on
 direct emissions, to the detriment of understanding of the system level or
 `embedded' emissions resulting from transport policies,
 such as road construction and vehicle
 manufacture \citep{Lenzen1999, Wee2005}.
 \item Because of the focus on the national level, papers in the area
 could be argued as offering little in the way of support to local and regional transport
 planners. This is an important oversight because local and regional level
 transport planners vastly outnumber national policy makers (in staff, if not
 in terms of political influence).
 \item The various scenarios of the future often appear to be overly academic,
 arbitrary and unrealistic. This is
 problematic because impenetrable models and scenarios
 may prevent engagement and interaction with the
 possible futures presented, by either the public at large or policy makers.
 To overcome this issue, participatory models
 such as that published online by the Department of Energy and Climate Change
 (\href{http://2050-calculator-tool.decc.gov.uk/pathways/11111111111111111111111111111111111111111111111111111/primary_energy_chart}
 {\color{blue} 2050-calculator-tool.decc.gov.uk}) have been advocated
 \citep{fulton2012exploring}.
\end{itemize}

Despite these issues, this thesis fits within the field: although the emissions
benefits are not calculated explicitly, it is not a large jump from energy
costs to emissions (CO$_{2eq}$ output
would be easy to estimate, based on the
emissions factors present in \cref{Chapter5}). The efforts to estimate
system level energy costs of different modes presented in the same chapter
are aimed at overcoming the focus on direct emissions alone, prevalent in the
transport-climate change literature. Regarding scale, in some ways it makes
sense that many of the studies in the area operate at a large scale
because climate change is inherently a global issue.
The problem is that there is an excess of studies that operate only at the
national level, with relatively little work focussing on larger or smaller geographical
unit of analysis. The methods
presented in this thesis are well-suited to smaller geographical unit areas, although
they can also be applied to nations (\cref{Chapter6}).
The methods presented in this thesis are not participatory (unless one is
willing to learn to code in R and apply it to spatial microsimulation!).
However, effort has been made to make the code and data underlying the
models as accessible as possible.\footnote{See
\url{http://rpubs.com/robinlovelace}, which contains links to
reproducible result, via sample code and data. Github has also
been used to make some experimental analyses available.
}

% \subsection{Assessing the energy impacts of political intervention} \label{spintervention}
% !!!

\section{The energy impacts of commuting} \label{sdisciplines}
The intersection between these two study areas, each large in its own
right and with substantial interaction, is surprisingly small.
As described in the previous two sections, major advances in understanding
commuting behaviour and energy use in transport have been made.
The problem is that these insights into commuting are often not
translated into energy use
estimates.\footnote{This step is in fact
relatively straightforward, once the energy use of different modes is well-known
(\cref{Chapter5}).
}
Or, conversely, existing estimates of energy use of different modes and
other personal variables are not combined with readily available
commuting statistics. The energy cost of commuting is not a
`pure' research area, in the sense that it relies on combining data from
sources that often are not linked.

The study that most closely fits the title of this section
was based on aggregated census data from Flanders. Without relying on
regression analysis  or sophisticated statistics \citet{Boussauw2009}
provided a detailed account of the factors linked to areas with
high and low average commuter energy costs. By mapping average
energy consumption per person per day (ranging from almost zero
to above 30 kWh/p/d) for small administrative zones, the impacts of
modal split (minimal), distance (``paramount'') and urban morphology and
infrastructure on energy use for commuting were determined. These are new
and important findings that need to be tested in other countries
and at different scales
before they are accepted as `universal' relationships that can form the basis of
policies worldwide. It was concluded that
``the energy performance of the transport system is an important approximate
indicator for the sustainability of a spatial structure'' \citep[590]{Boussauw2009}.
This observation was a major motivation for the subject matter of this thesis.
%!!! requote this elsewhere???
The political implications of the research are wide-ranging: the prevailing
focus on mode-split in Belgium (and in many other countries, including the
UK where uptake of cycling has become a major political issue) seems to be misguided.
Governments should instead focus on enabling their citizens to live closer to their
place of work.

\citet{Boussauw2009} did not provide `further research' type conclusions. However,
the arguments made throughout for a greater role for energy-based metrics of
transport system performance and sustainability clearly imply that more
research measuring energy use in commuting is needed. The paper therefore
provides a strong intellectual foundation on which this thesis is built.
The methodological guidance was limited as the analysis was quite simple.
From this was taken the importance of seeing method as a means to an end,
rather than an end in itself, an issue that has been debated in academia for
many years. 
%% surely there's another study that fits here???!!! looking good though.

% \subsection{Energy use and transport poverty} 
% One of the earliest investigations into the
% links between energy intensive modes and social and economic disadvantage
% was Ivan Illich's book \emph{Energy and Equity}
% \citep{Illich1974}. Since then, the causes identified by Illich for growing
% inequalities in personal mobilities and `power' --- the ever increasing speed
% with which the economic elite can travel, powered by fast cars --- has grown
% and plateaued. % would like to re-add this, at some point!!!
% 
% \citep{Sustrans2012}. %

% Commuting is an inherently spatial activity, as its purpose is % valuable!!!
% to transport people from A, their home to B, work.
% This, combined with readily available official data sources
% on commuter flows, make the phenomenon well suited for study
% within the field of transport geography. This field aims
% to explain ``the socioeconomic, industrial
% and settlement frameworks within which transport
% networks develop and transport systems operate''
% \citep{Hoyle1992modern}. %%% Citation added.
% The focus on explanation rather than mere description
% makes the field a rich source of ideas about \emph{why} transport systems,
% and more specifically commuter patterns, are the way the are.
% Plough on with examples from the literature

% Commute minimisation \citep{Buliung2002}
% 
% The concept of `activity space' is another key concept
% used by transport geographers... \citep{Buliung2006}.

% Transport geography is inherently multidisciplinary,
% and has contributed to wider debates started in other fields.

% \citep{Marique2013}
% 
% Despite the theoretical bias in Transport Geography noted at the beginning of
% this section, there have been substantial methodological contributions to the
% analysis of commuter patterns. The methods to evaluate different
% types of estimate of route distance (GPS, shortest route algorithms via
% GIS and straight-line distance) presented by \citep{Stigell2011}.
% Their result that self-report distance is the least reliable provides
% useful background for assessing the reliability of the input data
% used in this PhD. (Census data are straight-line distances by
% postcode, the survey data is self-reported route-distance.)

% \subsection{Non-academic contributions to commuting research}
% The above research is based predominantly in the realm of academia, written by
% University scholars and published in academic journals. The advantages of this
% `academic model' of research are clear, and have been implemented throughout
% this thesis. These include:
% \begin{itemize}
%  \item Traceability of sources of information --- hence the inevitable in-text
% references and lengthy reference lists that accompany academic research.
% \item Reproducibility of results --- academic writers must clearly state what
% they have done, why and report the results in a way that could be replicated,
% given the correct conditions.
% \item Impartial writing style --- academic writing is different from more
% creative styles, in that ideas must be ordered logically and artistic elements
% minimised in favour of clarity \citep{oshima1997introduction}.
% \end{itemize}
% 
% Because commuting is such an important part of the daily routine for millions
% of people, it has received much attention from outside academia too.
% Literacy has improved worldwide and
% opportunities to publish (most recently with the emergence of paperless
% E-books) have multiplied, so it is impossible to hope to have encountered even
% a tiny fraction of the totality of material written about
% commuting.\footnote{This
% could be interpreted as an additional advantage of the academic publishing
% model: it is relatively cohesive self referential compared with the tangled
% mass of non-academic writing. The latter is not ordered into neat searchable
% databases that is connected by a formal reference system in the same way that
% academic papers are.}
% What follows therefore is a concise summary of `lay' works on
% commuting that have been discovered, usually by coincidence, and that
% contribute information or understanding to the topic energy costs of commuting.
% 
% \citet{Orloff2002-60-second-commute} provide a self-help guide on telecommuting
% for those wishing save time and money.

While \citet{Boussauw2009} were writing from the perspective of transport
geography, the primary concern being spatial variation of energy costs,
the issue of energy costs has also been tackled from the perspective of
mainstream economics. \citet{Sexton2011} set out to
test a hypothesis: that the 2008 sub-prime mortgage crisis was triggered
by high liquid fuel prices. The mechanism for this was commuting energy costs ---
those who live closer to their place of work were found to be less affected.
This was shown through a number of maps illustrating the change in average
house prices over space. Areas furthest from employment centres had the greatest
falls, whereas house prices in more central locations were relatively unaffected.
This study demonstrates the importance of energy costs of commuting,
not just in abstract terms of environmental impact or global resource depletion,
but in terms of direct impacts on peoples' lives. No attempt is made to
replicate the economic methods used by \citet{Sexton2011} in this thesis.
However, \cref{svul} was heavily influenced by the paper. It takes from
\citet{Sexton2011} the need to assess potential future impacts of high oil prices
on different social groups.

\section{Commuting and energy use research: tools of the trade}
\label{s:tools}
The previous section illustrates that energy use in commuting can be seen in 
at least two different ways: a dependent variable influenced by geography, or
an explanatory variable affecting household expenditure.
Many other ways of looking at commuter energy use are possible and each
would suit different methods for describing and explaining
energy use. While research methods
and explanations can be closely bound
together,\footnote{\citet{Simini2012}, for
example, harness a vast commuter dataset covering the USA to support their
general numerical model of commuting: the model to a large extent
contains explanation implicitly.
}
different research methodologies
can also be used to investigate the problem from a single perspective.
For this reason the methods discussed below
are considered separately from the other sections of this literature review.
Theories are hypotheses about how the world \emph{should be}, based on
experience, concepts and intuition, while
the methods help uncover facts about how the world \emph{is}. This is the
standard model of science, which progresses by falsifying ideas which fail to
explain observed reality, and leads to the acceptance of systems that have most
explanatory power \citep{Popper1959}.

In some ways, this scientific approach
can be seen as a tool of the trade in itself: it provides a framework within
which competing theories can be impartially compared, and provides a mechanism
to discard ineffective explanations, `sorting the wheat from the chaff' in terms of
ideas about the world. For this reason the scientific method, as it has been
intermittently applied to research into commuting, is discussed as the primary,
and most broadly defined, tool of the trade.
% Spatial statistics, %(which have
% %only emerged since large spatial datasets became available during the 20)
% travel diaries, interviews,
Visualisation techniques have progressed alongside advances in data availability
and analysis are considered as a key method in the research area.
Finally, the `data deluge' precipitated by the
widespread adoption of handheld GPS devices and traffic monitoring technology
is briefly considered. This source of information may, one day,
rival official commuting statistics as a dataset from which to understand the
energy costs of work travel.

\subsection{`Scientific' approaches to energy and transport}
Science is a contested concept but has undoubtedly had a large impact on
methods of researching energy use in transport. Rather than be restricted
to Popper's narrow definition of science (as any knowledge that can produce
falsifiable hypotheses), the literature is more usefully seen as falling into a continuum,
ranging from ``scientific'' on the one side, to ``not scientific'' on the
other. This is not to make a value judgement about which research is `better'.
(Indeed, one could argue that commuting is not a research area
that is amenable to true science at all, due to the complexity of human decision
making and the impossibility of controlled experiments.) It is simply
to say that some methodological approaches borrow more heavily from the
formalisation of theory and emphasis on quantification and testability of
science than others. 

A well-established `scientific' theory about commuter patterns is the gravity
law. The law is falsifiable (and has been falsified on numerous occasions!)
because it predicts the number of trips ($T$) from location $i$ to location $j$
using the following formula:
\begin{equation}
 T_{ij} = \frac{m_{i}^{\alpha} n_{j} ^{\beta}} {f(r_{ij})}
\label{eq:gravl}
\end{equation}
where  $m_i$ and $n_j$ are the populations of the start and
destination settlements respectively, $r$ is the Euclidean or `straight line'
distance of the
journey, and $\alpha$ and $\beta$ are parameters to be calculated based on
evidence. The functional form of the denominator is open to interpretation,
making the gravity law more of a modelling framework. Proponents have
claimed that the
framework can predict commuter flows between two settlements, once the
functional form of \cref{eq:gravl} has been learnt.

This is quite a sweeping statement. Clearly, the model cannot be correct all
the time because it is deterministic. It can, however, produce a
sufficiently close fit with reality, across a number of transport flows, that it
has become ``the prevailing framework with which to predict population movement,
cargo shipping volume and inter-city phone calls, as well as bilateral trade
flows between nations'' \citep{Simini2012}. The gravity law has been applied to
commuting on a number of occasions with results pertinent to energy use.
\citet{gargiulo2012} presented a spatial interaction model based on the
gravity law. It was configured using a single parameter ($\beta$ in \cref{eq:gravl}),
and was used to calculate the probability of individuals travelling
from their home to workplace zones. Although no energy implications were investigated
by \citet{gargiulo2012}, the model could be used to
predict energy costs via trip counts between different zones.
In a related paper, \citet{Lenormandplosone2012} presented results of
a model that calculates commuter flows between zones about which the number
of incoming and outgoing commuters is already known. From this input dataset
could be estimated the flow between each zone pair, to a high degree of accuracy.
The authors tested their results against the radiation model and
found that theirs ``yields significantly better results''
\citep[p.~6]{Lenormandplosone2012}. It is to
this radiation model, another scientific approach to commuting, 
that attention is directed below.

The gravity law has been recently criticised by
\citet{Simini2012}, who proposed an alternative that they refer to as a
`radiation model'. In this model, the flow rate between two zones is defined
probabilistically. The average flux is estimated as follows:
\begin{equation}
\langle T_{ij} \rangle = T_i \frac{m_{i} n_{j}} {(m_i + s_{ij})(m_i + n_j + s_{ij}) }
\label{eq:radi}
\end{equation}
where $s_{ij}$ is defined as the total population living within a circle, the
centre of which lies in the centroid of zone $i$ and the radius of which is
the distance between zones $i$ and $j$. Thus, the greater the population
living within the commute distance, the lower the estimated flow rate.
This is key to the radiation model: it accounts not only for the characteristics
of the origin and destination zones, but also the surroundings.
Not only does this model have strong theoretical underpinnings, it also
performed well against commuting data from US counties: the flow between
each county pair was predicted with a high level of accuracy, based solely
on the population of each. The potential utility of this model in
energy applications is considerable: it is highly flexible so could be used in its
raw state, before adding refinements to explain the impact of infrastructure.
Also, the concept of impedance (introduced towards the end of \cref{sactive})
could be used to create modified versions of \cref{eq:radi}
for each commonly used form of transport. With both modifications in place,
such a model should be able to predict the energy implications for commuters of both
new settlements and new infrastructure.

% Another `scientific' model for understanding commuter flows was presented
% by \citet{gargiulo2012}. In this modified version of the gravity law, with a single
% parameter ($\beta$ in \cref{eq:gravl}), each commuter is allocated a workplace
% probabilistically, using randomised sampling. The model is slightly less
% simple than that presented by \citet{Simini2012}, but could be used in the same
% way as a predictor of energy costs via modal split and distance.
% The stochastic interpretation of the gravity law has also been used by
% \citet{Lenormand2011} to simulate commuter flows.
% There are many other spatial interaction models based on the gravity law to
% predict flow rates. A challenge is knowing which one is most suitable for commuting,
% suggesting a need for cross-comparisons. My hypothesis would be that the new radiation
% model provides the best starting point, and this could be tested by UK commuter
% flow data. 

Another area where the mathematical formalisation of theory has been useful in
energy-transport research is in the creation of future scenarios.
\citet{Kohler2009} used an agent-based model to create scenarios of behavioural
change and uptake of new transport technologies between the years 2000 and
2050. The novelty introduced by their model was use of different `agents' ---
people (`consumers') interacting with higher level `niches' and `regimes' to
determine the final outcome. The modelling framework is flexible, and allowed
for complex dynamic behaviour to be simulated. A downside of the model was
that it depended heavily on user input to set initial parameters. These
parameters were set in a
``scenario storyline of a successful transition'' \citep[p.~2988]{Kohler2009},
in which hydrogen fuel
cell cars become widely available by the 2040s. Clearly, this scenario of the
future is more the product of human imagination than the scientific method,
and the future may take an entirely different technological path than that
imposed by the authors.
However, the sophistication of the approach shows that scenario creation
can go beyond simple population models \citep{Lovelace2011-assessing} or
user-defined snapshots of the future \citep{Akerman2006}.

\subsection{Visualisation methods}
People tend to think visually and often lack the concentration or ability
to read through long verbal descriptions or understand mathematical formulae.
For this reason visualisation is important:
``A picture really can be worth a thousand words, and human beings are very adept
at extracting useful information from visual presentations'' \citep[p.~4]{kabacoff2011r}.
A list of some of the main visualisation techniques for representing
is therefore timely at the outset, to provide context and justification
for the use of figures in this thesis:
\begin{itemize}
 \item Choropleth maps are very common in geographical commuting research,
 providing an insight into the areas where particular behaviours are most
 prevalent. A minor difference between the maps used in most previous
 research and this is the use of continuous colour scales in this thesis,
 instead of bins for communicating energy costs (see \cref{Chapter6}).
 This can be problematic if a distribution is highly
 distorted by outliers, in which case bins would be preferable, but can provide
 additional information to the reader if neighbouring zones have values at the
 opposite ends of a single colour bin.
 \item Geographical flow maps, with thickness of lines joining origin-destination
 pairs proportional to the flow (e.g.~\citealp{Smith2009}).
 This technique is employed in \cref{s:workdes} to illustrate the important of
 knowing \emph{where} commuters are travelling to for local transport decisions
 that consider commuter energy use. Often these maps lack direction, however,
 leading to the use of arrows or asymmetries in lines being added
 (e.g.~\citep{Nielsen2008})
 \item On-line visualisations have become increasingly common as software such
 as Processing, OpenLayers (for maps) and an R package called Shiny have become
 increasingly available and user friendly. Although no on-line visualisations
 have been created for the main thesis, `Google Fusion Tables' and `Geoserver'
 options were considered to make the results more
 accessible.\footnote{A presentation
 on this topic was given by the author at the FOSS4G (Free Open Source
 Software for Geospatial) annual conference 2013.
 The slides can be viewed
 {\color{blue} \href{http://robinlovelace.net/visualisation/open\%20source/conferences/presentation/2013/09/22/foss4g-presentation.html}
 {online}}.
 }
\end{itemize}

% \subsection{Travel diaries}
% 
% \subsection{Interviews and participant observation}

\subsection{Harnessing the `data deluge'}
The increasing market penetration of hand-held GPS devices, in dedicated
packages \citep{Oliver2010} and more recently embedded within `smartphones'
\citep{Gong2011}, has lead to an `overabundance' of spatial data which must be
filtered, prioritised, ordered, sorted and analysed to provide meaningful
results.\footnote{%%%%%%%%%%%%%%%%%%%%%%%%%%%%%%%%%%%%%%%%%%%%%%%%%%%%%%%%
%%%%%%%%%%%%%%%%%%%%
This was the topic of
the Sixth International Workshop on ``Geographical Analysis,
Urban Modeling, Spatial Statistics'', held in Salvador de Bahia, Brazil,
June 2012. The problem neatly summarised on the conference's web-page:
``During the past decades the main problem in geographical
analysis was the lack of spatial data availability. Nowadays the wide diffusion
of electronic devices containing geo-referenced information generates a great
production of spatial data. Volunteered geographic information activities (e.g.
Wikimapia, OpenStreetMap), public initiatives (e.g. Spatial Data
Infrastructures, Geo-portals) and private projects (e.g. Google Earth, Microsoft
Virtual Earth, etc.) produced an overabundance of spatial data, which, in many
cases, does not help the efficiency of decision
processes''
(\url{http://www.unibas.it/utenti/murgante/geog_an_mod_11/index.html}, accessed
February 2012).
%%%%%%%%%%%%%%%%%%%%%%%%%%%%%%%%%%%%%%%%%%%%%%%%%%%%%%%%%%%%%%%%%%%%%%%%%%%%%%%%
}
This `data deluge' is still in its early stages \citep{Bell2009}, yet is
already having an effect on approaches to geospatial data analysis
\citep{Jiang2011}. The data analysed come from more
conventional sources (primarily the Census and official surveys). However, it is
important to be aware of the potential for this research to contribute to
knowledge about commuter energy use.

% Studies using GPS tracking devices worn by study participants can measure
% location, velocity, and route planning (). This research is still in its
% infancy yet has already shed light on commuting (get to the point - lit review
% to the max!)

\section{Concepts in energy and commuting} \label{skeyconcepts}
% A wide range of research has been presented in this chapter, and some of it
% may seem unrelated under first impressions.
The diversity of research on energy and commuting is great, yet within this
body of work lies a set of concepts that appear repeatedly. The purpose of
this short section is to summarise some of these ideas
and to help tie together the literature reviewed in this chapter.
The first two will act as points of reference in later sections.
% Summaries of
% these concepts will be quantified where possible, as an explanation of the
% assumptions made in future chapters, and to provide material for the discussion
% of the results.
% !!! Provide summary table at end with range ???
\begin{itemize}
 \item \emph{Circuity ($Q$)}: This is the ratio of network distance to Euclidean
distance
between two places \citep{Levinson2009}:
\begin{equation}
 Q(i,j) = \frac{dE(i,j)}{dR(i,j)}
\end{equation}
Circuity is important due to its impact on energy use \citep{Levinson2012}
and because other metrics of the transport network's performance can be
derived from it \citep{Barthelemy2011}.
Circuity impacts energy use because in highly circuitous
networks, more energy must be expended to go the same distance. In addition,
if circuity is low for energy intensive modes (e.g. the route
between settlements joined by a motorway), these modes will be preferred.

Circuity is also important practically:
the distance bins used to disseminate UK census data measure Euclidean distances,
whereas the actual distance travelled depends on network distance: to
calculate energy use, the circuity factor $Q$, must be used to translate
between the two. The second reason for circuity's importance
is that other useful metrics of transport system performance can be derived from
it. These  include the \emph{accessibility} of a location
 (how circuitous is the average route to that place), and the
\emph{global efficiency} of the network. These additional concepts
which grew out of the understanding of circuity have strict mathematical
definitions and could be used to quantify the impact of network
structure on scenarios of the future, including the likely resilience of
different parts of the travel network under scenarios of natural disaster
\citet{Barthelemy2011}. This is a research area with great potential for
the future. In this thesis, however, circuity is the only quantitative
description of the transport network to be implemented: in \ref{scircuity}
circuity is described as a mechanism to map the Euclidean
distances reported in the census to the route distances reported in survey data.

\item \emph{Efficiency ($\eta$)}: Efficiency is an important concept in
transport and energy studies. As with its everyday use, often its meaning
is not strictly defined in the transport literature. ``This is not an
efficient use of time'' is a typical use of the term, meaning that the
benefits (outputs) are low considering the time input. 

Regarding energy use, the meaning is the same, although the mathematical
definition allows for precision:
\begin{equation}
 \eta = \frac{E_{out}}{E_{in}}
\end{equation}
Where $E_{out}$ is energy that is useful (e.g. electricity), and $E_{in}$ is
the primary energy input (e.g.\ calorific content of petrol). Of course, the
definition of `useful' is open to interpretation\citep{Patterson1996}, leading
to various measures of efficiency, ranging from pure thermodynamic definitions
\footnote{The efficiency of electricity production, for example.}
through to
economic-thermodynamic definitions\footnote{For example, the efficiency of
freight transport can be defined as tonne-kilometres per unit energy input
(tkm/MJ) \citep{Simongati2010}. This hybrid economic-thermodynamic measure is
more commonly expressed as fuel economy of freight, its reciprocal
(MJ/tkm).},
to purely economic definitions\footnote{This is measured as the proportion of
an activity's monetary cost that is spent on energy --- the proportion of bus a
bus fare that goes towards diesel costs, for example.}. The concept of
efficiency --- and related concepts of fuel economy and energy
intensity --- is well established in research on the energy requirements
of freight transport \citep{Kamakate2009}. It has rarely been used to compare
the performance of different transport modes, however \citep{Fels1975,
Lovelace2011-assessing}.

A general principal of energy efficiency measures is that they should reflect
the purpose of the process they describe \citep{Patterson1996}. In commuting,
the transport of \emph{people} is the aim, so the commonly used fuel economy
metric (l/100 km) is not an appropriate measure of the  performance of the
system \citep{MacKay2009}. The preferred energy metric for this research is
therefore energy intensity:
\begin{equation}
 EI = \frac{MJ}{pkm}
\end{equation}
The energy intensity of passenger transport modes are described
(after a large body of evidence on the matter is considered) in \cref{sfinal}.
In everyday speak when transport modes are described as `efficient' people
are generally referring to energy intensity rather than thermodynamic
efficiency. Following this convention, `efficiency' when used in this thesis
also generally refers to energy intensity.

In terms of the energy costs of commuting, the preferred metric is the average
energy costs per commuter per two-way commuter trip (MJ/trp).
This is similar to the units of kWh/p/day used by \citet{Boussauw2009},
but the denominator is the number of commuters in this study, not the number
of people (making the results impervious to variable unemployment rates) here.
To translate MJ into kWh, multiply by 3.6.
The energy per trip results are presented in \cref{Chapter6} at a variety of scales.

\item \emph{Resilience}: this is measure of a system's capacity to function
after enduring external shocks \citep{Holling1973}\footnote{The seminal
definition of resilience is that it is ``a measure of the persistence
of systems and of their ability to absorb change and disturbances'', while
maintaining their functionality \citep[p. 14]{Holling1973}.
}.
Despite its origins in Ecology, the concept is applicable to any complex
system, and is especially relevant to the relationships between the
economy and the natural environment \citep{Holling2001}. In the sustainability
literature, the term is rarely quantified (see \citealp{Bridge2010}). However,
there has been progress in defining resilience mathematically for 
networks, which could theoretically be used to calculate the impacts of
large collapses, such as blackouts, or, by corollary, failure of the transport
network \citep{Barthelemy2011}. At present however, this quantitative branch of
the resilience concept lacks empirical application. The term is harnessed to
discuss the long term sustainability of commuter systems and their capacity to
function in the event of oil shortages.

\item \emph{Inertia}: in its original physical definition, inertia is the
characteristic of mass by which it ``endeavours to preserve [itself] in its
present state, whether it be of rest or of moving uniformly forward in a
straight line'' \citep[p. 73]{Newton1848}. In the context of transport systems,
inertia is used to describe `lock-in' to the current
transport system in the short term, and its resistance to change:
``Transport systems and urban lay-outs have great inertia and take years to
change'' \citep[p. 365]{Chapman2007}.
\end{itemize}

\section{Summary of the literature} \label{sc2sum}
This chapter has highlighted the range of methodologies and disciplinary
diversity of studies investigating the energy costs and
greenhouse gas emissions of personal travel.
The sustainable mobility paradigm provides a useful label that can be applied to
much of this research, differentiating it from the traditional supply-side
approach bemoaned in the opening quote. The majority of the literature in
transport and energy is not concerned with such high level discussion, however,
generally preferring to `let the facts speak for themselves'. The area of
study is quite new (except for a flurry of work following the
1970s oil shocks, exemplified by \citet{Fels1975}), perhaps explaining why
geographical studies into energy use for transport are still
largely descriptive (e.g.~\citealp{Marique2013, Boussauw2009}), content to
explain spatial variability intuitively rather than with the use of a
predictive model. This thesis takes a similar approach and is primarily
concerned with \emph{describing} the variability of commuter energy costs
at geographic and individual levels. This appears not to have been done
before in the UK.

Transport and energy use has been investigated from a wide range of disciplinary
perspectives, from psychology and economics through to engineering and physics.
This is because energy use depends not only on the efficiency of transport
technologies, but also the behavioural factors that determine how they are used.
Following this diversity, the research presented in this thesis is also
explicitly multi-disciplinary: claiming allegiance to any one discipline
would likely be at the expense of another, potentially 
hindering understanding of the complexity of factors at work.

The energy costs of transport, and their underlying causes, have been explored at a
range of different scales. Individual factors including family
and career commitments have an important role to play, but whether or not
these can be modelled using quantitative data from surveys remains to be seen.
At the regional level, geographical factors influencing energy use in transport
have been explored with reference to the `compact city' hypothesis. CO$_2$
emissions and energy studies have tended to operate at large national or
regional levels, despite the fact that most transport planners and other decision
makers implement policies (especially in the realm of active travel) at the
local level. This suggests a gap in the literature and highlights the need
for energy and transport studies focussed more locally. Moreover,
because the factors affecting commuting behaviour operate at many levels,
there is a need for further development of methods that allow factors operating
at individual and geographical levels to be taken into account simultaneously.

% and have focussed
% on description and scenario modelling rather than explanation.
% Despite the fact that most   A research gap





% \section{Conclusions: knowledge gaps and research directions}


%  \citep{Ballas2005}
 % Literature review

% Chapter 3

\chapter{Spatial microsimulation and its application to transport problems}
\label{Chapter3}
% \lhead{Chapter 3. \emph{Spatial microsimulation and % Definitely worth writing
% % as a stand-alone paper
% its application to transport problems}} % Write in your own chapter title to set
\fancyhead[RO,LE]{Chapter 3. Spatial microsimulation} %2side
\fancyhead[RE,LO]{\thepage}

\begin{quote}
\textit{The modellers' task is to predict how people and organisations will live
in
`good' and `sustainable' cities; how the infrastructure will, or should, grow;
and how activities and traffic flows are, where appropriate, best managed,
priced and regulated.}
\flushright{\citep[p.~3]{Wilson1998-past}}
\end{quote}

Microsimulation can have variable meanings depending on whether you are a
geographer, transport planner, or economist
(see \citealp{Ballas2005b, Liu2006, Bourguignon2006} for examples).
This chapter reviews existing work that uses individual level data and modelling
techniques to investigate transport and related problems.
It also introduces static spatial microsimulation, a particular type of
microsimulation that is central to the thesis. The method enables
individual level and geographical variation in commuting behaviour
to be analysed in tandem. Operational definitions, based on
established research, are important for clarity, repeatability and to
show how the work presented here builds on past research. A number of key terms will be
frequently used throughout the thesis, so this chapter begins with
definitions.
This is followed by an overview of the history (\cref{s:history}) and current
state of the art (\cref{s:sotart}) of the technique as it relates to transport
issues such as travel to work.

As implied in the quotation above, transport does not happen in isolation from
other phenomena. It is part of the complex web of social relations, the environment,
infrastructures, economics, policies and decisions
that define modern settlements. From this
perspective, spatial microsimulation for transport applications is just one
branch of a long-standing tradition of urban modelling
\citep{Wilson1970, batty1976urban, batty2007cities}.
Other branches include dedicated transport modelling techniques
(e.g.~\citealp{SATURN2012}), integrated land-use transport  models
\citep{Wegener2009} and agent-based approaches \citep{Gilbert2008-abm}.
These research areas are related to the thesis and in some cases have
the potential to build on its results. In this chapter they are
are grouped together under the broad term `urban modelling' and
discussed in \cref{s:urbanmodel}. The final section of
this chapter (\cref{s:bigdata-gps}) summarises the literature and
explains how it relates to methods implemented in the thesis.

% considers new research directions that have
% been made possible by  the availability of `big data' harvested from the
% internet or alternative sources such as GPS loans to survey participants.
% New approaches are already
% making substantial contributions to understandings of personal travel.
% Harnessed correctly, these new data sources have the potential to improve the
% energy use estimates resulting from microsimulation models and shed insight into
% the validity of assumptions made. %%%!!! Not any more!

\section{Definitions: what is spatial microsimulation?}
\label{s:defs}
\emph{Microsimulation}, as its name suggests, refers to the modelling of
individual units --- e.g.~people, household, companies --- which operate in a
wider system. Used in this sense, the term originates in economics, where it
signified a theoretical turn away from aggregate level analyses and towards a
focus on individual behaviour. ``This shift of focus, from sectors of the
economy to the individual decision making units is the basis of all
microsimulation work that has followed from Orcutt's work''
(\citealp[p.~145]{Holm1987}; see \cref{s:digirev} for further reference to
this work). Microsimulation overall therefore has a wide meaning, from
individual vehicles in a transport model \citep{Liu2006, Ferguson2012}
to the inventories of
competing firms over time \citep{Bergmann1990a}.
The term has a narrower definition in this
thesis, however, that is more concerned with modelling the distribution of
behaviours of individuals over space than over time. This thesis is predominantly
concerned with only one subset of microsimulation: spatial
microsimulation, modelling the distribution of individuals over
space. Within the category of \emph{spatial} microsimulation, different types
can be specified (\cref{types-msim}).

\emph{Spatial microsimulation} of the static kind can be formally
defined as
follows: the simulation of individual level variables within the geographic
zones under investigation \citep{ballas2003microsimulation-30-years, Ballas2007simb}.
The models that perform this operation have also
been referred to as `population synthesizers' \citep{Mohammadian2010}. This
term is useful in the context of transport applications, because small area
micro-population generation is only one stage of a wider process of
individual level transport modelling (\citealp{Pritchard2012}; \cref{f:msim-schematic}).

\begin{figure}[h]
 \centering
 \includegraphics[width=16 cm]{msim-schematic}
 \caption[Schematic a transport simulation model]{Schematic
of the components of a complete transport simulation model such as
TRANSIMS, after \citet{nagel1999transims} and \citet{Mohammadian2010}.
This thesis is primarily concerned with the first two stages.}
 \label{f:msim-schematic}
\end{figure}

During static spatial microsimulation individuals are sampled from a non-geographical
dataset via  reweighting, based on what have become known as `constraint variables'
from early combinational optimisation work \citep{Williamson1998}. The key
feature of these variables is that they are present in
both individual level and geographically aggregated data
sets.\footnote{Constraint
variables must be categorical variables (such as
`male', `age: 16 to 19' or `works 0 to 2 km away from home')
that are shared between the micro level data and
known geographical aggregates, usually from the census.
Continuous variables have not been used in the microsimualtion
literature reviewed, although they could theoretically be
used, by constraining variables' spread, skewness
and central tendency.
}
% The next 2 paras are taken from ints3.tex
\Cref{f:msim-schematic} shows the technique in the wider context of
transport modelling. Spatial microsimulation
here refers to only the top two stages in the diagram. It represents a
computationally small but important (for social analysis at least)
part of the wider simulation process. It is important to clarify this
distinction, as the meaning of `spatial microsimulation' can be ambiguous.
It can refer
either to the process of population synthesis
\citep{chin2006regional, Ballas2005c, Hynes2008},
or the entire urban modelling process that
builds on the spatial microdata \citep{Wegener2011}.
Spatial microsimulation here refers to the former case. The results could
thus be harnessed as inputs into more complex dynamic models in which
individuals interact with each other and other entities in a wider urban model.
The terms \emph{dynamic spatial microsimulation} or \emph{agent-based models}
will be used to refer to the wider modelling process.

Static spatial microsimulation (generally and henceforth referred to simply as
spatial microsimulation) involves sampling rows of
survey data (one row per individual, household, or company) to generate lists of
individuals (or weights) for geographic zones that expand the survey to the
population of each geographic zone considered. The
problem that it overcomes is that most publicly available
census datasets are aggregated, whereas individual level data are generally
much more detailed \citep{ballas2003microsimulation-30-years}.
The ecological fallacy, whereby relationships found at one level are 
incorrectly assumed to apply to all others \citep{Openshaw1983}, for example, can be tackled
to some extent using individual level data allocated to geographical zones
\citep{Hermes2012a}. This `spatial' or `small area' microdata is the output
of spatial microsimultion.

Despite its ability to output geolocated individuals,
spatial microsimulation should never be seen as a
replacement for the `gold standard' of real,
small area microdata \citep[p.~4]{Martin2002}. From the perspective of social
scientists, it would be preferable for governments around the world to follow
Sweden's example and release such small area microdata anonymously. However, this
prospect is unlikely to materialise in the UK in the short term,
adding importance to the process
of model validation. In any case, the experience of spatial microsimulation
development and testing can help prepare researchers for the analysis of real
spatial microdata. Also, the technique's links to modelling make spatial
microsimulation useful for investigating the impacts of policy or other
changes in the real spatial microdata \citep{Holm1987}.
The method's practical usefulness (see \citealp{Tomintz2008})
and testability \citep{Edwards2009} are beyond doubt.

Assuming that the survey microdataset is representative of the
individuals living in the zones under investigation,\footnote{The suitability
of this assumption is further discussed in \cref{Chapter8}.} the
challenge can be reduced to that of optimising the fit between
the aggregated results of simulated
spatial microdata and aggregated census variables such as age
and sex \citep{Williamson1998}. These variables are often
referred to as `constraint variables' or `small area constraints'
\citep{Hermes2012a}. The term `linking variables' can also be used, as they
\emph{link} aggregate and survey data.
Based on the literature, the technique seems to have been used for five main
purposes, to:
\begin{itemize}
 \item model variables whose spatial distribution at the aggregate level is
otherwise unknown (e.g.~\citealp{Ballas1999}).
\item estimate the individual level distributions of variables within small
areas about which only aggregate counts or summary statistics are known (e.g.
distance travelled to work)
\item understand the spatial distribution of discrete behaviours
(such as visiting `stop smoking' centres --- \citealp{Tomintz2008})
and thus the likely local level effects of policy change \citep{Ballas2001}
\item project future changes at the local level, based on past trends
\citep{Ballas2005}
\item provide a foundation for agent-based models, which rely on
discrete individuals \citep{Ballas2007simb, Pritchard2012, Wu2010}
\end{itemize}
The main purposes of spatial microsimulation here are
related to bullet points one
and two above. However, elements from each will be harnessed at some
point.
In essence, spatial microsimulation merges individual level data (a list
of individuals, each with their own ID) with geographical data (a list of
zones, each with its own ID). It therefore relies on two types of input data:

The \emph{microdataset} is the individual level data from which individuals are
weighted or probabilistically selected. It is referred to as the survey
dataset \citep{Wu2008} or simply as `individual data' \citep{Simpson2005}.
The input microdata should be as representative of the zones being studied as
possible\footnote{For
example, the date of survey data collection should be
close to date of at which the zonal data was collected. Also, the survey data
should preferably be from the same geographic region as the zones under
investigation, or at least weighted so that individuals from the region under
investigation are more likely to be sampled \citep{Ballas2005-ireland}. An alternative
way of making the survey dataset more representative is to preferentially
sample individuals from areas with the same classification as the 
their zone being modelled.
}
and sufficiently diverse. %%%!!! More here!

The \emph{constraint variables}, `small area constraints' or `linking
variables' are the aggregate level variables that link the zonal and individual
datasets together. They must (for current methods, at least)
be categorical and the categories in the two
datasets must be the same (re-categorisations may be needed).

\emph{Target variables} are the variables that spatial microsimulation seeks
to estimate. Typically they are not reported at all at the small area level
(e.g.~income), leading to the term `small area estimation' being used
to describe spatial microsimulation when it is used to estimate the
average values of unreported variables for small areas. But spatial
microsimulation can also be used to simulate the distribution of variables that
are already known. Thus, although distance is a constraint variable in
our model, it is also in some ways a target variable: 
little is likely to be known about its distribution within each distance bin. 
Finally, counts of interaction
variables (e.g. male, over 50,
high social class and car driver) are typically not reported from the Census.
These can therefore also be referred to as target variables. Overall,
target variables is the term given to the information targeted for
estimation by the spatial microsimulation model.

\emph{Reweighting} is the process by which individuals are assigned a weight
for each of the zones under investigation. \citet{harland2012} provide
an overview of the methods available for this process, which is
also known as `population synthesis'. The higher the weight for a
particular area, the more representative is the individual of that area,
compared with the rest of the survey dataset. Combinational optimisation
and deterministic reweighting 
are the two main methods for reweighting \citep{Hermes2012a}.

\emph{Combinatorial optimisation} \index{combinatorial optimisation} is an
approach to reweighting that uses repeated randomised sampling to
repeatedly select individuals from the survey microdataset and allocate them to
zones \citep{Williamson1998, Voas2000}. Based on the fit between simulated
and known aggregate counts after each
iteration, the parameters of the resampling algorithm can be adjusted (e.g.~via
simulated annealing).

\emph{Deterministic reweighting} refers to non-random methods of allocating
weights to individual-zone combinations \citep{Ballas2007simb, Tomintz2008}.
Iterative proportional fitting (IPF)
is a widely used deterministic reweighting algorithm and is used in the
spatial microsimulation model throughout. Whole cases
are generated using integerisation.

\emph{Integerisation} is the process by which integer weights are generated
from the non-integer weight matrix (see \cref{s:integerisation}).

\emph{Cloned individuals} \index{cloning} are rows in the survey microdataset
that have been replicated more than once in the spatial microdataset for a
particular area \citep{Smith2009}.
The cloning of individuals can be represented by an integer
weight above one, or simply by repeating identical rows multiple times. In
practice these two forms of representing data are interchangeable; the latter
takes up more disk space \citep{Holm1996} but may make certain types of analysis
easier.


\section{The history of spatial microsimulation}
\label{s:history}
This section outlines the history of spatial microsimulation. It would
be easy to repeat past work here.\footnote{Readers interested in a
comprehensive history of the field are directed towards
\citet{Ballas2009-sage}.} To avoid this, the focus is on
developments that influence the way spatial microsimulation
is and can be used for transport applications. These include:
\begin{itemize}
 \item the influence of location on individual behaviour via
 transport costs 
 \item the question of data vs theory driven approaches
 \item converting a spatial microdataset into a behavioural model
 \item the impact of rapidly advancing computers and data sources
\end{itemize}
These themes are present throughout the section, which is ordered
roughly chronologically.

\subsection{Pre-computer origins}
The theoretical origins of spatial microsimulation stretch back to before the
turn of the 20$^{th}$ century. It was only with the emergence of large scale
data sets, methods of analysis and and conventions of mathematical notation that
quantitative analysis of variables that vary over time and space could actually
occur \citep{Ballas2009-sage}.
Despite (or perhaps partly because of) the absence of these pre-requisites for
the analysis and simulation of large populations at the individual level, much
progress was made in thinking about how individuals behave within environments
that vary in predictable ways over space before computers were available.
Consideration of travel costs (which were much higher before most
people travelled by motorised modes) was integral to both Christaller's
central place theory and Von Th\"{u}nen's concentric agricultural zones.
Lacking reliable data with which to test their ideas, the
early quantitative geographers had to make do by
developing theories based on personal observation.
Some of these theories are still influential today \citep{Clarke1985}.
Ideas developed in the pre-computer age can be seen as the theoretical
forefathers of the microsimulation models of transport behaviour, and frameworks
for interpreting the results, that are in use today.

One explanation for the greater theoretical focus of pre-computer work
is that empirical data seldom fit into any neat model and therefore
distract from explanation.
This point was made as early as the 1970s, accompanied by the
warning that the
accelerating deluge of new datasets and quantitative methods was leading some
to conflate quantification with theory \citep{Wilson1972-theoretical}.
Much theoretical work has been done since this cautionary tale. Yet the
same problems of being blinded by new information (to the detriment of
deductive thinking) face modellers now, probably to a greater extent.
This, in combination with the fame enjoyed by early theoretical geographers
(as opposed to more recent empirical geographers who modified or
rejected their work), goes a long way to explain why researchers continue to cast back to
the pre-computer age for theoretical
insight. Two of the early theories that are most pertinent to
simulation of travel patterns are Von Th\"{u}nen's, on the
spatial distribution of
agricultural activity and Christaller's central place theory.

Von Th\"{u}nen's work in the early 1800s is a seminal example of this early
theoretical thinking. His model of concentric zones of agriculture was
described verbally and in the evolving language of mathematics but rarely tested
on real data \citep{Moore1895-thesis}.\footnote{For
example, ``although [Von
Th\"{u}nen] claims that his advantage over Ricardo consists in his ability to
reduce the co-operation of capital to terms of labour, the validity of that
claim has not been tested'' \citep[p.~126]{Moore1895-thesis}.
} 
Von Th\"{u}nen's work exerts a strong
influence, even in the 21$^{st}$ century (e.g.~\citealp{lankoski2008bioenergy}) 
due to its use of geographically defined variables, strictly 
defined assumptions and extensibility \citep{sasaki2003agent}. 
The approach describes individual units based in
Cartesian space, that can be seen both as discrete zones, or as
a continuous variable (as an input into the cost of travel)
\citep{Stevens1968a}. The model's insight into the variability of
individual level behaviour depending on their zone of habitation can therefore
be seen as a direct precursor to spatial microsimulation models. These also
seek to describe the characteristics and behaviour of individual units living in
geographical zones.
%%%!!! Please god give me time to make
% A general version of the Von Thunen model and place it on github -> rinseage
% Rinse this section; concise; go back if needed!!!

Walter Christaller's central place theory of the 1930s provided an integrated
theory of spatially variable behaviour (primarily shopping) and the location of
settlements of varying sizes \citep{matthews2008geography}. Based on the
assumption of a continuous and even geographical space ready for urban growth,
the theory proved fertile for hypothesis testing and extension to other sectors.
% based on increasingly complex mathematics .
Following Von Th\"{u}nen, Christaller
attempted a `scientific' explanation of the behaviour of individuals based on
where they live. The mechanistic nature of the approach has since been
superseded by more advanced and probabilistic models
yet central place theory continues to influence
many areas of spatial modelling \citep{Wilson1972-theoretical, Sonis2005,
Farooq2012-integreted}. Applied to commuting, the theory provides a ready made
model about where people travel to work: the settlement that can provide the
best pay, minus travel costs. Of course, both pay and travel costs vary greatly
depending on a number of individual and geographic variables that cannot be known
in every case. However estimates can be made (even in the absence of now
readily available data) and applied stochastically. 
This theoretical approach has subsequently helped
explain spatial distributions in travel to work patterns, using models based on
Christaller's ideas \citep{Tabuchi2006-commuting-costs}. Christaller was
a major advocate of explaining theories in
mathematics: ``the equilibrium of the location system ... can only be
represented by a system of equations'' (Christaller, 1933; quoted in
\citealp[p.~35]{Wilson1972-theoretical}).
More recent research suggests that urban systems are rarely in equilibrium
\citep{batty2007cities}. In any case, Christaller provided a hypothesis about
why some settlements grow more than others, attracting more people, trade and
commuters.
% Whether or not this statement is true,
% and its potential applicability to commuting patterns, is
% discussed in \cref{Chapter8}. % No, it's not!
More prosaically, Christaller's theory also helps explain why
long-distance commuting appears to be more common into large cities than small ones
(see \cref{Chapter6}).
% !!! REallly? + maybe more precisely located cross ref.

The preceding discussion provides only a small snapshot of pre-computational
spatial analysis, based on two influential thinkers. The
focus was on deductive reasoning, rather than inductive methods, whereby large
amounts of data are processed in the hope of finding some underlying pattern.
This emphasis can provide a lesson for the future: despite the clear
disadvantages faced by researchers before the digital revolution, one
advantage they seem to have had was a clear theoretical focus and this may
have been due in part to absence of large and distracting datasets and
computers.
The danger that this historical perspective flags is that the masses of
micro level data now available could distract from explanation. As
\citet{Wilson1972-theoretical} emphasised, it is explanation and theory
development, not mere description, that enables a discipline to progress. 

Despite this risk, the emergence of powerful computers have allowed theories to
be developed and tested in ways that were previously impossible. The digital
revolution can thus be seen as the single most important event in the history
of spatial simulation.

\subsection{The digital revolution}
\label{s:digirev}
\begin{quote}
 At the present time, the speed and capacity
of electronic computers would still put economic limits on the number of units
that could be handled in the above fashion.
\flushright{\citep[p.~120]{Orcutt1957-new-type}}
\end{quote}

After World War II a number of factors drove interest in modelling human
behaviour and transport. Important among these were a couple of influential new
technologies: the mass produced car and electronic computers. The former
expanded rapidly in the West before the oil price shocks of the
1970s, during a sustained period of stability and economic growth. Nowhere
was this more apparent than in the USA, where the rapid uptake of the car was
forcing planners to reconsider city layouts in order to cope with the influx.
Linked to this pressure, the broadly defined art/science of `Urban Modelling'
also
began, originating in the USA \citep{batty1976urban} and continuing to this
day in a paradigm that can be described as the `science of cities' \citep{Batty2012}.

In the early phase of this research program, planning for the future
of cities in a resource-constrained world was a research priority for some,
even before the severity of environmental problems such as climate change was
fully understood \citep{Rouse1975}. The potential of numerical models to tackle
the mismatch between economic development and resource and energy issues was
not overlooked, although models were also used to investigate
how best to accommodate anticipated growth in populations, economies and
car use \citep{Irwin1973-simulation}.
Still, there were calls to harness these newly discovered methods
for consideration of the relative performance of
radically different options from first principles \citep{manheim1968search, TUI1972}.

Beyond changing mobility patterns (the impact of which was largely to provide
motivation, but not method), it was the appearance of computers that drove
forward and facilitated progress in the field. Although many now
take fast and efficient processors for granted, for example by using hand-held computers
to play `Angry Birds' and check Facebook accounts, computers increasingly are used in vital areas of
daily life, from
education to the design of traffic lights. The digital revolution should not
be seen as a single transformative event: it is an ongoing and accelerating
set of changes in the way information is stored, processed and communicated.
Combined with the internet, the digital revolution has ongoing
impacts on society \citep{Rushkoff2011}, including travel to work patterns
\citep{Orloff2003} and of course the methods available to investigate human
behaviour over space.

As with other areas of rapid technological progress, there is no
fixed point at which there is `enough' computing power to solve the
most pressing issues: an interesting phenomenon with computing power
is that, much like the problem of roads driving demand for driving up, the more
there is the more demand grows. Throughout the 20$^{th}$ century computing power
was often seen
as \emph{the} limiting factor preventing accurate simulation of social
systems.\footnote{This
is well illustrated by the quote that begins this section. To
put the quote into its proper context, consider the following: the
IBM 704 had the equivalent of 18,432 bytes of RAM. This was the first mass
produced computer and was
considered as the state of the art at the time of Orcutt's paper:
subsequently in the article it was referred to as a `powerful giant'
\citep{Orcutt1957-new-type}. Now one can purchase a laptop with 16 Gigabytes of
RAM for approximately 5\% of average UK wages (\pounds1,000). This is
1,000,000,000
times more memory than was available to the IBM, operating millions of times
faster and costing thousands of times less in real terms. Yet still people
complain about lack of computing power! In other words, as computing power has
advanced exponentially, approximately by Moore's law --- which accurately
predicts the exponential shrinkage of electronic components, by a factor of 0.7
every 3 years \citep{kish2002end} --- our hunger for more and faster processing
has increased even faster.
}
This is no longer the case: ``Modern computing is now sufficiently powerful to
deal with most [urban] models ... models based on individuals are
now feasible both in terms of their computation and their representation
using new programming languages'' \citep[p.~5]{batty2007cities}.


Regardless of our insatiable thirst for processing power, these external factors
--- the digital revolution and wider societal changes embodied in the
car --- undoubtedly drove forward research seeking to understand and model
transport systems in detail. The aim was to harness the marvel of computing
power to better understand the rapid shifts taking place.
This was most apparent in applied urban modelling: ``Increasing car
ownership during the 1940s and early 1950s led to the growing realisation
that cities with their traditional physical form could simply not cope
with the new mobility'' \citep[p.~6]{batty1976urban}. The new methods formed an
important tool for enabling planners to deal with this shift.
Some of the descendants of this early transport modelling work are described in
\cref{s:dedicated}.
%%% equations!
% after 3.2.3 drafted!

\subsection{Statistical methods for estimation}
In statistics too, more
sophisticated methods were being considered during and after World War II.
Increasingly large and complex
datasets were an additional driver of advancement here: the increased automation
and rigour of data collection led to new data management
problems. Placing his seminal work on iterative proportional fitting (IPF) in
context, \citet[p.~427]{Deming1940} provides the following example of this
data-driven methodological development:
``in the 1940 census of population a problem of adjustment arises from the fact
that although there will be a complete count of certain characters for the
individuals in the population, considerations of efficiency will limit to a
sample many of the cross-tabulations (joint distributions) of these
characters.'' In other words, IPF was developed not to simulate populations but
to fill in empty cells in situations where storing all possible cross-tabulations of
categorical data was not feasible or where internal cells needed to be
updated based on new marginal constraints:
``The iterative proportional fitting method was originally
developed not for fitting
an unsaturated model to a single body of data but for combining the information
from two or more sets of data'' \citep[p.~97]{bishop2007discrete}. To provide a
concrete example of this ``classical'' use of IPF, \citet{bishop2007discrete}
reproduce \citet{Friedlander1961-ipf} who updated cross-tabulations of
counts of women by age and marital status from the complete 1957 table by 1958
margins. More than 50 years later, IPF was still in use, to
tackle the same issue \citep{Jirousek1995}.

\index{entropy maximisation}
Parallel to these developments the concept of `entropy maximisation' emerged.
This method aims to ``produce the maximum-likelihood estimate --- the distribution [of cell
values] that is most likely to occur given no other constraints [on their
marginal totals] than those imposed'' \citep[p.~95]{johnston1985geography}.
Originally proposed and formalised mathematically in the field of statistical
mechanics \citep{jaynes1957information}, the concept was used to estimate
probability distributions that satisfy all conditions without making any
further assumptions about the data. ``Mathematically, the maximum entropy
distribution has the important property that no possibility is ignored; it
assigns positive weight to every situation that is not absolutely excluded by
the given information'' \citep[p.~623]{jaynes1957information}.
This definition is very similar to the maximum likelihood estimate attained
through iterative proportional fitting. The mathematics underlying entropy maximisation
is complex,
% (for a geographer at least!),
involving Lagrangian multipliers and
a series of interrelated equations containing exponentials
\citep{jaynes1957information}. Its relevance here is that it is a way of
estimating unknown probability distributions, based on a limited set of
constraints. In the language of spatial microsimulation, this means calculating
internal cell values based on marginal constraints. Thus entropy maximisation
can be used to estimate the maximum likelihood of individual level attributes
for areas about which only counts are available. Because of this, iterative
proportional fitting has been shown to be a specific form of
entropy maximisation \citep{Beckman1996, ye2009methodology, Rich2012}.

It was not until the
1990s
that IPF (and, often unconsciously, entropy maximisation) was discovered by
human geographers and
`put on the researcher's desk' \citep{Norman1999a} for spatial
microsimulation.\footnote{There
were a few earlier exceptions, including its application to
model the diffusion of Dutch Elm disease in the UK \citep{sarre1978diffusion}.
}
An early advocate was \citet{Wong1992}; early applications that produced
spatial microdata included \citet{Birkin1988}, who used IPF in combination with
Monte Carlo sampling to create completely synthetic microdata.
\citet{Ballas1999} used IPF to allocate individual level survey data to
areas.\citet{Mitchell2002} used IPF to create cross-tabulations
of categorical marginal totals to investigate the changing geography of health
inequalities in the UK.

Deming's methodological innovation was not especially outstanding in the context of
rapidly advancing 1940s statistics, but it is worth considering in more detail. The
IPF procedure that it was built upon (Deming, 1940) 
is now frequently used in spatial microsimulation models
 for automatically allocating individuals from a survey
dataset to the zones for which they are most representative.
New applications and refinements to Deming's method continued in the
proceeding years within statistics
\citep{stephan1942iterative,Friedlander1961-ipf}, although the term `iterative
proportional fitting' was only used to describe it after
\citet{Fienberg1970}. Since then, IPF has continued to be refined and applied to
various statistical problems involving the estimation of missing data, but
these advances are generally contained in a literature that is separate from
the body of work that is the focus of this
chapter.\footnote{As
a relevant
aside, history of IPF provides an interesting example of fragmentation in
academic research, as the statistical community continued to use Deming and
Stephen's method of estimating internal cell values based on known marginal
subtotals,
but using a totally different name: ``The methodology became known as `raking'
and found widespread application in sampling, especially at the US Census Bureau
and other national statistical offices''  \citet{Fienberg2007}. It is important
to note this divergence, as the statistical uses of IPF (or `raking') have the
potential to aid the technique's usage in spatial microsimulation.
}
The reasons for using IPF instead of combinatorial optimisation or other related
methods of discrete multivariate analysis described
in \citet{bishop2007discrete} include 
speed of computation, simplicity and
the guarantee of convergence \citep{Deming1940, Mosteller1968,
Fienberg1970, Wong1992, Pritchard2012}.
\citet{Rich2012} endorsed IPF over alternatives in the context of transport modelling.
Summarising past literature, they state that IPF can arrive at the same
(maximum likelihood) result as other maximum entropy (ME) approaches,
but faster: ``The popularity of the IPF is therefore mainly due to the
fact that it provides a solution which is equivalent to that of the ME approaches,
but attained in a much more computationally efficient way'' \citep{Rich2012}.

It was only with the intervention of Guy Orcutt that such methodological
advancements were combined with new computing capabilities to provide new
possibilities for social science, based on the simulation of individuals.
Although Orcutt is often cited as one of the founders of social simulation,
arguably his most important contribution was to place computerised methods in a
wider conceptual framework of policy analysis. Instead of using a
single `representative agent' with averaged values, the microsimulation method
enabled the evolution of multiple micro units to be traced, under different
scenarios \citep[p.~176]{mitton2000microsimulation}.
This helps explain why Orcutt (\citeyear{Orcutt1957-new-type,
orcutt1961microanalysis}) is frequently cited as one of the founding fathers of
the field
(e.g.~\citealp{Clarke+Longley1989-UK-housing-sim,Wu2008,
Ballas2013-4policy-analysis}). Granted, he successfully exported the concept of
manipulating individual level variables based on estimated
probabilities of change, but Orcutt was not particularly interested in
spatial analysis.\footnote{Although
Orcutt was instrumental in advocating and demonstrating
micro level methods for policy evaluation, he was more concerned with time than
he was with
space. %%%!!!referred
Neither IPF nor combinational optimisation, two of the main tools used for
generating spatial microdata in spatial microsimulation research today,
%!!!verify
are mentioned in his seminal works
\citep{Orcutt1957-new-type,orcutt1961microanalysis}.
Instead, he laid down the tantalizing possibilities of simulating society, in
very general (and seldom validated) terms, using the newly available
mainframe computers. The following is a typical example of the clarity,
enthusiasm and sense of purpose of his vision: ``The following method is
feasible, readily comprehensible and may serve to illustrate still further the
proposed model. Using this approach the model would be simulated on a large
electronic machine, such as the IBM 704 or the UNIVAC II, or some improved
successor to these powerful giants'' \citep[p.~119]{Orcutt1957-new-type}.
}
Building on Orcutt's methods, simulation grew popular in the
increasingly quantitative social sciences. Uptake was
greatest in economics, where the technique
gained a strong following as a method for evaluating the impact of
changing policy and economic conditions at the individual level
(see \citealp{Merz1994} for an overview).
The branch of microsimulation associated with spatial problems emerged later
\citep{Tanton2013-intro}, although it has clear links with earlier shifts
towards modelling within the wider field of quantitative geography
(e.g.~\citealp{Clarke1985}).

The shift to the practical application of microsimulation to explicitly
\emph{spatial} problems was not to happen until around 30 years after the
1960s applications. This can partly be attributed to the
computational limits emphasised by Guy Orcutt at the outset of this chapter, but
partly also to a disinterest in quantitative models on the part of geographers.
A seminal paper \citep{Holm1987} reviewed the limited experience of
microsimulation models for
spatial applications up to that point. The authors warned of ``the
possibility of the method being reinvented by different
researchers independently'' if the new techniques continued to be ignored by
geographers \citep[p.~145]{Holm1987} and provided a coherent argument in favour
spatial microsimulation, culminating in the following conclusion:
``With micro-modelling it is possible to use and formulate theoretical concepts
and hypotheses about social action on at least the same level of detail as
sometimes found in other social sciences  without neglecting the apparent and
important elements of spatial interdependence seldom found in studies outside
geography'' \citep[p.~163]{Holm1987}.
Thus the gauntlet was laid down to future
researchers entering this emerging field: develop spatial microsimulation models
to take advantage of newly available computers, programming languages and
datasets. Since then ``the speed of development has gathered
pace''\citep[p.~259]{clarke2013conclusions}. Spatial microsimulation is now a
field of social and spatial analysis in its own right, with an expanding range
of applications. 

\subsection{Modern spatial microsimulation}
Geographers are not generally taught computer programming.
This, and the `erosion of quantitative literacy' \citep{ESRC2013}
helps explain why spatial microsimulation has been limited to a
small field within geography and related disciplines. Spatial
microsimulation now constitutes ``a relatively small
community'' that can be considered a field in its own right
(Wilson, in \citealp[p.~vi]{Tanton2013}).

This community can roughly be identified as those with links
to the International Microsimulation Association (IMA), 
who publish spatial microsimulation work in peer reviewed
journals\footnote{The following journals are common places for the
publication of spatial microsimulation research:
\emph{Computers, Environment and Urban Systems},
\emph{The international Journal of Microsimulation},
\emph{Journal of Artificial Societies and Social Simulation} and
\emph{Environment and Planning A}. Applied spatial microsimulation
research is also published in a wide range of regional science
and geography journals.
}
and whose work is referred to in recent overviews of the field
\citep{Tanton2013, O'Donoghue2013}.
In summary, spatial microsimulation has emerged
from pre-computer origins and mid 20$^{th}$ century theoretical quantitative
geography to tackle the research challenge set out by
\citet{Holm1987}. Since powerful computers became available at the turn
of the 21$^{st}$ century, methods and applications have
proliferated and accelerated. Spatial microsimulation now
provides small-area estimates
of individual level variables and projections of future change.
Transport, along with a number of other phenomena, has been
identified as an area for future application of the modelling framework
\citep{clarke2013conclusions}.
% yet there is much scope for
% Microsimulation models clearly had great potential yet many of the researchers who
% could benefit most from them lacked the skills to `get stuck in' and build
% spatial microsimulation models from scratch. Even if scripts were made available, it
% would have been difficult to find suitable computers and programming expertise to
% run them until the turn of the century, by which time approaches to spatial microsimulation
% were moving on. Spatial microsimulation is undoubtedly a fast moving field, in which
% a number of key papers have had a large effect on subsequent studies. To gain an understanding
% of the reasons behind the current state of the art \cref{s:sotart}, seminal papers that
% have helped to consolidate and define the field are presented below:
% \begin{itemize}
%  \item \citet{Holm1987}
%  \item \citet{Williamson1998}
%  \item \citet{}
% \end{itemize}
% 
% \citep{Birkin1989}
% \citep{ballas2003microsimulation-30-years}
% % Tell a story: it started in Africa, uptake by geographers
% \citep{Wong1992}
% \citep{Johnston1993}
% 
% The first national dynamic spatial microsimulation model was the System for
% Visualising Economic and Regional Influences in Governing the
% Environment (SVERIGE) \citep{vencatasawmy1999building}. This model, unlike the
% static approach used in this PhD, is fully dynamic and includes modules which
% calculate fertility, mortality, employment and other life events including
% migration. SVERIGE focuses on agent life events, rather than the geographical
% distribution of their individual level attributes and how they were allocated to
% different
% places.\footnote{In fact, SVERIGE did not need to allocate any individuals from
% an a-spatial dataset to zones based on linking variables. This is because a
% micro level dataset containing all Swedish individuals is available to
% researchers \citep{Ballas2005-ireland}. This removes the need for
% spatial microsimulation via reweighting as defined above. 
% 
% } The dynamic nature of 
% SVERIGE and its basis in real microdata make it a powerful tool for evaluating
% the impacts of national policy changes \citep{ Ballas2013-4policy-analysis}
% 
% an early spatial agent-based model (see
% \cref{s:agent-based}), rather than as an early example of dynamic spatial
% microsimulation as described by .

\section{Spatial microsimulation: state of the art}
\label{s:sotart}
Spatial microsimulation can now be seen as a field in its own right, with roots
in Economics, Geography, Statistics and Regional Science.  It is
evolving, so any rigid definition of the `state of the art' is likely to become
obsolete quickly. Instead, the scope of spatial
microsimulation is explained below in terms of the types and applications of models
in use, the variety of reweighting algorithms and recent transport applications.

\subsection{Types of spatial microsimulation models} %%% Add applications 
\label{types-msim}
%%% Taken directly from ints paper
The wide range of methods available for spatial microsimulation can be divided
into static, dynamic, deterministic and probabilistic approaches (Table
\ref{typology}). Static approaches generate small
area microdata for one point in time. These can be classified as
either probabilistic methods which use a random number generator and
deterministic reweighting methods, which do not. The latter produce
fractional weights. Dynamic approaches project small
area microdata into the future. They typically involve modelling of
life events such as births, deaths and migration on the basis of random
sampling from known probabilities on such events \citep{Ballas2005c,
Vidyattama2010}; more advanced agent-based techniques, such as spatial
interaction models and household level phenomena, can be added to this basic
framework \citep{Wu2008, Wu2010}. There
are also `implicitly dynamic' models, which employ a static
approach to reweight an existing microdata set to match
projected change in aggregate level variables
(e.g.~\citealp{Ballas2005-ireland}).

\begin{table}[h]
\centerline{}
\caption{Typology of spatial microsimulation methods}
\vspace{0.25 cm}
\footnotesize{
\begin{tabular}{p{1.6cm}p{2.4cm}p{3.5cm}p{3.0cm}p{2cm}}
\toprule
{Type} & {Reweighting technique} & {Pros} & {Cons} &
{Example} \\ \midrule
\multirow{3}{2cm}{\vspace{0.3cm} \\ Determ-  inistic\\\vspace{0.3cm}
Re-\\weighting} & Iterative proportional fitting (IPF) & Simple, fast, accurate,
avoids local optima and random numbers & Non-integer weights &
\citep{Tomintz2008}⁠⁠ \\ \cmidrule{2- 5}
& Integerised IPF & Builds on IPF, provides integer
weights & Integerisation reduces model fit & \citep{Ballas2005c}⁠ \\
\cmidrule{2- 5}
&  GREGWT, generalised reweighting
& Fast, accurate,
avoids local optima and random numbers & Non-integer weights  &
\citep{Miranti2010}⁠ \\ \midrule
\multirow{2}{2cm}{ \\ Probab- ilistic  Combin-
atorial optim-
isation} & Hill climbing approach & The simplest solution to a combinatorial
optimisation, integer results & Can get stuck in local optima, slow &
\citep{Williamson1998}⁠ \\ \cmidrule{2-5}
& Simulated annealing & Avoids local minima, widely
used, multi level constraints & Computationally intensive
& \citep{kavroudakis2012}⁠  \\ \midrule
\multirow{2}{2cm}{\vspace{0.3cm} \\ Dynamic} & Monte Carlo
randomisation to simulate ageing  & Realistic treatement of stochastic
life events such as death & Depends on accurate estimates of life event
probabilities & \citep{Vidyattama2010}⁠ \\
\cmidrule{2- 5}
& Implicitly dynamic & Simplicity, low
computational demands & Crude, must project constraint
variables & \citep{Ballas2005b}⁠ \\ \bottomrule
\end{tabular}
}
\label{typology}
\end{table}

In practice, the typology presented in \cref{typology} is an
oversimplification. The spatial microdata generated during the same spatial
microsimulation project can be used for both static and dynamic applications and
different reweighting algorithms can be applied to the same dataset with
similar results. Spatial microsimulation can thus be seen as an evolving
process rather than a `once-through' analysis. A typical spatial microsimulation
project, for example, may involve some or all of the following four steps
(the first four are from \citealp{ballas2003microsimulation-30-years}):
\begin{itemize}
 \item construct a micro-dataset, usually from surveys
 \item reweight the individual level data to create a spatial microdataset
 \item static what-if scenarios (implicitly dynamic scenarios in
\cref{typology}) to assess the impact of instantaneous change
 \item agent-based modelling, to better understand how the individuals in each
zone interact with the environment and each other
\end{itemize}

% The applications of spatial microsimulation can also be usefully categorised
% to inform discussion on the method's expanding range of uses.
% 
% \citet{van2012multifunctional} harnessed a spatial microsimulation model
% to provide input data for a logit model of shopping preferences.
% This modelling exercise provided estimates of the
% impact of new a supermarkets on surrounding stores: it was calculated
% how much more or less would be spent in each. This novel use of
% spatial microsimulation for scenario-testing is highly policy-relevant
% (and politicised), and could be made more so by calculating distributional
% impacts.

\subsection{Reweighting algorithms} %%% Could easily put the IPF paper here
\label{sreweight}
To run a spatial microsimulation model, a prerequisite is a mechanism by
which individuals from the survey are selected to `populate' the areas under
investigation. For the technique to be worthwhile, it is vital that individuals
who are in some way representative of each area should be selected
\citep{Ballas2005}⁠. Doing this manually is clearly not feasible, so a number
of computerised techniques have been developed to create weight matrices
automatically. This section provides an overview of the reweighting techniques
that have been used in published research; the findings fit directly into the
choice of microsimulation model used in this research. %(Illustration here???)

Reweighting algorithms allocate individuals counts or weights for target
areas based on a number of matching or linking variables that are shared between
area and survey datasets. A number of options are available and these can be
broken down into the following categories:
deterministic/randomised, integer/ratio and count/weight.
The option used in this thesis is deterministic
sampling based on IPF. This reweighting procedure was
chosen due to the repeatability of the
results,\footnote{``One
advantage of a deterministic model is that the estimated population
distributions will be the same each time the model is run'' \citep{Smith2009}⁠.
Thus, the results of any model to be replicated
without the need to ``set the seed'' of a known list of
Pseodo-random numbers \citep{Robert2009}: this makes results easier to
test and update when new data emerges.}
relative simplicity and past experience with
the technique.

Randomised (combinatorial optimisation) sampling strategies have the advantage
of robustness against local
optima, which may mean that deterministic models may not always arrive at the
optimal solution \citep{Williamson1998}.
Also, a combinatorial optimisation sampling strategy has the inherent advantage of
keeping individuals as integers (as opposed to deterministic reweighting,
which results in fractional weights). This makes it easier to
understand the simulated population, analyse the results 
(e.g.~ the Gini Index calculation
is more straightforward if integer weights are used) and select
subsets of the simulated population with certain characteristics.
In addition, integer weights are needed for agent-based models. On the other
hand, integer results can be associated with large differences between simulated
and actual cell values \citep{Ballas2005}⁠.

In order to calculate the probabilities of survey individuals appearing in
statistical areas, iterative proportional fitting (IPF) has been used. By
altering the cell values in a 2 dimensional matrix, IPF is used to match
``disaggregated data from one source with the aggregated data from another''
\citep[p.~1]{Norman1999a}⁠. This is done iteratively: each iteration brings the column and
row totals of the simulated dataset closer to those of area in question.

Another, more fundamental, disadvantage of IPF is its inability to simulate
individuals based on data at multiple levels, for example household and
individual: ``it can control either for agent level or for group level
attributes but not for both simultaneously'' \citep[p.~5]{Muller2010}.
This problem has long challenged researchers because ``working at the
household/family and person levels simultaneously can introduce conflicts
between the competing goals of achieving good fit at both levels''
\citet[p.~694]{Pritchard2012}. \citet{Pritchard2012} have tackled this problem
by matching either individuals to known family attributes, for example 
based on conditional probabilities of the spouse sharing given
attributes (age, level of education). These results offer the promise of allowing
family level microdata generation from deterministic reweighting algorithms
such as IPF.

Despite the wide range of reweighting options available and even wider
range of implementations, there has been relatively little work comparing
different approaches. Most model experiments evaluate goodness-of-fit for
only a subset of reweighting algorithms, changing just one or two variables
at a time \citep{Voas2000, Smith2009, Rahman2010}. Another problem is the
wide range of evaluation tools on offer, leading to confusion about
which method is appropriate for a given application:
``Different researchers use different methods to test the
reliability of their results. This makes it more difficult for `outsiders'
to evaluate the value of a model or set of artificial population data''
\citep[p.282]{Hermes2012a}. This issue is tackled with respect
to the problem of integerisation in \cref{s:integerisation}
and discussed in more general terms in \cref{meval}. One group
of `outsiders' that could benefit from more accessible code and
reproducible testing of it is the transport community, who are increasingly
turning to spatial microsimulation to meet the need to include
social factors in scenario evaluation.

\subsection{Transport applications}
%%% This is the section I need to write next 4 this chap. First I'd like some results.
It was mentioned in \cref{s:defs} that `population synthesis' is a synonym for
(static) spatial microsimulation. The term is used by transport modellers
to describe the
process of generating individuals as inputs into wider transport models.
Thus spatial microsimulation is used in transport applications.
Whether to classify any given transport study as spatial microsimulation for
transport analysis, or a transport model with spatial microsimulation
`bolted on', is a question of semantics not dwelt on
here.\footnote{\citet{Ballas2013-4policy-analysis}
treat activity-based
transport models as an add-on to spatial microsimulation methods.
The approach taken here is to deal with 
spatial microsimulation models that have some transport
considerations added-on (this section) separately from
dedicated transport models  that also simulate individuals
(\cref{s:dedicated}).}
In any case, there is clearly a
large degree of overlap between
the two approaches. This section describes transport research that focuses on the
individual (human, not vehicle) level, primarily through spatial microsimulation.
\Cref{s:dedicated} outlines dedicated transport models, which can also harness
spatial microsimulation data as an addition to assess social impacts.

Transport modelling has a long history with strong links to engineering
,\footnote{The strength of engineers'
influence is emphasised in the following passage: ``If the main brief
of the planners is to recommend the `shape' of cities,
then it is usually left to the engineers to design, build and manage the
transport systems. Engineers,
therefore, can use models as design tools: for predicting loads ...
network optimisation ... they will have concerns with project
appraisal'' \citep[p.~16]{Wilson1998-past}.}
strategic planing \citep{Wilson1998-past} and hence large contracts.
Aggregate economic return on income has thus played a central role in
project evaluation and has become a focus of various modelling efforts
\citep{Masser1992}.
Perhaps due to this narrow technical and economic heritage, traffic models
have tended to omit people from the analysis. Technical
questions, such as `how much congestion will intervention x alleviate?',
predominate, rather
than social questions more common in spatial microsimulation research
such as `which groups will benefit most from intervention x?'.
Thus it has been rare for socio-economic variables to be included in
the model-based evaluation of transport projects, although
social impacts are increasingly considered \citep{Masser1992, Tribby2012}.
This explains growing interest
in spatial microsimulation for transport applications.
It is in this context --- a divide between the transport community, with its focus
on traffic and aggregate economic performance and the spatial microsimulation
community, with its focus on distributional impacts and public policy --- that these
studies are conducted. 


\citet{Pritchard2012} advocated harnessing spatial microsimulation
for methodological reasons, including the computational benefits of sparse data storage
for transport models.\footnote{Sparse storage here refers to data structures
whereby only non-zero values are stored and replication
weights are used instead of repeating statistically identical individuals multiple
times. This also avoids problems associated with arbitrary categories, e.g.~for
age: ``Complete array storage is proportional to the number of categories used
for each attributes, while the sparse storage scheme is not affected by the
categorization of the attributes'' \citep[p.~691]{Pritchard2012}.} These
efficient % could add table 1 from here - cool!!!
data structures have origins in early spatial microsimulation
research \citep{Holm1987, Williamson1998} and have the additional benefit
of providing ready-made inputs into agent-based transport models such as
ILUTE (see \cref{s:dedicated}).


\emph{PopGen} is a program used
to generate spatial micro-data on the characteristics of individuals
living, and using transport services, in the study region
\citep{Ravulaparthy2011}. It \index{PopGen}
is essentially a static spatial microsimulation model that combines non-spatial
survey data with `marginal tables'
(constraint variables). Three input files can be used at each level ---
individual, household and optional `groupquarters'
(these are generally students living away from home) --- leading to a high level of
detail. The use of iterative proportional updating (IPU) is key to the
ability of PopGen to simultaneously match individual and household level
characteristics, during the process of allocating individuals to household
\citep{ye2009methodology}. PopGen is made freely available to
anyone from Arizona State University and has been used as a population
synthesizer for other tranpsort studies \citep{pendyala2012application}.

\emph{Popgen-T} is a different (albeit confusingly similar in name) population
synthesiser developed specifically for the purpose of analysing the
distributional impacts of new transport schemes such as congestion charges
\citep{Bonsall2005}. The method uses IPF to combine data from a very wide range
of sources, although the exact mechanism is not
explained.\footnote{In the 2005 paper, the following information on
data sources was provided: ``The data sources used in this application include the
Household Census, the National Travel Survey, the Journey to Work Census,
the Household Income Survey, The Household Expenditure Survey,
the New Earnings Survey and a number of local travel surveys'' \citep[p.~410]{Bonsall2005}.
The data are further explained in a 2002 working paper, but this could not
be found.} Since the 2005 paper, no further implementations of the Popgen-T method
could be found.




\section{Microsimulation in urban modelling}
\label{s:urbanmodel}
Urban modelling goes beyond the estimation of individual level
characteristics, as performed in spatial microsimulation.
It attempts to include influential factors from the entirety of
urban experience, from house prices and the labour market to
the transport network and land-use. It is therefore inherently
an ambitious project, that could claim to encapsulate transport models
and explain travel to work patterns in their wider context.
Only recently have data and computational power emerged to
make this `dream' reality; many of the approaches to urban
modelling are related to this research. The most relevant are outlined below.

Five entities central to any urban model have been identified
% by Wilson (2000),
by \citet{wilson2000complex} and
it is the interaction between these that determines
the final model outcome. The importance of each
for influencing commuter flows, level of data availability and
ease of incorporation into quantitative models is presented in
\cref{t:entities}. Ultimately, these considerations should determine
whether, and at what stage, each of these entities are included in urban models.
\begin{table}[htbp]
\caption{Five entities central to urban modelling, after Wilson (2000)}
\begin{tabular}{p{2cm}p{3.5cm}p{3.5cm}p{3.5cm}} \toprule
Entity & Data availability & Importance for commuter flows & Ease of model inclusion \\ \midrule
People & High: commuting data collected in the Census and surveys & High: personal behaviour & High: individuals are basic unit of analysis \\
Organisations & Low: rapid change (especially in private sector operators) and poor accountability in many cases & Medium: councils and companies influence travel patterns & Low: organisations often
diffuse bodies \\
Commodities, goods, services & Low: petrol sales and bus ticket data not publicly available & Medium: travel is effected by price of fuel & Medium: can be defined by price of oil; depends on
elasticity \\
Land & High: maps of terrain and land use readily available & Medium: network distance and terrain alter travel behaviour & Medium: via influence of topology and distance \\
Infrastructure & Medium: Open Street Map and Ordnance Survey data & High: personal travel depends on infrastructure & Medium: can influence local travel decisions \\
\bottomrule
\end{tabular}
\label{t:entities}
\end{table}
Based on the basic multi-criteria analysis presented in
\cref{t:entities}, the following
hierarchy of entities for inclusion was established, in descending order
order of priority:

$people > infrastructure > land > commodities > organisations$.

This priority list was considered when 
compiling the data in \cref{Chapter4},
although only the first and second are
included in the methods of this thesis. Due to the importance of road network
planning, much of the research in the broader field of urban modelling is
dedicated to the development of dedicated transport models, which focus on
the second element of Wilson's (2000) list.
% The available data are therefore presented in this order, after a
% brief discussion of an additional data source that is
% vital to the research problem: energy use in transport.

% \subsection{Agent based models} %!!! Add this at some point
\label{s:agent-based}
\subsection{Dedicated transport models}
\label{s:dedicated}
Transport modelling is a large field within the wider framework of
urban modelling. It has a long history, but has undergone a rapid
evolution in the last decade, largely due to the emergence of the
internet, which allows large collaborative software projects to flourish.
Three dedicated transport models, of increasing levels of sophistication
have been selected from the
vast array of options to illustrate the state of transport modelling
and its relation to this thesis
(see \citealp{Rasouli2012} for a technical review).

\emph{SATURN} is a commercial transport model, originally
developed at the University of
Leeds \citep{boxill2000evaluation}. Its current incarnation
is version 11, a stable package running only on Windows \citep{SATURN2012}.
The SATURN model is a mature tool for determining traffic loads on road
networks given a known origin-destination flow matrix,
and is used for this purpose in local authorities in the UK
\citep{boyce2005urban}.

\emph{OpenTraffic} addresses many of the issues arising from commercial,
closed-source traffic simulation models such as SATURN: ``Most commercial
traffic simulation packages primarily offer only ready-to-use
functionality and do not facilitate the addition of new
functionality by users or provide a transparent picture of how the
underlying components are implemented'' \citep[44]{Tamminga2012}.
This recently developed simulation framework has a modular design
and is therefore useful in a wide range of applications, from
`car follow' to activity planning \citep{Tamminga2012}.

\emph{MATSim} is a more mature open source transport model that improves
on previous transport modelling programmes in a number of ways
\citep{rieser2007agent}. \index{MATSim}
The model allows individual attributes to be maintained throughout
agent-based simulation and ensures that trips made throughout the
day are realistically inter-dependent (see \cref{fmatsim-schema} for the
model's structure). For example, being late for one trip
will have an impact on the start-time of the next \citep{Balmer2009}.
Since the project was first made available as a free open source project
in 2006 (see \href{http://sourceforge.net/projects/matsim/files/MATSim/}{sourceforge.net}),
uptake has been rapid with applications ranging from agent-based
modelling of trips for leisure and shopping \citep{horni2009location}
to intensive
performance testing, in which MATSim is shown to accurately
model real world travel patterns \citep{balmer2008agenta, gao2010comparison}.
MATSim has also been used to model commuter patterns in Pretoria, South Africa,
incorporating previously omitted trip-chaining behaviours \citep{van2011agent}.

\begin{figure}[hb] \centerline{
\includegraphics[width=14cm]{matsim-schema}}
\caption[MATSim schema (permission: Michael Balmer)]{Schema of the MATSim
model \citep{Balmer2009}. Thanks to Michael Balmer for permission.}
\label{fmatsim-schema}
\end{figure}

Because MATSim builds on Kai Nagel's experience as a computer scientist, who
also developed the highly successful TRANSIMS model (described below),
it has several advantages over competitors. These include:
\begin{itemize}
 \item ``MATSim is consistently constructed around the notion that travellers
 (and possibly other objects of the simulation, such as traffic lights) are `agents',
 which means that all information for the agent should always kept together in
 the simulation at one place'' \citep[p.~9]{Balmer2009}. This allows demographic
 data on each traveller to be instantly available, rather than being completely
 unavailable (as in most transport models),
 or available in a fractured file system (as in TRANSIMS).
 \item MATSim is fast to run in comparison with other transport
 models with similar specifications.
 \item Strong user community. As of May 2013, there is a comprehensive
 new tutorial on how to install and use MATSim (see
 \href{http://www.matsim.org/docs/tutorials/}{MATSim.org's tutorials site}),
 and daily commits to the source
 code (see \href{http://sourceforge.net/apps/trac/matsim/timeline}{sourceforge.net}).
\end{itemize}
For these reasons, and due to its accessibility to anyone with a modern computer,
MATSim has been identified as the most appropriate pre-existing model for
interacting with the data and methods presented in this thesis. MATSim
was carefully designed from the ground up to be the most powerful, user-friendly
and fast agent-based transport model available. It is important to recognise
that in order to avoid trying to `re-invent the
wheel'.\footnote{See \cref{s:workdes}
for a crude attempt to integrate the road network in the spatial microsimulation
--- a MATSim implementation may have been more appropriate given sufficient
time.
}
% Insert here an image of MATSim (Balmer2009) and where my model fits in 

\subsection{Land-use transport models}
% ``The first generation urban models were designed and implemented in
% North America mainly during the years 1959-68, years which coincided
% with the launching of large-scale land-use-transportation studies in major
% metropolitan areas.''
Researchers now have decades of experience modelling individual agents
\citep{Ortuzar1982},
transport flows \citep{Wilson1970} and the land-uses that lead personal
transit to take place \citep{batty1976urban}.
Of course, each of these elements depends to some extent on the others, so
integrated land-use transport models have long been regarded as the holy grail
in urban modelling. It is only recently that the computational
requirements of this task have been
available.\footnote{The memory requirements alone of storing a detailed
transport network in RAM are large. Combining this with complex polygons
defining administrative zones, a detailed microdataset and then performing
calculations defining how each model object changes from one moment to the next
in high temporal resolution is clearly a taxing computational task.
}
Despite the daunting complexity and data and computational requirements of such
models, their design and implementation has been theorised and attempted
since the 1960s, with limited levels of success
\citep{timmermans2003saga}. The author of this critical review went so far as
to suggest that the costs invested in ambitious land-use transport models
generally outweigh the benefits.
% ``There seems insufficient support for
% such investment, especially because the field has not succeeded
% in commercialising the short-term forecasts of traffic flows''
% \citep[p.~3]{timmermans2003saga}.
On the other hand, some have argued that it is only with modern computers and software
that integrated land-use transport models can move from a mere
`dream' \citep{timmermans2003saga}  into reality: ``recently,
the development of large-scale integrated land-use and
transportation microsimulation systems such as ILUTE ... ILUMASS
... and UrbanSim has generated a new excitement in the field'' \citep[p.~935]{Pinjari2011}.
These models, and TRANSIMS, are outlined below.

% A number of integrated models have been developed
% \emph{RAMBLAS}

\emph{ILUTE} \index{ILUTE} represents the `third wave' of transport-land use
models based on individual level data:
``[it] represents an experiment in the development of a
fully microsimulation modelling
framework for the comprehensive, integrated modelling of urban transportation-land use
interactions and, among other outputs, the environmental impacts of these interactions''
\citep[p.~15]{timmermans2003saga}. Thus ILUTE can can be used to analyse a wide range
of phenomena: it is an integrated urban model in the fullest sense of the word
and has been even been used to analyse the distribution of house prices
in and large city over time \citep{Farooq2012-integreted}.

%\emph{TRESIS}
\emph{UrbanSim}, like ILUTE, is a micro level integrated land-use transport
model, aimed at \index{UrbanSim}
``incorporating the interactions between land use, transportation, the economy,
and the environment'' (\href{http://www.urbansim.org/Main/WebHome}{urbansim.org}, 2012).
The source code (written in Java and Python)
is open source and remains under continued development \citep{Nicolai2012-matsim}.
Perhaps because the software is free for anyone to download, use and modify,
it has been used for a range of applications including as a tool
to aid planners in the evaluation of transport projects \citep{Borning2008}.
Although UrbanSim does not contain an advanced transport module,
work has been done to integrate the dedicated transport MATSim model (see
\cref{s:dedicated}) into it,
via a plug-in \citep{Nicolai2012-matsim}.

\emph{TRANSIMS} \index{TRANSIMS} was developed at the Los Alamos National
Laboratory with an
ambitious objective mirroring that of ILUTE:
``to model all aspects of human behaviour related to
transport in one consistent simulation framework''
\citep[p.~1]{nagel1999transims}.  The model, which is based on cellular
automata, has been given a public licence (the NASA Open Source Agreement
Version 1.3), is cross-platform (with Windows and Linux binaries) and has been
widely adopted.\footnote{``TRANSIMS''
was cited 166 time in Google
Scholar in 2012 publications, many of which implemented the model for their own
applications.}
% It would be good to have ``n. times cited, 2012'' as a variable in comp. tab.
The encouragement of community contributions and an experienced development team
has led the model to be extended various ways. For example, TRANSIMS can be
configured to take advantage of parallel processing (in which one CPU is
allocated to each area being modelled) \citep{nagel2001parallel}, or external
programs for the visualisation of results
(http://sourceforge.net/projects/transimsstudio). The sub-modules of TRANSIMS
include a micro level population synthesizer, a trip generator, route planner
and microsimulator (which determines the location and behaviour of each
individual at each time step). The model is being increasingly adopted by
Municipal Planning Organizations (MPOs) in the USA \citep{lawe2009transims,
ullah2011travel} and has successfully simulated the entirety of Swiss travel
flows (around 10 million trips), using a `Beowulf cluster' of parallel computers
\citep{Raney2003}.

The modular design of TRANSIMS means it can be used in conjunction with the
spatial microsimulation methods presented in this paper. The small area
microdata could, when allocated home-work pairs, be used as an input forming
the baseline situation at time zero. The potential for combining the spatial
microsimulation methods presented in this thesis with additional modelling
tools is described in chapter 8.
% !!! Include this !!!

\section{Summary: research directions and applications}
\label{s:bigdata-gps}
Over time the uses of spatial microsimulation, in its broadest sense,
have expanded from a way of
providing quantitative geographers and others with individual level data, into a more
general modelling strategy harnessed to tackle many problems.
In this thesis, however, a narrower definition is used:
spatial microsimulation here refers to the process of generating spatial
microdata, analogous to `population synthesis' in transport models.
As in many fields, the
rate of change has also increased, due to increased availability of
sophisticated software, large datasets and powerful
computers. One could make the argument that the
uses of spatial microsimulation, as defined above, have become more specialised
as it is adopted by various fields for their own purposes, sometimes under
different names. This fragmentation is aggravated by the fact that
many do not make the code used for their analysis available, a
practice prevalent across the sciences \citep{Ince2012}.
However, there are also signs of integration. With the continued growth of
open source software and the greater dissemination of code
(e.g.~through sites such as Github), a kind of evolutionary process can be observed:
winners are picked and then generalised to be applied to a range of
problems.\footnote{A
good example of this positive-feedback process of picking winners, whereby
the most promising projects receive much new attention and then grow most
rapidly as a result (of peer feedback and new collaborators), is MATSim.
Released as an open source project in 2006, the project has rapidly gained
users, contributors and policy applications. MATSim also illustrates the
wide appeal of microsimulation software, finding applications as ranging from
a `plugin' to pre-existing urban simulation models to a framework for
modelling leisure and shopping trips \citep{Nicolai2012-matsim, horni2009location}.
}
% \citep{Clarke2013-concs}

The rate of change is fast, yet it is important to make use of more than 30 years
. Looking back, it is possible to reflect
on what works and what does not work so well in spatial microsimulation
research. Summarising a large body of experience,
\citet[p.~197]{Holm2013-design-principles} created the following `wish list' of
factors that future spatial microsimulation researchers should consider
when creating new, or updating existing, models:
\begin{itemize}
\item  use the most modern software
\item  use standard methods, shared by many users
\item  backward compatibility (so keeping our old models and subsystems running)
\item  avoid relearning
\item  develop solutions that are theoretically well designed
\item  transfer knowledge and know-how to new colleagues
\end{itemize}
It is interesting to note that this list could have been as applicable 30 years
ago as it is now, indicating key areas of continuity in the field.
Effort has been invested throughout to comply with these
principles. It is hoped that the focus on the final point, dissemination of methods,
will enable spatial microsimulation to be used by policy
makers.\footnote{To
this end, experiments to improve the performance of IPF and some other
script files that may be of use to others
have been put online via the dissemination portals www.rpubs.com/robinlovelace
and www.github.com/robinlovelace . Knowledge transfer was also behind the
publication of a user manual alongside \citet{Lovelace2013-trs}.
}
Indeed,
its potential for policy evaluation, at individual and local levels, was
one of the major reasons for choosing the spatial microsimulation approach
to tackle the problem, helping to fill the `scale gap' between academic
studies and policy interventions described in \cref{Chapter2}.

The literature summarised in this chapter should make it clear that
the methods used are not new: researchers have been modelling
transport problems at the individual level over two decades \citep{Ortuzar1982},
and developing the theory behind individual level behaviour for even longer
\citep{Wilson1970}.
The novel
contribution made in this thesis is the practical \emph{application} of
the existing method of spatial microsimulation to the problem of unsustainable
commuting. Approaching the issue from a quantitative geography and spatial
microsimulation perspective allows the focus on spatial
variability and social inequalities in transport energy use, highlighted in
\cref{Chapter6} to \cref{Chapter8} of this thesis. This is in contrast to the 
transport modelling perspective, which is still largely traffic-orientated.
Before proceeding to apply
the method, however, it is vital to understand precisely how the
spatial microsimulation model used in this thesis works and the input data.
That is the task of the next chapter.



%!!! include these!!!
% \subsection{`Big data'}
% \subsection{GPS-loan surveys}
% \subsection{Applications for model validation and enhancement}
\label{s:applications}

%%%%%%%%%%%%%%%%%%%%%%%%%%%%%% Deshets
% In economics, spatial microsimulation has been used to \citep{Cullinan2011}
 % Msim introduction

\input{./Chapters/Chapter4} % Data and methods

 \input{./Chapters/Chapter5} % Experiment 2
% %
 % Chapter 6

\chapter{The energy costs of commuting}
\label{Chapter6}
%The volatility of oil prices \cref{fig:oilprices}).
% This is the money shot chapter: maps of behaviour culminating in energy maps
% generated by msim
% Compare energy use estimates derived through msim vs agg methods (it allows
% estimates at a small geo. scale right???!!!)
% Put national maps in national section
% \lhead{Chapter 6. \emph{The energy costs of commuting}}
\fancyhead[RO,LE]{Chapter 6. The energy costs of commuting} %2side
\fancyhead[RE,LO]{\thepage}
The preceding two chapters have demonstrated that there are both detailed
data (at various levels) on travel to work in the UK \emph{and} methods
that can be used to convert this information on behaviour
into estimates of energy use. Based
on these foundations, this chapter illustrates the main results, in terms
of overall energy use. 
Estimates of
energy use at national (\cref{snational}), regional (\cref{sregional}) and
in comparison with other sectors (\cref{stotalcomp}) levels are presented.
The approach follows the
principle of Occam's razor, whereby additional complexity is only added when
necessary, in contrast to agent-based approaches, where complexity is inherent at the
outset \citep{batty2012perspectives}.
Therefore the high level results are based on the simpler
aggregate level methods. Results that emerge from spatial microsimulation
(and which would be inaccessible using aggregate level methods alone) are presented
later on, for a smaller case study region. South Yorkshire is used here
as the case study region here and in subsequent chapters for consistency
(\cref{sindvar}).\footnote{The
reasons for choosing this case study area explained in \cref{soyoref}.
}
In this section the spatial distribution of energy use for commuting is
illustrated at a low level. Indicators of how the energy use
in each zone is distributed between different members of society are also
presented.
The international applicability of the
methods for calculating the energy costs of work travel
is tested in \cref{sinternational}, which
compares the energy intensity of commuting in England and the Netherlands.
% Estimates of changes in the energy costs of work travel over time
% are presented in \cref{stime}.
% 
% After presenting these estimates, attention is directed towards
% factors accounting for the variability observed at all levels.
% \Cref{sexplanation}, explores possible causes of energy intensive
% commuter patterns with reference to the literature, the energy use estimates
% present in this chapter and 
% additional data sources. The aim here is to move beyond description and
% towards explanation, to
% % so \cref{sexplanation} sets !!! re-add
% set out the scene for policy implications:
% it is the \emph{understanding} of an issue that allows it to be tackled
% successfully, rather than mere knowledge of its existence
% (see \citealp{Berners-Lee2013}).
In the final section the results are discussed with
reference to the debate on energy use and urban form, introduced in
\cref{s:energy}. %!!! really?
% \section{Commuting behaviour}
% Although available transport technologies and infrastructure affect the energy
% use of travel to work patterns over space and time, as presented in
% \cref{Chapter5}, it is behaviour that determines how much energy using devices
% are used. This section therefore analyses the aggregate level transport to work
% data, to guide the subsequent analysis of energy costs.

\section{Commuter energy use at the national level} \label{snational}
Based on the data and discussion of it presented until now, we are
well-placed to perform a preliminary estimate of energy use at the aggregate level.
This approach, starting simple to understand the fundamentals and most important
factors influencing the system before later adding details, follows the
recommendation of \citet{batty1976urban}.

Having considered the limitations of the data, and weighed up the
costs and benefits of complexity, it was decided to
primarily calculate $ET$ at the aggregate level,
as a function of only two parameters: mode and distance travelled.
(These are the cross-tabulated categorical variables provided as geographically
aggregated count data at administrative levels down to ST Wards ---
see \cref{t:agdata}). This can be expressed for any particular area as
\begin{equation}
 ET = \sum_m \sum_d{2dR_{(d,m)} \times E_m}
 \label{eqet1}
\end{equation}
where ET is the total work-day energy costs for all commuter trips that happen
in that area, d and m are distance and mode categories, dR is the mean average
route distance inferred from the mode-distance combination and E an
estimate of the
energy cost per unit distance (direct or indirect), presented for each mode
in \cref{tfinale}.

An alternative way to express this would be based on commuter flow data.
If one know the approximate origins (i) and destinations (j) of every commuter
trip, this can be expressed in a different way:
\begin{equation}
 Et_i = \sum_j \sum_m {n_{(i,j)} \times 2Q \times dE_{(i,j)} \times Ef_m}
 \label{eqet2}
\end{equation}
where $Q$ is the circuity factor which translates the Euclidean distance between
two places into an approximation of the network distance, defined by \cref{eq:circ}.
Summing Et for all the origin areas in the region of interest would provide
an overall estimate of energy costs.

Clearly, neither \cref{eqet1} nor \cref{eqet2}
tell the entire story, as they omit frequency of travel: how many days
per week people travel to work (this is covered in \cref{sfreq}).
They also omit a number of other complicating factors that are discussed in
the previous chapter.  However, they are enough to begin with, to create maps
that capture the spatial variability of energy costs of commuting at
a coarse geographical resolution.
The approach summarised by \cref{eqet1} is used, because the input data
is much simpler, smaller and easier to manage. (\Cref{eqet2} could be used
to verify the estimates.)

The input variable into \cref{eqet1} that has not yet been quantified is
dR. Route distance by mode and distance band
is needed to account for the fact that Census data on distance is presented in
categories (with breaks at 2, 5, 10, 20, 30, 40 and 60 km), whereas distance
itself is continuous. The simplest way around this problem would be to
assume that route distance sits in the centre of the bins (i.e.~1, 3.5, 7.5, ...
km). However, this would be a very gross simplification because the route distance
is certain to be greater than the Euclidean distances calculated from
home-work postcode pairs. Also, because each mode has a different
distance-frequency distribution,
% !!! figure???
it is safe to say that the average route distance will also vary depending
on the mode of travel.\footnote{One would, for example,
expect people who walk 2 to 5
km in Euclidean distance to travel on average less far than those who drive
between 2 to 5 km, as `impedance' of walking rises rapidly after the first
kilometre whereas the additional personal effort of driving
an extra kilometre or two is much lower
\citep{Iacono2010}, discussed in \cref{Chapter2}.
}
To take this into account, distance data from Understanding Society was used.
% !!! major *** should have used NTS --- do after = priority!
First, the values were converted into estimates of Euclidean distance and
split into the Census bands. Next, these were re-converted into the original
route distances, and the average was taken for each distance band/mode
combination. The results, which are presented in \cref{tdboxes} and visualised in 
\cref{fdboxes} and \cref{fdboxes2} for motorised and non-motorised modes,
provide strong evidence of inter-mode variation in distance travelled within
the same distance band. However, these results are problematic due to the
low quality of the input data (n = 5,000 but less than 5 individuals
were present for unusual
categories such as people walking more than 5 km to work) and were not entirely
as expected. The anomalies are summarised as follows:
\begin{itemize}
 \item Bus journeys appear to be longer than the equivalent journeys by train,
 which was expected to be associated with the longest trips (although train
 journeys are in second place).
 \item The average bicycle trip was expected to be longer than walking trips
 in all cases. This did apply in the 0-2 and 2-5 km categories, but after that
 the trend reversed. This can be explained by sample size: a few unusual people
 walk far to work, whereas cyclists, as expected, tend to cluster around the lower
 ends of the 5-10 and 10-20 km bins.
 \item The `inverse U' shape of the bottom graphs in both cases were unexpected.
 This could be explained by the tendency of people to round to 10: the
 average distance travelled in the 30-40 km bin was the closest to the upper
 bound in all cases, perhaps a result of people rounding to 25 miles for many
 trip distances in the 20s (just under 40 km in Euclidean distance).
\end{itemize}
It would be desirable to corroborate these findings with other individual level
data on travel to work. For the purposes of assessing the relative energy
costs of commuting in different areas, however, these estimates suffice:
the concepts and code behind the estimates would produce slightly different
values given different input data, but, at present, this is not our concern.
With evidence-based estimates of $dR_{(d,m)}$ in place,
we can proceed to estimate the relative energy costs of commuting in different
places. 

\begin{table}[htbp]
\caption[Average distance travelled by mode and distance band]
{Average distance travelled by mode and distance band (km),
from USd data.}\label{tdboxes}
\begin{center}
\begin{tabular}{lrrrrrrrr}
\toprule
Upper limit & 2.0 & 5.0 & 10.0 & 20.0 & 30.0 & 40.0 & 60.0 & 250.0 \\ \midrule
Car driver & 1.6 & 3.9 & 7.9 & 15.0 & 26.0 & 35.8 & 50.3 & 102.6 \\
Car passenger & 1.5 & 3.9 & 7.9 & 15.2 & 26.5 & 36.4 & 48.0 & 95.0 \\
Motorbike & 1.4 & 4.1 & 7.0 & 15.2 & 23.5 & 36.0 & \multicolumn{1}{l}{NA} & \multicolumn{1}{l}{NA} \\
Bus & 1.8 & 3.8 & 7.7 & 13.9 & 27.7 & 40.0 & 56.0 & 110.5 \\
Train & 1.5 & 4.2 & 8.1 & 15.1 & 26.4 & 37.6 & 53.2 & 98.8 \\
Metro & 1.7 & 4.0 & 8.1 & 14.7 & 25.8 & \multicolumn{1}{l}{NA} & \multicolumn{1}{l}{NA} & 65.0 \\
Cycle & 1.5 & 3.9 & 7.5 & 11.5 & \multicolumn{1}{l}{NA} & \multicolumn{1}{l}{NA} & \multicolumn{1}{l}{NA} & \multicolumn{1}{l}{NA} \\
Walk & 1.2 & 3.5 & 8.0 & 13.7 & 25.0 & \multicolumn{1}{l}{NA} & \multicolumn{1}{l}{NA} & \multicolumn{1}{l}{NA} \\
Other & 1.0 & 4.3 & 7.6 & 13.5 & 27.8 & 37.5 & 42.0 & 130.0 \\
Taxi & 1.7 & 3.0 & 9.0 & 12.0 & \multicolumn{1}{l}{NA} & \multicolumn{1}{l}{NA} & \multicolumn{1}{l}{NA} & \multicolumn{1}{l}{NA} \\
\bottomrule
\end{tabular}
\end{center}
\end{table}

\begin{figure}[htbp]
\begin{center}
    \includegraphics[width=9cm]{dboxes}\end{center}
  \caption[Distance bands and average distance travelled for motorised modes]
  {Distance bands and average distance travelled for motorised modes, expressed
  as the relationship between lower bound and average distance (top)
  and that between lower bound and the ratio of upper bound to average distance
  (below), from Understanding Society data. `card' and `carp' refer to
car driver and car passenger respectively.} %%% could update with NTS!!! should have used centre...
  \label{fdboxes}
\end{figure}

\begin{figure}[htbp]
\begin{center}
    \includegraphics[width=9cm]{dboxes-walk}  \end{center}
  \caption[Distance bands and average distance travelled for active modes]
  {Distance bands and average distance travelled for non-motorised modes, expressed
  as the relationship between lower bound and average distance (top)
  and that between lower bound and the ratio of upper bound to average distance
  (below).} %%% could update with NTS!!! should have used centre...
 \label{fdboxes2}
\end{figure}

% As a result, it can be assumed that the number of commuter occupants in each
% car is 1 (so $EI_car$ = 2.98, not 2.4) for all those who
% ticked the option ``...'': Those who car share will be captured in option x:
% car passenger trips are counted a unique mode of transport to work. Calculated
% at the individual level, a sample of the input data and resulting values
% for Et are illustrated in table xx and xx for xxx, the first MSOA area in
% the input dataset.
% 
% Even if more accurate results can be obtained by, and assumptions behind, the data
% used to calculate energy costs per trip

Based on these categories, and the values of Ef reported in the previous section,
the 99 distance-mode variables of the cross-tabulated
census table ST121 can each be allocated an average energy
costs.
% !!! figure here right!
Originally the energy cost associated with the number of people in
each distance/mode category was calculated using the LibreOffice Calc
spreadsheet software. However, this soon became unwieldy so the analysis
was transferred into R. The main script file used to convert the raw
count data (\cref{frcount}) into energy estimates is available in the
{\color{blue} \href{https://github.com/Robinlovelace/thesis-reproducible}
{thesis-reproducible}} folder associated with this
thesis.\footnote{Code %!!! make available!!! - in intro!
and output were also embedded in RMarkdown, to show the output from R.
Every step of this process is illustrated on the author's RPubs website
(\href{http://rpubs.com/RobinLovelace/7178}{rpubs.com/robinlovelace)}.}
The benefit of this script is that it can take input data of the type
displayed in \cref{frcount}, regardless of the number or scale of the
geographic units.
% Starting at the largest (regional) geography, the results
% are displayed in \cref{fgoren} to \cref{fwarden}.

At the national level, the distribution of trips by mode and distance is
displayed in \cref{fengmodedis}. This graph shows the
dominance of car drivers for all trip distances, except for the 0-2 km bin.
As expected, bicycle and walking trips are dominant in the lowest
distance categories and tail off to essentially zero after the 20 km mark.
Another result that was expected was the tendency of train journeys to
be longer, probably due to the possibility of working on the train and
the use of this mode by high-income workers travelling to London.

\begin{figure}[htbp]
    \centering{\includegraphics[width=12cm]{England-ttw-mode-distances}}
  \caption[Mode and distance categories of commute in England]
  {Mode and distance categories of commuter trips in England, 2001.}
  \label{fengmodedis}
\end{figure} %!!! re-add? (hopefully!)

According to the methodology described above,
this data was translated into energy costs at the national level of
Wales and England (the data table ``ST121'' is unavailable for Scotland and
Northern Ireland). As illustrated in \cref{few}, the energy costs of commuting
in Wales are higher per trip, by 10\% (34.5 MJ in England, 38.0 in Wales).
In practice, it is probably not worth plotting this information geographically,
as there is very little geographical information to report:
the values are aggregated over a very wide area, so a choropleth map
of the results makes little sense. However, the purpose of  \cref{few}
is primarily to introduce the subsequent geographical plots, which
are of increasingly small geographic zones.

\begin{figure}
 \centering{
 \includegraphics[width=8cm]{ew} }
 \caption{Comparison of commute energy costs between England and Wales.}
 \label{few}
\end{figure}



\section{Regional and sub-regional patterns} \label{sregional}
The average energy costs of commuter trips in England
are illustrated at the regional level in \cref{fgoren}, to provide an
overall impression of its spatial variability at the coarsest geography.
The high degree of geographical aggregation masks much of the variability,
yet there is still a substantial difference between regions. As expected,
London is the region with the lowest energy costs per commute at
20.8 MJ per one-way trip or 40\% below the average for all regions.
Excluding London, energy costs were lowest in the North West
and highest in the East of England (closely followed by the South East).
The variability between these regions was less noticeable:
they were 10\% below and 12\% below the national average respectively.


\begin{figure}[htbp]
\begin{center}
    \includegraphics[width=10cm]{goren}  \end{center}
  \caption[Average energy use per trip (Etrp, in MJ) in English regions]
  {Average energy use per trip (Etrp, in MJ) in English regions, based
  on cross-tabulated distance/mode geographically aggregated count data.}
 \label{fgoren}
\end{figure}

\begin{figure}[htbp]
\begin{center}
    \includegraphics[width=13cm]{rawcount}  \end{center}
  \caption[Raw count data of commuters by mode and distance.]
  {Raw count data of commuters by mode and distance, the first 5 columns of
  regional level data, from Casweb table ST121. Data displayed in RMarkdown
  format, illustrating the reproducibility of the results (see
  \href{http://rpubs.com/robinlovelace}{www.RPubs.com}).}
 \label{frcount}
\end{figure}

To gain more insight into the spatial pattern of commuter energy costs,
the same data was re-plotted at lower geographical scales, down to the ward
level for the nation. \Cref{fcountyen} shows the distribution of energy costs
at the county level, constituting 88 polygons (42 counties and an additional 46
Local Authorities to make-up areas not covered by counties). This is a useful
level for identifying case study cities and areas that have unusually high
or low levels of energy use, given their surroundings. As a general pattern,
large and high-density urban areas tend to have lower energy use, with
the three largest built-up areas in England (Inner London, Greater Manchester
and the West Midlands built-up area) all having average commuter energy costs
below 30 MJ (the mean is 36). Another pattern that emerges is the relationship
between the very low energy costs of commuting in London, and the relatively high
costs of areas within a $\sim$100 km radius surrounding the centre: commuters in Bedford, Essex and
Kent, all of which contain `commuter belts' feeding London, for example, use
on average 45 MJ per trip to work. The highest and lowest (outside London)
values are found in Rutland (the geographic centroid of which is located 109
km from central London, and which was the last county in England to have
a direct trainline to London) and the City of Kingston upon Hull, respectively.
Comparison of these two counties could make an interesting case study
to explore the reasons for underlying reasons behind high and low
energy costs of commuting in England.

\begin{figure}[htbp]
\begin{center}
    \includegraphics[width=13cm]{countyen}  \end{center}
  \caption[Average energy use per commuter trip at the county level]
  {Average energy use per commuter trip at the county level. The
  letter strings are abbreviations of the full county names (e.g. Dv is Devon).}
 \label{fcountyen}
\end{figure}

The results for districts, of which there are 308 in England,
are presented in \cref{fdistricten}.
As is apparent from the large and relatively homogeneous
area of bright green in London (and
knowing its high population density), the districts with the lowest
commuter energy costs are found in the capital. In fact,
9 out of 10 of the districts with the lowest energy costs per
commuter trip are located in London (the lowest is found in the
Isles of Scilly, with an average of 7.6 MJ/trip). The district with the
highest energy use per commuter trip (60 MJ/trip, 10\% more than the second
highest zone) is South Northamptonshire, visible
in \cref{fdistricten} as the red zone in the far south corner of the
East Midlands. The standard deviation of average energy use per trip at
this level of geographic aggregation was 9.0 MJ, 50\% higher than the
6.0 MJ/trip standard deviation observed at the regional level.

\begin{figure}[htbp]
\begin{center}
    \includegraphics[width=13cm]{districten}  \end{center}
  \caption[Average energy use per commuter trip at the district level]
  {Average energy use per commuter trip at the district level.}
 \label{fdistricten}
\end{figure}

The same results are presented in \cref{fengplotnm}, at the ward level.
Here, much greater variability is apparent (note the increased range of
values represented in the colour scale). The standard deviation is 11.6
and values range all the way from 5.1 to 88 MJ per trip.
It is interesting to note where these extreme values are found:
the former is located in the central London ward of
Portsoken, where walking is the most common mode of travel to work,
followed closely by catching the tram. The latter was
found in Park Farm North,
a suburban ward located in the far South East of England, just south of
Ashford, where car drivers account for 68\% of all commutes. The complex
patchwork of average  commuter energy costs displayed in \cref{fengplotnm}
suggests that regional level assessments, such as those
presented in \cref{fgoren}, are not able to capture the full geographical
variability of the variable at all well: there is much more variability
within zones than between them. One pattern that stands out from the ward level
analysis is the tendency of settlements to be directly surrounded by green areas
associated with low energy costs. Although only large cities (those with
populations in excess of 100,000) are displayed in
\cref{fengplotnm}, it seems that many towns and cities are immediately surrounded
by areas of low commuter energy costs. Haverhill (located in the East of
England, roughly half-way between Cambridge and Chelmsford),
Hereford (in the south-west of the West Midlands) and a number of coastal
towns such as  Sheringham ($\sim$40 km north of Norwich) and Scarborough
(in Yorkshire and the Humber) are examples of this.
%%% Link here to analysis showing cor between remoteness and ecosts
\begin{figure}[htbp]
\begin{center}
    \includegraphics[width=16cm]{engplotnm}  \end{center}
  \caption[Average energy use per trip (Etrp, in MJ) in English wards]
  {Average energy use per trip (Etrp, in MJ) in English wards.
  The black dots are large (100,000 people or more) cities (from
  \citet{Brownrigg2013}).}
 \label{fengplotnm}
\end{figure}

The method used to calculate energy costs creates estimates
that are disaggregated by mode and distance. This allows the
aggregate energy use result in each area to be subdivided.
A policy-relevant example of this would be those areas in which
short-distance car journey constitute
a large proportion of the energy costs of work travel (these areas
may benefit from improved walking and cycling infrastructure). Another example
is the proportion of commuter trip energy use
in each area used by trains. The result is interesting in itself, and
provides confidence that the calculations are working correctly:
it is clear from \cref{ftrainen} that there is a tendency for
areas located close to railways
to be associated with a high proportion of per trip energy
use to be composed of rail travel. Also as expected, areas with fast rail
connections to London seem to have high energy use for this mode of travel.

\begin{figure}[htbp]
\begin{center}
    \includegraphics[width=15cm]{trainen}  \end{center}
  \caption[Proportion of energy use caused by train trips]
  {Proportion of energy use caused by train trips, plotted alongside the rail
network (black lines). Only areas above the national average (3\%) are
plotted.}
 \label{ftrainen}
\end{figure}

\section{Total commuting energy use and comparisons with other sectors}
\label{stotalcomp}
In \cref{Chapter5}, reasons and methods for calculating commuter energy use
on an annual level were laid out. In this section, total energy use
for commuting is presented, based on the average frequency counts presented
in \cref{tdistable} and the assumption that people work on average for
44 weeks per year. As acknowledged in \cref{sfreq}, these are quite crude
assumptions that could be updated if the true distribution of part and
full time jobs in each area were known and using spatial microdata.
However, geographical breakdowns of energy use from other sectors are
provided only at coarse levels of aggregation, so using the spatial
microsimulation approach in this case seemed unnecessary. Moreover, total
energy use for commuting is something that would be useful to estimate at
the national level, something which the spatial microsimulation methods
described in \cref{Chapter4} cannot
handle.\footnote{If small samples of the
spatial microdata were used (e.g. a 1\% sample), a spatial microsimulation
model would be possible for the whole of England, although the loss of
information from sampling may negate the benefits.
}

Using the script file `districten-yr', the total energy costs of commuting
across all of England in 2001 was estimated to be 220 PJ, or 61 TWh.
To put these large numbers into context, total electricity usage in the UK
(not just England) is 400 TWh \citep{MacKay2009}. Overall, this represented
4.1\% of total energy in England from all sectors and 14.4\% of total transport
energy use, based on the DECC's 2003 NUTS level 4
estimates.\footnote{This
dataset is available from {\color{blue} \href{https://www.gov.uk/government/statistical-data-sets/total-final-energy-consumption-at-regional-and-local-authority level-2005-to-2010}{https://www.gov.uk/government/statistical-data-sets/}}
and includes breakdowns of energy use by sector (industry \& commercial, domestic and transport)
and primary energy source (from coal to renewables).
Because the national level commuting dataset I was using operated at the
Local Authority level, while the DECC data was presented as NUT 4 zones,
which are slightly different. Joining by zone name, 16 of the 354 Local
Authorities were left blank, as shown in \cref{fpropten}.
}
As expected, commuting was found to be a large energy user.

Because commuter energy use scales with population, it was decided to represent
total energy use not in absolute terms, but relative to total energy use,
in each area. \Cref{fpropten} illustrates the spatial distribution of
the proportion of energy use across England. It shows that although the
average is just over 4\%, in some areas it approaches 10\%. Four areas
were identified in which commuter energy use accounted for over 9\% of total
energy use: Castle Point (a wealthy area in South
Essex),\footnote{Hints to its high commuter energy use, relative to its total
can be found on its Wikipedia page:
``Levels of home and car ownership in Hadleigh and Canvey are very high,
social deprivation is relatively low.'' `Commuters' are also
identified as a major economic group in the area {\color{blue}
\href{http://tinyurl.com/qfkb9ta}
{(see wikipedia link embedded in pdf)}}.
}
Maldon (another wealthy zone in Essex),
Rushmore (East Hampshire) and Tamworth (an urban area on the Northern
outskirts of Birmingham). Whether or not these areas can be classified as
`commuter belts' or if there are other reasons for their high energy use was
not explored and remains an interesting question for future research.
The only two Local Authorities in which commuting was found to account
for less than 1\% of total energy use were both in Central London.
A similar picture is painted when the proportion of total transport
energy use consumed by commuting is plotted (\cref{fproptrans}).
It inspires confidence that when total transport energy use was plotted
against commuter energy use, there was a strong positive correlation
(r = 0.75). This correlation was slightly higher than when
the simpler energy use per trip (Etrp) metric was used.
This correlation increased slightly when compared with
total road energy use. Surprisingly, the correlation
was even greater between total commuting energy use and total energy use
(r = 0.82). No explanation for this finding could be found.

\begin{figure}
 \centering{
 \includegraphics[width=15 cm]{prop-total-energy}}
 \caption[Proportion of total energy use in the UK consumed by commuting]
 {Proportion of total energy use in the UK consumed by commuting.
 Grey areas represent zones for which the DECC `NUTS 4' level did not coincide with
 Local Authorities from the census.
 }
 \label{fpropten}
\end{figure}

\begin{figure}
 \centering{
 \includegraphics[width=15 cm]{prop-trans-energy}}
 \caption{Proportion of transport energy use in the UK consumed by commuting.}
 \label{fproptrans}
\end{figure}


It is also interesting to compare the energy use estimates presented in the
previous section with official emission data, which have recently been
released as 2005 estimates (the closest to 2001 available) at the Local Authority
level.\footnote{These datasets can be accessed from {\color{blue}
\href{https://www.gov.uk/government/publications/local-authority-emissions-estimates}
{https://www.gov.uk/government/publications}}.
}
It was found that the total per trip costs were closely correlated to the
official estimate of total transport energy (r = 0.78) and that emissions
from minor roads were most closely correlated (\cref{tco2cor}). 
It is interesting to note that the variable most highly correlated
with per person energy commuter energy costs was transport emissions
from motorways. This can be explained by considering that areas near to
motorways tend to have longer commutes. There was also a fairly strong
positive correlation (r = 0.48) between per capita commuter energy use
and per capita transport use.

\begin{table}[htbp]
\caption[Correlation matrix of energy use for commuting and emissions]
{Correlation matrix of energy use for commuting and emissions at the Local
Authority level in England. ET and EAV are total and per capita commuter energy
costs, respectively.}
\begin{tabular}{lrrrrrr}
\toprule
 & \multicolumn{1}{l}{ET} & \multicolumn{1}{l}{EAV} & \multicolumn{1}{l}{A roads} & \multicolumn{1}{l}{M ways} & \multicolumn{1}{l}{Minor roads} & \multicolumn{1}{l}{Trans. Total} \\
 \midrule
ET & 1 &  &  &  &  &  \\
ETrp & 0.06 & 1 &  &  &  &  \\
A roads & 0.62 & 0.13 & 1 &  &  &  \\
M ways & 0.36 & 0.25 & 0.16 & 1 &  &  \\
Minor roads & 0.85 & -0.08 & 0.55 & 0.25  & 1 &  \\
Trans. Total & 0.78 & 0.18 & 0.71 & 0.74 & 0.74 & 1 \\
\bottomrule
\end{tabular}
\label{tco2cor}
\end{table}

In the policy context, commuter energy use has been
quantified at the national level and disaggregated by Local Authority.
It appears to be closely correlated with official data on transport energy
use and emissions. In the intuitive units recommended by \citet{MacKay2009},
commuting has been found to use, on average, 7.9 kWh/p/d for each commuter or
3.7 kWh/p/d for every man, woman and child living in England. In terms of
the total energy use figures developed by David MacKay (which includes
embodied energy and services such as defence), this equates to
only 1.9\% of per capita energy use. (The system boundaries in the
DECC analysis are far narrower, accounting for the differences between
MacKay's figures and theirs.) Even without including the system level
energy costs of commuting described in \cref{Chapter5}, this is a large
energy user for something that is so integral to a functioning society as
getting to work. However, the aggregate level is limited, and masks the
large differences that exist within statistical zones.
For this reason, the next section investigates the variability of commuter
energy costs at the individual level.


\section{Local and individual level variability} \label{sindvar}

As with any research in which geographical zones are the unit of analysis,
the maps of energy use presented above mask individual level variability within
zones. If interpreted incorrectly, conclusions resulting from such analyses
may be `ecological fallacies', where knowledge generated
at one level of understanding is incorrectly applied to another.
To provide an example, the strength of the correlation between wealth and the energy
costs of work travel at the ward level is unlikely to be the same as the
strength of the correlation at the level of individuals. The process of
geographic aggregation smooths relationships, often making correlations seem
greater and simpler that they really are \citep{Openshaw1983}. 

Spatial microsimulation can also be used to generate estimates of
geographically aggregated variables such as income, hence the use of the term
`small area estimation' used to describe some spatial microsimulation models
(see \cref{Chapter3}). Regarding the energy use of travel to work, spatial
microsimulation can help overcome a major data constraint at some geographical
levels: energy use is roughly a function of mode and distance of travel, yet
in some cases no cross tabulations on this matter are provided.
Even if average distances of travel to work are provided, it may be
impossible to know which modes of travel are responsible for high values.
When distance band and mode of travel are known but no cross-tabulations
are provided between them (as is the case with Super Output Area administrative
geographical levels from the data portal Casweb),
spatial microsimulation can be used to `fill in the gaps'.


A final potential issue with the ward level analysis of the entire nation, as presented
above, is the assumption that relationships are constant over space.
In many cases this assumption may justified (e.g. for the relationship between
population density and travel-to-work distance, which can be assumed to be
more-or-less universal), but sometimes relationships vary
substantially from place to place. This is a central motivation behind
geographically weighted regression \citep{Fotheringham2002}.

\subsection{A case study from South Yorkshire} \label{soyoref}
To illustrate the results of the spatial microsimulation model
in terms of energy use, a case study of South Yorkshire is used.
This county case study is used rather than the entirety of England because
processing time and memory demands were found to be problematic for
larger areas.\footnote{The
model was run for
Yorkshire and the Humber, which contains just over 2 million commuters.
Results were generated (as shown in \cref{svul}), but the time between
IPF iterations, and the tendency of the computer to lock-up after all
available RAM had been used --- on a computer with 12 Gb ---
led to a smaller case study region being selected.
}
The reasons for selecting South Yorkshire over other counties included the
clearly defined cities of Sheffield and Barnsley, as well as the region
between Sheffield, Rotherham and Doncaster that may be described as
the `South Yorkshire conurbation' \citep{barker1978perthes} --- it has
a diverse range of settlements from rural to urban and suburban.
In addition, social inequalities are quite clearly inbuilt into South Yorkshire's
geography. One can see, for example, where traits associated
with wealthy (to the west of Sheffield city centre, bordering
the Peak District) and more deprived (in the South-East of Sheffield,
for example) are located by visual inspection. The final reason
is that the author is well-acquainted with this area of England,
although a different case study region could equally have been used:
the purpose is to show the kinds of result that the
spatial microsimulation method can generate.
For continuity, 
South Yorkshire is also used as a case study region in the subsequent chapters.
% Because the majority of the analysis was done in R (with the R's
% improving geo-spatial data handling packages, all of the analysis
% could be done in this environment), some of the results are
% reported as listings, to illustrate how the results are accessed. %wtf?

After running the spatial microsimulation model outlined in
\cref{Chapter4}, constraining by age/sex, mode, distance of commute and
social class, an R object called a list is created. The list is a collection
of data tables, one for each administrative zone; each contains a number of
rows corresponding to the number of commuters in the area of interest.
The results for the first six individual in the first MSOA area
in South Yorkshire in the list (``Barnsley 001'') are displayed in
\cref{tintallh}.

\begin{table}[htbp]
\caption[Sample of the spatial microsimulation model output]
{Sample of the spatial microsimulation model output for South
Yorkshire. The table was saved as a comma-delimited file with the command
``intall[[1]]'', which refers to the data table corresponding to the
first zone in Sheffield. In total, the R object
``intall'' contains 532,130 individuals from 176 MSOA zones.}
\begin{tabular}{rrrrlrrlllr}
\toprule
\multicolumn{1}{l}{} & \multicolumn{1}{l}{a\_hidp} & \multicolumn{1}{l}{a\_pno} &
\multicolumn{1}{l}{pidp} & sex & \multicolumn{1}{l}{age} & \multicolumn{1}{l}{dis}
& mode & nssec8 & urb & \multicolumn{1}{l}{ncars} \\
\midrule
18 & 68041483 & 2 & 68041491 & male & 35 & 71 & Car (d) & Other & rural & 2 \\
18 & 68041483 & 2 & 68041491 & male & 35 & 71 & Car (d) & Other & rural & 2 \\
200 & 68303283 & 1 & 68303287 & male & 41 & 125 & Car (d) & Other & urban & 1 \\
200 & 68303283 & 1 & 68303287 & male & 41 & 125 & Car (d) & Other & urban & 1 \\
219 & 68323003 & 1 & 68323007 & male & 53 & 71 & Car (d) & Other & urban & 1 \\
219 & 68323003 & 1 & 68323007 & male & 53 & 71 & Car (d) & Other & urban & 1 \\
\bottomrule
\end{tabular}
\label{tintallh}
\end{table}

From the household and personal ids (a\_hidp and a\_pidp) can be joined a
wide range of additional variables (\cref{tintcar}).
Binding the information representing
in \cref{tintallh} for all 176 zones (using the command \verb do.call() )
results in a single table representing all five hundred thousand commuters
in South Yorkshire. From here, energy use data can be produced for each
individual, using the same technique described for the calculation of
aggregate energy use. The additional refinement added at this individual level
was the size of car: large cars were allocated a higher value (3.9 MJ/km)
than small cars (2.5 MJ/km).\footnote{13.6\% of
responses to this question were ``inapplicable'' or some other `NA' value,
even amongst those who drove a car. In these cases the energy costs were
set equal to those of a medium-sized car.}
\begin{table}[htbp]
\caption[Sample of individual level spatial microsimulation output]
{Sample of individual level microsimulation output. The number of cars
in the individuals' household and the engine size of their primary car
are extracted using the merge() ~function applied to the ID codes,
that are also present in \cref{tintallh}}
\begin{tabular}{rrrrrlr}
\toprule
\multicolumn{1}{l}{} & \multicolumn{1}{l}{a\_hidp} & \multicolumn{1}{l}{a\_pno} & \multicolumn{1}{l}{pidp} & \multicolumn{1}{l}{N.~cars} & Engine size & \multicolumn{1}{l}{Et} \\ \midrule
18 & 68041483 & 2 & 68041491 & 2 Sheffield& medium engine - 1.4 - 1.9999 & 312.3 \\
18 & 68041483 & 2 & 68041491 & 2 & small engine - 1.0 - 1.3999 & 268.3 \\
200 & 68303283 & 1 & 68303287 & 1 & inapplicable & 743.8 \\
200 & 68303283 & 1 & 68303287 & 1 & small engine - 1.0 - 1.3999 & 471.7 \\
219 & 68323003 & 1 & 68323007 & 1 & inapplicable & 423.0 \\
219 & 68323003 & 1 & 68323007 & 1 & medium engine - 1.4 - 1.9999 & 312.3 \\
\bottomrule
\end{tabular}
\label{tintcar}
\end{table}

The impact of car engine size on the relative average energy use of each
zone was found to be very small and the correlation between values calculated
that did not take car size into account and values that did was very high
(r = 0.9985). The resulting spatial distribution of energy costs of
commuting at the MSOA level is plotted in \cref{fsoyoen}. This illustrates how
spatial microsimulation can be used to create estimates of energy use at the
aggregate level when cross-tabulated distance/mode datasets are unavailable.
At the individual level, the standard deviation in per trip
energy use is much greater than at the geographical level in this
case study: 95 MJ between individuals compared with only 11 MJ between
MSOA areas. This reflects the impact of geographical smoothing and also
provides an indication of the high level of inequality in energy use for
work travel between commuters living in the same area.

\begin{figure}[htbp]
\begin{center}
    \includegraphics[width=14cm]{soyoen}  \end{center}
  \caption[Commuter energy use in South Yorkshire.]
  {Energy use (direct and indirect) per commuter trip at the
  MSOA level in South Yorkshire.}
 \label{fsoyoen}
\end{figure}

\begin{figure}[htbp]
\begin{center}
    \includegraphics[width=14cm]{prop20-top}  \end{center}
  \caption[Proportion of energy used for commuting by the top 20\%]
  {Proportion of energy used for commuting by the top 20\% of commuters.
  Highest and lowest areas labelled for future reference.}
 \label{fineq20}
\end{figure}

The individual level results are well-illustrated by plotting the proportion
of energy use consumed by different groups. The example plotted in
\cref{fineq20} represents the proportion of energy use for commuting
consumed by the 20\% most energy-intensive commuters, which is also a
proxy for inequality. This plot shows a very clear spatial pattern,
with city centres being associated with the most unequal distribution
of commuter energy costs. We will return to this point in the subsequent
chapter --- for now suffice to say it is an interesting result.
To illustrate the method's ability to disaggregate
by socio-economic categories, \cref{ftopprop} shows the ratio of energy
used for commuting by the top social classes (1.1 and 1.2) compared with
the average energy cost per commute in each area. It is interesting to
note that in all areas the value is above 1.4, reaching more than 3 times
the average in some areas.

In fact, one can use the simulated spatial microdata
to cross-tabulate any combination of variables within any area.
This is illustrated in \cref{tindenergy}, which shows the
link between socio-economic class and commuter energy use
for 3 geographical zones: South Yorkshire overall, as well as the same
relationships in the most and least unequal areas, defined in \cref{fineq20}.
The results indicate that in the centre of Sheffield (`Sheffield 031'),
the lowest classes tend to work closer to home, on average, than the averages for
their class overall and that distance travelled is highly unequally distributed.
In North Stocksbridge (`Sheffield 001'), by contrast, there is much less
difference between different classes. It is also interesting to note that
the average energy intensity of trips in the city centre is lower for all
classes than in Stocksbridge. This can be explained by the proximity to
tram and rail stations and the higher proportion of walking and cycling.
We build on these insights in \cref{Chapter7} to further explore the
inequalities in commuting and commuter energy use in the study region.

\begin{table}[htbp]
\caption[Commuter energy use in South Yorkshire areas by class]
{Average commuter energy use (MJ/trip), distance (km) and energy intensity (MJ/km)
in South Yorkshire (SOYO) by socio-economic class.
The three areas are SOYO and the most and least
unequal zones in terms of the distribution of individual energy use (see
\cref{fineq20})}
\begin{tabular}{rl|rrr|rrr|rrr}
\toprule
\multicolumn{2}{c}{Area $\rightarrow$} &  \multicolumn{3}{|c|}{SOYO}  & \multicolumn{3}{c}{Shef 031} & \multicolumn{3}{|c}{Shef 001} \\
\multicolumn{1}{l}{} & Employment class & \multicolumn{1}{l}{Etrp} & \multicolumn{1}{l}{Dis} & \multicolumn{1}{l}{EI} & \multicolumn{1}{|l}{Etrp} & \multicolumn{1}{l}{Dis} & \multicolumn{1}{l}{EI} & \multicolumn{1}{|l}{Etrp} & \multicolumn{1}{l}{Dis} & \multicolumn{1}{l}{EI } \\
\midrule
% 1 & other & 62 & 14.5 & 4.3 & 87 & 21.0 & 4.2 & 60 & 15.2 & 3.9 \\
 & large employers  & 111 & 27.5 & 4.1 & 141 & 39.6 & 3.6 & 119 & 28.4 & 4.2 \\
 & higher professional & 73 & 17.8 & 4.1 & 102 & 27.3 & 3.7 & 86 & 19.9 & 4.3 \\
 & lower management  & 56 & 14.5 & 3.8 & 66 & 21.9 & 3.0 & 59 & 16.3 & 3.6 \\
 & intermediate & 29 & 8.1 & 3.6 & 17 & 7.5 & 2.3 & 47 & 12.1 & 3.9 \\
 & lower supervisory  & 39 & 10.5 & 3.7 & 16 & 8.8 & 1.9 & 58 & 14.8 & 3.9 \\
 & semi-routine & 20 & 8.4 & 2.4 & 10 & 11.1 & 0.9 & 28 & 13.7 & 2.0 \\
 & routine & 26 & 8.1 & 3.2 & 9 & 5.8 & 1.6 & 42 & 12.7 & 3.3 \\
 \bottomrule
\end{tabular}
\label{tindenergy}
\end{table}

More detailed analysis at the individual
level is presented in \cref{Chapter7}. The results presented in this section
demonstrate that individual level variability in commuter energy use
is important and in some cases potentially more so than inter-zone variation.



\begin{figure}[htbp]
\begin{center}
    \includegraphics[width=14cm]{topprop}  \end{center}
  \caption[Relative energy use by top social classes]
  {Relative energy use by top social classes in South Yorkshire.}
 \label{ftopprop}
\end{figure}

\section{A comparison of commuter energy use in England and the Netherlands}
\label{sinternational}
In order to demonstrate that the methods can be used internationally,
this section provides a short case study, comparing the energy costs
of home-work travel in England and the Netherlands. These countries
were chosen for the following reasons:
\begin{itemize}
 \item Geographically aggregated data could be found for both.
 \item There are reasons to expect the Netherlands to have commuting energy costs
 substantially different from those in England. The working hypothesis we
 set out to test was that the Netherlands would have lower energy costs, primarily
 due to the high uptake of cycling, for which the nation is famous.
 \item The countries are similar `on paper', in terms of population density,
 GDP per capita and culture.
\end{itemize}
The final point is illustrated in \cref{tcompare}, which shows the extent to
which England and the Netherlands are similar according to a handful of basic
measures. One major difference between the two countries is in terms of
income inequality, with England being substantially more unequal.
If only \cref{tcompare} were considered, one would assume that the energy
costs of commuting would be roughly the same in the two countries. However,
a couple of factors led to the hypothesis that commuting in the Netherlands
would be less energy-intensive: its relative size (42,000 km$^2$ vs 130,000 km$^2$
for England) and its famously high rate of cycling, which account for
27\% of trips nationwide and above 50\% of trips in some cities
\citep{Pucher2008}.

\begin{table}[htbp]
\caption{Comparison of basic national attributes in England and the Netherlands}
\begin{center}
\begin{tabular}{llrl}
\toprule
Attribute & England & \multicolumn{1}{l}{Netherlands } & Units \\
\midrule
Population density & \multicolumn{1}{r}{407} & 406 & ppl/km2 \\
GDP & \multicolumn{1}{r}{50000} & 46000 & \$/capita \\
Income inequality & 34 (UK) & 31 & Gini Index \\
Wellbeing & 0.875 (UK) & 0.921 & UN HDI \\
\bottomrule
\end{tabular}\end{center}
\label{tcompare}
\end{table}

\subsection{Data, method and results} \label{sdutchdata}
The input dataset for the Netherlands came in a different form from
that of England. The English data, downloaded from the Census,
provided 88 key columns from which energy values were generated:
8 distance bins for 11 modes of transport. Based on average route distances
estimated for each of the 8 Euclidean distance bins for the 8 modes whose
energy costs are described in \cref{sfinal}, the energy costs per one-way
trip were calculated for each cell in all of the 88 columns. The values in
each of the cells of the English data are people counts, constraining the
number of people in each distance/mode category.
The Dutch dataset, on the other hand, provided proportions, average distances
and average times for 8 modes of transport in a wide format (\cref{tdutch}).
The first challenge upon receiving this dataset was to understand the
table's structure and translate the column headings into English.
Another issue was finding geographical data for Dutch provinces and their
populations (this allowed for the energy costs per province to be weighted,
to provide an accurate estimate of average energy costs per commuter trips
nationwide). This data was provided by the open-data initiative
Natural Earth.\footnote{\href{http://www.naturalearthdata.com/}
{http://www.naturalearthdata.com/}
}

\begin{table}[htbp]
\caption[Sample of the raw Dutch commuting data]
{Sample of the first 4 columns of the raw Dutch commuting data. A further 54
columns on the proportions travelling by and average time and distances of
trips by 9 modes of transport are not shown.}
\centering{\begin{tabular}{lrrr}
\toprule
Perioden & 2010 & 2010 & 2010 \\
Vervoerwijzen & \multicolumn{1}{l}{Totaal} & \multicolumn{1}{l}{Auto (bestuurder)} & \multicolumn{1}{l}{Auto (passagier)} \\
Regio's & \multicolumn{1}{l}{aantal} & \multicolumn{1}{l}{aantal} & \multicolumn{1}{l}{aantal} \\
\midrule
Nederland & 0.48 & 0.25 & 0.03 \\
Groningen (PV) & 0.44 & 0.22 & 0.03 \\
Friesland (PV) & 0.45 & 0.24 & 0.02 \\
Drenthe (PV) & 0.46 & 0.29 & 0.03 \\
Overijssel (PV) & 0.48 & 0.26 & 0.03 \\
Flevoland (PV) & 0.51 & 0.28 & 0.04 \\
Gelderland (PV) & 0.47 & 0.26 & 0.03 \\
Utrecht (PV) & 0.5 & 0.23 & 0.03 \\
Noord-Holland (PV) & 0.48 & 0.22 & 0.03 \\
Zuid-Holland (PV) & 0.49 & 0.23 & 0.03 \\
Zeeland (PV) & 0.47 & 0.27 & 0.03 \\
Noord-Brabant (PV) & 0.47 & 0.28 & 0.04 \\
Limburg (PV) & 0.46 & 0.28 & 0.03 \\
\bottomrule
\end{tabular}}
\label{tdutch}
\end{table}

Finally, the commuting dataset was matched to the geographical shapefile
data in %!!! add reference of the files where this is done.
R.\footnote{Initially
this stage was problematic, as was discovered when the regions were
plotted with their name codes highlighted: the names were not associated
with the correct geographical areas. The R code used was reviewed at
each stage and it was discovered that the error was introduced through
the ``merge()'' function, which allocated the tabular data to the
geographical data by matching the zone codes. It was found that the
default (silent) default argument of ``merge()''~is ``sort=TRUE'' .
This meant that the function was re-ordering the geographical data
alphabetically. Adding ``sort=F'' ~into the command solved the problem.
}
Despite these data preparation issues, the Dutch dataset
was in fact easier to convert into average energy costs per trip than
the UK data, as it was simply the product of mode efficiency ($Ef$), average
route distance ($dR$) and modal split ($p$) for each mode:
\begin{equation}
 Etrp = \sum_m p_m \times Ef_m \times \bar{dR}_m
\end{equation}
This formula was applied to Dutch regional data, and aggregate energy costs
were calculated for England using the method described in \cref{snational}.
The results, illustrated in \cref{fdutchen}, came as a surprise:
energy use for commuting is \emph{higher} in the Netherlands, which is relatively
small, bicycle-friendly and has a low GDP, than in England. The difference
is not as great as that represented in \cref{fdutchen} (a 14\% difference,
when energy use per trip is averaged across all zones), because the zones
are not of equal population or size. When commuter energy costs are
weighted by population, the overall average energy cost per commuter trip
is still higher in the Netherlands, but less so --- 8\%:
37.5 MJ/trip in the Netherlands against 34.5 MJ/trip in England.

\begin{figure}
\centering{
\includegraphics[width=12cm]{dutchen}}
 \caption[Comparison of commuter energy use in England and the Netherlands]
 {Comparison of commuter energy use in England and the
Netherlands.} \label{fdutchen}
\end{figure}
\subsection{Explaining Dutch commuter energy use}
To explore this non-intuitive result, the first stage was to look at the modal
split of commuting in England and the Netherlands (\cref{fdutchmode}).
As expected, Dutch commuters are far more likely to travel to work by bicycle.
However, they are also less likely to travel to work by walking, as a car
passenger or by metro (due primarily to the London Underground) --- all low-energy
modes --- than UK commuters. The proportion of people travelling by car,
the most energy-intensive personal travel mode, is  only slightly lower in the
Netherlands (57\%) than in England (60\%) despite the 27\% of trips made by
bicycle. Modal split cannot account for
unexpectedly high Dutch commuter energy costs.

\begin{figure}
\centering{
\includegraphics[width=12cm]{envsnl-modesplits}}
 \caption[Modal split of commuter trips in England and the Netherlands]
 {Modal split of commuter trips in England and the
Netherlands.}\label{fdutchmode}
\end{figure}

The next variable explored was distance. The average Dutch
commute for the major forms of transport %%% The overall average is 17.6km!!!
is 1 km further than the English average at 15.5 km, from the data.
This may seem like a small amount, yet it is almost 7\% further, accounting for
most of the variability in energy use. When we
break this figure down by mode, as in \cref{favdistnl}, it becomes clear that
car trips are the reason for the increased distance of travel to work
in the Netherlands: all other modes are associated with shorter trip distances,
whereas the average commuter trip by car, the most energy intensive transport
mode, is \emph{30\%} further than in England (24.6 km in
the Netherlands, compared with 18.7 km). It therefore seems that
the prevalence of one particular trip type --- long car trips --- explains why
commuter energy use in the Netherlands is greater, per person, than in the UK.

To explore the underlying reason for these high-distance car commutes,
the length of motorway in each country was found. In the Netherlands
there are 2631 km of motorways whereas in the
England there are 3673 (Eurostat, 2013, via the UK Data Service). These
values equate to roughly 150 km of motorway per million people in the Netherlands,
compared with only 70 km per million in England, less than half. Despite this
advanced road network, and the bicycle infrastructure for which Holland is
famous, road congestion is a known problem \citep{OECD2010}. The average time for commutes
in the Netherlands is longer than for any other nation in the Organisation for
Economic Cooperation and Development, something that has been attributed to
high population density and a rigid housing market: ``more than just transport
policies are required to solve these problems'' \citep[p.~8]{OECD2010}.
% !!! spatial distribution???

\begin{figure}
\centering{
\includegraphics[width=12cm]{avdist-nl-en}}
 \caption[Distance of commuting by mode, England and the Netherlands]
 {Average distance of commuter trip by mode in England and the
Netherlands.}\label{favdistnl}
\end{figure}

Regarding the spatial distribution of energy-intensive commuting,
there is no clear pattern at this coarse level of geographical aggregation.
A pattern does emerge when energy use is plotted against
population density (\cref{fepdensnl}), which shows a strong negative correlation
(r = -0.7, p $<$ 0.001) between the two variables. The two clear outliers in
terms of energy use are London (20.8 MJ/trip) and Flevoland (54.8 MJ/trip),
which are also on opposite ends of the population density scale.
\Cref{fepdensnl} is also useful as it shows there is a large amount of
overlap in commuter energy between the two countries, even at this high
level of geographical aggregation. Three English regions
(the South East, East of England and the East Midlands) have average
commuter energy costs above the Dutch national average; interestingly
each of these zones is quite wealthy, with strong links to London
(implying commuting to London may be a cause of high energy use here).
The only Dutch province with average commuter energy costs below the
English average is Zuid (meaning South) Holland. This area has a very high
population density and includes large cities including the Hague and
Rotterdam.

\begin{figure}
 \centering{\includegraphics[width=12cm]{epdensnl}}
 \caption[Population density against commuter energy use]
 {Population density against commuter energy use, in Netherlands and England.}
 \label{fepdensnl}
\end{figure}

\subsection{Data inconsistencies and caveats}
A problem with the preceding national level comparison is that the
data come from different years, 2001 and 2010 for England and the Netherlands
respectively. One could argue that this is not an issue
from the perspective of demonstrating the international applicability of the
methods. However, it is a major problem if the aim is to use the empirical results
to inform policy.
for example to argue that a focus on modal split alone may not be  effective at
increasing the sustainability of personal travel, if distance is not considered
as well. %!!! add a link here
That energy use per commute is greater in the Netherlands
than in England is an interesting result in itself and merits
corroboration with additional data to confirm this result.

\Cref{fcommuterdistime} shows that the length of commuter trips in Great
Britain (including Wales and Scotland) has remained steady over
time. It increased by only 5\% between 1995/1997 and 2009 and only by
1\% between 2002 (the closest data point to 2001) and 2009. In addition,
\cref{fmode-time-dft2011} demonstrates that the modal split of commuter trips
has also been relatively steady, with slight declines in car use suggesting that
energy use may have even declined. 

\begin{figure}
 \centering{\includegraphics[width=12cm]{commuter-trip-dis-time}}
 \caption[Average commuter trip distance over time in Great Britain]
 {Average commuter trip distance over time in Great Britain. Data from
 \citet[table 9]{DfT2011-commuting}, n $>$ 15,000 for every year.} \label{fcommuterdistime}
\end{figure}

\begin{figure}
 \centering{\includegraphics[width=12cm]{mode-time-dft2011}}
 \caption[Modal split of commuter trips, Great Britain 1995 - 2009]
 {Modal split of commuter trips, Great Britain 1995 - 2009. Data from
 \citet[table 9]{DfT2011-commuting}, n $>$ 15,000 for every year.} \label{fmode-time-dft2011}
\end{figure}

Another issue is data quality. While both datasets
are from official sources, the Dutch dataset is far less detailed and provides
only two significant figures for the proportions of people travelling by each
mode (e.g. 0.01). Thus, error up to 0.5\% in these figures is possible.
Further, average distances were not provided for all modes of transport in all
areas, in which case the mode's average figure for the areas that were reported
were used to fill in the gap. Finally, the figures for the proportion of people
travelling by train seemed very low, given that the Netherlands has an
advanced rail network. As outlined in \cref{Chapter4}, %!!! really???
there are also issues with the UK dataset. The translation of
Euclidean distance
categories into average route distances is a particularly risky
activity and may introduce error in excess of the difference between
Dutch and English average commuter trip energy costs reported above.

In light of these caveats, it is concluded that a 
more robust dataset from the Netherlands is needed to resolve the
enigma of high Dutch commuter energy use. The basic method used to calculate
energy costs has been shown to be applicable to another country,
although more refinements (e.g.~alterations in the average energy
intensity of Dutch cars) will be needed if this result is to be
seen as robust. If it holds up to further investigation, it is an interesting
and policy relevant result: it would illustrate that promotion of urban
cycling alone is not enough to reduce the overall energy costs of personal
transport nationwide.


% \section{Changes over time}
% % \section{Changes in energy use over time} \label{stime}
% % Mothballed for now - now that interesting, and not needed atm
% As seen in the gradual improvements in car fleet efficiencies reported in
% \cref{Chapter5}, the rate of energy use in transport systems is in constant
% flux due to changing technologies. While efficiency improvements tend
% to be gradual and predictable (due to regulation, and the long lead times
% in car model development), human behaviour is not. It can respond rapidly
% and unexpectedly to external factors, such as the oil price
% spike of 2008 \citep{Sexton2011} and can be shifted
% by emergency measures such as food rationing, conscription and
% enforced agricultural and mine labour policies in World War II.
% 
% More mundane shifts in behaviour have affected the energy costs of commuting
% in recent years, including the rise of `telecommuting', `flexi-time', increased
% labour mobility and the tendency of people to start families far from their
% home. Over the last 300 years the most important change
% in commuting behaviour has been its emergence as a common activity:
% before the industrial revolution and
% the emergence of centralised factories,
% many people worked in `cottage industries' at home. A common arrangement in rural
% areas in the
% 1800s, for example, was for the men to work long hours (50 + per week) on
% neighbouring farmland, while women worked at home \citep{groves1949sharpen}.
% From the historical literature
% it is clear that the energy costs of transport to work
% before the industrial revolution were very low indeed, consisting primarily of
% people walking a mile or two to work and a relatively small number of
% horse-drawn carriages for the wealthy whilst the majority of
% of the population worked from or very near to home.
% 
% To gain insight into commuting energy use before the advent of
% large official databases on work and travel to work
% behaviour, anecdotal and archival evidence must be relied upon. Over the
% past 100 years there has been a trend towards official
% data collection, analysis, storage and dissemination. In the last 50 years,
% this tendency has been greatly accelerated and automated by the `digital
% revolution', as mentioned in \cref{Chapter3}.
% Still, before 1971 (when travel to work is first made available as a
% Census variable) the best source of commuting data that could be
% found was from a retrospective survey of elderly people, asked about their
% past travel to work habits. %!!!
% The longest available dataset on commuting that could be found was collected by
% \citet{Turnbull2000}, in a retrospective questionnaire about commuting habits over
% the past 100 years...

% \section{Explaining high and low energy use} \label{sexplanation} %!!! re-add
% \index{decomposition framework}
% Regardless of technology and the various complicating factors discussed
% in \cref{svariable}, the primary determinants of the energy costs of
% personal transport are mode and distance. Focussing simply on one or the other
% (as other authors have done) omits a substantial part of the picture because it
% is the combination of an energy intensive mode and an energy intensive trip that
% leads to high energy costs. In England as a whole, the most energy intensive
% form of transport (single occupancy cars) is the most common form of transport
% to work for all but the shortest trips \cref{fengmodedis}.

% The aggregate estimate of energy use is interesting in itself, allowing
% commuting energy used to be placed in context of other phenomena.
% % It should be clear that commuting is more important than many other
% % ``energy issues'' such as the efficiency of lights and solar panels,
% % which have received relatively more attention from an energy perspective from
% % policy makers and academicis. !!! put commuter energy use in perspective - fig
% However, one of the aims (A1.2, \cref{s:aims}) was also to explain why energy
% use in transport to work is as it is. As illustrated in previous literature
% \cref{Chapter2},
% many factors contribute to the total energy use in transport systems.
% To recap, these include number, distance and frequency of trips, occupancy,
% mode, fleet efficiencies, infrastructure impacts and behavioural factors such
% as driving style. %!!! refs.
% 
% This section will formalise these considerations using a framework: the
% decomposition framework \citep{}. In terms of total energy use in the economy,
% three main factors (activity, structural and energy intensity effects)
% can be modelled to explain and project growing or declining energy use
% \citep{farla2000physical}.
% This framework can be applied to the decomposition of energy use in, as
% illustrated in \cref{f:components}. Starting with decomposition by mode, the
% formulae to describe each element of the decomposition analysis are formalised
% in equations \ref{e:component1} to \ref{e:componentn}.
% 
% \begin{equation}
%  Etot = {\displaystyle \sum^m_{m=1}
% N_{trips,m} \times \overline{d}_{R,m} \times \overline{E}_{F,m}}
% \end{equation}  


% \section{Uncertainties and complications} \label{suncertainties}
% 
% ``correctness is usually expensive, and high correctness is often
% \emph{disproportionately} more expensive'' \citep[p.~153]{janert2010data}.

% !!! Links with section in ch.4.

% \subsection{The relevance of flow data for the energy costs of travel to work}
% The route taken by a given vehicle can have a large impact on its energy use.
% This can be illustrated by considering cases of bus travel, on various
% locations of the energy efficiency spectrum:
% \begin{itemize}
%  \item dhf
% 
% \end{itemize}
% 
% On one hand,  extreme example could be the
% comparative efficiency per person of an old London bus carrying 20 people
% through clogged streets c

\section{Discussion}
In this chapter the methods and data presented in
\cref{Chapter4} have been combined with the estimates of energy use by mode presented in
\cref{Chapter5} to calculate the energy costs of commuting at a range of scales.
The main unit of measurement used to present these results is
energy use per one-way commuter trip. This is a useful measure, as it is
robust to variations in the employment rate and makes no assumptions about
frequency of trip. If the aim is to compare commuting with other energy-using
activities, however, the results would be more usefully presented as energy
costs per person per day. This approach was undertaken
by \citet{Boussauw2009}, which would allow
direct comparisons between commuter energy use and other `essential'
energy costs such as electricity and gas use in the house and (depending on
data availability), other travel costs. %!!! Add here results of e.use/zone!!!

Despite these limitations, the findings are still useful in their own right.
From inspection of the district and ward level maps, it is clear that dense
urban areas tend to have lower average commuting costs than the countryside.
London is the extreme manifestation of this tendency, and has achieved
commuting energy costs below the national average throughout most of its
wards. However, many of the areas within roughly 100 km but outside
Greater London have unusually high  average energy costs per commute.
This is likely to be due to long-distance commuters and `commuter belts'
which serve London's vast service sector. It is concluded from this
pattern that citywide personal transport costs should not be evaluated
only in terms of the internal flows within them: flows from the surrounding
areas should also be considered.

The results presented in this chapter provide much scope for further research.
The pattern of London as a centre of relative commuting sustainability surrounded by
a ring of high energy costs, for example, raises the following question:
are cities, overall, associated
with lower commuting energy costs than rural settlements, once long-distance
commuting has been taken into account? This question feeds into the ongoing
debate about compact cities and urban forms that are conducive to reduced energy
use \citep{Levinson2012}. Moreover, the descriptive results require explanation.
Is there a model that can successfully explain the variability in energy
use observed, based solely on population distribution and infrastructure?
If so, this would have implications for planning policy, as the energy impacts
of new settlements (e.g.~housing estates) and transport infrastructure could
be predicted.

This potential for policy relevance leads on to the tentative
finding that Dutch commuter trips are, on average, more energy intensive than
English ones. This, if it was confirmed, would strongly suggest that simply
trying to emulate the Netherlands in terms of rates of urban cycling
would not guarantee environmental and other benefits of lowered energy use.
The finding supports the conclusion of \citet{Boussauw2009}, that
interventions aiming to reduce the distance between
home and work may be more effective than those aimed at changing
modal split.

Before exploring some of these broad policy-relevant questions 
in \cref{Chapter8}, the next chapter zooms-in, to a single case-study area.
This is to illustrate the ability of the spatial microsimulation approach to
explore local commuting patterns and evaluate specific transport interventions.

 % Results and Discussion

% Chapter 7

\chapter{Social and spatial inequalities in commuter energy use} % Write in 
\label{Chapter7}
% \lhead{Chapter 7. \emph{Social and spatial inequalities in commuter energy
% use}}
\fancyhead[RO,LE]{Chapter 7. Social and spatial inequalities} %2side
\fancyhead[RE,LO]{\thepage}
% Latest plan (May) --- add complete vulnerability paper in here to save time
\begin{quotation}
\textit{ There  are many options open  for manipulation of the transportation
system,
and many  impacts on different groups which must
be considered. Prediction of the impacts associated with
a particular set of options requires prediction of the corresponding
pattern of flows which will occur in the multimodal transportation
network, using a complex system of models.}
\end{quotation} {\flushright \citep{manheim1968search}}

\section{The importance of distributional impacts in transport studies}
At the sub-national level, the relative costs and benefits of climate change-related
policies are highly uneven. It has been calculated, for example that the bottom
10\% of households by income will benefit least from the government's domestic
energy policies such as those contained in the Green Deal \citep{JRF2013-distributions}.
This, the authors point out, is unfair on three levels: poor people are least
able to deal with the impacts of climate change; they pay proportionally more for the
mitigation strategies; yet they have contributed least to the problem: the top
10\% emit 3 times more emissions than the bottom 10\%, excluding indirect emissions
caused by the products and services they consume.

At the aggregate level, literature shows that behaviour varies depending
on a range of factors including distance to employment
centres,
transport infrastructure and the number of local employment opportunities.
Social characteristics are also closely linked with commuting behaviour,
as illustrated by DfT data on the average distance travelled to work
by mode, cross tabulated by household income (\cref{fig:income-dis}). Transport
modelling, and especially the related discipline of transport engineering, have
tended to be `hard' subjects, focussed only on the technological performance of
transport interventions. However, as implied by the quote that begins this
chapter, \emph{all} transport interventions will have some kind of
distributional impacts, either favouring certain places more than others or
certain groups of people.

The dangers of omitting such social considerations from the analysis were
recognised early in the history of transport and urban modelling. In fact,
ignorance of distributional impacts was implicated as
one of the reasons for the perceived failure of the first generation of urban
models in the 1960s: ``disillusionment with technology began to grow as
planners and politicians began to realise that long-term planning of
transportation and land use [which the models focussed on] had little or nothing
to do with more immediate problems of poverty and inequality''
\citep[p~10]{batty1976urban}. This problem continues today (see
\citealp{Tribby2012} for one example), providing a strong remit for this
chapter and its focus on including social factors in the evaluation of travel
patterns and future interventions. Before moving on to the core results of this
chapter --- a case study of inequalities in commuting patterns and energy used
in South Yorkshire --- it is worth considering a few national statistics on
the relationship between socio-economic variables and transport to work, for
context.

\begin{figure}[htbp]
  \centerline{
    \includegraphics[width = 14 cm]{./Figures/Income-dis-GB}}
    \rule{35em}{0.5pt}
  \caption[Trip distance and mode by household income]{Average
distance of commute by mode by income quintiles in Great Britain
in 2009. Data: \citep[Table 6]{DfT2011-commuting}.}
  \label{fig:income-dis}
\end{figure}

\Cref{fig:income-dis} illustrates that social inequalities are manifested
not only in income and material goods but also in terms of the
daily trip to work. Workers in the top 20\% of households by income
commuted on average 8 times further during 2009 than those from the
bottom 20\%. From one income quintile to the next, average distance
almost doubles in every case, with the difference slowing only slightly towards
the
top quintiles.\footnote{Distance
travelled to work increased by a
factor of 1.8, 2.0, 1.5 and 1.4 between Q1 and Q2 in the
first instance to Q4 and Q5 in the last.
}
It is notable from \cref{fig:income-dis} that wealthier people
also tend to use more energy-intensive modes. However,
the variability in mode of transport is far lower than the
variability in distance (\cref{income-heatmap}).

\begin{figure}[htbp]
  \centerline{
    \includegraphics[width = 14 cm]{./Figures/income-heatmap}}
    \rule{35em}{0.5pt}
  \caption[Heatmap of mode of travel by income group]{Proportions
of trips made by mode of transport in Great Britain, 2009.
Data: \citep[Table 6]{DfT2011-commuting}.}
  \label{income-heatmap}
\end{figure}

These overall findings provide a strong message to policy makers:
policies encouraging behavioural change may be most effective
if they target particular groups of commuters.
This differs from blanket policies such as efficiency-related
tax bands which inherently assume commuter patterns are homogeneous.
At sub-national level, such variability depending on socio-economic
status should also be taken into account by local planners.
However, in many cases, the data or analysis capabilities are
not available to target particular groups living in particular areas.

With these motivations, the present chapter builds on the kind of breakdowns in
commuter behaviour
by socio-economic variables illustrated in \cref{fig:income-dis}, but
at lower levels. This is where the simulated individuals provided by
spatial microsimulation really come into their own, as aggregate data tell
us little about the socio-economic attributes of the individuals
that make up aggregate commuter patterns.

The following presents results which tackle these issues.
Because the spatial microsimulation model assigns characteristics to every single
working person in the study area, the analysis becomes unwieldy when applied to
very large areas. (The IPF model took 30 minutes per iteration when applied
to the 2 million commuters of Yorkshire and the Humber on an Intel i5 `Sandy
Bridge' computer with 12 Gb RAM). Age/sex, mode,
distance and social class categories were used as the constraints, from which
a wide range of simulated results were generated.

% \section{A case study of commuting in South Yorkshire}
%%% Try inserting all JTRG paper here - done (ish) how do I reference this???
As noted in chapters 1 and 2, commuting is a major reason for personal travel,
and a broad research area within transport geography. In many cases zonally
aggregated census statistics --- often the most reliable source of
information about spatial variation in commuter patterns --- form the basis of
geographical commuting research
\citep{Horner2002,Titheridge2006}.
Advances in data availability
and computational methods  have facilitated
the analysis at the individual level, as outlined in \cref{Chapter4}.
This trend --- towards micro level social and spatial analysis
--- has several potential benefits
for decision makers. It is the aim of this section to highlight these benefits
and provide useful insights into the link between socio-economic attributes
and commuter behaviour. The case study region of South Yorkshire is the same
as that used in \cref{Chapter6}, for continuity. The results showcase the
potential benefits of spatial microsimulation:
\begin{itemize}
 \item the ability to target specific \emph{types} of commuters
\item the possibility of modelling the impacts of small scale interventions
(e.g.~a new bicycle path or bus lane) on individuals living in the local area
\item higher spatial resolution than is provided by aggregate data for certain
cross-tabulated variables (e.g.~mode and distance). This could provide insight
into the impacts of change on network usage (e.g.~identify likely points of
congestion)
\item a foundation for agent-based and dynamic microsimulation models.
\end{itemize}
% Maybe add refs here.

The shift towards micro level analysis also has some potential
disadvantages. These limitations,
and strategies to overcome them, can be summarised as follows:
\begin{itemize}
 \item The individual level results are simulated, and are unlikely to be
 totally representative of the zones in question. We can have confidence
 in the constrained variables (although large bin sizes for continuous
 attributes such as age may not fully capture unusual
 distributions),\footnote{The
 distance bins presented in Table \ref{t:constraints}, for example,
 are quite widely spaced. In a situation where many people
 travelled a distance close to the edges of one of these bins --- for example
 due to a factory located 11 km from an employment centre --- the results,
 which would represent an even distribution of
 all individuals in the sample who 10 to 20 km to work, would
 be inaccurate.}
 but the target variables are simply the result of their relationship with
 constraint variables at the national level. This can be tackled through
 validation methods (see \citealp{Edwards2009}, and
 below) or, in the long run, through increased access to real spatially
 disaggregated
 microdata.\footnote{For example, a dataset of geo-coded
individuals and their workplaces provided by Finnish government allows
destination/origin analysis and insights into the directions of flow
\citep{Helminen2007}
}
In fact, awareness of the policy insights offered to researchers
by spatial microdata could encourage the release of real
geographically disaggregated microdata (see \citep{Lee2009}).
% \item Better energy data - limitations with EU ways of measuring things,
% discrepancies between DfT data and DECC data.
\item Lack of accurate distance travelled estimates in the main model
(currently broad distance categories are used). This could be overcome by creating
more accurate origin-destination pairs for individuals. Lower level commuter
flow data (compared with the data presented in Fig.~\ref{fig:sflow}) is
available to do this.\footnote{Commuter flow
datasets of the type presented in Fig.~\ref{fig:sflow} are available at the much
smaller Output Area level
(from
\href{http://www.ons.gov.uk/ons/guide-method/census/census-2001/data-and-products/data-and-product-catalogue/origin---destination-statistics/output-areas/index.html}
{the Office of National Statistics}). However, the data are available only on a DVD, with the following proviso:
``analysis [of the Output Area commuter flow data] requires the use of
specialist software, which is not supplied with the product, but which is
available from intermediary organisations (for more information contact Census
Customer Services).''
}
Also, undertaking network analysis of roads, railways, and walkways (see
Fig.~\ref{fig:agent} for an example) for all individuals could allow more
accurate estimates of route distance. However, this is computationally
challenging, although increasing feasible \citep{gao2010comparison}.
\item Omission of explanatory variables such as car parks, the quality of paths,
and even the provision of showers for cyclists at work destinations. These
variables can be included by appropriate survey questions \citep{Buehler2012}
or analysis of environmental variables \citep{Rietveld2004}.
\end{itemize}
Each issue presents a major methodological challenge, but none
of them invalidates spatial microsimulation as a modelling tool to
better understand travel behaviour. These issues are partly tackled in
Section \ref{s:valid} and their implications discussed in the final section of
this case study.

These include greatly increased computational requirements for
analysis, lack of available software or expertise, and the pitfalls of
overcomplexity. As \cref{Chapter3} shows, new techniques for spatial
microsimulation, which model individual characteristics and behaviour,
can overcome the majority of these problems.
A more fundamental barrier preventing the use of micro level
methods in many contexts is that accurate, geocoded
microdata are simply unavailable. In the UK, for example, census-derived
microdata are made available only as a Sample of Anonymised Records (SARs) at
coarse geographical levels
\citep{Dale2002}.\footnote{The
SARs are divided into two parts: the 2\% SAR, which
allocates each individual to a geographic region with a population
size of at least 120,000 (narrowing-down the results to one or more Local
Authorities), and the 1\% sample, which allocates each individual
to countries \citep{Dale2002}.
}
More specific surveys (such as the UK's National Travel Survey) can provide
further insight into travel patterns at the
individual level but these also omit high resolution geographical
information to protect participants' anonymity.
% \footnote{The Understanding Society
% dataset referred to in this paper, for example, only provides detail on the
% region in which each individual lives.}
% 
% Spatial microsimulation techniques, of the type described
% in this paper, hold great potential benefits for transport planners and
% policy makers who lack access to official, geocoded microdata (individual level
% data allocated to small areas).
% With such `spatial microdata', new analysis options are created, including
% route choice between origin destination pairs, localised intervention evaluations
% and cross-tabulated contingency tables. These applications should also be
% of use in the rare (yet increasingly common) situations
% where official geolocated microdata are provided.
% In the UK, as in many other countries, spatial microdata must be simulated, as
% reliable secondary data sources are limited to 1) zonally aggregated census data,
% and 2) non-geographical, individual level ‘microdata’ from national surveys.
% % This raises the following question: how can micro level analyses of
% % commuter behaviour be conducted under such data constraints?
% This paper builds on the pioneering theoretical work on spatial microsimulation
% and applies it to the issue of commuting.


% The spatial microsimulation model used in this paper is described
% in detail in \citet{Lovelace2013}. The code is written in the open source
% statistical language R \citep{R2013}, so the results are
% reproducible.\footnote{Provided the same input data and
% suitable computer, the results presented in this paper can be reproduced by anyone.
% Supplementary datasets are not published alongside the paper in this case, however,
% due to conditions of
% use. People wishing to reproduce the results presented in this paper are
% therefore asked to contact the authors by email. Reproducibility has
% huge potential to increase not only the transparency of academic research,
% but also that of the planning process. Testing of the methods is also
% encouraged for educational purposes.
% }
% The methods are illustrated using a case
% study from South Yorkshire in the UK.
% The aim
The more practical aim of this section is to bring micro level analysis
within reach for transport planners and
researchers already acquainted with aggregated census data on commuting.
Detailed non-geographical microdatasets on commuting already exist,
but many analyses for evaluating the
impact of commuting policies require \emph{spatial} microdata. As indicated
above, there are a number of  reasons why such spatial microdata may be
needed: planning for more sustainable commuting is a complex problem that
operates on a range of scales, including that of individuals
\citep{Vega2012, Verhetsel2010}. In the words of \citet[p.~313]{Li2012}, ``a more spatially
disaggregated method is needed''. To summarise the research problem,
tools to aid the design and evaluation of policies affecting commuters are needed.
These tools should be flexible, able to operate at a range of
levels and shed light on various issues, from the potential of telecommuting
(where internet access facilitates working from home, saving transport fuel)
to levels of access to public transport, walkways and cycle paths.

% The remainder of this chapter is organised as follows:
% % Section \ref{s:litrev} reviews relevant literature on commuting,
% % transport modelling and
% % spatial microsimulation, highlighting the  potential benefits of incorporating
% % individual level socio-demographic data into transport studies.
% % ; it also briefly introduces spatial microsimulation and highlights the
% % potential for applications in transport.
% Section \ref{Methods} outlines the data and methods required to fulfil this
% potential, and and shows how spatial microsimulation has been implemented
% in this chapter. Section \ref{results} presents some outputs from the
% spatial microsimulation model. The purpose is to illustrate the new types
% of analysis opened-up and policy relevance of distributional impacts.
% Finally, in Section \ref{discuss-jtrg}, these results are discussed
% and placed in the context of current practice in transport planning and
% policy evaluation.
% 
% \subsection{Literature} \label{s:litrev}
% % \subsection{Understanding commuting} !!!delete this section?
% Commuting has been a topic of research for many decades,
% reflecting its role in relation to economy, to
% individual and household well-being and, increasingly, to environment.
% From this extensive literature, it is apparent that commuting should, in theory,
% be relatively easy to model. This is because journeys to work tend to be:
% \begin{itemize} %!!! Maybe re-add these bullet points - still, it's lit rev.
%  \item Regular, occurring on a near-daily basis for most people and following
% predictable hourly, weekly and annual patterns \citep{Akkerman2000}.
% \item Non-discretionary --- work trips, unlike trips made for socializing
% and holidays, are an essential part of daily working life. In
% other words, the demand for commuter travel is non-elastic, and responds slowly
% to changes in the cost of travel \citep{Depalma2012}.
% \item Destination-constrained. It is often challenging to change one's
% work location (e.g.~after moving house), as embodied in the common
%  assumption of fixed workplaces \citep{Vega2009}.
% \end{itemize}
% These characteristics mean that commuting flows should follow more regular
% patterns over space and time than travel for other purposes, such as holidays
% or shopping. In addition, commuting statistics are widely available from
% national censuses, which often contain a question on travel to work.
% This data availability and relative predictability has made commuting
% well-suited to academic research, and a number of methodological advances
% have been demonstrated using travel to work statistics.
% 
% This is well illustrated by comparing the methods of \citet{Ibeas2012}
% with those employed 16 years earlier by \citet{Forrest1996}. In the
% former, four (increasingly complex) spatial econometric models were harnessed to
% investigate links between house prices and commuter accessibility.
% The latter used a single linear regression model to explore the house-price
% accessibility relationship with respect to a case study of Metrolink,
% a light rail scheme in Manchester. Increased range and complexity of
% methodologies can also be seen by
% comparing the descriptive methods used by \citet{Knowles1996} with the
% statistical tests employed by
% \citet{Senior2009} for exploring the transport impacts of the same
% scheme.\footnote{The former study harnessed
% descriptive statistics based on primary data and hand-crafted maps to
% investigate the transport impacts of Metrolink. The
% latter employed multiple regression and chi-squared tests of survey data to
% identify longer-term changes in behaviour attributed to the light rail system.}
% 
% The most recent major methodological advance to use commuting data is the
% radiation model \citep{Simini2012}. Based on census-derived
% inter-county commuter flow data
% across the USA, \citet{Simini2012} developed a probabilistic method of
% predicting the flows between any two zones, based only on knowledge of
% population and employment. If the claims stand up to further tests, this
%  could represent
% a step forward in the modelling capabilities of transport geographers
% \citep{Brockmann2012}, for example by allowing individual trips to be
% predicted and by providing realistic estimates of commuter flows in
% areas where no flow data is available.
% In general, however, methods for investigating commuting have advanced
% gradually, in-line within the `normal science' of transport geography.
% In addition, most modelling efforts have been constrained to the geographical
% scale at which data is made available.
% 
% Despite the advances outlined above many geographic approaches
% for analysing commuting patterns
% operate only at a single level of analysis. This is often the lowest geographical
% level for which the required data are available. Indeed, prior to the 21$^{st}$
% century, personal transport
% models tended to be simplistic, assuming `mono-centric' cities
% (see Fig.~\ref{fig:dis-msoa}) and taking little or no account of
% geographic factors beyond distance \citep{Akkerman2000, Horner2002}.
% This was problematic for practitioners aiming to evaluate interventions,
% the impacts of which may be geographically heterogeneous and highly localised
% (e.g.~bicycle paths) or focused on
% specific socio-economic groups (e.g.~telecommuting).
% Due to data, software and computing limitations, evaluations of the
% impacts of policies affecting personal transport have tended to be
% over-simplistic, considering only a single scale of
% analysis.\footnote{See, for example, \citet{Lovelace2011-assessing} for
% a non-geograhical example of city level aggregation,
% and \citet{Li2012} or \citet{Titheridge2006} for analyses that use
% only a single geographical level of analysis.
% }
% Ideally, however, macro
% (geographic) \emph{and} micro (individual level) factors would be
% included. The efforts towards such an approach ``that integrates [spatial]
% demographic microsimulation with urban simulation and travel demand'' are
% making progress and could signify a major step forward for personal transport models
% for policy evaluation \citep[p.~4]{Ravulaparthy2011}.
% % \citep{Vega2011}
% Increasingly, newly available micro level datasets are being incorporated
% into geographical analyses of personal travel and commuting in particular
% (the next section provides examples of this work).
% 
% \subsection{Incorporating the micro level}
% % This micro level section needs a bit of work mate (5th Sept)
% % E.g. make GPS stuff hang with previous paragraph with next...
% \label{s:lit}
% % At the global level, dynamic models of transport-related energy use have been
% % developed. These take advantage of the fact that, at high levels of
% % aggregation, transport behaviour is relatively predictable proportion. For
% % example, it has been found that 15 to 20\% of income is spent on personal
% % travel in most industrialised nations; in poorer countries with reduced access
% % to cars, the proportion drops to 3 to 5\% \citep{Schafer1999}. This knowledge
% % allows transport energy costs and emissions to be projected, and broken-down
% % world region, based on future income estimates. The averaging effect of large
% % scale aggregation has also been used to calculate the future energy use of
% % existing infrastructure overall \citep{Cald2010}; transport infrastructure
% % contributes $\approx$ 23\% to the total (2/3 of this due to roads). A more
% % complex model, based on assumptions about the costs and availability of
% % next-generation fuels and vehicles, implies that climate change mitigation in
% % the transport sector requires policy intervention \citep{Gul2009}. In each
% % of these studies the geographic variability of transport systems is
% % noted\footnote{The stagnating tendencies for transport demand in the West
% % contrast with the rapid rates of car ownership and transport-related energy use
% % in the developing world.} yet lower level factors such as international
% % variation in work-home distance are not considered. No global model quantified
% % the variability in the main reasons for personal transport demand (e.g.~the
% % large impact of leisure travel and flying in high income nations). Studies
% % investigating the energy implications of commuting begin only at the
% % national level.
% As described in \cref{Chapter4}, modern computers facilitate the simulation of
% hundreds of thousands
% of simultaneous trips. A good recent example illustrating this
% is the work of \citet{Ferguson2012}, who used microdata on
% company location in combination with the road network to produce
% traffic simulations at high spatial and temporal resolutions. A major advantage
% of such detail is the opportunity to test our understanding of transport systems
% directly, through prediction and corroboration. The close fit between simulated
% and independent observations made of commercial vehicles by
% \citet{Ferguson2012}, in both space
% and time dimensions, illustrates the potential of combining microdata with
% geographical inputs for policy analysis. In the realm of public transport,
% \citet{Tribby2012} combined demographic data of small areas with bus and
% walking networks for Albuquerque, New Mexico. The results
% of this study (which is further discussed in section \ref{discuss-jtrg})
% were used to evaluate the accessibility impacts
% of new bus routes. It was found that the impacts varied greatly between
% neighbourhoods and, crucially for social justice, that
% disadvantaged groups benefited \emph{least} from the intervention.
% From a methodological perspective, \citet{Tribby2012} used the study to
% highlight the importance of geographical \emph{and} socio-economic
% disaggregation of results. While the preceding literature is new, it
% is worth noting that the benefits of including spatial and non-spatial
% factors in personal travel analysis have been expounded since the 1970s
% \citep{Horowitz1986}. What is new is the
% widespread availability of data, computers and software to meet the challenge.
% 
% A couple of national level studies serve here to illustrate the utility of
% analysing spatial microdata for the geographical investigation of commuting
% patterns. \citet{Helminen2007} investigate the relationship
% between distance from workplace and telecommuting in Finland. They used an
% individual level geolocated database of all 2 million workers to calculate
% average trip distances and total annual distance travelled. As the authors
% note, ``distance is a basic characteristic of the spatial pattern of
% commuting'' \citep[p.~333]{Helminen2007}, yet it is difficult to calculate
% accurately in practice: Distance data are usually `Euclidean' (provided as a
% straight line between home and work), yet the actual route distance travelled is
% almost always longer and invariably difficult to calculate. Network analysis
% methods have recently emerged to overcome this problem  \citep{Ehrgott2012,
% Levinson2009}. However, these methods would be difficult to conduct at
% the national scale: \citet{Helminen2007} tackle this issue by explicitly using
% Euclidean distance and citing estimates of circuity (the ratio of route distance
% to Euclidean distance).
% % As we shall see, the discrete agents provided by spatial
% % microsimulation can be used as the basis for estimating route distance, based on
% % knowledge of Euclidean distances and inter-zone commuter flows from the census
% % (section \ref{s:workdes}).
% The results illustrate the utility of geographically disaggregated
% microdata.\footnote{In this case
% for calculating the transport impacts of telecommuting in Finland
% and identifying the characteristics of telecommuters
% \citep{Helminen2007}.
% }
% 
% % Data
% % on place of work was harnessed to investigate the direction of flows.
% Another recent application of individual level geolocated census data to
% commuting policy was the investigation of the impact of location (relative to
% railway stations and bus stops) on sustainability of work travel in
% Flanders \citep{Verhetsel2010}. As with \citet{Helminen2007}, a problem
% encountered was the sheer size of the raw commuting database: 1.2 million
% individuals. This problem was overcome by aggregating the results into small
% areas (each containing around 130 people). The diversity of the data was
% tackled by classifying small areas into 5 groups, depending on the number of
% train stops made in each per day. Simplifying classifications may be an
% important way of interpreting complex spatial data, as will be seen in Section
% \ref{Methods}.
% 
% With the increasing availability of individual level transport
% data geographical methods
% for analysing them, that are accessible to transport planners,
% have (in general) struggled to keep up.
% Notable exceptions include the work of \citet{Bhat2004}, who presented
% an econometric microsimulation approach
% to modelling daily travel patterns, and \citet{Guo2007},
% who refined the iterative proportion
% fitting procedure of \citet{Beckman1996} to
% create accurate synthetic microdata for
% transport modelling applications. However, in neither case
% are methods for the \emph{geographic}
% analysis of the microdata results presented.
% \citet{Buliung2006} addressed this problem by developing bespoke
% extensions to ArcGIS software.
% Their toolkit facilitated the geographical analysis
% of the travel spaces of households based on a detailed travel-diary
% dataset. The research illustrates the potential for
% new software to pose relevant hypotheses and visualise travel patterns.
% The research agenda pursued by \citet{Buliung2006}
% raises the following questions: Can the
% behaviour of \emph{all} citizens in a study area
% be simulated (rather than just the survey respondents)?
% How can methods of individual level transport analysis be
% presented and disseminated such that they are used by
% others?\footnote{The ArcGIS-based methods of individual level analysis and
% visualisation advocated by \citet{Buliung2006} appear, based on
% the academic literature, not to have adopted by researchers using microdata.
% None of the 59 articles
% citing \citet{Buliung2006} in Google Scholar
% % (\url{
% % http://scholar.google.co.uk/scholar?hl=en&lr=&cites=12668077024949976043&um=1&
% % ie =UTF-8&sa=X&ei=oXpYUP_KIcem0QW65ID4DA&ved=0CCsQzgIwAA})
% (September 2012) reported using their software to investigate travel patterns
% using microdata, instead dealing with
% the broader concept of activity spaces. This is despite the efforts
% made to ensure the software was user friendly,
% with the addition of a graphical user interface
% \citep{Buliung2006}.}
% \citet{Goulias2005} presented an activity-based
% approach to the analysis of travel demand and
% travel schedules taking into account household characteristics.


% Spatial microsimulation allocates individual level data to areas (see
% \citealp{Ballas2005b} for an overview).
\section{Model implementation} \label{Methods}
The method requires both aggregate and individual
level datasets described in \cref{Chapter4} to
share at least one `linking variable'. These linking (or constraint)
variables, described in Table \ref{t:constraints},
preferentially sampled representative
individuals, in this case via IPF, which was introduced in
\cref{Chapter3}. The target variables (Table
\ref{t:indata}) are thus simulated.

\begin{table}[htbp]
\caption[Aggregate level inputs into the spatial microsimulation model]
{The four constraint variables and their associated categories used as
the aggregate level inputs into the spatial microsimulation model.
The category notation for numeric variables follows
the International Organization for Standardization
\href{http://www.iso.org/iso/catalogue_detail?csnumber=31887}
{(ISO) 80000-2:2009}:
Square brackets indicate that the endpoint is not included in the set,
curved brackets indicate that the endpoint is included.}

\begin{center}
\begin{tabular}{lrp{3cm}p{8cm}}\toprule
Variable & \multicolumn{1}{l}{N. } & Categories/bin breaks & Comments \\ \midrule
Age/sex & 12 & (16,20] (20,25] (25,35] (35,55] (55,100] & Female and male categories, in employment (excludes full-time students) \\
Mode & 11 & mfh       metro     train     bus       moto      car.d     car.p
taxi      cycle     walk      other & “Main mode of travel to work” (no data on variability of mode choice) \\
Distance & 8 & (0,2] (2,5] (5,10] (10,20] (20,30] (30,40] (40,60] (60,250] &
Euclidean distance between respondents' home postcode and their main place of work (does not capture multiple work destinations) \\
NS-SEC & 9 & NS-SEC 1.1, 1.2 2, 3, 4, 5, 6, 7 and other & Classes range
from higher managerial (NS-SEC  1.1) to routine occupations (NS-SEC 7) --- see
\citep{chandola2000new} and on the ONS website
(\href{http://www.ons.gov.uk/ons/guide-method/classifications/current-standard-classifications/soc2010/soc2010-volume-3-ns-sec--rebased-on-soc2010--user-manual/index.html}{www.ons.gov.uk}) \\
\bottomrule
\end{tabular}\end{center}
\label{t:constraints}
\end{table}

% The linking variables: age,
% mode of travel to work, distance travelled to work and socio-economic
% status. Tables \ref{t:link} and \ref{t:zone} show a sample of the
% linking variables at the
% individual level and census cross-tabulations respectively.
The mathematics \citep{Fienberg1970} and code
\citep[Supplementary Information]{Lovelace2013-trs} used to implement IPF are
described in detail in \cref{Chapter4}.
% (See the full paper for full details) % !!! add this !!!
% how the model works. Tables \ref{t:link} and \ref{t:zone} provide samples of
% the raw constraint data, on aggregate and individual levels respectively.
% The spatial microsimulation model works by adjusting a large array of weights ---
% rows corresponding to individuals and columns
% corresponding to the geographic zones under investigation --- iteratively, to
% maximise the fit between simulated and census data. Assuming temporarily that
% only the four individuals represented in Table \ref{t:link} were used,
% constraining by the distance variable in \ref{t:zone} would lead the individual
% with an ID of 2 to be allocated a weight of 914 for zone 1, 665 for zone 2 etc,
% as they are the only person who fits into that category. Clearly, many other
% individuals, with other characteristics would fit into the 5-10 km distance
% category in the entire microdataset, and this diversity is what allows
% the weights to converge towards a single result for each individual-zone
% combination (Fig.~\ref{f:fit-plot}).
% 
% \begin{table}[htbp]
% \caption{Sample of linking variables at the individual level (USd)}
% \begin{tabular}{|r|l|l|r|l|}
% \hline
% \multicolumn{1}{|l|}{ID} & Age/sex  & Mode  & \multicolumn{1}{l|}{Distance
% (km) } &
% NS-SEC  \\ \hline
% 1 & Male, 59 & Car driver & 3 & lower management \\ \hline
% 2 & Female, 51 & Car driver & 9 & higher professional \\ \hline
% 3 & Male, 31 & Car driver & 2 & other \\ \hline
% 4 & Female, 24 & Walk & 1 & lower management \\ \hline
% \end{tabular}
% \label{t:link}
% \end{table}
% 
% \begin{table}[htbp]
% \caption{Sample of linking variable values for zones. The population of the most
% populous category is presented for each variable.}
% \begin{tabular}{|l|r|r|r|r|}
% \hline
% Variable $\Rightarrow$ & Age/sex  & Mode  & \multicolumn{1}{l|}{Distance
% (km) } &
% NS-SEC  \\ \hline
% Area code & \multicolumn{1}{l|}{Males, 35-54} & \multicolumn{1}{l|}{Car drivers}
% & \multicolumn{1}{l|}{5-10 km} & \multicolumn{1}{l|}{lower management} \\ \hline
% E02001509 & 116 & 1616 & 914 & 499 \\ \hline
% E02001510 & 94 & 1430 & 665 & 402 \\ \hline
% E02001511 & 82 & 1467 & 848 & 340 \\ \hline
% E02001512 & 152 & 2280 & 573 & 791 \\ \hline
% \end{tabular}
% \label{t:zone}
% \end{table}
% % Possible discussion of IPF and integerisation here
% % The technique used to allocate representative individuals to each zone,
% % iterative proportional fitting (IPF), results in non-integer weights
% 
% % The spatial microsimulation model works by adjusting an array of weights ---
% % with rows corresponding to individuals in the microdataset and columns
% % corresponding to the geographic zones under investigation --- iteratively, to
% % maximise the fit between simulated and census data.
% % Iterative proportional fitting
% % (IPF) is the technique used to alter the weights \citep{Wong1992,
% % Pritchard2012}.
To ensure the model is working, the simulated micro-data are
aggregated and then compared with census data. Total absolute error (TAE), a
simple and effective goodness-of-fit metric \citep{Williamson1998, Voas2001}, was
calculated after constraining for linking variable and after each complete
iteration (Fig.~\ref{f:fit-plot}). Further validation tests are described in
section \ref{s:valid}.
% % % % % Maybe include this later...
% The model was written in the statistical language R, and is
% described in more detail, with a worked example, in Lovelace and Ballas (under
% review). All the software and code behind the model has been made open source,
% following best practice recommendations for transparency and dissemination of
% scientific analysis \citet{Ince2012}.

\begin{figure}
\includegraphics[width = 6.5 cm]{tae-it-plot}
\includegraphics[width = 6.5 cm]{perc-it-plot}
\caption[Fit between simulated and census data]
{Improving fit between simulated and census data across all 4 constraint
variables outlined in \cref{Chapter4}, as illustrated by decreasing values of
the total absolute error (TAE) (left) and decreases in the proportion of
simulated aggregate cell values that differ from census data by more than 5\%
(right) after each constraint and iteration. The horizontal black lines
represent 0 error and 5\% of cell values, respectively. }
\label{f:fit-plot}
\end{figure}

The weighted data provided by IPF-based spatial microsimulation is bulky
(containing rows even for individuals who contribute very little: whose
weight is close to zero), making many types of analysis more difficult
(e.g.~contingency tables and~Gini Lorenz curves). To tackle
this problem, and provide a single
dataset for analysis using various techniques (e.g.~individual level,
geographic, or agent-based methods), the `truncate, replicate, sample' method
of integerisation was used \citet{Lovelace2013-trs}. Still, the final output
dataset contained 532,130 rows, representing every commuter in South Yorkshire.

\section{Assigning work location}
\label{s:workdes}
%  Original intro: weak
% What is the spatial distribution of energy intensive commuter patterns? What
% types of places are conducive to low-energy modes such as walking and cycling?
% Intuition would suggest an urban-rural divide, and that transport
% infrastructure would influence commuter patterns over space.
% This subsection provides more rigorous methods to investigate the spatial
% correlates of energy intensive commuter patterns.
The spatial microsimulation model results in a large dataset containing
hundreds of individuals for each zone under investigation. For micro level
spatial analysis, origin-destination pairs are needed: simulated
places of home and work need to be geotagged. The simplest solution
to this problem is to allocate all individuals in each zone home coordinates
corresponding to the zone's population-weighted centroid. Likewise, work coordinates
can be set to the nearest employment centre. This method allows
for simple analyses such as the proxy for geographic
isolation presented in Fig.~\ref{fig:dis-msoa}.

Rather than assuming that work centres are always located in the city
centre, a more realistic approach is to acknowledge that a variety of
employment centres exist, and that the relative importance of each varies from
place to place. This is illustrated in Fig.~\ref{fig:sflow}, a ward level
flow diagram of
the work locations of commuters based on the outskirts of Sheffield. Although
Barnsley is the closest city centre to Stocksbridge (see
Fig.~\ref{fig:dis-msoa}), this analysis makes it clear that Sheffield is the
primary non-home workplace.

\begin{figure}
\centering{
 \includegraphics[width = 10 cm]{qgis-image-small}}
\caption[Employment density at the local level in Sheffield]
{Employment density at the local level in Sheffield (n is the number of
employees registered to each zone). These results
were generated by summing all incoming flows to all of Sheffield's 1,744
Output Area (OA) administrative zones. Data provided on a CD, on request from
http://www.nomisweb.co.uk/ .}
\label{fig:sworkdens}
\end{figure}
At an even finer geographical level, it is possible to discern the localities
within each city and ward where people are most likely to work based on UK
census data. This is illustrated in Fig.~\ref{fig:sworkdens}. Although this
level of geographic detail was not used in the final
results due to aggregation
issues,\footnote{The
Output Area flow data presented in \ref{fig:sworkdens} is difficult
to work with for individuals allocated to specific zones, because any
number between 1 and 4 is randomly set as either 0 or 3. This makes
the flow data essentially probabilistic for single Output Area pairs,
hence our limitation to aggregate level analysis of this dataset here.
}
it demonstrates the potential for highly localised work allocation based on
census-derived flow data.

\begin{figure}
\centering{
 \includegraphics[width = 10 cm]{sflows}}
\caption[Flow diagram of commuter destinations from Stocksbridge]
{Flow diagram illustrating popular commuter destinations for citizens
of Stocksbridge. The thickness of the lines is proportional to the number of
people who travel there (for reference, 661 people travel to
the centre of Sheffield --- illustrated by the thickest line ---
and 2036 people work in Stocksbridge --- illustrated by the dot
from which all lines radiate. n = 6,338).}
\label{fig:sflow}
\end{figure}

The analyses presented in both Fig.~\ref{fig:dis-msoa} and Fig.~\ref{fig:sflow}
both greatly oversimplify trip routes. The straight lines underestimate travel
distance, completely ignoring the transport network. A more
realistic method is to randomly allocate each individual to a unique home
location based on
% data on houses, and work
% locations based on commuter flow data and the distance they travel to work
population density (or, potentially, local area classification) and estimate
the route taken using shortest trip algorithms dependent on the mode of
transport used (Fig.~\ref{fig:agent}). This latter method allows for the
calculation of route distances by mode, but is more complex and difficult to
implement over large areas.
%\footnote{?}

%%% This section is completely out of place...
% The results of spatial microsimulation include both continuous and
% categorical variables, posing a visualisation challenge during spatial
% analysis. Energy use, for example, depends on mode and distance travelled, so
% both variables should be seen together. Driving is the most popular form
% of commuting in all areas. Instead, the second most popular form of
% travel --- or lack of travel for those who work mainly from home (MFH) ---
% can be used to indicate variability in mode choice (Table \ref{t:cont}).

% \begin{figure}
%  \includegraphics[width = 13 cm]{map-lines-mode-dis}
% \caption{Second most popular mode of transport and average distance travelled
% for MSOA zones in South Yorkshire (add transport infrastructure and
% settlements).}
% \label{fig:dis-mode}
% \end{figure}
\begin{table}[htbp]
\caption[The 2nd most common mode of commuting compared with other factors]
{Contingency table illustrating the link between 2nd most common mode of
TTW in an area and average values for other variables.}
\begin{center}
\begin{tabular}{lrrrrr}
\toprule
2nd mode & \multicolumn{1}{l}{N. zones} & \multicolumn{1}{l}{Total (\%)} &
\multicolumn{1}{l}{$\overline{D}$ (km) } &
\multicolumn{1}{l}{$\overline{P}car$ (\%)} &
\multicolumn{1}{l}{$\overline{D}ens$ (People/km$^2$)} \\
\midrule
MFH & 18 & 10 & 17.0 & 68 & 31 \\ 
Tram & 4 & 2 & 10.8 & 53 & 179 \\ 
Bus & 95 & 55 & 11.2 & 54 & 106 \\ 
Car (p) & 10 & 6 & 13.5 & 63 & 40 \\ 
Foot & 46 & 27 & 13.2 & 53 & 112 \\
\bottomrule
\end{tabular}\end{center}
\label{t:cont}
\end{table}

These methods of spatial analysis provide great insight into the meaning of
aggregate statistics for groups of individuals at the city level of policy
intervention. However, to gain insight into the impacts of schemes on
individuals and local communities, agent based models may be needed.
In particular, there is great potential to link the work
presented here with relevant agent-based simulation work in the social
sciences (e.g.~\citealp{Gilbert2005a, Gilbert2007}) and
attempts to add a geographical dimension to this work
(see \citealp{Wu2008}).

To this
end Fig.~\ref{fig:agent} presents the simulated route choice of the 18
commuters selected from the spatial microsimulation model, and
contains both socio-demographic and geographic
detail.\footnote{For
example, the simulated car passenger who commutes to
central Sheffield in Fig.~\ref{fig:agent} is 16 years old, is classified as
class `other', and lives in a family that has access to 5 cars. These, and
further simulated details such as income, could, once validated, contribute
towards transport interventions targeting specific commuter groups.
}
% \footnote{This analysis results from 20 randomly selected
% individuals (2 of
% whom worked at home) from the spatial microsimulation model output for
% Stocksbridge and allocated origin-destination points based on known distance
% bands, ward-ward commuter flows, and the population density distribution of
% Stocksbridge.
% }
\begin{figure}
 \includegraphics[width = 13 cm]{agents4}
\caption[Simulated route choice for 20 randomly selected individuals]
{Simulated route choice for 20 randomly selected individuals
from the spatial simulation model. Destinations were determined by 1) subsetting
destination wards by distance from Stocksbridge centre, 2) assigning
probabilities of working in each ward for each distance band (based on flow
data presented in Fig.~\ref{fig:sflow}) and 3) randomly selecting points
within the resulting destination wards. (Workplaces of 3 people who work from
home are not mapped).}
\label{fig:agent}
\end{figure}

The distances travelled along the transport network are clearly substantially
further than represented by simple straight lines. This concept can be
defined formally as \emph{circuity}, the ratio of straight-line distance
to route distance \citep{Ballou2002}. Fig.~\ref{f:circ} illustrates the impact
of the road network on distance travelled. Overall, the route distance
represented in Fig.~\ref{fig:agent} is 223 km, 24\% further
 than the straight-line distance (179 km) for the 17 commutes. As in previous
studies, circuity tends to decrease approximately logarithmically as a function
of distance \citep{Levinson2009}. The spatial microsimulation method holds
great potential for investigating the impact of the travel network, especially
when combined with new tools for batch-processing of shortest-route
algorithms.\footnote{The
analysis conducted one trip at a time, using the
QGIS plugin ``Road Path'' for a simple solution with a user-friendly interface.
(\href{http://docs.qgis.org/2.0/html/en/docs/user_manual/plugins/plugins_road_graph.html}
{http://plugins.qgis.org/} ).
To automate the process, Routino (http://www.routino.org/), PGRouting
(http://pgrouting.org/) or the recently released R package osmar
(http://cran.r-project.org/web/packages/osm) could be used.
The rapid evolution of transport network data and software
provides avenues for methodological advance.
}
\begin{figure}
\begin{center}
 \includegraphics[width = 8 cm]{cirqui17}\end{center}
\caption[Circuity as a function of distance in Sheffield]
{The circuity of the route distance as a function of the
straight-line distance for 17 commuter trips modelled in Stocksbridge.}
\label{f:circ}
\end{figure}

\section{Model validation} \label{s:valid}
% % Internal vs external validity; comparison
Due to the dangers of using incorrect model data to inform policy,
the importance of validation has been emphasised repeatedly in the
spatial microsimulation literature \citep{Holm1987, chin2006regional,
Smith2009,Clarke2010-valid, Ballas2013-4policy-analysis}.
Because the outputs of spatial microsimulation
are by nature detailed and provided at the individual level, validation
is challenging: ``such detailed information is virtually never
available at the disaggregate level for an entire region''
\citep[p.~37]{Ravulaparthy2011}. In fact, one could argue that
if individual microdata were made available at the small area level,
spatial microsimulation would be obsolete.

Researchers using spatial microsimulation have been innovative at
overcoming this `catch 22' situation, using a variety of methods.
In broad terms, there are two types of strategy available:
internal and external validation \citep{edwards2013validation}.
The first of this is relatively
straightforward: the aggregated constraint variables are compared with the
aggregated results of the spatial microsimulation model for the same variables.
In our model, the results of this test were reassuring: the correlation between the
aggregate counts from the census and those generated in our spatial microsimulation
were 0.9989 overall for all 6,920 data points (40 categories by 173 zones).
However, the quality of the fit was better for some constraint variables than
for others: the r$^2$ values for the distance and mode variables were
0.9993 and 0.9983, primarily due to the inaccuracy or our estimates of
individuals who work mainly from home (mfh) (Fig.~\ref{finvalid}).


\begin{figure} \begin{center}
    \includegraphics[width=12cm]{invalid}
 \end{center}
 \caption[Comparison of census and simulated results at the aggregate level]
 {Comparison of census and simulated results at the aggregate level
 for a selection of six categories from the mode and distance constraints.
 The 20 category, for example, refers to the number of people travelling
 10 to 20 km to work.} \label{finvalid}
\end{figure}


This internal validation
result is less impressive when one considers that IPF always converges
towards the optimal result for known constraint variables:
it is the unknown cross-tabulations and
target variables that are the most useful result,
so external validation should, in many cases, be the focus
\citep{Morrissey2008, edwards2013validation}.
Four methods of corroborating spatial microsimulation results with external
data were identified:
\begin{itemize}
 \item Compare simulation results with real spatial microdata.\footnote{Income,
for example, is collected by the Census, but is not disseminated at aggregate
levels, let alone the individual level geocoded data required to validate the
individual level results of the spatial microsimulation model. Access to such
sensitive real microdata limits the applicability of this method.}
\item Collect primary data from specific areas against which the simulated
results can be tested.\footnote{In some cases (e.g.~environmental attitudes)
this may be the only reliable validation option, as the information is simply not
collected in geo-coded surveys.}
\item Compare simulation results at the aggregate level with estimates
from a dataset external to the model \citep{Morrissey2013}.
\item Aggregate-up the small area estimates provided by spatial microsimulation
to compare the results with real data that \emph{is} provided at higher
geographies \citep{Edwards2009}.
% \footnote{To provide another example,
% domestic energy use, which is provided in the Understanding Society
% dataset, is disseminated at the
% LA level by the Department of Energy and Climate Change (Decc), which
% constitute 40 to 50 Medium Super Output Areas (MSOA) combined.}
% \item Compare aggregate level results of the model with census variables that
% were not constrained for. % Include this if needs be, or in thesis.
\end{itemize}
Each of these options was considered for our case study,
but data constraints meant that only one, comparison of aggregate data
on a target variable with a reliable external dataset, was deemed viable.
The target variable chosen for this was income; Neighbourhood Statistics
provides estimates of this at the MSOA level, allowing
for direct comparison with our results (Fig.~\ref{fincome-scatter}). The results show
high levels of correlation ($r^2 = 0.93$) between simulated incomes and official
estimates, although the spread of the values resulting from spatial microsimulation
underestimated the true level of inter-zone variation in average incomes.

\begin{figure}[h*]
 \centering
\includegraphics[width=13.5cm]{income-scatter}
 \caption[Mean equivalised household income from official and simulated data]
 {Scatter graph of mean equivalised household income produced as an
 output from the spatial microsimulation model (y axis) and official estimates
 from the Office of National Statistics for the 173 Medium Super Output Areas
 of South Yorkshire. Maximum and minimum official estimates labelled in blue.}
 \label{fincome-scatter}
\end{figure}

\section{Results} \label{c7results}
The results show that, at the aggregate level, South Yorkshire's commuting
behaviour is comparable to the national average. Nevertheless, the
microdata illustrate substantial
inter- and intra- zone variability. Table \ref{t:sum} illustrates the
cross-tabulations (contingency tables) that are made possible
when spatial microdata are used. Univariate statistics are available on
mode of transport, age and number of cars but the interaction between these
variables remains hidden in aggregated Census data.

\begin{table}[h]
\caption[Summary statistics of the commuting behaviour of in South Yorkshire]
{Summary statistics of the commuting behaviour of individuals in South
Yorkshire disaggregated by mode. (Motorbike, taxi, metro and `other'
modes have been removed for brevity).}
\begin{center}
\begin{tabular}{lrrrrrr}
\toprule
Mode & N.  & \%  & \% National & Age & Distance (km) &
Ncars \\
\midrule
Bus & 31486 & 7.2 & 7.4 & 38.3 & 7.5 & 0.5 \\ 
Car (d) & 268496 & 61.1 & 54.6 & 40.1 & 14.3 & 1.9 \\ 
Car (p) & 38233 & 8.7 & 5.9 & 33.5 & 14.5 & 1.5 \\ 
Cyc & 4498 & 1.0 & 2.6 & 38.3 & 5.0 & 1.1 \\ 
% Metro & 692 & 0.2 & 3.8 & 32.0 & 8.7 & 0.8 \\ \hline
MFH & 45326 & 10.3 & 9.3 & 40.0 & 0.0 & 1.9 \\
% Moto & 4844 & 1.1 & 1.1 & 37.0 & 11.0 & 1.3 \\ \hline
% Other & 692 & 0.2 & 0.5 & 42.8 & 8.8 & 1.5 \\ \hline
% Taxi & 1211 & 0.3 & 0.5 & 38.9 & 8.5 & 0.3 \\ \hline
Train & 5709 & 1.3 & 4.6 & 36.9 & 24.6 & 1.2 \\ 
Walk & 38406 & 8.7 & 9.7 & 36.6 & 3.1 & 0.8 \\ 
Average & - & - & - & 39.0 & 11.3 & 1.6 \\
\bottomrule
%%% sort out walking distance too!
\end{tabular}\end{center}
\label{t:sum}
\end{table}

Beyond illustrating the capability of spatial micrsimulation to provide
estimated cross-tabulations of aggregate level data,
Table \ref{t:sum} also provides substantive information
about commuting patterns that could be applied to transport policy:
\begin{itemize}
 \item Cars dominate travel to work in South Yorkshire, to an
even greater extent than in England as a whole.
\item The dominance of cars is even greater when measuring travel
to work in terms of distance travelled: car commuters travel on
average further than all other types of commuters bar those who commute
by train.
\item There are also substantial differences in the age profiles of
different commuting modes: walking, which is often associated with older
members of society, appears to be more prevalent amongst the young. Bicycle commuters,
who are sometimes stereotyped as young \citep{Daley2011}, are not much younger
than the average. Car drivers and home workers tend to be slightly older.
\item Car ownership, which is seldom factored into transport policy assessments,
\citep{Kay2011}
varies with the mode of travel to work. Those who catch the bus or walk are least
likely to own a car, while a those who drive to work or work from home own on
average almost 2 cars per household.
\end{itemize}


As in England as a whole, it is clear that cars, in round numbers, constitute
70\% of trips (61\% of commuters drive to work; 9\%
are passengers in other peoples' cars). The utility of the individual level
results is illustrated at this aggregate level by observing differences in
average age and distance of commute between modes: car drivers and bus
passengers are on average older than those who walk to work.
Unsurprisingly there are also differences in the average distance
travelled. Train passengers travel 13 km further than average;
those travelling by bus or non-motorised modes tend to live closer
to home.
A predictable, yet rarely investigated, result from Table \ref{t:sum},
is the high variability in the average number of cars in households of different
types of commuters: bus passengers appear to have the fewest cars per household
of all modes. Each model result has the potential to inform policy. The final one,
for example, provides support for the argument
 that public transport policies are currently failing to
 ``lure car users out of the car''
\citep[p.~193]{Davison2006}.\footnote{As
with the
other non-constrained variables target variables described in Table
\ref{t:indata}, this model result should be validated by additional data before
strong conclusions are drawn.}
% Cite the sdc article here !!!

From this, total distance travelled and
energy use by mode per year can be calculated.
Fig.~\ref{fig:proportions} presents these model
results (of which distance is most robust, as it is constrained by Census data)
for the average and range for all 694 MSOA zones in Yorkshire and the Humber.

\begin{figure}[h*]
 \centering
\includegraphics[width=10cm]{proportions}
 \caption[Proportion of trips, distance, and energy by mode]
 {Proportion of trips, distance, and energy use accounted for by
different commuter modes. The error bars represent the range of values within
MSOA areas in Yorkshire and the Humber.}
 \label{fig:proportions}
\end{figure}

The proportion of energy
used by cars for transport to work is 95.6\%: \emph{this is more than 20 times
the energy costs of all other modes of transport put together}.

An illustration of the increasing dominance of cars as one moves from trip
number, through distance travelled, and then energy use metrics, is provided in
Fig.~\ref{fig:proportions}. Note that in some regions car drivers account for
less than a third of all commuter trips. Yet in terms of energy use, cars
consume more than 85\% of all energy consumed for getting people to work and
back.

The results show a strong relationship between location and
 distance travelled.
The role of location, and distance to employment centres
more specifically as a cause of distant
commutes was explored using travel to work (TTW) zones, defined by the Office
for National Statistics at the wider regional level of Yorkshire and the Humber
(Fig.~\ref{fig:map1}).\footnote{The
wider regional level of analysis of
Yorkshire and the Humber (see Fig.~\ref{f:scales}) was used in this case
because TTW zones are large: only 3 are found in South Yorkshire
(Fig.~\ref{fig:dis-msoa}), so a larger area is useful to see the overall
pattern. Travel to work zones are defined as ``zones with a self- containment of
at least 75\% (which is to say that less than 25\% of those who work in an area
live outside it, and less than 25\% of the employed residents of that area
commute to workplaces outside the same area)'' \citep{Coombes1982}.
}
\begin{figure}[h]
 \centering
 % \includegraphics[width=14cm]{dismap2}
 \caption[Average distance travelled to work in Yorkshire and the Humber]
 {Average distance travelled to work in Yorkshire and the Humber by MSOA
zone. Black lines represent TTW zones.}
 \label{fig:map1}
\end{figure}
Fig.~\ref{fig:map1} shows that MSOA areas located in and
around the conurbations surrounding Bradford, Sheffield and Hull tend to have
low average commuter distances, while rural locations such as the North
York Moors are associated with long average commutes. This result differs
from that of suburban USA (where urban sprawl accounts for high commuting costs
even within major conurbations), but it is hardly new or surprising
\citep{Marshall2008, Sexton2011}.
An unexpected result is the tendency of city centres to be associated
with high average commuter distances. This can be seen in red patches surrounded
by a sea of green in the centres of Bradford, Leeds, Scarborough and Sheffield.
(One hypothesis to explain this is as follows: some city centres attract
wealthy individuals, who tend to commute further, often by train.)
Energy costs are directly proportional to distance travelled for all
modes. It is therefore unsurprising that average energy cost of commuter trips in each area
are closely related to the distance of commute (r = 0.97).
Distance is the most important driver of energy costs at the MSOA level within
Yorkshire and the Humber; the correlation between average distance and average
energy use per commuter trip is 0.97.
The
geographical causes of energy intensive commuting are therefore the same as the
causes of high average commuter distances at the MSOA level.

\begin{figure}
\begin{center}
  \includegraphics[width = 13 cm]{distances-msoa2}
\end{center}
\caption[Average distance to employment centre in South Yorkshire]{Average
distance to employment centre in South Yorkshire. The
left-hand map illustrates how distance was calculated (using the command
 nncross() in the R package `spatstat'). The right-hand map illustrates
the results --- Sheffield and Rotherham are grouped together in the same
travel to work zone.}
\label{fig:dis-msoa}
\end{figure}

To explore this link further, the average distance from employment
centre
was calculated (\ref{fig:dis-msoa})
and plotted against the average energy cost of transport to work in each MSOA,
see dots in Fig.~\ref{fig:dis-e}. The reversal of slope in the
tick-shaped curve of the relationship between distance to employment centre and
energy use suggests that the link between these variables is not as simple as
one might expect: other factors are at play, possibly linked to
individual level variables such as income.

\begin{figure}[h*]
 \centering
\includegraphics[width=12cm]{Dist-vs-Et2}
 \caption[Scatter plot of distance vs energy costs in Yorkshire and the Humber]
 {The relationship between distance to employment centre and average
energy costs of commute for MSOAs in Yorkshire and the Humber. The blue and
black lines are smoothed moving quantiles (Q1 and Q3 represent the 25$^{th}$ and 75$^{th}$
percentiles respectively), which indicate central tendency and heteroscedasticity.}
 \label{fig:dis-e}
\end{figure}

Spatial microsimulation allows one to `drill down' to the individual
level, target specific groups and model who (in addition to where) is most
likely to benefit from specific interventions. Table \ref{t:msim-res}, for
example, shows simulated differences in commuting patterns between high
and low income citizens in South Yorkshire as a whole.\footnote{The categories
``very poor'' to ``affluent'' used here are defined in \citep{Ballas2005b}.
Statistical bins are defined as proportions of the median income, with
breaks at 50\%, 75\%, 100\% and 125\% of the median \citep[p.~91]{Ballas2005b}.}
Because the individual
microdata are also geocoded, the same analyses could be conducted for
specific zones.
Table \ref{t:msim-res2} illustrates how the results of spatial microsimulation
allow inter- and intra-zone analysis to be combined. Table \ref{t:msim-res2}
indicates that Sheffield028 (an MSOA zone) is more unequal in terms of
income and distance travelled to work than Stocksbridge (a statistical Ward)
(see Fig.~\ref{fig:agent} to see their respective locations).
These results, which can be compared with the regional data presented
in Table \ref{t:msim-res}, or re-calculated for smaller zones, are thus
(to the extent that administrative boundaries allow)
`frame independent' \citep{Horner2002}.

To further explore differences in intra-zone inequality,
commuter work travel distances were plotted as Lorenz curves
(Fig.~\ref{fig:cat-vars}b).
These provide further insight into commuter patterns in each of the zones
described in Table \ref{t:msim-res2}, and illustrate that a small proportion of
the population living in Crookes accounts for a large part of the average trip
distance. Stocksbridge, by contrast, has a more even distribution of commuter
patterns.

Regarding the categorical target variables described in Table \ref{t:indata},
the results imply that wealthy commuters in South Yorkshire drive larger cars,
use the internet more frequently, and may be less likely to want to move
than those with low incomes (Fig.~\ref{fig:cat-vars}a).

\begin{table}[h*]
\caption[Contingency table of variables related to commuting]
{Contingency table of average values for continuous variables related to commuting,
cross-tabulated by income bands, based on the spatial microsimulation
model for South Yorkshire (n = 531,282).}
\begin{center}
\begin{tabular}{lrrrrrr}
\hline
Income group & \multicolumn{1}{l}{Proportion} & \multicolumn{1}{l}{Age} &
\multicolumn{1}{l}{Dis (km)} & \multicolumn{1}{l}{N.cars} &
\multicolumn{1}{l}{Income (\pounds/yr)} & \multicolumn{1}{l}{N.child} \\
\midrule
v.poor & 10\% & 38 & 5.8 & 1.2 & 5519 & 0.9 \\ 
poor & 18\% & 39 & 8.1 & 1.2 & 10158 & 1.0 \\ 
below.av & 22\% & 39 & 8.3 & 1.4 & 13974 & 0.8 \\ 
above.av & 18\% & 39 & 8.9 & 1.6 & 17902 & 0.6 \\ 
affluent & 32\% & 40 & 16.5 & 1.9 & 29448 & 0.5 \\ 
\end{tabular}\end{center}
\label{t:msim-res}
\end{table}

\begin{table}[h*]
\caption[Commuting characteristics cross-tabulated by income bands]
{Contingency table of average values for continuous variables related to commuting,
cross-tabulated by income bands, based on the spatial microsimulation
model for the Ward of Stocksbridge (n = 6,338) and MSOA Sheffield028, which
corresponds to Crookes (n = 2,470).}
\begin{center}
\begin{tabular}{lrrrrrr}
\toprule
Income group & \multicolumn{1}{l}{Proportion} & \multicolumn{1}{l}{Age} &
\multicolumn{1}{l}{Dis (km)} & \multicolumn{1}{l}{N.cars} &
\multicolumn{1}{l}{Income(\pounds/yr)} & \multicolumn{1}{l}{N.child} \\ \midrule
\multicolumn{7}{c}{Stocksbridge (13 km from centre)} \\ \midrule
v.poor & 10\% & 39 & 9.5 & 1.2 & 5886 & 1.0 \\ 
poor & 21\% & 38 & 12.3 & 1.0 & 10571 & 0.9 \\ 
below.av & 19\% & 39 & 12.3 & 1.5 & 14560 & 0.7 \\ 
above.av & 20\% & 39 & 12.9 & 1.8 & 18513 & 0.5 \\ 
affluent & 30\% & 40 & 17.1 & 2.0 & 29198 & 0.5 \\ \midrule
\multicolumn{7}{c}{Crookes (2 km from centre)} \\ \midrule
v.poor &  10\% & 32 & 4.0 & 1.1 & 5208 & 0.8 \\ 
poor & 16\% & 33 & 5.7 & 0.9 & 9972 & 0.9 \\ 
below.av & 23\% & 31 & 7.4 & 1.1 & 14145 & 0.5 \\ 
above.av & 14\% & 34 & 8.7 & 1.5 & 17914 & 0.5 \\ 
affluent & 37\% & 36 & 25.0 & 1.8 & 29932 & 0.4 \\
\bottomrule
\end{tabular}             \end{center}
\label{t:msim-res2}
\end{table}

\begin{figure}[h*]
 \centering
\includegraphics[width=8.2cm]{cat-vars}
\includegraphics[width=8.2cm]{Lorenz3}
 \caption[a) Income and household traits; b) Lorenz curves of commute distances]
 {\textbf{a)} Variability of vehicles (proportion of primary cars in household
whose engine size is 2.0 litres or more), internet use (proportion of
commuters who use the internet daily or weekly) and desire to move home
depending on equivalised income. These categorical target variables are
described in Table \ref{t:indata}. \textbf{b)} Lorenz curves illustrating the
individual level variability in
commuter distances for 3 zones. The Gini indices associated with these curves
are 0.278, 0.294 and 0.305 for Stocksbridge, South Yorkshire and Sheffield028
respectively.}
 \label{fig:cat-vars}
\end{figure}

\section{Discussion}
\label{discuss-jtrg}
This chapter has demonstrated how spatial microsimulation can be
used to model commuter patterns in concrete case study.
Whole individuals from a detailed national survey were
allocated to geographic zones at various levels; this provided further insight
into intra-zone variability of commuting than is available from the use of
aggregated census data alone. In addition, the careful selection of target
variables not included in the census provided insight into the relationships
between commuting behaviour and a variety of `target variables' such as income,
internet use, desire to move home, type of car and number of children.

From the perspective of data-constrained policy makers, these results are attractive:
they provide a level of detail that is inaccessible for analyses
based on geographically aggregated census data alone. The ability
to explore the commuter behaviour of subsets of individuals based on age,
distance travelled and class (constraint variables) or other variables
including size of car or income (target variables) will be useful in
various applications:
being able to simulate the \emph{characteristics} of commuters who are most
likely to benefit from certain interventions and identifying \emph{where} these
people live and work clearly has huge potential for transport planning and policy.
To illustrate the point, the
distribution of low-income households reliant on buses can be simulated and mapped
at the county level to help inform the location of new bus routes (Fig.
\ref{fig:busmap}). For example, if this type of analysis had been properly
conducted and validated during the planning stages of the recently implemented
rapid bus routes in Albuquerque mentioned in \citet{Tribby2012},
the system could have been
designed such that low income residents benefited from faster access to the city
centre. In fact, relatively wealthy households (who probably have more transport
options already) benefited most from the scheme \citep{Tribby2012}. This
illustrates the importance of considering not only aggregate level impacts, but
also taking into account the local and micro level distributional effects of
intervention.

\begin{figure}[h*]
 \centering
% \includegraphics[width=13.5cm]{busmap}
 \caption[Low-income car-free families and bus-stops in South Yorkshire]
 {Proportion of population which earns less than 50\% of South
Yorkshire's median income \emph{and} lives in a car free household within the 173 MSOA
boundaries of the metropolitan county, according to the spatial
microsimulation model. Translucent red dots represent bus
stops (data from {\color{blue}\href{http://data.gov.uk/dataset/nptdr}{data.gov.uk/dataset/nptdr}}).}
 \label{fig:busmap}
\end{figure}

The spatial microsimulation approach to modelling commuter
patterns outlined in this section provides a foundation for investigating such
effects.  In addition, it has been shown that spatial microsimulation methods
can enrich transport models with policy relevant socio-economic variables at
individual and small-area levels.
% In broader terms,
% the  promotes a closer collaboration between the fields of
% transport modelling, spatial microsimulation and spatial microsimulation.

Despite these possibilities, it is important to remember that the results are
\emph{simulated}. Consequently, 
linking variables --- these
are constrained by known census aggregates and are therefore trustworthy ---
must be distinguished from target variables, which are more tentative
estimates based on correlations between target and linking
variables at the national level. Target variable
estimates rely on an often unstated assumption: that the relationships
between variables at the national level (e.g.~between distance travelled to
work and income) tend to remain at local levels. This assumption cannot be
expected to hold everywhere, so results arising from target variables are
expected to underplay the true level of spatial variability. Where possible,
target variable results should be corroborated against independent datasets
(\citealp{Edwards2009}).

Many transport interventions have wide-ranging impacts on commuters.
These depend on geographical \emph{and} individual level factors,
and the importance of the latter especially is often overlooked in transport policy
 (e.g.~\citealp{Tribby2012}).
The micro level methods presented in this chapter therefore have great potential,
to enable researchers and transport planners to better model and predict the
impacts arising from various interventions.
With the current focus on energy and sustainability in transport
\citep{Chapman2007}, there is a risk that distributional impacts continue
to receive little or no attention. Spatial microsimulation
has the potential to address this issue, by helping
decision makers to design sustainable transport measures
that are both effective and fair.
 % Conclusion

\input{./Chapters/Chapter8} % Conclusion

% Chapter 9

\chapter{Conclusions} % Write in your own chapter title
\label{Chapter9}
% \lhead{Chapter 9. \emph{Conclusions}} % Write in your own chapter title to set
\fancyhead[RO,LE]{Chapter 9. Conclusions} %2side
\fancyhead[RE,LO]{\thepage}
% - What has been done?
% - What has been learned?
% - What can the new methods provide?
% - What are the political implications of this study?
% What does this thesis add to %!!!
% the rapidly growing and diverse field of transport and energy?
This thesis has investigated the energy costs of commuting and how they vary
between people and over space.
Motivated by the major problems of climate change, peak oil and
social inequality, the research set out to offer
evidence, and tools, to policy makers tackling these issues in the realm
of personal travel.
To complete the task, the methodology had to provide insight into the
spatial distribution of commuter energy use,
inequalities in its social distribution and the likely social and spatial
impacts of different intervention options.
Based on reviews of previous transport studies (in \cref{Chapter2})
and individual-level methodologies (\cref{Chapter3}), it was decided that
a \emph{spatial microsimulation approach} was most appropriate,
due to the maturity of the techniques involved, flexibility of application and
ease of use.
A spatial microsimulation model was developed and tested, building on previous
work and implemented in the free and open source programming
language R (\cref{Chapter4}). The model
was used to combine geographically aggregated count data from the UK's 2001 National Census
with individual-level data from the national Understanding Society dataset,
resulting in simulated \emph{spatial microdata}: individual
records which have been selectively sampled based on `constraint variables' shared
between the individual and aggregate level datasets.

Spatial microdata form the foundation of the spatial microsimulation approach.
Yet it is during the subsequent \emph{analysis} of this spatial microdata that
value for decision makers is generated: the interrogation of spatial microdata enables
calculation of energy costs at high geographical resolution (\cref{sindvar}),
analysis of social and spatial inequalities in the distribution of this energy
use (\cref{Chapter7}) and the development of quantitative
`what if' scenarios to model the impacts of change (\cref{Chapter8}).
Thus the spatial microsimulation approach developed here
includes not only the generation of spatial microdata but
analysis, visualisation, testing and modelling also.
% To produce estimates of energy use at the
% individual-level and at small geographical scales a spatial microsimulation
% model was developed. This model

This thesis provides, for the first time,
estimates of the energy costs of commuting at a range of geographic scales in
the UK, and an exploration of its social and spatial
variability. Some of the methods
used to achieve this result are already well established. What is new
methodologically is the way that these methods, and datasets on which
they depend, have been integrated with one another in novel ways
to provide results that are reproducible
and consistent regardless of the scale of analysis.

This chapter summarises what has
been learned during the course of the research project:
methodological  contribution (\cref{smethcont}), its policy relevance
(\cref{sprel}) and the central findings (\cref{sumfind}). The research opens many
new pathways for further research which are discussed in \cref{sfurther}.
Finally, the thesis is evaluated in terms of the original aims and objectives,
in \cref{ssummary}. It is worth reflecting on the conclusions in the
context of the two
main aims of the thesis, introduced in \cref{s:aims}:
\begin{itemize}
 \item[A1] Investigate the energy cost of transport to work, its variability
at individual and geographic levels, drivers, and policy implications.
  \item[A2] Explore and evaluate the potential of spatial microsimulation
models for the social and spatial analysis of the energy costs of commuting.
\end{itemize}

\section{Methodological contribution} \label{smethcont}
The main methodological contribution of this thesis is the application of spatial
microsimulation to the social and spatial analysis of the energy costs of commuting.
It is concluded that commuting
research is an area that can benefit from this increasingly accessible
technique. Individual-level analysis is becoming the norm in
transport modelling (\cref{Chapter3}) but often these
omit distributional impacts of new policies.
From the geographical literature, the vast majority of analysis into the spatial variability
of transport energy use and commuting patterns operates solely at aggregate levels.
Spatial microsimulation has several practical advantages over these
aggregate approaches, enabling outcomes that are otherwise inaccessible.
% Three central advantages of the approach, from the perspective of
% evaluating the energy costs of commuting patters from UK census data, are as follows:
More specifically, the four central methodological achievements of the
work are as follows:
\begin{itemize}
 \item The development and testing of algorithms to `integerise' the 
 weight matrices generated by iterative proportional fitting, allowing analysis
 to be conducted on whole individuals rather than fractions of individuals
(\cref{s:integerisation}).
 \item The calculation of energy costs per commuter trip in zones for which
distance/mode cross tabulated count data are unavailable (e.g.~output area levels)
from official sources.
 \item Insight into the intra-zone variability of commuting energy costs and
the links between commuter energy use and other socio-demographic variables,
based on analysis of spatial microdata.
\item The manipulation of this dataset to achieve goals outside the reach of
aggregate-level studies, such as the targeting of specific groups in what-if
scenarios of the future, and assessment of the distributional impacts of
localised transport interventions.
\end{itemize}
Each of these points highlights the advantages of the spatial microsimulation
to analysing the energy costs of commuting and modelling travel to work.
Although spatial microsimulation has not been used to generate every energy
cost estimate presented in this thesis (it has been demonstrated that per trip
energy use can be estimated based on geographical data that provides
mode/distance cross-tabulations), the approach has been critical to achieving the four
outcomes listed above. These are arguably the most important outcomes
from a policy and methods perspective, hence the title of this thesis as a
\emph{spatial microsimulation} approach. During some sections (the
national-level results presented in parts \cref{Chapter6} and \cref{Chapter8}),
a simpler `spatial approach' has been used to assess energy costs.
Yet, as illustrated in \cref{smshift}, the two approaches are not incompatible.
On the contrary, the scenario of modal shift shows that aggregate-level
analysis can be useful for a rapid assessment of the basic determinants of
change (in this case mode and distance categories) and for generating
national level results (which would be overly resource consuming
using spatial microsimulation). The progression from aggregate to micro-level
undertaken in this scenario illustrates the benefits of using a micro-level
approach in tandem with preliminary aggregate-level analysis.
The individual-level implementation of the scenario, based on spatial microsimulation,
allowed greater sophistication: new variables
(age and distance as a continuous variable in this case)
were taken into account when estimating the extent of modal shift;
the results were displayed at a higher resolution,
and information about the socio-demographics of those affected was generated.
% Do this!!!

In the process of moving from an aggregate to a micro-level model of
modal shift many new possibilities were opened up, not
all of which were implemented (\cref{stfurther}). The decision to commute,
how far and by what mode, is ultimately determined
by individuals (\cref{s:commuting}), so a micro-level approach makes sense in theory.
Of course, transport infrastructure and other geographic factors also have a major
influence, and the spatial microsimulation approach would enable the interaction
between geographical and individual-level factors to be included.
The reason for choosing the topic
were not only academic, but related to issues that
require an urgent policy response. Policy-makers often lack the tools and skills needed to
evaluate which policies would actually work to reduce energy
use and emissions, let alone at local levels and taking consideration
of the social distribution of these changes \citep{Banister2008, Tribby2012}.

In light of the evidence presented throughout the thesis, the kinds of question that
the spatial microsimulation approach helps answer
seem to be precisely those that policy makers should be asking before implementing new
strategies to meet climate change targets in fair way. Will the policy work?
Are there more effective alternatives? and which types of areas will be most
affected, and is this fair? The thesis cannot answer these questions in general terms,
but the results show that the 
methods can provide important evidence to aid the evaluation process, if the policy options are
clearly defined. The policy relevance of this work is one of its major strengths.

\section{Policy relevance and limitations} \label{sprel}
Climate change, resource depletion and standard of living provide the underlying
motivation for this research. One of the broad conclusions is that
methods of calculating energy costs of everyday activities are
highly relevant to policy makers concerned with sustainability. The
`sustainable mobility' paradigm requires new tools of assessment
as well as new concepts if it is to move out of pure academic discussions and
into practice around the world \citep{Banister2008}. In this respect, the
research presented in this thesis has much to offer. Too often, academic
research into the energy and climate impacts of transportation operates solely
at the level of entire nations or regions (\cref{s:energy}). Yet actual
transport
policies are often implemented locally.\footnote{The recently announced
\pounds77 million funding to promote cycling in cities and
national parks has been allocated to 7 specific urban areas and particular
routes within 4 national parks
\citep{RimeMinister'sOfficea}. \pounds 20 million of this funding is
allocated to Manchester alone, for 56 km of new cycle paths, amongst other
facilities. The question of where to invest these funds for the greatest social
and environmental benefit is of great policy importance.
}
The spatial microsimulation approach can help bridge such a `scale gap'
between academics and practitioners, by making individual and local-level
analysis of personal travel patterns accessible.

% Granted, not all transport
% planners or other local decision makers can be expected to tailor dozens of
% lines of R code to their particular requirements. And granted, commuters will
% only represent a fraction of total traffic, let alone energy use, in any
% particular area. % Scrapped by dad
Not all local transport policy makers will have the time, skills or
desire to apply the methods advocated in this thesis to their local areas
and problems. However, some may be prepared to use techniques, with potential gains
in their ability to evaluate different scenarios of change. Would
increasing the cycling rate have greater impacts in location A or B? This kind
of question can be answered using the simple what-if scenarios presented in
\cref{Chapter8}, and refined to provide insight into the distributional impacts
using spatial microdata.

The spatial microsimulation approach is not without limitations: it is
it is complex,\footnote{Spatial
microsimulation is complex relative to
simplistic cost-benefit scenarios, but \emph{not}
compared with some transport models currently used in local government such as
SATURN \citep{SATURN2012}.
}
requires specialist knowledge to implement and produces simulated results that
may be prohibitively expensive to verify. For these reasons, it has been
emphasised that spatial microsimulation results should build on, rather than
replace, simpler aggregate level analyses for corroboration. There is a real
danger that, without proper understanding of the assumptions on which spatial
microsimulation is based, the approach could lead to incorrect interpretation
of results or, in worst case scenarios, fudging of results for political
purposes \citep{Openshaw1978}.
For this reason the reproducibility of the method and results is of utmost
importance if spatial microsimulation does
become widespread for evaluating real (and not just hypothetical)
interventions in transport systems. Following best practice guidelines
\citep{Peng2006}, government or private analyses can be made both transparent
and reproducible. Using free, open source and cross-platform programs such as R
can give analyses on which transport decisions are made attributes
vitally important in the democratic system: accessibility and transparency.
% Results generated
% through properly commented code and publicly available data should, in theory,
% be reproducible by anyone willing to learn how the model works.
% Another potential benefit is the use of local analyses in education. 

\section{Summary of findings} \label{sumfind}
Returning to energy in transport, a range of interesting results have been
generated using the methods developed during the PhD project.
No single,
overriding factor that determines commuter energy has been found. In broad
terms the findings presented in \cref{Chapter6}
support the conclusions of past research that energy use in transport is
complex, varies on a range of scales, and appears to be affected by many
factors, especially urban form \citep{Levtnson1997, smith2011polycentricity,
Levinson2012}.
More specifically, it has been found that at the regional level London
is the `greenest' area in terms of commuter energy use, but that this is
partly offset by the surrounding regions which have the nation's most
energy intensive average commute. This finding provides tentative support
to the `compact city' hypothesis \citep{Breheny1995}, but suggests that
the energy use in surrounding areas may be pushed up beyond the average
due to long-distance commuting to concentrated employment centres.

Nationally, it was calculated that
commuting uses 4.1\% of direct energy use in England. Commuting was found to
account for almost 15\% of transport energy use, representing an
important and relatively inelastic contribution to the total.
Individual-level variability was also explored in the
same chapter (\cref{sindvar}). It was found that in urban centres
the 20\% top energy consuming commuters can account for over 90\% of commuter
energy use, a very high level of inequality.

At lower geographical levels, the variability in average commuting energy
costs increases as would be expected, and a clear spatial pattern, in which
urban centres and their direct surroundings have low energy costs compared
with the rural surroundings. However, commuting energy costs still vary
greatly between many areas that are similar `on paper' at the level of
statistical wards (\cref{sregional}). At the local level, the pattern appears
to be more complex still, with a tendency for large city centres to be associated
with above commuter energy costs greater than their surroundings in South
Yorkshire. Later, in \cref{c7results} this finding is replicated in terms
of the relationship between areas' distance to the nearest employment
centre and average energy costs across Yorkshire and the Humber, adding
further evidence to suggest that the compact city hypothesis, in its simplest
form, is over simplistic. 

% Say what you actually found here!!!
In agreement with \citet{Boussauw2010}, the average distance between home and work,
which in itself depends on a range of social and geographical factors,
seems to be the major driver of energy intensive commuting: when distances
are large, the possibilities for modal shift are greatly reduced, and
telecommuting can only be seen as a realistic solution for certain types of
jobs, many of which are out of the reach of the most vulnerable
(\cref{Chapter8}).
% Correlation between distance and energy use! !!!
Further modelling work could contribute to the debate about the factors
underlying transport energy use, providing statistical evidence about the range
of factors at play. But the focus here has been policy, not theory.
To summarise, the most important policy relevant findings are as follows:
\begin{itemize}
 \item Energy use for commuting varies at all geographical levels
 and is distributed highly unevenly between individuals in most zones.
% The gini index of commuting is higher than for income %!!!
 Even between areas
 that appear to have similar levels of energy use at the aggregate level, there
 are great differences in how commuter energy use is divided up between their
 inhabitants (\cref{c7results}).
 \item At the scale of cities,
 there is a tendency for highest energy costs to appear furthest from the city (around
 60 km in the case of London), which tends to fall towards the city centre, but then
 rising again in the city centre (\cref{fig:dis-e}).
 \item At the international level, England appears to have lower per-trip energy
 costs than the Netherlands, despite Holland's reputation for excellence in
 environmentally benign transport planning.
 \item In terms of modes of travel, cars were found to completely dominate the
energy costs of commuting in most areas. This can be easily overlooked based on
existing statistics that focus on modal split by number of trips and distance.
In Yorkshire and the Humber over 95\% of energy use for commuting was found to
be due to cars (\cref{c7results}), implying that environmentally aware policy
makers there should focus on reducing private car use as a priority rather than
the current focus on modal shift. 
 \item The energy impacts of an ambitious scenario of modal shift from cars to
 bicycles would be relatively modest, compared with telecommuting, which is
 rarely framed as a transport policy. Active travel policies need to be supplemented
 by policies encouraging car sharing, reducing demand for long-distance travel and,
 in the long-term, reducing average home-work distances.
\end{itemize}

Each of these findings has implications for transport planning strategies
in the UK in broad terms. Exploring what these implications are on a
case-by-case basis is outside the scope of this thesis, and further exploration
of the most policy relevant overall findings provides a strong
incentive for further work at the local level in different case study areas.
Because of the applied nature of this research, it is suggested that much of
it is conducted by policy makers. In terms of opportunities for
building on the thesis in the academic context, there is also much scope for further
work, as outlined below.

\section{Further work} \label{sfurther}
The work undertaken has provided new contributions to knowledge, both empirical
and methodological. The latter contribution, used appropriately,
could outlast the former: the spatial microsimulation approach has the potential
to generate many more interesting results than are presented in the preceding
chapters. The empirical results also raise important research questions, by
challenging conventional wisdom about the energy costs of commuting and how
these costs can be best be reduced.

It is therefore hoped that the thesis is not seen simply as an `end product' or `final
result' but as a tool for stimulating and enabling further lines of study into
energy and transportation. It is up to other researchers to
judge how best to use the methods for their own purposes, so the concluding
remarks in this section are intended to provide general guidance, rather than a
prescriptive research agenda. It was decided that the following research areas,
in rough descending order of priority, would benefit from further investigation,
building on the methods and findings presented in this thesis:
\begin{itemize}
 \item The use of spatial microdata as an input into agent-based
transport models: the recent advances in microsimulation in urban and transport
models outlined in \cref{s:urbanmodel} make modelling techniques simultaneously
more accessible to transport planners and much more
powerful.\footnote{In this
regard MATSim in particular
seems to hold great promise for `open sourcing' transport modelling for the
evaluation of specific schemes, due to its uptake by US planning authorities.
Yet environmental/energy and distributional impacts are still under-reported in
scheme evaluation. Combining the socio-demographic variables contained within
simulated spatial microdata with models such as MATSim therefore has great
potential to further enhance the use of models for practitioners.
}
Starting from
the other side of the spatial microsimulation versus transport
planning/modelling
divide, the addition of agent-based models with inbuilt capability to load and
interpret the road network (e.g.~from Open Street Map data), has the potential
to vastly improve the ease with which infrastructure interventions can be
assessed by academics already acquainted with spatial microsimulation. This
approach could be far more advanced (and potentially user
friendly) than the crude methods presented in \cref{s:workdes}.
\item Extend the spatial microsimulation methods presented in \cref{Chapter4} so
that they are capable of classifying individuals into family units
(\citealp{Pritchard2012}, see \cref{sreweight}) and allocating their home and
work locations to precise geographical coordinates (as described in
\cref{s:workdes}).
\item Development of more realistic and localised `what if' scenarios: 
the modal shift scenario presented in \cref{smshift} is useful to gauge the
potential magnitude and spatial distribution of cycling uptake in the UK, but
is unlikely to be realistic as the same proportion of short-distance car
drivers are expected to shift in every area. In reality, most transport
interventions are localised. The recent allocation of \pounds 77 million to
cycling cities schemes \citep{BBc2013-cycling}, for example, will inevitably be
spent locally. Localised scenarios of different expenditure options could help
planners maximise the benefits resulting from this expenditure.
\item Prediction of energy use: variation in energy use variable has been
explained intuitively as the result of a few key factors: wealth, distance to
employment centre and the nature of the surrounding transport network all seem
to have an influence (\cref{Chapter6}). The next logical step forward would
be the creation of a predictive model to estimate energy use based on
underlying geographical drivers. This could include flow data
\citep{Simini2012} as well as more conventional explanatory variables such as
topology, wealth and connectivity measures. Such a predictive model would be
useful academically, enhancing understanding of the geographical drivers of
energy use \citep{Steemers2003} and practically, as a basis to project the
energy impacts of future change.
\item The application of the method to more countries at more time periods, to
investigate the generality of the findings and provide further guidance to
policy makers based on the international evidence.
\end{itemize}
This is a diverse set of recommendations that can be explored using a variety
of methods. It is therefore suggested that resulting research does not 
need to fit into the `spatial microsimulation approach' advocated
in this thesis to build on its findings. However, approach
may offer certain advantages as a way of framing the research methodologically.
Returning to the central policy issue of energy use in transport
it is recommend, if an overriding agenda or
paradigm is deemed beneficial at all (it may not be), that future research
in this area uses the sustainable mobility paradigm \citet{Banister2008}.

\section{Thesis evaluation and summary} \label{ssummary}
To evaluate the thesis by its own standards, we return to the aims and
objectives introduced at the end of the opening chapter (\cref{s:aims}), and
discuss to what extent they have been accomplished.  The first aim (A1) was to
``Investigate the energy cost of transport to work, its variability
at individual and geographic levels, drivers, and policy implications.'' This
aim was mostly accomplished in \cref{Chapter6}, in which national commuter
energy costs were estimated in terms of both energy use per trip and energy use
per year per commuter. In the same chapter commuter energy use was also found
to vary at all geographical scales, with the range of average values
unsurprisingly increasing at lower geographies and the spatial pattern becoming
more complex at the local level. In terms of individual-level variability, it
was shown in \cref{sindvar} and throughout \cref{Chapter7} that the distribution
of energy use across the population varies greatly from place to place and that
socio-economic factors play an important role in determining an individual's
use of energy to travel to work that is likely to be missed in analyses that
operate only at the aggregate level. 

Sub aims 1.1, 1.2 and 1.3 relate to the variability of commuting energy costs;
the factors most closely associated with high and low energy use; and how the
spatial microsimulation approach can be used to inform policies using scenarios
of scenarios of change, respectively. The following bullet points summarise
progress in achieving these aims:
\begin{itemize}
 \item The quantification of the variability of commuter energy costs at various
levels has been a major output of the research, as detailed above. However, the
variability over time has received less attention due to data
constraints.\footnote{The
observation that energy costs have increased tenfold
over the past century (\cref{s:eff-imps}, \cref{flongip}) was based on a small
sample and crude assumptions about average distances travelled by, and
efficiencies of, different modes of transport. Still, this is an interesting
result. Also, the changing distribution of car dominance for the trip to work,
illustrated in \cref{f1981}, is an interesting finding that likely relates to
changes in the spatial distribution of energy
intensive commuting over time.
}
Aim 1.1 was also to investigate household-level variability. This has not been
achieved in the thesis, although pointers of how to do this have been
suggested.\footnote{See
the second bullet point in the list of further research
in the previous section.
}
\item The explanation of this variability set out in aim 1.2 was largely
achieved. At the aggregate level, distance from employment centre was found to
account for much of the variability in average commuter energy use, although
this was not formalised as a predictive model or linked to additional
geographical factors such as the road network. At the individual level it has
been shown that average commuting behaviour also varies depending on age,
number of cars in household and, more importantly for policy makers, by
socio-economic class and income (\cref{c7results}).
\item Regarding the formulation of models for change (Aim 1.3), a number of
`what if' scenarios were considered in \cref{Chapter8}. Only 2 of
these (high cycling and telecommuting scenarios, based on evidence from Holland
and Finland) were quantified, but the results were interesting, policy relevant
and surprising. As stated in the previous section, there is great potential for
further research in this area.
\end{itemize}
The second main aim was methodological, to test the potential of spatial
microsimulation for the ``social and spatial analysis of the energy costs of
commuting.'' It is concluded that the thesis has succeeded in meeting this aim:
spatial microsimulation has for the first time been applied to the
investigation of this issue and the methodology has been developed in a way
that should be reproducible by others based on code and documentation that has
been made available to others.\footnote{In
the
{\color{blue}\href{https://github.com/Robinlovelace/thesis-reproducible}
{`thesis-reproducible'}} repository
and other personal repositories hosted on the social coding site github.com
}
It is also concluded that the benefits of using the spatial microsimulation
approach outweigh the additional complexity, computing and time costs of the
individual-level methodologies compared with more common aggregate-level
approaches. The ability to target specific groups in scenarios of change, to
explore the interaction of individual and geographical factors in influencing
travel behaviours and to investigate the distributional impacts of change
suggests the approach has great potential as a tool for policy makers and
academics. Overall the thesis has achieved most aspects of all of its original
aims, although further work is needed to include household-level impacts and
better explain the variability of energy use based on a wider range of
variables than those used here.

In summary, this thesis has contributed methods and findings to the emerging
area of energy use in transport. The research was motivated by the seemingly
intractable socio-environmental problems of climate change and resource
depletion, leading to a focus on pragmatic policy relevance rather than theory.
The methodological innovations of integerisation and allocation of home-work
locations in the context of spatial microsimulation are relatively minor
achievements academically, yet their application to real-world transport
planning decisions could yield major benefits for policy makers.

Some of the findings were unexpected and challenge conventional wisdom about
what constitutes `good' transport policy environmentally. The current
emphasis on bicycles, for example, is at-odds with its relatively
minor potential for large emissions cuts (although health and social
considerations should also play their part in transport policy, areas
in which the bicycle has much more to offer).
The key message for policy-makers wanting to reduce fossil fuel dependence
is that policies that can reduce the consumption
of the most energy intensive areas and individuals (such as telecommuting)
should take priority over policies that will further reduce energy use in
places that are already quite energy efficient in terms of travel to work.
This finding was reinforced by the comparison between 
commuter energy use in England and the Netherlands, where the Dutch were
unexpectedly found to use \emph{more} energy for commuting. 

These findings not only challenge wishful thinking in the area of energy and
transport, they lay the
foundations for further work from which additional results can be
generated. The findings are also important in their own right: they
provide insight into the interventions that would be needed if reducing energy
use in personal transport
becomes a political priority. The impacts of this research may thus
depend more on the extent to which the approach is adopted by practitioners,
than its direct influence in
academia. In terms of social and environmental impact, a single well-designed
intervention in the transport system resulting from this research could be worth
several thousand words.

% The methodological `black boxes', that process the
% input data and generate results may only be seen as a means to an end, but they
% are fundamental to the results and how reality is represented. It is therefore
% important that they
% are critically placed in the spotlight.
% % Reproducible research is one of the
% % cornerstones
% % of scientific advancement, so this is probably
% % the most important chapter in the Thesis from an academic perspective.
% % Others can build on the analysis and take it further.
% Too many times have researchers had to `start from scratch', to implement
% methods that are already widely used. This problem is accute in Transport
% Modelling, where a small number of proprietary (and extremely expensive)
% software packages dominate the market, reducing potential for transparency,
% reproducibility and inter-organsiational collaboration
% \citep{Tamminga2012}. These problems can largely be overcome
% by an open source approach, to both data and methods \citep{Ince2012}.
% Therefore, in addition to presenting the methods clearly and concisely,
% a sub-aim of this chapter is to ensure that all of the results are reproducible
% based only on data referred to in this thesis. % Conclusion

%% ----------------------------------------------------------------
% Now begin the Appendices, including them as separate files

\addtocontents{toc}{\vspace{2em}} % Add a gap in the Contents, for aesthetics

\appendix % Cue to tell LaTeX that the following 'chapters' are Appendices

\addtocontents{toc}{\vspace{2em}}  % Add a gap in the Contents, for aesthetics
\backmatter

%% ----------------------------------------------------------------
\label{Bibliography}
% \lhead{\emph{Bibliography}}  % Change the left side page header to
% \fancyhead[LO,RE]{\emph{Bibliography}}

\fancyhead[RO,LE]{Bibliography} %2side
\fancyhead[RE,LO]{\thepage}
\fancyfoot{}

\bibliographystyle{model2-names}  % Use the "unsrtnat" BibTeX style for
% \bibliography{library, lincluded}  % The references (bibliography) information are stored
\bibliography{custom,link-geo,link-pstar,library}  % The
% Replace w. link-pstar or link-ps depending


\addtocontents{toc}{\vspace{2em}}  % Add a gap in the Contents, for aesthetics

%% -----------------------------------------------------------

\printindex
\label{index}
\phantomsection
\addcontentsline{toc}{chapter}{Index}
\end{document}  % The End
%% -------------------------------------------------------
